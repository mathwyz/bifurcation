\section{分布问题}

我们仍用\emph{如}表示点\emph{工}=0的V次迭代,并且令
岡=4立如 (4.1)
在本节中,我们讨论四的渐近性质.更确切地讲,我们将证明下面 的基本定理.
\emph{定}理4.1 对于前面定义的丄中几乎所有的参数,序列
仏广的"、极限产是鲍对连续的:毎=Mr,其中〃,对所
有\emph{P\textless{}2}成立.

附注4.2这个定理表明,测度产不是很特殊的,并且临界点
的轨道关于测度``遍历,为证明这一定理,我们将依照§ 3中所得
到的基本性质分步进行讨论.我们首先讨论在广上的分
布,然后讨论在[---\emph{i,U\textbackslash{}r}上的分布,由后面的讨论我们知道,
\textsc{e-}i,uy 上的分布不难从广上的分布得到.因此,讨论清楚
厂上的性质是重要的.

我们已皎知道,\&8), 丄以三种形式返回到厂,自由返
回,非本质自由返回和有界返回.我们首先证明在r上自由返回
x\textsubscript{B}( = 的分布具有有界密度.其次,用类似的方法可以证明非
本质返回"{£»,}的分布具有有界密度.我们最后证明有界返回{少丿
的分布具有密度g £ \emph{Lf,p\textless{}2.}

定理4,1的证明

1.自由返回的分布问题的讨论.

记4 = 4,不妨设\emph{A}的Lebesgue测度为1,即臨=1.这样,
我们可以把它看成概率测度.在此基础上,我们可以引入期望E和 条件概率等概念.

取定1 MM ••• M V,和間必m,定义

==化 w e \emph{I \%、j} 一 1,2,---,z\},

或者财=\textsc{a}由前面的讨•论我们知道,这种形式的参数集合可以分
成一列子集\# 3),其中每个x (0相应于一条不同的``路径''i,该
路径在第%次自由返回时,通过区间\emph{1\%}我们定义

4, = \{a £ \# \emph{\textbackslash{} x„(a)} 6 \emph{h),}

= \{a \& " I 弓(a) £ \emph{I\textsubscript{u},} x„\textsubscript{+1}(a)
£ 丄,路径 i\},\\
心=1)収\&*).

对每个路径\emph{i,}由(3.13)\emph{丄} 可扩增为l个区间4,(0,其中

\begin{quote}
lAXi) I =乙(用)2 弦卩)=exp(---\textsc{2k*).} (4.2)
\end{quote}

由分类的意义,我们知道珏(D可以分成一些区间孔的并,由前面 的引理3.
5,我们得到

. IZ I

m\%(D M const (4.3)

为完成情形1的证明,我们首先给出两个事实,并把它们归纳 为下面两个引理.

引理4. 3给定\# = 或U 4•有一个与"无关的

常数Q,使得如果对某常数\emph{Q \textgreater{}} c°,下式

mA* M 2Q \textbar{} A, \textbar{} \emph{mA}

对所有的v成立,那么,对所有``我们有

瑚'户 MQIWmA

证明由(4. 3)我们有

mA`` ---必)、\emph{raA\textsubscript{llfl}(i') +}

\emph{v,i} KlMYz)

曰41 S\sout{况n}〉mW).

心\textgreater{}g) 、' ' \emph{k}

因为

亞烏沒心)3山零,

并注意到、心\textbar{}/殓0 V+ 8,所以适当选择g,我们有 S制§m处⑺M糸m\# .

对\emph{,}\textless{}
g,存在一个常数氏,使得\emph{L(.\textsubscript{V},i)}法\emph{L\textsubscript{o}.}于是

N 瑚*) 加 8瑚祯)具争財m\# ,

认Yq\textgreater{} S 3佑 \emph{乙}

其中% 2 (2c/Z\textsubscript{0}).于是

(号+号\textbar{}成冋\#京心\textbar{}m\#. \textbar{}

\begin{quote}
附注4.4引理4.3给岀从V到``的``转移概率''的一个估计. 引理4.
3的一种特殊情况为
\end{quote}

引9 4.5假如

An = \{a w \textsc{AJzrGi)} e \emph{I„\},n} = 1,2,---

那么对所有的如,

mA«Wco\textbar{}A,\textbar{}.

证明我们用归纳法来证明这个引理.当n=l时,由引理 \emph{2.3}可得

M \%!匕 I,\\
如果上述不等式对«时成立,下面证明\\
mA"x+l 具帛//.

事实上,如果我们记x = U/网,那么

\begin{quote}
\textsuperscript{A}» n+l =
\end{quote}

由归纳法假设,

mAnMcolAJ \textless{}Qc\textsubscript{0}\textbar{}Z,,\textbar{}m\^{} ,

其中Q= (m")T.于是,由引理4. 3,我们立得

mAn+1 =庭七 W ■j'QcolAJniNMcolAJ. \textbar{}
下面我们开始估计{\%}的分布函数.令不含0)是长度为
©的任意固定区间,取\#=4,并考虑

\textsc{E(X»(\^{}m))} =--- {[} XsStGz)) da,

J人

其中担,是示性函数,A可以依照不同的路径分解为4, =对每个£,由引理3.
5,為(a)在九中的分布有一致有界 密度.因此

\includegraphics[width=0.37361in,height=0.45972in]{media/image83.png}(4.4)

那么

\textsc{E(Fn)} W const £.

下面我们任意取定一个充分大的正整数五,并考虑

\emph{Eg) =} S N-叶 Xa(气)••%(勺)da.

\begin{quote}
弗``*N 八
\end{quote}

我们要证明

\emph{E\textless{}F£)} \textless{} (const 6)\textsuperscript{h} + Owb.

E(F幻可依下标数组(力,\ldots{},水)分成两部分.因为

CardfOt, ---,j\textsubscript{h}) I min\textbar{}j\textsubscript{ft} ---
j\textsubscript{£}\textbar{} \textless{} -/N} 并注意到 '

丄為,(叼卩\ldots{}么愆``)da W 1,

因此,在(4. 6)中相应的这些项全体为。3遴).E(H)中剩余的 项可做如下估计.取A
\textless{} \emph{k \textless{}}\ldots{}V水W N,满足力+1 -

/N, *=1,・・・,力一1.并考虑一个区间3UA 假设与\textsc{s\&omW}
,,那么我们在引理4.3中选择龙=力以及\emph{i=\{ae} \&吃 e
L,sW2\}.对这样的取法,引理4. 3中的假设条件显然成立.此
时,我们在该引理中取2Q= \textsc{IAJt,}令

4片,=\{a \& 兩 \textbar{}zj,+*(a) £ 丄\}.

由引理4. 3,我们有

\emph{\textless{} Qm\^{}i \textbackslash{}1\^{}\textbackslash{}.}

归纳地应用引理4: 3,在\emph{k = j\textsubscript{l+l} -
h}次迭代后,只要\emph{N}充分大,使
得\textbar{}侦\textgreater{}qT2-E,我们推得

mA件 \textless{} \emph{2\textasciitilde{}\^{}\textsuperscript{{]}} • Q
•} \textsc{•JIM Co\textbar{}/\^{}\textbar{}hi-\^{},.}

因此

于是

\emph{m=h} \textless{} (c°\textbar{}L\textbar{})k

如果我们进一步要求,对每个项``z厶£ = 1,・・・,九,那么由

\begin{quote}
\emph{fti h}
\end{quote}

引理3.5,上述测度的下降比例为M无(参考(4.4)).因此,当 N充分大时,

E(理)M(*)"+O(NT).

于是,对几乎所有4\_

\begin{quote}
\textbar{}\textbar{} limF/le M 勺 e. (4.7)

N-*8
\end{quote}

(4. 7)式意味着自由返回的极限分布在x = 0外有有界密度(界为
c°).由于在x\#:O处密度一致有界,以及对所有的'',为尹0,所以 x =
0处的情况可以不必考虑.

利用非本质返回的定义和性质,类似地可以证明,非本质返回
的极限分布具有有界密度.

2.有界返回的分布问题的讨论.

我们要讨论有界返回⑶?,« \emph{= l,2,-,N,j} = 1,2,-的分
布问题.这个问题的讨论可廿分成二步进行,我们首先讨论对
\emph{J(J}固定)的分布,然后对所有\emph{j\textgreater{}J}迸行分析.

对 令%•表示X-0点第j次返回的时刻,并且记旳(4)

=称此返回是奇(偶)的,如果相应的揉序列(Kneading
sequence)是奇(偶)的.任给定Q和,\textgreater{} 0,并令©(a)是区间,定义 为

a(a)\_ J\textsuperscript{(M}\textgreater{} \emph{---
p\textasciitilde{}l,uj ---} p), 奇返回,

\emph{+ P、*} + p + 2),偶返回. 我们考虑

\emph{破 =H\textsubscript{N}(,a)}=布力九3)(外*)).

\emph{n=\textbackslash{}}

令U是W中的一个参数区间.假设巧S)是0点 的0次迭代,对a €
U,我们考虑马伝)=理,3,a),其中b = (0,
±Ft)二\textgreater{}八.由中值定理和(3. 5)式,我们有

\emph{\textbackslash{}ffj\textbackslash{}} const \textbar{}tf(a)
\textbar{}\textsuperscript{2} exp \emph{qj} \textgreater{} const
\textbar{}ff(a) \textbar{}\textsuperscript{2} expji, (4.8)
这里我们要特别指出,如果{\%}徂i是一个奇(偶)揉序列,那么
的右(左)端点为f,如图6-1所示.

\includegraphics[width=2.4in,height=0.84028in]{media/image84.png}

图6-1

因此,一个点石6孔的第丿次有界返回属于3(a)的``条件概
率'',由引理3.5,满足下面的不等式,

点尸編可'BA'

b, I \textless{} p.

于是,由(4.8)式以及x\textsubscript{B} e \emph{1„,}立得

\begin{quote}
fiCZoifa) (jnXti)) \textless{} const \emph{I
e\textasciitilde{}\textsuperscript{J}'\textsuperscript{3}}
\end{quote}

MM \emph{\textsuperscript{C}P} 其中

"=(g I c; N const \emph{p} eT*\textsuperscript{5}"\} •

显然復\textbar{} I" \textgreater{}同UM.因此

\begin{quote}
\_1 \_ i\_ 4
\end{quote}

\&\%也3)(为,(a)) W const p 刃.

从而得

E(H£(a)) W const \emph{p\textasciitilde{}\^{}l} e"\textsuperscript{-}捉.

我们考虑\emph{(HE*}的期望E((H女a))'').它可以表为下面和的 形式

\emph{N\_h} S E(為sg(a))・``gg(a)))・

如同对自由返回情况的分析,我们将上述和式依照min\textbar{}n\textsubscript{s+1}-n\textsubscript{i}\textbar{}

\begin{quote}
\emph{S}
\end{quote}

\textless{} /N和 厲+1---标 \textgreater{}
加分成两部分.由一个类似的讨论得 到,对充分大的h,我们有

\begin{quote}
J J. I J
\end{quote}

E((H\&(a))A) W (const "〜以`厂,)* + \emph{0\textless{}N-}芝).(4. 9)
由(4. 9)式我们得到,相应于少j,极限分布对几乎所有的a E A有 密度幻,它满足
'

\begin{quote}
{[} g/(a) ds M const Z 厂 M ''厂(4.10)

Jfv(a)
\end{quote}

对J \textgreater{} J,我们要证明相应\emph{j\textgreater{}
J}的全体有界返回的数目(分 布)是小的,记

n ---1

\begin{quote}
\emph{8}
\end{quote}

其中织=2气,而气(a)定义为

丿7+1

\_ J1. 了勻(a)存在,

\begin{quote}
饥8)=[皿 火,\&)不存在.
\end{quote}

如果外6 \emph{WA、}由(3. 8)中有关步的估计推得

\begin{quote}
\emph{8}
\end{quote}

号 gl\^{}w g\textbar{}「

\begin{quote}
J=1
\end{quote}

因此仞M
l``l.由上式还可以得到,只有I川\emph{\textgreater{}J+}1时,们才可能
不为0,从而立得

\begin{quote}
\textless{} const 2

点 \textsc{hJ+1}
\end{quote}

因此E(TQ W eT C 与前面对自由返回情况的讨论相同,我们
考虑ECTQ时分两种情况:minlsi-啪W 加和1\%+1-色\textbar{}

\emph{S}

\emph{\textgreater{} m} 并由此容易证明,对几乎所有的

尽7\textbackslash{}(a) VeTQ

\begin{quote}
N-*w
\end{quote}

总结上述的结论,我们可以将;•上的分布情况写成下面的定理.

定理4.6对几乎所有的\emph{aEA,}自由返回和非本质自由返回
的极限分布具有一致有界的密度,而有界返回极限分布具有密度 gU).
2满足不等式

gGr) Wcon就[习厂第侦旳---工)+习e「第*甄工---\emph{uj)J,\\
}\%odd 勺even

(4.11)

其中

\begin{quote}
•z V0.
\end{quote}

3.在厂外的分布问题的讨论.

到目前为止,我们仅在r内讨论了分布问题.令{岡愆))丄】
表示轨道{5(a)}:=i回到广的点.那么(4.1)的冶,可以表示为

1 a 1 \textsuperscript{M n} \textsubscript{d} , "

\emph{上}=4力%时=时\%)=吏沼),\\
v=l \emph{v=\textbackslash{}} 卜 1 \emph{u} i=0

(4.12)

其中求和号Z'表示,如果F'Jq \emph{\& r,j= l,2,-,k,}那么

\begin{quote}
V
\end{quote}

\^{}(\textsubscript{Uv})属于该和式・

\begin{quote}
令产表示(他住】的``•-极限.我们已经证明 皿聞=川厂£ '', P V 2.

,---*8 J
\end{quote}

下面我们讨论{成>}丄1.由于风W ",卩=1,\ldots{},AC所以\\
supp /4° U {[}1 ---泌2,{]}丄

给定小区间3U 口一渺S1{]},那么F-4由广中两个对称小区间
。和一0组成.由(4.12)我们有

\begin{quote}
\&3)=必>3)=\emph{成3} U (-。)).
\end{quote}

因此

\emph{ju(co)} < f A(x) di,

\begin{quote}
8 expl --- •\textbar{}\textsuperscript{\_}搭)
\end{quote}

其中ft(x) = const 2 .做变量替换y = FS),那么

\begin{quote}
\textbf{I \& ---z\textbar{}E}
\end{quote}

上式的右端具形式

\textbf{, (.2}

Hz) dx = 2>(F尸Cr))(F「> dz ,

JnU(-fl) J® ;±7

其中{FQ捻是FT的两个分支,我们证明

\begin{quote}
2
\end{quote}

J\textgreater{}(F尸(z))(F湼),(Z)£ 〃,/\textgreater{} V 2.

事实上

IL* "一3)的F-3I风= j:\textbar{}y \textbar{},云帀血\\
const \textbar{}\textsubscript{o} \textbar{} A (x) \textbar{} *•
\emph{Ax} \textless{} const \textgreater{} £. \&

因为

\begin{quote}
\includegraphics[width=1.65972in,height=0.52639in]{media/image85.png}L
\emph{P}
\end{quote}

\& + G 、, 卩=[击+1
其中E和G为常数.记上式右端的两项为£和S''为证明(4.13)
的收敛性,我们只要估计\&和£的阶.利用\textbar{}旳\textbar{}
2e\textasciitilde{}",我们 有\textbar{}x-\textsubscript{M}J
\textgreater{}齢財,Y W]口 E丄.因此得S的估计式

\begin{quote}
7
\end{quote}

S] W const]号七一伊'\textbar{}心1*2丄和祯W const e(邱T")・

\begin{quote}
\emph{y=\textbackslash{}}
\end{quote}

用积分估计式,我们容易证明

Sgf const 、厂亍\\
T/勺+1 V

7 1丄

\begin{quote}
V const (1 +戸來)厂野七
\end{quote}

由上述对\&和另的估计立得(4.13)的收敛性.因此,V \emph{P\textless{}2.}

闵(1-折,1) £ (4.14)

现在令a是(一 1,1 - 中的一个区间,那么

\emph{n}

为(a) =亀〉3)・ (4.15)

*=2

我们已经知道对于z£ (1 ---抻,D,如果FXQ冬广`` =1,2,
\emph{-,k-l,}那么存在A\textgreater{} 1,使得

\textsc{\textbar{}3hF*(z)\textbar{}} \textgreater{}A*. (4.16)

用此事实以及完全类似前面的方法,我们有下面的估计

W j g*(z)dx, \emph{h =} 2,3,---

其中kill? \textless{} const A\textasciitilde{}\textbackslash{}
2,由此推得

\begin{quote}
L8
\end{quote}

ZJgAO)\&,

\emph{k=2}

8 , '

其中习由此我们证明了定理4.1. \textbar{}

*=2

小話4.7以上我们已经详细地讨论了 F(z,a) = 1 ---心2在 \emph{a =}
2附近的动力学行为.证明过程也说明,[Yo]中阐述的关于当
前如何研究非双曲系统的看法的实用性.下面我们简单地介绍,关
于一般单峰睞射族兀S)的新结果,以此作为对上面特例的总结.

定义4.8设\#是一个参数空间\emph{.Y艾 口},设兀(工)是区 间1= [一
1,口上的映射.称{兀}住"是正则映射族,如果

(i) 是关于Gc,a)的 C,映射;

(ii) c\textsubscript{0} = 0是九的唯一临界点,兀在[一 1,0)单调上升而在
(0,口单调下降\emph{,A(oxo\textless{}f\textsubscript{a}(o),fl(o)KJl(o),}并且对所有
X 6 (--- 1,0)匕(z) \textgreater{} X\textbar{} •"

(iii)存在正数4: ,A; ,C*和r\^{}2,使得对所有a任\#和所 有愆,少G \emph{I}

A; \textbar{}硏7三庖兀愆)丨\textbar{}x\textbar{}\textsuperscript{r}-i,

以及

I \emph{aja}S) I v { p. \emph{X} \textsubscript{1}1 \textbackslash{}\\
TV\^{})T\textsuperscript{Cexp}l\textsuperscript{c}
7\textasciitilde{}\textsuperscript{1}I)-

附注4. 9请读者分析F(H,a)是否是正则映射族.

记{c„(a)l-\textsubscript{=1}表示临界点c° = 0的轨道.下面我们凫义可扰
动参数的概念•

定义4.10设{兀}抵``是正则映射族.一个参数a .称为可扰
动參數,如果它满足如下条件:

\textless{}M)存在L \textgreater{}Q,y 并且兀.没有稳定

周期轨;

(CE。)存在l \textgreater{}0,使得对每个de\textsc{(0,l)}和''2、如果
\emph{H £1} 满足 n.(釦 \$ 并且尺 *3) e \emph{(-S,3),i = 0,l,}

•••,* --- 1,那么 (X)\textbar{} \textgreater{} e"\textsubscript{(}

\begin{quote}
M (qd))
\end{quote}

\textsuperscript{(T)}战叫(")) \textsuperscript{=Q}'\^{}°-

附注4.11 (M)条件即通常所说的Misiurewicz条件,它要求
临界点勺=0是非回复的;面九.没有稳定周期轨可以保证在临界
点的任意邻域外的任何足够长南轨道段的指数增长.(CEQ条件
是证明过程中技术上的要求,它保证轨道的增长指数(比方说A)
与临界点邻域的选取无关.(T)条件是橫截条件,它保证九橫截 地穿过0 .对兀愆)=1
一技回言,丄=2是可扰的.可以证明 对此映射

定义4.12 参数队称为一个Borel集JJ £或的Lebesgue正 91点,如果

,-{\textbar{}。D(4. --- e,a* + e) \textbar{}\\
}T I\# n (a. --- e\textgreater{}a. + e) \textbar{}

其中 冋表示集合。的"besgue測度.特别,如果该极限值为1,
则称a•,为Lebesgue全?I点.

附注4.13显然全稠性意味着正稠性.在对兀=1 一山的 讨论中,我们仅证明了。. =
2是正稠点■-实际上将证明稲加改动, 可以证明。.=2是全稠点.

对一般的具可扰参数山的正则族{兀槌A我们有

定理4.14令{兀}亦/是正则族.对每一个可扰参数a* E
行,存在正常数a和用使得%是满足以下诸条件的参数a构 成的集合"的全稠点.

(NS)兀无稳定周期轨;

(ER) V « \textgreater{} 1. \textbar{}/S(0) \textbar{} \textgreater{} e
exp(--- na)\textsubscript{f}

(CE1) V ''20,\textbar{}勺務(兀(0))\textbar{} \textgreater{} A exp («A);

(CE2)对V«\textgreater{}1,如果x6 [-1.1]满足充愆)尹0对于 \emph{k} ==
0,1,--- ,n --- 1 成立,并且務(£)= 0,那么 \textbar{} \textgreater{}

\emph{k} exp(n人).

由于对每个a E \emph{D,f\textsubscript{a}}满足条件(NS),(CE1)和(CE2),利用
CNS]的结果,立得对\emph{f\textsubscript{a}}存在一个关于Lebesgue测度绝对连续的
不变概率测度产.

近年来,在非双曲系统研究中不断有较深刻的结果出现.我们
在本章中不准备介绍了.但有一点要指出,我们在这里介绍的
Benedic屁和Carleson的证明思想(简称BC方法),近几年来在这个
方向的研究中起着十分关键的作用.事实上,无论是对一般区间映
射族还是对平面间宿相切(例如Henon映射),鞍结分岔甚至高维
的同宿分岔现象所得到的主要结果的证明方法,基本上是BC方
法,或者是从这一方法中派生出来的.

即使我们只考虑R"上的向量场.在讨论分岔的余维数时,也
需要在全体光滑向量场所成的(无穷维)空间中考虑某些子流形的
余维数,以及光滑映射与这个子流形的横截相交性等.为此,我们
在附录A-C中介绍一些有关微分流形与微分拓扑的概念、名词
和重要结果,以便使读者减少査找参考书的麻烦.对这些结果本
身感兴趣的读者,可以参考有关的文献,例如

{[}Al{]},和{[}Zg{]}等.

附录\textbf{A Banach}流形和流形间的映射

微分流形是欧氏空间中光滑曲面概念的抽象和推广.它的基
本思想是,先在这个对象的局部通过与Banach空间的一个开集建
立微分同胚而引进相应的代数和拓扑结构,然后再把这些局部结
构光滑地粘接起来.因此,我们可以把Banach空间中的运算推广 到微分流形上.

微分流形的定义

定义A. 1 设M是一个连通的Hausdorff空间,B是一个
Banach空间.假设U是M的开子集怦是从U到8中开子集\emph{构)}
的同胚映射,则祢(U瘁)是\emph{M}的一个坐标卡.

\emph{M}的两个坐标卡(U,时,\emph{(V,\^{}}祢为是\emph{C}相容的,如果当\emph{u}
n V尹。时,映射

,。广 1叫而,p(C7 fl V) - \emph{\textless{}P(JJ} f\textbar{} V)

是8上的仃微分同胚,

\emph{M}的坐标卡集\emph{必}=\{(亿洪)\textbar{}a£ A\}
\emph{(A}是一个指标集)称 为一个L坐标系,如果

(1) \emph{\{U\textsubscript{a}\textbackslash{}aeA\}}是M的一个开覆盖;

(2) \emph{庭}中的任意两个坐标卡都是r相容的.

\emph{M}的两个\emph{C}坐标系\#冃庭勢为等价的,如果也U M还是
M的(7坐标系.M上(7坐标系的一个等价类£称作M的一个仃 微分结构.
2内所有坐标系的并集\#=U \{方1\^{}65\}称作M
的一个极大仃坐标系,而(V,时6丿称作一个容许坐标卡.

如果在M上给定了一个<7■微分结构\&删称S= (M,£)是一
个仃微分流形.一般常把M与f等同,简称M为微分流形,为了标
明卡映射\emph{甲}的取值空间B,可称M为装备在\emph{B}上的Banach流孵
(注意,由\emph{M}的连通性和坐标卡的相容性易知\emph{,B}与坐标卡的选取
无关).特别,当B为Hilbert空间时,称M为Hilbert流形湎当B
为有限维空间(例如R")时■,称M为3维)微分流形.

附注A.2应该指出,一旦给定了診的一个坐标系行,就可
以把与M等价的全部坐标系合起来而得到一个极大坐标系,从而
生成\emph{M}上的一个微分结构.因此,只须给定\emph{M}上一个特定的坐标
系,就可以决定这个微分流形.

附注A.3如果把定义A.1中的仃全部换成C",则我们相
应地得到b微分流形.C"微分流形也称为光滑流形.

例A.4 (1)设B是一个Banach空间,则可取坐标卡(B,id),
(id是B的恒同映射),这个坐标卡显然就构成B上的一个b坐标
系.因此,任何Banach空间都是装备在它自身上的一个光滑 Banach 流形.

⑵S" = \{工£ R"\textsuperscript{+1}1 U\textbar{}\textbar{} =
1)是一个m维流形.事实上,记 \emph{N =} \{1,O,-,O\},S=
\{-1,0,•••,()\}分别是S''的北极与南极,取
坐标卡(S"\textbackslash{}\{N\}, " (S»\textbackslash{}\{S\}, ft),其中

"S"\textbackslash{}\{N\}fRK,啊=(\sout{]三僅}, 卻
S\textsuperscript{n}\textbackslash{}\{S\} ---
R",你如,\ldots{},命+1)=(\sout{].务}, 容易得出

\^{}?T\textsuperscript{,!} R\textsuperscript{n}\textbackslash{}\{0\}
-\textgreater{}R"\textbackslash{}\{0\},件矿伝)=赤
是C\textasciitilde{}微分同胚.

流形间的映射

定义A.5设M,N分别是装备在Banach空间A,B±的役'
流形.\emph{ftM\^{}N}称为是(/映射,如果\textsc{VhEM,}及N上任一容许
坐标卡\textless{}V,Q J3)6 V, 3必中的容许坐标卡(U,机工G \emph{U, f(U)
U} V,使得映射(叫作\emph{f}的局部表示)

\emph{f\^{} = \textless{}!\textgreater{}\textsuperscript{a} f -f'*}
中(U) U A --- \emph{\textless{}p(V)} U B 是仃的.

容易证明下面的两个定理.

定理A.6设\emph{f*MfN}是流形间的连续映射,则/■是顷的,
当且仅当对M与N上任意取定的坐标系而言,相关的局部表示都 是仃的.\textbar{}

定理A.7设\emph{M,N,P}都是仃流形都
是尸映射,则复合映射也是(7的.\textbar{}

定义A.8设是\textless{}7■微分流形,称須,MfN为仃微分
同胚,如果它是一一的仃映射,并且其逆映射M也是C5
的.如果两个流形间存在一个r微分同胚,则称这两个流形C微 分同胚.

子流形与积流形

类似于向量空间的子空间与乘积空间,微分流形也存在子流

形与积流形.

设N是微分流形M的一个开子集,则把M上的微分结构限
制到N上,就自然得到\emph{N}上的一个微分结构,从而使\emph{N}成为一个
与M同维数的流珍这时,称N是M的一个开子流形.俱是M前
亠个闭子集,例为R''+i中的闭子集3"也可以成为一个流形(见例
A.4).注鬻,此时程R"+i中存在坐标卡〔U,乎),使祁J (AS'')成为
R"的子集.这引出如下的一般定义

定义A.9设M是裝备在Banach流形理上的微分流形.M的
一个子集\emph{N称为M}的子流形.如果\emph{戏 £N,3M}的容许坐标卡
(。疗),和\emph{B}的宜和分解B = 8】击0,使得戸£ U,并且

卩《7\textbar{}"\textbar{}村)=中《7) 0(4 X {0}). (A.1)

可对这个定义给出直观的几何解释\emph{N NC} 容许坐标

卡(U,时,使邻域U fl
N经甲作用展平在\emph{B}的子空间\emph{B\textsubscript{t}}上.

定理A.UJ设N是微分流形M的子流形,则N自身也是一
个微分流形,并且它的微分结构可由下面的坐标系生成:

{(C/r\textbar{}N, pBnM):(U,P)是M的容许坐标卡,

它满足条件(A.1)}. \textbar{}

在附录B中,我们将介绍通过浸入或浸盖来构造(或鉴别)子 流形的方法.

定义A.11设是微分流形,它们分别有坐标系{(。心
"\textbar{}a£4}和{V"侏B},删集\emph{合MXN}在由坐标卡{(亿 X皿,饱X
sMu°x*)\textbar{}a £ A,尸务B }生成的微分结构下作成微
分流形,称它为肱与N的积流形,仍记为\emph{MXN.}

附录B切丛与切映射,向量场及其流,漫入与浸盖

向量丛是积流形的推广,而流形上的向量场,则是作为一个特
殊的向量丛(即切丛的截面)而定义的.利用切映射,我们可以把

Banach空间中的隐函数(反函数)定理以及局部浸入和浸盖定理
推广到流形上,从而得到判断(构造)子流形的有效手段.下面先 定义向量丛.

向■丛

定义B.1设M是仃微分流形,B是\&皿ch空间\emph{\textsubscript{t}UCM}是
开集,称17X3为局部向量丛,并称U为底空间,它可以等同于(7 X
\{0\},后者称为同部向量丛的\#截面.V«6 u,称3\} X B为过
«的鉀维,它可以众\emph{B}获取Banach结构.\emph{Y u£U,bW B,}由R(``,
方)="所定义的映射\emph{\^{}UXB\^{}U}称为投影..

注意,过``的纤维就是而U X召为积流形\emph{MXB}的 一个开子流形.

定义B. 2设\emph{UXB}和5 X B都是局部向量丛.如果映射 \emph{"U 7}和%:E7f
)都是(7的,则由

\emph{强u,b)=}(阳(``),¥\%(``)• 6)

所定义的映射\emph{fUXB-\^{}U'} x \emph{B'}称为r局部向量丛映射.如果
这个映射还是一一的,则称它为仃局部向量丛同构.

注意,一个局部向量丛映射把纤维S\} X五线性地映到纤维 («(»)\} XB*.
一个局部的向量丛映射是局部向量丛同构,当且仅 当VuGU,
\%(«)是Banach空间器与否之间的同构映射.

类似于微分流形的定义,我们可以把局部向量丛粘接起来,得
出整体的向量丛结构.

定义B.3设S是集合.称(W,时是S的一个局都向量丛卡, 如果IV U
S,甲是从W到一个<7局部向量丛\emph{UXB}的一一映射
(U,B可能与p有关).称这样的卡集留=\{(W\^{},\%)\textbar{}a£7i\}是S
的b向■丛坐标系,如果

(1) \emph{\{W\textsubscript{a}\textbackslash{}aeA)}覆盖 S;

(2) 若("》,代),(w``啊)e第,吼,n巧声尹。,则%(叽n
巫\#)是局部向量丛,且阳°缶'\^{}(W\textsubscript{O}AW\textsuperscript{,}\textsubscript{\textbar{})})是b局部向量丛同构.
称S的两个向量丛坐标系幽1与绥将价,如果躋1U漆必是
一个向量丛坐标系.向量丛坐标系的一个等价类称为S上的一个
C°■向量丛结构.称\emph{Ef} )是一个b向暈丛,这里S是一个集
合,戸是S上的一个r向量丛结构.与微分流形类似,通常也把E
与S等同,并把源中的任意局部向量丛卡称为一个容许向■丛卡.
向量丛有如下性质'

(1) 仃向量丛\emph{E}是一个侦微分流形.

(2) 对向量丛E,定义其專截面(或称为E的底空间)

\emph{M \{p e E\textbackslash{} 3} 容许向量丛卡使 p = 它是E的子流形.

⑶ 设怦)是一个容许向量丛卡\emph{,弈WfU X} 设pF

W,满足\emph{甲}(/>) = 3,0),令集合

\emph{Eg} = ?"'(\{``\} X BQ,

它可以通过中从B诱导Banach结构,并以\emph{p}为零元素.如果(Wi,
ft)和(巫``啊)是两个容许向量丛卡,P G W\^{}i Cl W2,则\emph{Eg} 与
E',%线性拓扑同构(作为线性空网是同构的,作为拓扑空间*同
胚的).在这个意义下,可以认为Eg与甲无关,可记为\emph{E\textsubscript{P}.}

(4)v«eE,3唯一的\emph{\textsc{p}}e
\textsc{m}(底空间),使得ee\emph{\textsc{e\textsubscript{p}.}}由此,
可定义投彩

\emph{E f M, = p,}

它是仃的满映射,并且旷 W)=耳称为过\emph{peM}的纤维,它具
有的Banach结构称为纤维型.在有些书上,把上面的性质作为向
量丛的定义.我们有时也把向量丛记为或wEf M,以 标明投影映射m和底空间Af.

从几何上粗略地说,以流形M为底空间的向量丛,就是在\emph{M}
上的每一点``附着''一个以该点为零元素的Banach空间,而不同点
上附着的Banach空间是彼此同构的.例如,在S?上,我们可以用下

列两种方式构造出不同的向量丛:V,€ S2,取过力的法线为纤维
E,;或取过\emph{p}的切平面为\emph{Ep.}前者的纤维型是畔,而后者是\emph{R\textsuperscript{s}.}

切空间与切丛

我们可以通过坐标卡把Banach空间中曲线相切的概念诱导
到流形上,从而建立切空间与切丛.

定义B.4设M是仃微分流形设\$\textgreater{}0,开区间Z =(一畧3).称。映射cJf
M为M上的一条曲线「如果''0)=
P,则称曲线以\emph{P}为基点.设上,电是以\emph{P}为基点的两条曲线,并且
(U怦)是一容许坐标卡\emph{\textsc{,p}} e \emph{U.}如果

D(p。C])(0) --- D(p。@)(0)

(即Banach空间中的曲线处q与中。勺在。点相切),则称M上的
曲线勺与6在p点相切,

注意,利用流形M上坐标卡的相容性,%与位在声点的相切性
与容许坐标卡的选取无关,事实上,设0,對),(叫决)是两个坐
标卡,/\textgreater{} 6 n \emph{u\textsubscript{2}.}设 D3。2(0) =
D(p。C2)(O).由于•

矽。G = (©。p\textsuperscript{\_1})-(中。c;), i = 1,2,

所以

D(© ° \textless{}:i)(0) = D(0 » 亍')(卩3)) • D(p- cD(0)

=D@ -伊f)(p (/\textgreater{})) - D(p» c\textsubscript{z})(0) =
D(。。C2)(0).

这样,我们在同基点的曲线之间规定了一个等价关系m〜◎
Qq与§在*点相切.记c在戸的等价类为称它为流形M在 P点的一个切向■

定义B.5设M是U微分流形(r\textgreater{}l),/\textgreater{}eAf.在》点的全
体切向量之集合

T»W) = 是M上以?为基点的曲线}
称为流形M在p点的切空间.并称\emph{M}上全体切空间的集合

为M的切丛.

定SB.6切空间是一个Banach空间,而切丛TM在 投影

\emph{jrtTMM (jr\^{}cjp --- p)}

之下作成一个向量丛

证明先证明\emph{T?(M)}是一个Banach空间.取包含7■点的一
个坐标卡■(!/,甲),则以/■为基点的曲线c有局部表示D3u)(0)W 0
(R,Span?07)).记旦=(7?,Spanp(\{7)),它是一个 Banach 空间•这样.v
Ed\textsubscript{f} er\textsubscript{?}(A/),有'' \textsc{=
d(\textgreater{}c)(o)}与之对应. 反之,\textsc{Vp} € 耳,令 c«)=广(敏'')+
拔),则 p\textsuperscript{o} c(r)=\emph{秩i\textgreater{}) + vt,}
从而D("Q(O)=a,从c可得于是,得到TpW)到品的一
一对应.这样,就可把\emph{B\textsubscript{r}}的Banach空间结构通过局部表示
D0\textgreater{}。c)(0)而迁移到\emph{T政M)}上.

再来证明是向量丛,其中投影\emph{mTM f M}通过
巩[員捞=/•规定.取仃微分流形M的一个坐标系==\emph{\{(U\textsubscript{a},}
物)\textbar{}a£A\},则由下文的定理B.10可知\emph{,\^{}=\{(TU\textsubscript{a},T\textless{}p\textsubscript{a})\textbackslash{}\textsubscript{a}eA\textbackslash{}}
就构成了 TM上一个(7-\textsuperscript{1}向量丛坐标系,其中7Va =
\emph{U\^{}Tp(M),}而丁削丁(我UQ)是如下定义的切映射.

切映射

设是微分流形J:A/f N是仃映射5法1).设q宀是
财上在"点相切的两条曲线,则/。勺和 E 是N上在/(/.)点
相切的曲线.事实上,设(U,时,(V\textsubscript{)S}i)分别是\emph{M.N}上的容许坐标
卡\emph{,p € u,fip')e} v,y(u)uv,则 \textsc{d}(少勺``。)=D(wy)(o). 注意

\emph{\textless{}l\textgreater{}° f ° c,---} 3。y。p-') ° (p ° C,),
\emph{i} --- 1,2, 则由Banach空间中导算子的链式法则可得

D(0 ° \emph{f} - c,)(0) = D(° " \emph{f}。广)(中(0)) -
D(中■\textgreater{} c;)(0),

(B. 1)

从而

\begin{quote}
D@ \emph{* f。}勺)(0) = D((J。\emph{f} ° c\textsubscript{z})(0).
\end{quote}

由此,我们给出如下定义

\begin{quote}
定义臥7如果是流形间的C\textsuperscript{1}映射,则称映射 T/: \emph{TM
-\textgreater{} TN,} 7y({[}c3 = {[}/。理3
\end{quote}

\emph{为f}的切映射.有时也把7Y记为d/或兀.

如果(U建),(y,Q是上而所说的坐标卡,则由(B.1)式可知, 序有如下表示式

\emph{Tfi} 3,a) 1 (須(/\textgreater{}), D3 -
\emph{f}。广)(?\textless{}/\textgreater{})) • 0),

(B.2) 其中p表示D(p。c) (0).由此,可从Banach空间中导算子的性质,
得出切映射的如下性质,

(1)若\emph{f:MfN}是仃的,则\emph{TfWN}是(7T的.

⑵ T"M) - Tp(N)是线性的.

\begin{enumerate}
\def\labelenumi{(\arabic{enumi})}
\setcounter{enumi}{2}
\item
  若\emph{AMf,} K是流形间的仃映射,则
\end{enumerate}

\emph{T(g,f)=TgF}

是\emph{TM\^{}TK}的(7T映射.

\begin{enumerate}
\def\labelenumi{(\arabic{enumi})}
\setcounter{enumi}{3}
\item
  若五:是恒同映射,则\emph{Th:TMfTM}也是恒同映 射.
\item
  若BMf N是微分同胚,则\emph{Tf'TMfTN}是单、满映 射,且
\end{enumerate}

引理B.8 设W是Banach空间B的开子集(从而是 b
Banach流形B的开子流形),则7W同构于局部向量丛W X
\emph{尊}(因此,我们在下文中把加与"X B等同).

证明设c是W上以在为基点的曲线,则存在唯一的i\&B, 使得由

\emph{Cp,b(.t) --- p 1b}
所定义的曲线在\emph{P}点与\emph{c}褂切.事实上,以(0)是名\emph{(R,B)}中唯一
的线性映射,使得与曲线,在在点相切的另一曲线具有形 式

\emph{g(t) = p +} Dc(0) • \emph{t.}

令g = \textsc{Cm,}则\emph{b} = DC(0) - 1是唯一存在的.定义映射

A' IV X B-► \emph{TW, h\{p,b\} --- {[}Cp,b\}\textsubscript{P},}

则上而的结论表明\emph{h}是一一的.我们可以在\emph{TW}上建立局部向量
丛结构.例如,取\emph{K\{p\}XB)}为过\emph{pEW}的纤维,则儿恰是7W与
\emph{WXB}的局部向量丛同构映射.\textbar{}

引理B.9设W和W\textsuperscript{7}分别是Banach空间B和3的开子集,
\emph{f,W\^{}W'}是仃微分同胚,则\emph{TfiWxB\^{}W' XB'}是局部向量
丛同构映射.

证明因为

773,8)=(須(力),D\_f。)E)

(见(B. 2)式,取务少为恒同映射),所以\emph{Ff是 L 局部}向量丛映
射(见定义B. 2).又因为『是仃微分同胚,因此\emph{(T* = TH}
也是一个<7一'局部向量丛映射,从而"是一向量丛同构映射.\textbar{}
现在可以证明

定理B.10设财是</微分流形\&梁1),\#=\{(儿用)楂£
A\}是M的一个仃坐标系,则\emph{T庭}=((T(U\textsubscript{a}) ,T皿))S £⑴是
切丛\emph{TM}的C-\textsuperscript{1}向量丛坐标系,从而TM是(7T向量丛.'此外,如
果M是"维流形,则\emph{TM}是\emph{2n}维流形.

证明 (U T«j\textsubscript{a}) \textbar{}« e A\} Z) 7W.
T«J\textsubscript{O}) n

\begin{quote}
A.则由引理B.8,
\end{quote}

\textsc{t}物 cr(s)n \textsc{t(uq)=t(\textsubscript{p}}n T(a(s))

是局部向量丛,并且

\emph{T\%} °
(Tfj,)\textsuperscript{-1}1码(7\textbackslash{}\%)(1崇``)=T(師。缶')

(见定义.B. 7下的性质(3)和(5)).注意驿。传'是\%(C/\textsubscript{Q})所在的

Banach空间E中的开子集你T(S) f\textbar{} \emph{TW)}到自身的CT映射.
由引理B.
9立得,(T"。(亍铅厂是C-\textsuperscript{1}向量丛同构映射.因此,
帛曳口.9申的柔侔U)和⑵坞成立.有关维数的论断,是引理B.8
的自然推论(注意做S)是它所在Banach空间的开子集).\textbar{}

附注B. 11前面已经提到,流形间的切映射是Banach空间
中导算子的推广•因此,有时把它称为导映射•设M和N是充分光
滑的微分流形,则定理B. 10说明,TM和TN是向量丛,从而是微
分流形(见向量丛的性质⑴),并且 TM因此,可以继

续求二阶导映射\emph{T(Tf)
=T\textsuperscript{z}f\textgreater{}T(TM)\^{}T(TN),}并可递推地定
义高阶导映射.

向量场及其流

定义B.12 设(E,心M)是一个向量丛.如果映射

满足?rM = idM,则称它为向量丛的一■个截面M称为是(7连续的,
如果它作为Banach流形间的映射Af-£是(7的・

定义B. 13设M是一个C\textsuperscript{5}微分流形,其切丛上
的一个C7rMsW8)截面\emph{XiM\^{}TM,}称为M上的一个仃向
量场.\emph{M}上一切(7向量场的集合记为缶飞M).

附注B.M设\emph{X'M\^{}TM}为一向量场,按定义QX = idM, 即V \emph{P
\&M,}有X3)€ \emph{TpM.}换句话说,M上的向量场就是把M
上的每一点赋予一个(在该点切空间内的)切向量,从而形成一个
M上的切向量场.

现在设(―】,1)\textasciitilde{}M是M上的一条§曲线(sWr),
我们要规定曲线上每一点的切向量•注蠢由映射\emph{f M}可得切 映射

\emph{Tai} 77 = / X R--- \emph{TM.}

取切丛77的一个截面A-/-TI, A(t)-a,D,Vt\&7,则复合映 射

\emph{Ta -} A\textgreater{} 7-\textgreater{}TAf
是上的一条C\textsuperscript{1}"\textsuperscript{1}曲线.记廿=•
Ta。人,称\emph{QTM\^{}M}
上的曲线a在p点的切向董.如果取/上的坐标卡(W,id)和M上 的坐标卡(。并)(£ E
\emph{W,\textsubscript{a}(t) EU),}则由公式(B. 2)可知,映射/ 有局部表示

\begin{quote}
*(f) = Ta(t,l) = (a(t), D(0。a)(r)). (B. 3)
\end{quote}

注意,上面的1是R'中的单位映射.

定义B.1S 设f £亥'(M)JURi是开区间,0 e \emph{I.}称a:/ f
M是S的过A6M的积分曲线或流,如果\emph{Yt\&r,}

\begin{quote}
J \textless{}/(f) =£(a«)),

1 a(0) \emph{= p„.}
\end{quote}

称上述流是极大的,如果£过A,的任一流俄J U Ri f M (J是包
含0的开区间),都有\emph{JUI,}且a\textbar{}j = \&

取M上的坐标卡\emph{(U,\^{},\textsubscript{fio}eU\textsubscript{,}}设向量场TM有局部
表示其中\emph{sU---B} (Banach空间).与上面/(c)的局部
表示相对照,就得到Banach空间中关于3。a)的微分方程初值问 题

\emph{j} D(p ° a) «) --- \emph{v(a(t)})=为。少-i ((中。a) (f)),

I(0。a)(0) --- \emph{\textless{}p 3°).}

利用坐标卡的c\textsuperscript{1}■相容性,此方程与坐标卡的选取无关.就是说,如
果取另一坐标卡\emph{(V仲HpM} V,则只须把上面微分方程和初值条
件中的\emph{\textless{}P°} a换成0。a即可.

利用Banach空间中微分方程初值问题解的存在和唯一性定 理,可以得到

定理B. 16 设?€ l.jgjf过力的极大流

是存在且唯一的. I

由附注B. 14可知,求? \& 过户的流,就是在M
上找一条过*的曲线,使它在每一点的切向量刚好与S在该点给
出的切向量相吻合,这与欧氏空间中向量场的流的概念相一致.

附注B. 17 如果Af = U是Banach空间B的一个开子集,则

u上的向量场就是一个映射x:i/f a x 8,它具有如下形式

X(H)=(工 VGc)).

我们称V是X的主部(principal part).显然,我们可以把X与V等
同,简单地认为向量场X就是映射们\emph{U\^{}B.}此时,如果曲线吋)
满足微分方程

Da(f) =U(g),

则它就是\emph{X}的流(在定义B. 15中仲=id).

如果M是一个U微分流形,(U,p)是它的「个坐标卡,\emph{啊U}
fiy'UB,厕M上的向量场X诱导出8上的一个向量场.

\begin{quote}
云=7V・X(?ri(z)),
\end{quote}

它称为X的局部表示.

设是一曲线,。是亲的主部,如果

D5(r) =V(«(0),

其中\emph{三=罕5、}则a就是M上的积分曲线.

特别地,当B = R''时,京的主部0具有形式(叭。),\ldots{},
吃(``)),z£R".设曲线a(£)的局部表示为(勺《),\ldots{},a''Q)),则
它成为积分曲线的条件是

因此可以说,对于«维流形上的向量场而言,其积分曲线的局部表
示满足R"中的微分方程组.

沒入与浸盖

隐函数定理是微分学中最重要的定理之一•设f R''是 a'的.当\emph{m =
n}并且导算于\textsc{D/(x\textsubscript{d})}是单、满映射时J给出x\textsubscript{0}的邻
域到K(女)的某邻域之间的微分同胚.,如果D/(x\textsubscript{0})仅是单射(设
\emph{m
<}沥,则存在R''中/(x\textsubscript{0})附近的局部微分同胚幻使得在%点
附近有

\emph{g ° f\textsuperscript{:}}(命,...,办)H*
(气,\ldots{},%,。,•••,()),

这在微分几何中称为1局部)正则浸入.另一方面,若W(他)仅是
一个满射(设\emph{m>n),}则存在R"中归点附近的局部微分同胚矿使
在玄点附近有

\emph{f°\textsuperscript{h}-}(气,\ldots{},弓,%+1,.``,工京 i
(而,\ldots{},不), 这称为(局部)正则浸盖(投影).

容易把隐函数定理推广到Banach空间.现在我们把上面有关
技入和複盖的结果就Banach空间的一般情形给岀证明,然后再通
过坐标卡推广到微分流形上,并由此得到构造(或判断)子流形的
方法.对于无穷维的Banach空间,仅有D/(x)的单射(或满射)> '
件是不够的,要附加适当的可裂性条件.首先给出下面的..代"「

定义B. 18设E是Banach空间,田是E的闭子空间.妬棄容
在\emph{E}的团子空间码,使E = g £爲,则称瓦是可裂的(split).

附注B.19如果E是有限维的,则它的任何子空间都是闭且
可裂的;如果日是Hilbert空间,G匚E是闭子空间,则E = G®
G丄,因此Hilbert空间的闭子空间都是可裂的.但存在无限维
Banach空间,它有不可裂的闭子空间.一般而言,E的闭子空间\emph{F}
可裂的充要条件是;寻\emph{P3} (E,E),使= 且「=脸£
\emph{E\textbackslash{}Pe=\^{}e\}.}

定理B.20 (局部浸入:定理) 设E,F是Banach空间\emph{,八UU
EfF}是仃映射\emph{,r\^{}l,u\textsubscript{a}EU.}设''g),Ef
F是单射,并且 D/(«o)的值域码=烫是闭且可裂的.从而存在闭子空 间
\emph{F\textsubscript{1} U F,F} =呂 £ \textsc{Fr.}(当 E = R'',F =
R"时,只须设 rank(D/■(阳))=m.)则存在开集 VCF (/(«„) € V)和 归
UE W码,以及仃微分同胚?> V-> W,使得(卩。\emph{f)(e)} = (e,0),Y e

£ V (1 (E X \{0\}) UE.

证明 由条件可知,线性映射D/(«\textsubscript{o} )是£到码= 阕(''(%))
UF的代数与拓扑同构.令

g: U X 码 U E X 卩2 f F = F1 ㊉码,g('',t\textgreater{}) = /(a) + v,
其中\emph{\textsubscript{u}eu,ve F\textsubscript{s}.}注意

\begin{quote}
D/(u\textsubscript{0}) 0 '
\end{quote}

DggO) = „ ,

(J

是\emph{U
XF\textsubscript{2}\^{}F\textsubscript{t}®F\textsubscript{t}\^{}F}的U微分同胚'由隐函数定理,存在开
集卩,归,使(\#。,0)e 及仃微分

同胚 \emph{f\textgreater{}:V} f W,P\textsuperscript{\_,} =
\emph{g\textbackslash{}\textsubscript{w}.}因此,v (e,0) e v, (p • y)(e)=
3。g)(e,O) --- (e,0). I

定理B.21 (局部浸盖定理)设E,F为Banach空间\emph{,f\textgreater{}U(Z} Ef
F为(7映射,r\textgreater{}l,«\textsubscript{0} e
tA设\textsc{D/(m\textsubscript{0})}是满射,并且战=
ker(D/(\textsubscript{Ko}))是可裂的,E = \& \$ Ez(当 E = R-,F =
R"时,只须 设rank(D/m))=如),则存在开子集U和V,\% £17, UC7,VU
F£E,以及(7微分同胚\emph{岫 f邛}使得(«, u) e V.

证明定义映射

g: t7 C £j ® E\textsubscript{2} -* F® E\textsubscript{2}, g(``i,``z)---
(/(a\textsubscript{l},tt\textsubscript{a}),w\textsubscript{z}). 因此

(Dj/Cwo) D\textsubscript{2}/(a\textsubscript{0})'

Dg(uo,O)= ,

\begin{quote}
。 \textsuperscript{j£}2
\end{quote}

故Dg(\textsubscript{Mo},0)eGX(£,K®E\textsubscript{a}),由隐函数定理,存在开集V,

成UC7,VUF㊉旦,以及仃微分同胚SV-*U,,使得\textless{}!\textgreater{}-'
\emph{---g\textbackslash{}v-}因此

a,勿=(g。,)(心力=\textsc{(/w«tv),\^{}\textsubscript{3}(w,v)),}

其中。■山X处,即间(心a) = 5 \emph{e Q} 0)(14加)=*・I

现在把上面的结果推广到微分流形之间的映射\emph{f'M f N.}
注意及\emph{\textsc{T\textsubscript{m}(N)}}都是Banach空间,因此下面的可裂性条
件是合理的.

定义B.22设M和N是Banach流形\emph{J、MfN}是C7映射.
称/■在\emph{PEM\^{}C-}局部浸入,如果Tp/■是单射,而且它的象集在
\emph{T\textsubscript{\}W}N}中是闭的可裂集.称;'在\emph{P\&M}是b局部浸盖,如果
是满射,而且ker(Tyf)作为丁双 的闭子空间是可裂的.在流形
\emph{M}上每一点都是C局部浸入(局部浸盖)的映射称为\emph{C}浸入(浸 盖).

定理B.23设是Banach流形,\emph{f:M\^{}N}是仃映射(r \textgreater{}
1),则下列三种陈述是等价的:

\begin{enumerate}
\def\labelenumi{(\arabic{enumi})}
\item
  /在,EM是局部浸入;
\item
  存在坐标卡\emph{(U,G兩,P £ U ,fUV冷U fU、 5Vfir} x u',且K3)=0,使得
  o。須。0T:。'---成 x V' 是 包含映射Pi 3,。);
\end{enumerate}

⑶ 存在少的邻域U,使得是N中的子流形,且flu是 U到/W)的微分同胚.

证明(1)与(2)的等价性可由定理B.20得到.(2)与(3)的
等价性可由子流形定义A.9得到,注意V是N中的开集.\textbar{}

附注B.24定理B.23说明,若在力是局部浸入,则存在/\textgreater{}
的邻域U,使\emph{f(U)}是\emph{N}中的子流形.要特别注意这个结论的局部
性.卽使須在M的每一点都是局部接入,也不能断言/'(M)是N
的子流形.事实上,浸入的局部单射性不能保证整体的单射性.
例如,由平面极坐标方程r = COS2W定义的映射f IT是一个
浸入,但不是单射,且/(5\textsuperscript{1})不是W的子流形(见图1).即使浸入
\emph{『MfN}在整体上是单射,仍不足以保证須(M)是N的子流形.
一个反例是由极坐标方程\emph{r} = sin田定义的映射/■ (0,2\^{}) U R f
R2(见图2).上面两例中的问题都出在W原点附近的邻域内、

\includegraphics[width=2.07361in,height=0.9in]{media/image86.png}\includegraphics[width=0.11319in,height=1in]{media/image87.png}\includegraphics[width=0.29306in,height=0.22014in]{media/image88.png}\includegraphics[width=0.33333in,height=0.96667in]{media/image89.png}

图L 图2

利用須从M诱导的拓扑与7XM)作为晔的子集而获得的拓
扑是不同的.在前一种拓扑下,/"(M)作成一微分流形,因而有时
称单射溪入的象集\_f€M)是一个漫入子流形;但是在后一种拓扑
下,它可能不是\emph{N}的子流形.此时M与\emph{HMK}作为\emph{N}的子空间)
不同胚.这就引出了下面的

定义B.25设\emph{北M-N}是寝入,并且是M到/'(M)(在N的
相关拓扑下)的同胚,则称它是一个嵌入.此时,称是N的 嵌入子流孵.

容易证明下面的

定SB. 26 设\emph{戶MfN}是单值的浸入,若它是\emph{M}到\emph{f(M)}
开映射(或闭映射),则■/是一个嵌入.I

定理B. 27 设是微分流形\emph{,f\^{}M -\textgreater{} N}是仃映射\emph{,q €
N,S} =广W).若■/■在S上每一点都是局部浸盖,则S是M中的 一个仃子流形.

证明 由定理B.21及子流形的定义即得.\textbf{I}

附注B.28 设是仃映射.点\emph{qgN}称为f的正则
值,如果\emph{Y作广5项}是满射且其核在与M中是可裂的.记
为■为•/的一切正则值的集合,则定理B. 27有如下等价的陈述:

"定理B.29 若则广】(q)是M中的子流形.\textbar{}

在讨论一个映射的水平集时,这种说法是方便的.

附录\textbf{C Thom}横截定理

在寻找结构不稳定向量场的普适开折时,有时要确定开折中 的通有族(generic
family),或称为一般族,非退化族.为此,需要利
用Thom横截定理(它是从著名的Sard定理导出的),事实上,为了
把无穷维问题简化到有穷维,我们需要的是jet形式的Thom横截
定理.因此,我们先建立映射空间中的拓扑,再引进成流形,最后
介绍Thom橫截定理.

映射空间的拓扑 '

记(7(M,N)为\emph{C}微分流形M与N之间的(7映射所成的集
合.我们在\emph{C\textbackslash{}M,N)}中引迸拓扑,使它成为拓扑空间.

定义C.1 (弱拓扑,即compact-open拓扑) 设/' £ C\emph{(.M, N),}
(t/瘁)和(丫仲)分别是M和N的容许坐标卡;令\emph{KUU}是紧
集,使\emph{f(K) U} 丫;令e为正实数.定义弱子基邻域

彳(m7,9\textgreater{}),W,0),K,e) = \{g £ (7(5/0)
\textbar{}g(K) U U,

sup \textbar{}\textbar{}。(。° \emph{f}。pT)(z) ---。(4\&。\emph{g
°} \textless{} e\}.

C7M,N)中的U弱拓扑就是由这种形式的集合所生成的.任何
包含有限个这种集合交集的集合都是\emph{f}的一个邻城.所得的拓扑 空间记为

定义C.2 (强拓扑,或fine拓扑,WHiney拓扑). 令* =
\{(亿,铀)\textbar{}a任4\}是M的一个局部有限坐标系,即\emph{M}的每一点都
有一个邻域,它只与有限个相交;记火=A\},K.U 亿是M上的紧集;图=\{(皿*)愣£
B\}是N的坐标系.对任一 正数集合£ = (\&a\textbar{}a£ A\},定义

,(儿岳,Q - \{g£(T(Af,N)\textbar{}g(KQU*(小

sup ID0S)° \emph{f} \textsuperscript{0} 妃)(Z)

\emph{---成由3 ° g "} Q(Z)II \textless{}£\textsubscript{a}. V a e
\emph{A\},
\textsc{cxm,n)}}上的r强拓扑就是以上面类型的集合为拓扑基(开集)
所生成的.记所得的拓扑空间为\emph{CslM'N).}

附注C3上面我们假设r\textless{}8.为了定义C\$(Af,N)(或
C;W,N)),只须把包含映射 \emph{\textsc{c8(m,ns(m,n)(}}或
\emph{N机}诱导的拓扑对所有有限的r合起来即得.

附注C.4 q(Af,N)有很好的性质.例如,它可以赋予完备
的度量,并有可数基'当\emph{M}为紧流形时\emph{与皿M,N)}
一致,它们是Banach空间.当肱不紧时,弱拓扑不能很好地控制
U映射在``无穷远''的性质,强拓扑就成为需要的了'注意,在M
非紧时,强拓扑在任一点都没有可数基,因而它不可度量化.但
是,在强拓扑下応(M,N)的任一弱闭子空间都是Baim空间(剩余
集在其中稠密).这对研究通有性质是很重要的.如无特别声明,
下文中都取强拓扑,并把応(M,N)简记为C7M,N).

下面几个定理反映了 \emph{CXM'N)}中函数类的性质.

定3 C.5 设为U微分流形,1 M s M+ 8, ICrCs.dimM \textless{}+oo.则
G\textbackslash{}M,N )在 )中稠密,其中

\emph{U CXM,N)}表示如下任何一个映射类:

微分同胚、嵌入、闭嵌入、浸入、浸盖、真映射(即紧集的原象是 紧集). I

定理C.6 G)设则任何U镰分流形U微分同 胚于一个C"■微分流形'

(2)设lWrVsW+8,若两个C•微分流形(7微分同胚,则
它们C\textsuperscript{1}微分同胚.\textbar{}

定理C.7 (Whitney定理)设lW『W+8,则任何n维的C
微分流形都仃微分同胚于RE'的一个闭子流形.I

附注G8 从定理C6可知,如果把所论流形的光滑性提高
到C",并不是一个很严重的事情.今后我们将经常作这样的假定.
虽然》维微分流形是我维欧氏空间非常一般的推广,但定理C 7说
明,它反过来又可作为嵌入子流形放到R\textsuperscript{2}"\textsuperscript{+1}维欧氏空间中.

射式W)流形

设是U微分流形.我们把三元组\emph{愆,了,U)}的等价类母,
\emph{f,U\textbackslash{}}称为从肱到\emph{N}的一个r-jet,其中。U
\emph{M}是开集,h 6 \emph{U ,f} 6
\emph{C\textbackslash{}U、N机}等价关系\emph{(.x,f,U\textbackslash{}}〜S'
\emph{,f ,U'\textbackslash{}}是指注=/ ,并
存在M与N中的容许坐标卡(W,仞和\emph{(V,\textless{}!\textgreater{}-),}使WUU
(\textasciitilde{}1 t? UM,并且局部表示

\emph{\textsc{\textgreater{}p} - £ * p'} 与。。尹。广:Wf /(iy)n
\emph{f}(W) 在z的前r阶(包括0阶)导数均相同.

显然,上面的定义与坐标卡的选取无关,从而也与。的选取无 关.因此记

称\emph{徵}为映射r\emph{在工}点的r*,并称Z为起点,■/■»)为终点.

记\emph{尸(M,N)}为全体从肱到\emph{N}的r-jet之集合'我们把\emph{J\textsuperscript{r}(M,}
N)中起点在x的子集记为终点在y的子集记为
\emph{J試M,N)*}而把再)与\emph{J\^{}M,N})的交集记为

现在考虑一个特殊情形,M = R\textsuperscript{m}\textsubscript{(}N =
R".此时简记

\begin{quote}
J(R\textsuperscript{m},R\textsuperscript{n}) =
\end{quote}

设uuRm是开集,/er(u,R\textsuperscript{B}),则f在点的广阶丁卽辰多项
式就给出/■在\#点的。jet 一种自然的表示.这个从R"劉R"的多
项式映射可由『在纟点直到r阶(包括。阶)导算子所唯一决定.这
些导算子合起来,属于向量空间

FGng) = R\textsuperscript{n} X \%R七R*),

*=1

其中Sf
*(R\textsuperscript{ra},R")表示从R™到RK的龙重对称线性映射所成的向
量空间.这说明,对\textsc{PGmm)}中的每一个元素,都有且仅有一个
中的元素与之对应,从而有下面的等同关系,

--- P'S,''),

= R™ X PXwig).

因此,当r有限时,尸(成是有限维的向量空间'如果UUR七V U
R"都是开集,则尸(U,V)是按S,'')的开子集.

\begin{quote}
现.在设\emph{M, N}分别是皿维和''维流形.若(U ,會),(V, °)分别是
上的容许坐标卡,则
\end{quote}

E \&/■ h* 疗Cr)(少。/"。广)

就给出了尸(U,V) f尸(戦。),択丫))的单、满映射,从上面的讨
论中得知是jr(m,'')中的开集,而jr(m两 与欧
氏空间同构'所以(8,尸(U,矿))就是尸3f,N)的一个坐标卡'这
样,就可以从的tT坐枚系导出\emph{J\textsuperscript{r}(.M,N)}的C。坐标系'当

是\emph{C\textsuperscript{+i}}流形时,尸(M,M)是一个\emph{甘}微分流形,称为从\emph{M}到
N的射式(jet)流形.

Thom横截定理

定义C9线性空间Z的两个线性子空间X与Y称为是横栽
的,如果它们之和是整个空间\emph{,L = X + Y.}

由于微分流形的切空间是线性空间,而切映射是切空间之间
的映射,所以可以定义两个子流形的横截性和流形间的映射与象
空间中某子流形的横截性.

定义C. 10 设\emph{A,B}是光滑Banach流形\emph{M}的两个光滑子流
形.称于流形A与8横截,如果A PI B = 0,或者V \emph{pEAdB,
T\textsubscript{P}A}与\emph{T\textsubscript{p}B}在\emph{TpM}中(在定义C.
9的意义下)横截.

定义C.11设M,N是光滑的Banach流形M是N的光滑子
流形.称C"(r\textgreater{}l)映射\emph{f:MfN在}点p £ M与A横截,如果
\emph{面 \& A,}或者/(/\textgreater{}) 6 A,并且满足下面的条件

• (1) \emph{(Tpf'XTpM) + T\textsubscript{fw}A = Tfs\textgreater{}N,而}且

(2) \emph{T\textsubscript{fw}A}
的原象\emph{.(T\textsubscript{f}\^{}A)}在 \emph{T\textsubscript{P}M}
中可裂. 在每一点都与A横截的映射f,称为与子流彩A■檎截.

注意,当M是Hilbert流形(特别地,是有限维流形)时,可裂
性条件(2)是自然成立的(见附注B. 20).

例C.12 (1)如果定JtC.il中子流形4上的每一点都是y 的正则值(见附注B.
28),则\emph{f与A}横截.

(2)如果 Af,N 是有穷维的,且 dim(M) 4-dini(\^{})\textless{}dimN,
则;■与A横截意味着\emph{f(.M) CIA =
0.}例如,把R\textsuperscript{1}嵌入R\textsuperscript{5}的映
射与R中的曲线\emph{7}横截K嵌入曲线与7在It,中无公共点'

由于横截相交性与坐标卡的选取无关,所以当\emph{M与N}都为有
穷维流形时,下面的结果显然成立.

定理C. 13设M和N分别为林维和孙维的C"微分流形, \emph{f
N}是可微映射,A是N的余维为5的\emph{C\textasciitilde{}}子流形.设\emph{M}在
血点附近有局部坐标扌,\ldots{},漬,\emph{N在fg} 点附近有局部坐标
护,\ldots{},俨.如果在『3。)的某邻域U内,点集A Ci U有表达式尸
=\ldots{}=尸=0,则映射『与子流形A横截的充分必要条件是 (尊\textbar{}
在⑶点的秩为s. I

下面设M和N是有限维光滑微分流形,且是第二可数的,人 是N的光滑子流形,记

\begin{quote}
= j/er(A/,N)\textbar{}r\textgreater{}l,/ 与 A 横截},
\end{quote}

虫Sard定理和附注C, 4,可以推出下面的结果.

定理C. 14 (Thom横截定理)\emph{W(M,N;A})是\emph{C\^{}M'N})中
的剩余集(即可数个开調集的交集),从而在\emph{UVM'N)}中稠密.如

果厶是闭子流形,则它还是开的.I

证明可参考[Hir,pp74 - 77丄这个定理说明,若厶是M的闭
子流形,则中与\emph{A}不横截的映射可以在任意小扰动下
成为横截,而原来横截的映射仍可保持横截,因此,利用橫截性可
以得出CTW'N)中的通有族,又称为一般族.

Thom横截定理可以推广到jet形式.这样就可把无限维空间
中的通有性问题,转化到有限维空间\emph{mM,N)}中的横 截问题.

定理C. 15 (Thom横截定理的jet形式) 设\emph{M,N是无}边界
的有限维光滑流形,A是的L子流形,1Mr Vs W 8. 则映射集合

\emph{事=}\{/eCCA/.JOI//与 A 横截\}

在C\textbackslash{}M,N)中是剩余集,从而是稠密集.如果4是团的,则質在
中还是开的.\textbar{}

证明可见[Hir,\textsubscript{PP}80 一 811下面的定理对于确定子流形的余
维数是重要的.

定理C.16设\emph{f'MfN}是5映射,A是N的子流形.如果
f与A横截,则尸'(4)是M的子流形.如果A在N中有有限余 维,则

codim(yTG4)) = codim(A). \textbar{}

特别地,当是浸盖时,条件勺'与4横截''总是溝足 的;而投影是特殊的浸盖.

参考文献

{[}Al{]} Arnold V I. Geometric Methods in the Theory of Ordinary
Differential Equations. Second Edition. New York: Springer-Verlag, 1983

{[}A2{]} Arnold V L Ten problems. Adv. in Sov. Math. 1990(1): 1---8*
数学 译林,1993(4):257-261

CAAIS{]}Afraimovich V \emph{S,} Arnold V I, H如shenko Yu
S\textsubscript{f} Sil\^{}iikov L P・ Bi­furcation Theoryj Dynamical
Systems V . New York: Springer-Ver­lag, 1988

{[}ALGM{]} Andronov A A, Leontovich E A, Gordon I I\textsubscript{t}
Maier A G. Theory of Bifurcations of Dynamic Systems on a Plane- New
York( Israel Program for Sci. Transl. , Wiley« 1973

{[}AMR{]} Abraham R, Marsden J E, Ratiu T. Manifolds, Tensor Analysis,
and Applications. London Amsterdam --- Tokyo; Addison-Wesley Publishing
Company, Inc. , 1983

{[}Ba{]} Bautin N N. On the number of limit cycles which appear with
variation of the coefficients from an equilibrium position of focus or
center type. Math. Sb. 1952, 30: 181 --- 196 (in Russian)i AMS Trans.
Series 1962, 5: 396-413 (in English)

{[}BC1{]} Benedicks M, Carleson L. On iteration of 1 --- «
x\textsuperscript{2} on ( --- 1,1). Ann. Math. 1985, 122: 1-25

{[}BC2{]} . The dynamics of the Hfenon map. Ann Math. 1991» 133: 73 ---

169

{[}Be{]} EeJIHHUKHft r P・ HopMaJILHHe gpMbh I HBapMaHMH H jTOWIbHblC
OTOGpaJKOHMfi- \textsc{Khcbi} Haykosa AyMKa, 1979

{[}BL{]} Bonin G, Legault J. Comparision de la methode des eonstantes de
Lia­punov et la bifurcation de Hopf. Canad. Math. Bulk 1988, 31(2): 200
-209

{[}Bol{]} Budanov R T. Versal deformations of a singular point of a
vector field

on the plane in the case of zero eigenvalues. Trudy Sem. Petrovsk. 1976,
2: 37---65 (in Russian) \emph{i} Sei. Math. Sov. 1981, 1\emph{1}
389---421 (in English)

{[}Bo2{]} ・ Bifurcations of the limit cycle of a family of plane vector
fields.

\begin{quote}
Trudy Sem. Petrovsk. 1976, 2; 23 --- 35 Gn Russain)i Sei. Math. Sov.
1981, \emph{l\textsubscript{t}} 373-387 (in English)
\end{quote}

\begin{enumerate}
\def\labelenumi{\Roman{enumi}.}
\setcounter{enumi}{99}
\item
  陈翔炎.含参数微分方程的周期僻与极限环.教学学报,1963, 13(4): 607 ---
  609
\end{enumerate}

{[}Cyl{]}曹永罗.关于非双曲奇异吸引子.北京大学博士论文,1994

{[}Cy2{]} Cao Yongluo. Strange attractor of H\^{}non map and its basin.
Scientia Sinica (Series A), 1995, 3839-35

{[}CH{]} Chow Shui-Nee, Hale J K. Methods of Bifurcation Theory. New
York: Springer-Verlag, 1982

{[}CLW{]} Chow Shui-Nee\textless{} Li Chengzhi, Wang Duo. Normal Forms
and Bifur­cation of Planar Vector Fields. New Yorkt Cambridge University
Press, 1994

{[}CM{]}蔡燧林,马晖.广义Lizard方程的奇点的中心焦点判定问题.浙江大
学学报.1991, 25(4): 562-589

{[}Cs{]}蔡燧林.二次系统研究近况.数学遂展,1989, 18(1): 5-21

{[}CS{]} Cushman R, Sanders J. A codimension two bifurcation with a
third or­der Picard-Fuchs equation. J・Diff・Eq. 1985, 59: 243 --- 256

{[}CW{]}陈兰嬴,王明二次彼分系统极限芥的相对位置和数目.败学学 报,1979,
22(6): 751-758

{[}CY{]}陈兰莉,叶奮谦.方程组等=-¥ + \&r + ZH,+ zy + p:,\$=z的
极限环的唯一性.数学学报,1975, 18. 219-222

ECZ{]}蔡燧林,张平光.二次系统极限环的唯一性.高校应用敛学学报, 1991,
6(3): 450-461

\begin{enumerate}
\def\labelenumi{\Alph{enumi}.}
\setcounter{enumi}{3}
\item
  Dulac H・ Sur ks cycles limites. Bull. Soc・ Math. Fr. 1923 ・ 511 45
  --- 188
\end{enumerate}

{[}DER{]} Dumortier F • El Morsalani M, Rousseau C. Hilbert\% 16th
problem for quadratic systems and cyclicity of elementary graphics, to
appear in

Nonlinearity

{[}DGZ{]} Drachman B\textsubscript{t} van Gils S A, Zhang Zhi-Fen.
Abelian integrals for quadratic vector fields. J. Reine Angew. Math.
1987, 3821 165---180

EDL{]} 丁同仁,李承治.常微分方程教程.北京:高等教育出版社,1991

{[}DLZ{]} Dumortier F, Li Chengzhi• Zhang Zhi-Fen. Unfolding of a
quadratic integrable system with two centers and two unbounded
hetroclinic loops. Preprint \textsubscript{f} 1996

{[}DRR1{]} Dumortier F\textsubscript{f} Roussarie R, Rousseau C.
Hilbert\% 16th problem for quadratic vector fields, J. Diff・ Eq・ 1994,
1101 86---133

{[}DRR2{]} ・ Elementary graphics of cyclicity one and two.
Nonlinearity,

1994, \emph{h} 86-133

{[}DRS1{]} Dumortier F\textsubscript{f} Roussarie R, Sotomayor J.
Generic 3-parameter family of vector fields on the plane, unfolding a
singularity with nilpotent linear part. The cusp case of codimension 3.
Ergodic Theory and Dy­namical Systems 1987, 7: 375---413

{[}DRS2{]} ・ Generic 3-parameter family o£ planar vector fields
\textsubscript{T} unfolding of

\begin{quote}
saddle, focus and elliptic singularities with nilpotent linear parts.
Lec­ture Notes in Math. 1991* 1480: 1---164
\end{quote}

{[}DZ{]}杜乃林,曾宪穢・计算焦点畳的一类通推公式.科学通报,1994, 39
(19): 1742-1744

{[}E {]} Ecalle E J・ Finitude des cycles limites et acc616ro-sommation
de \^{}application de re tour. Lecture Notes in Math. 1990・ 14551 74
--- 159

{[}EZ{]} Edmunds D E \textsubscript{t} Zheng Z. On the stable periodic
orbits of regular maps on a completely ordered invariant set. Preprint

{[}F{]}泻贝叶.临界情况下奇环的稿定性.数学学报,1990, 33(1): 113-134
{[}FLLL{]} Farr W W, Li Chengzhi\textsubscript{9} Labouriau I
S\textsubscript{t} Langford W F. Degenerete

\begin{quote}
Hopf bifurcation formulas and Hilbert七 16th problem. SIAM Math. Anal.
1989, 20: 13-30
\end{quote}

{[}GH{]} Guckenheimer J•, Holmes P. Nonlinear Oscillations, Dynamical
Systems and Bifurcations of Vector Fields. New York: Springer-Verlag
\textsubscript{v} 1983

{[}GaHo{]} Gavrilov L« Horoaov E・ Limit cycles and zero of Abelian
integrals satisfying third order Picard-Fuchs equations. Lecture Notes
in

Math. 1990, 1455: 160-196

{[}Go{]} \textsc{Pomosob} E IL 9KBMBaJieHTHocTE ceMeAcTB
4n44»eoMOp\^{}x*MOB KaHeMHoro K/iacca rjtaflKocTw- BecTH. XapbKOB.
yw-Ta- Cep. Max. -MaT, 1976, 134(41)、95--- 104

{[}Gw{]}高维新.僻析理论讲义(北京大学数学系用).

\begin{enumerate}
\def\labelenumi{\Alph{enumi}.}
\setcounter{enumi}{7}
\item
  Hilbert H D. M\&thmatische Probleme (lecture). The second
  Internation­al Congress of Mathematicians, Paris 1900» Gottinger
  Nachrichtenj 1900: 253-297
\end{enumerate}

{[}Ha{]} Hayashi S. On the solution of C\textsuperscript{l} stability
conjecture for flows. Preprint. {[}HI{]} Horozov E, Uiev I D. On
saddle-loop bifurcations of limit cycles in per­turbations o£ quadratic
Hamiltonian systems. J. Diff・ Eq. 1994, 113 (1): 84-105

{[}Hir{]} Hirsch M W. Differential Topology. New York Heidelberg
Berlin: Springer-V\^{}rlag, 1976

{[}HLZ{]}韩茂安,罗定军,朱幫明.奇闭轨分支出极限环的唯一性(I
\emph{)、5).} 数学学报,1992, 35(4): 541-548、35(5), 673-684

{[}Hm{]}韩茂安.周期扰动系统的不变环面与亚调和解的分支.中国科学(A
辑),1994(11), 1152-1160

{[}Ho{]} Horozov E. Versal deformations of equivariant vector fields in
the case of symmetry of order 2 and 3. Trans, of Petrov ski Seminar
1979, 5\emph{1} 163 \_192 (in Russian)

{[}Hw{]} Huang Wenzao. The bifurcation theory for nonlinear equations*
Lecture Notes in Pure and Applied Mathematics. VoL 109♦ New York, Marcel
Dekber. INC, 1987, 249---260.

{[}HWW{]} Huang Qichang\textsubscript{t} Wei Junjie. Wu Ji\&ngbong. Hopf
bifurcations oi some second-order FDEs with infinite delay and their
applications. Chinese Science Bulletin, 1995, 40(4):

{[}Hu{]} Hu Sen. A. proof of the C stability conjecture for
3-dimensional flows. Trans. AMS. 1994.

{[}HZ{]}韩茂安.朱德明.微分方程分支理论.北京:煤炭工业出版社,1994

\begin{enumerate}
\def\labelenumi{\Alph{enumi}.}
\setcounter{enumi}{8}
\item
  Il\textbackslash{}eshenko Yu S. Finiteness theorems for limit cycles.
  Russian Math. Surveys 1990, 40\textsubscript{1} 143-200
\end{enumerate}

{[}IL{]} UVashenko Yu S, Li Weigu. Nolocal Bifurcation\textsubscript{f}
to be published by AMS

{[}IY{]} UVashenko Yu S, Yakovenko S. Finitely smooth normal forms of
local families diffeomorphisms and vector fields. Russian Math. Surveys
1991, 46: 1-43

{[}Jajjakobson M V. Absolutely continuous invariant measure for
one-parame­ter family of one-dimensional map- Comm. Math. Phys. 1981*
81: 39 --- \emph{88}

{[}Jo{]} Joyal P. Generalized Hopf bifurcation and its dual generalized
hoinuclinic bifurcation. SIAM J. Math. 1988. 48: 481---496

{[}K{]} Khovansky A G・ Real analytic manifolds with finiteness
properties and complex Abelian integrals. Funct. Anal. Appl. 1984, 18:
119---128

{[}LI{]}廖山涛.微分动力系统的建性理论.北京,科学出版社,1992

{[}L2{]} Liao Shantao. Obstruction sets, Minimal rambling sets and their
applica­tions. In: Oiinese Mathematics into 21st Century. Peking
University Press, 1992.

{[}Lc{]}李承治.关于平面二次系统的两个向風 中国科学(A1»)・1982Q2),
1087-1096

{[}LbZ{]} Li Bao-Yi・ Zhang Zhi-Fen. A note on a result of G・ S・
Petrov about the weakened 16th Hilbert problem. JMAA 1995, 190:
489\^{}516

{[}LH{]}李继彬,黄其明.平面三次微分系统的极限环复眼分支.数学年刊(B
辑),1987, 8; 391-403

{[}LHZ{]}罗定军,韩茂安.朱德明.奇闭轨分支出极限环的唯一性(I ).数学
学报,1992, \emph{35(3)\textsubscript{;} 407-417}

{[}LR1{]} Li Chengzhi, Rousseau C. A system with three limit cycles
appearing in a Hopf bifurcation and dying in a homoclinic bifurcation
\textsubscript{t} the cusp of order 4. J. Diff. Eq. 1989, 79: 132-167

{[}LR2{]} ・ Codimension \emph{2} symmetric homoclinic bifurcation. Cam
J.

Math. 1990, 42: 191 一212

{[}Lw{]} Li Weigu. The bifurcation of "eight figure" of separatrix of
saddle with zero saddle value in the plane» Preprint of Peking
University \textsubscript{f} Research Report \textsubscript{f} No 46,
1995

{[}Lz{]}梁肇军.多项式律分系统全简分析导引.武汉,华中师范大学出版社. 1989

{[}LZ{]} Li Chengzhi, Zhang Zhi-Fen. A criterion for determing the
monotonici­ty of the ratio of two Abelian integrals. J. Diff. Eq・,1996,
124\$ 407--- 424

{[}M{]} MaM R・ A proof of the C\textsuperscript{1} stability
conjecture. Inst. Hautes. Sci・ Publ. Math. 1987- 66, 161-210

{[}Maj Mardesic P. The Number of limit cycles of polynomial deformations
of a Hamiltonian vector field. Ergod. Th・ \& Dynam. Sys. , 1990, 10;
523 \_529

{[}Mo{]} Mourtada A. Degenerate and non-trivial hyperbolic polycycles
with two vertices. J・ Diff. Eq. , 1994, 113: 68---83

{[}MV{]} Mora L, Viana L・ Abundance of strange attractor. Acta. Math.
1993, 170, 1-63

{[}Ma{]}马知恩.种群生态学的数学建模与研究.合肥,安律教育出版社,1996
{[}NS{]} Nowicki T, Strien V S. Absolutely continuous invariant measures
forC* unimodal maps satisfying the Collet-Eckmann condition- Inv・ Math.
1988, 93: 619---635

{[}Pl{]} Petrov G S・ Number of zeros of complete elliptic integrah.
Funct. Anal. AppL 1988. 18, 148---149

{[}P2{]} ・ The Chebyshev property of elliptic integrals. Funct. Anah
AppL

1988. 22: 72-73

{[}Pa{]} Palis J, A proof of the \emph{Q-} stability conjecture. Inst,
Hautes. Sci. Pubh Math. 1987, 66, 211-218

{[}Pon{]} nOHTTJITHH JI Q O flMHlMHTeCKHX CHCTeMaX, \textsc{GjIMSKIDC} K
EMHJTVTOHODbJM. \^{}ypHUi SKcnepHMeHTLrrbHoA \textsc{h} TeopeTHwecKofi
\textless{}t)HamcK, 1934* 4: 883 --- 885

{[}PT{]} Palis J, Taken,F. Hyperbolicity and sensitive chaotic dynamics
at ho­moclinic bifurcations. Cambridge University Press« 1992

{[}Q{]}秦元«h律分方程所定义的积分曲线,上、下册,北京*科学出版社,
1956C959

{[}QL{]}秦元勛,刘尊全.律分方程公式的机器推导(■)・科学通报,1981,7:
388-391

{[}Rl{]} Roussarie R. Weak and continuous equivahnces for families of
line dif- feomorphisms. In \^{}Dynamical Systems and Bifurcation
Theory*, Cama­cho, Pacifico ed.\emph{,}Longman, Scientific and
Technical, Pitman Research Notes in Math. Series 160, 1987* 377-385

{[}R2{]} ■ On the number of limit cycles which appear by perturbation of

\begin{quote}
separatrix loop of phnar vector fields, BoL Soc・ Bras. Mat. 1986, 17:
67-101
\end{quote}

{[}Rc{]} Rousseau C- Universal unfolding of a singularity of a symmetric
vector field with 7-jet C™- equivalent to + (± J:\textsuperscript{3} ±
\emph{冨・} Lecture Notes in Math. 1989, 1455\$ 334---354

{[}RS{]} Rousseau C; Schlotniuk D. Generalized Hopf bifurcations and
applica­tions to planar quadratic systems. Ann. Polon・ Math. 1988, 49*
1 --- 16

{[}RT{]} Ruelle D, Tokens F. On the nature of turbulence. Common. Math.
Phys. 1971. 20: 167-192

{[}S{]} Singer D・ Stable orbits and bifurcation of maps of the
interval. SIAM Ap- pL Math. 1978, 35: 260-267

{[}Sh{]} Shoshitaishvili A N・ Bifurcation of topok\^{}ical type of
singular points of parameterized vector fields. Funct. Anal. AppL
1972(2): 169 --- 170

{[}Si{]} Sibirskii K S. On the number of limit cycles in a neighborhood
of singular points. Diff. Eq. 1965, 11 36---47 (in Russian)

{[}Sij{]} Sijbrand J. Properties of center manifold亀 Trans. AMS 1985,
289; 431 -469

{[}Sill{]} Sil\^{}iikov L P. On a Poincart-Birkhoff problem. Math USSR
Sb. 3 i 353-371 .

{[}Sil2{]} . On the generation of a periodic motion from a trajectory
doubly

\begin{quote}
asymptotic to an eqilibrium state of saddle type. Math USSR Sb 1968, 6:
428-438
\end{quote}

{[}Sil 3{]}--- ・ A contribution to the problem of the structure of an
extended neighborhood of a rough equilibrium state of saddh-focus type.
Math USSR Sb 1970. 10, 91-102

{[}SJ{]} Shen Jaqi, Jing Zujun. A new detecting method of conditions for
the ex-

istence of Hopf bifurcation. In "Dynamical Systems* Nankai Series in
Pure, Apl. Math, and Theor・ Phy. , Vol 4, eds. S*T Liao,T-R Ding and
Y-Q Ye, World Scientific Publishing, Singapore» 1993\$ 188---203
{[}Sm{]} Smale S. Dynamics retrospective, great problems, attempts that
failed.

\begin{quote}
Nonlinear Science, The Next Decade.见"数学译林'',动力系统学的回
顾;宣大问题■失败的尝试.1993(4),262-269''
\end{quote}

{[}\%{]}史松龄.平面二次系统存在四个极限环的具体例子.中国科学,1979

(11), 1051-1056

{[}T{]} Takens F. Forced oscillations and bifurcations: Applications of
global analysis I . In Commun. Math. Vol 3 Inst. Rtjksuniv. Utrecht. ,
1974 {[}TTY{]} Thieullen P, Tresser C, Young L-S・ Positive Liapunov
exponent for gereric one-parameter families of unimodal maps. Preprint

\begin{enumerate}
\def\labelenumi{\Roman{enumi}.}
\setcounter{enumi}{4}
\item
  Vanderbauwhede A. Center manifold私 normal forms and elementary
  bi­furcations. Tn Dynamics Reported, Vol 2, ed\& U. Kirchgraber and
  O・ Walther, New York. Wiley, 1989, 89-169
\end{enumerate}

{[}V\&{]} Varchenko A N. Estimate of the number of zeros of an Abelian
integral depending on a parameter and limt cycles. Funct. AnaL AppL
1984, 18: 98-108

{[}W{]} Wen Lan, On the C stability conjecture for flows. {]}. Diff. Eq-
1996, 129(2): 334---357

{[}Wd{]} Wang Duo- An introduction to the normal form theory of odinary
dif­ferential equations. Advances in Math. 1990, 19(1)\$ 38 --- 71

{[}Wil{]} Wiggins S・ Introduction to Applied Nonlinear Dynamic成
Systems and Chaos. New York Berlin Heidelberg: Springer-Verlag,1990

{[}Wi2{]} , Global Bifurcations and Chaos \textsubscript{T} Analytical
Methods. New York

Berlin Heidelberg ■ Springer-Verlag»198 8

{[}Wl{]}王兰宇.多峰映射的动力学.北京大学博士论文.1996

\begin{enumerate}
\def\labelenumi{\Roman{enumi}.}
\setcounter{enumi}{9}
\item
  肖冬梅.一类余维3鞍点型平面向員场的分支,中国科学(A辑).1993
\end{enumerate}

(3); 252------262

LY1{]}叶彦谦等.极限环论■第二版.上海:上海科学技术出版社,1984
{[}Y2{]}叶彦3L多项式微分系统定性理论.上海,上梅科学技术出版社,1995
\textsc{{[}Yq{]} Yoccoz} J C. Recent developments in dynamics. Plenary
Address of the

TCM 勺4

{[}YY{]}杨信安,叶彦谦.方程糸=-】 + \&■ + \#+玲+陌,咨=工的
极限环的唯一性.福州大学学报,1978(2): 122-127

{[}Z{]} Zheng Zhiming. On the abundance of chaotic behavior for generic
one-pa­rameter families of maps. Acta Math- Sinica, 1996(12); 398---412

{[}Zdl{]} Zhu Deming, Melnikov vector and heteroclinic manifolds.
Science tn China, Ser.A, 1994. 37(6), 673-682.

{[}Zd2{]} ・ Melnikov-type vectors and principal normals. Science in
China,

Se``A, 1994, 37(7); 814-822.

{[}Zd3{]} . Transversal hetroclinic orbits in general degenerate cases.
Science

in China, Sen A, 1996, 39(2): U2-12L

EZDHD{]}张芷芬.丁同仁,黄文灶.董镇喜.微分方程定性理论.北京:科学
出版社.1985

:Zg{]}张恭庆.临界点理论及其应用.上海,上海科学技术出版社.1986

{[}Zj{]}张嫦炎.常微分方程几何理论与分支问庖(修订本).北京,北京大学出
版社,1987

{[}ZJ{]} Zeng Xianwu, Jing Zujun, Monotonicity and critical points of
period.

Prepress in Natural Science, 1996, 6(4) t 401 --- 407

{[}ZQ{]}张锦炎,钱敏.微分动力系统导引.北京,北京大学出版社.1991

{[}Zol{]} Zoladek H・ On the versality of certain family o£ vector
fields on the plane. Math. USSR Sb. 1984, 48: 463-492

{[}Zo2{]} ・ Bifurcations of certain family of planar vector fields
tangent to

axes. J. Diff, Eq. 1987 ・ 671 1 --- 55

{[}N{]}涅梅斯基B B.四十年来的苏联数学(1917-1957),常做分方程部分.
饶生忠译.北京:科学出版社,I960

{[}Zzfl{]} Zhang Zhi-fen・ On the uniqueness of the limit cycles of some
nonlinear oscillation equations. Doki. Acad. Nauk SSSR» 1958, 119\$ 659
--- 662 (in Russian )

{[}Zzf2{]} . Proof of the uniqueness theorem of limit cycles of
generalized

Li\&nard equations. Applicable Analysis, 1986, 23\$ 63---67.

囱晩{]}张筑生.微分动力系统原理.北京,科学出版社,1987

中文词条按首字的笔画排列,西文开头的词条按字母顺序排列.

\begin{longtable}[]{@{}llll@{}}
\toprule
\endhead
\begin{minipage}[t]{0.22\columnwidth}\raggedright
\begin{quote}
{---}
\end{quote}\strut
\end{minipage} & \begin{minipage}[t]{0.22\columnwidth}\raggedright
画\strut
\end{minipage} & \begin{minipage}[t]{0.22\columnwidth}\raggedright
\begin{quote}
五
\end{quote}\strut
\end{minipage} & \begin{minipage}[t]{0.22\columnwidth}\raggedright
\strut
\end{minipage}\tabularnewline
子流形 & 7,46-50,286 & 正规形 & 34,40\tabularnewline
\begin{minipage}[t]{0.22\columnwidth}\raggedright
马蹄映射\strut
\end{minipage} & \begin{minipage}[t]{0.22\columnwidth}\raggedright
\begin{quote}
170
\end{quote}\strut
\end{minipage} & \begin{minipage}[t]{0.22\columnwidth}\raggedright
正则奇点\strut
\end{minipage} & \begin{minipage}[t]{0.22\columnwidth}\raggedright
117,118\strut
\end{minipage}\tabularnewline
\begin{minipage}[t]{0.22\columnwidth}\raggedright
马蹄存在定理\strut
\end{minipage} & \begin{minipage}[t]{0.22\columnwidth}\raggedright
\begin{quote}
173
\end{quote}\strut
\end{minipage} & \begin{minipage}[t]{0.22\columnwidth}\raggedright
正则映射族\strut
\end{minipage} & \begin{minipage}[t]{0.22\columnwidth}\raggedright
281\strut
\end{minipage}\tabularnewline
\begin{minipage}[t]{0.22\columnwidth}\raggedright
\begin{quote}
四
\end{quote}\strut
\end{minipage} & \begin{minipage}[t]{0.22\columnwidth}\raggedright
19\strut
\end{minipage} & \begin{minipage}[t]{0.22\columnwidth}\raggedright
对参数一致的Hopf分岔定理 82\strut
\end{minipage} & \begin{minipage}[t]{0.22\columnwidth}\raggedright
\strut
\end{minipage}\tabularnewline
\begin{minipage}[t]{0.22\columnwidth}\raggedright
双曲奇点\strut
\end{minipage} & \begin{minipage}[t]{0.22\columnwidth}\raggedright
\begin{quote}
4
\end{quote}\strut
\end{minipage} & \begin{minipage}[t]{0.22\columnwidth}\raggedright
可扰动参数\strut
\end{minipage} & \begin{minipage}[t]{0.22\columnwidth}\raggedright
282\strut
\end{minipage}\tabularnewline
\begin{minipage}[t]{0.22\columnwidth}\raggedright
双曲闭轨\strut
\end{minipage} & \begin{minipage}[t]{0.22\columnwidth}\raggedright
\begin{quote}
4
\end{quote}\strut
\end{minipage} & \begin{minipage}[t]{0.22\columnwidth}\raggedright
可裂\strut
\end{minipage} & \begin{minipage}[t]{0.22\columnwidth}\raggedright
297\strut
\end{minipage}\tabularnewline
\begin{minipage}[t]{0.22\columnwidth}\raggedright
双曲不动点定理\strut
\end{minipage} & \begin{minipage}[t]{0.22\columnwidth}\raggedright
\begin{quote}
160
\end{quote}\strut
\end{minipage} & \begin{minipage}[t]{0.22\columnwidth}\raggedright
边界的水平部分\strut
\end{minipage} & \begin{minipage}[t]{0.22\columnwidth}\raggedright
159,198,200\strut
\end{minipage}\tabularnewline
\begin{minipage}[t]{0.22\columnwidth}\raggedright
分岔\strut
\end{minipage} & \begin{minipage}[t]{0.22\columnwidth}\raggedright
\begin{quote}
9
\end{quote}\strut
\end{minipage} & \begin{minipage}[t]{0.22\columnwidth}\raggedright
边界的垂直部分\strut
\end{minipage} & \begin{minipage}[t]{0.22\columnwidth}\raggedright
159,198,200\strut
\end{minipage}\tabularnewline
\begin{minipage}[t]{0.22\columnwidth}\raggedright
分岔集\strut
\end{minipage} & \begin{minipage}[t]{0.22\columnwidth}\raggedright
\begin{quote}
9
\end{quote}\strut
\end{minipage} & \begin{minipage}[t]{0.22\columnwidth}\raggedright
边界条件\strut
\end{minipage} & \begin{minipage}[t]{0.22\columnwidth}\raggedright
160\strut
\end{minipage}\tabularnewline
\begin{minipage}[t]{0.22\columnwidth}\raggedright
分宿值\strut
\end{minipage} & \begin{minipage}[t]{0.22\columnwidth}\raggedright
\begin{quote}
9
\end{quote}\strut
\end{minipage} & \begin{minipage}[t]{0.22\columnwidth}\raggedright
主稳定方向\strut
\end{minipage} & \begin{minipage}[t]{0.22\columnwidth}\raggedright
191\strut
\end{minipage}\tabularnewline
\begin{minipage}[t]{0.22\columnwidth}\raggedright
分岔图\strut
\end{minipage} & \begin{minipage}[t]{0.22\columnwidth}\raggedright
19,47\strut
\end{minipage} & \begin{minipage}[t]{0.22\columnwidth}\raggedright
\begin{quote}
六
\end{quote}\strut
\end{minipage} & \begin{minipage}[t]{0.22\columnwidth}\raggedright
19\strut
\end{minipage}\tabularnewline
\begin{minipage}[t]{0.22\columnwidth}\raggedright
分岔曲线\strut
\end{minipage} & \begin{minipage}[t]{0.22\columnwidth}\raggedright
\begin{quote}
19
\end{quote}\strut
\end{minipage} & \begin{minipage}[t]{0.22\columnwidth}\raggedright
轨道\strut
\end{minipage} & \begin{minipage}[t]{0.22\columnwidth}\raggedright
2,3.168\strut
\end{minipage}\tabularnewline
分宿方程 & 62,63 & 同宿轨 & 10\tabularnewline
分岔函数 & 60,63 & 同宿点 & 168\tabularnewline
\begin{minipage}[t]{0.22\columnwidth}\raggedright
分岔的余维\strut
\end{minipage} & \begin{minipage}[t]{0.22\columnwidth}\raggedright
\begin{quote}
46
\end{quote}\strut
\end{minipage} & \begin{minipage}[t]{0.22\columnwidth}\raggedright
同宿分岔\strut
\end{minipage} & \begin{minipage}[t]{0.22\columnwidth}\raggedright
15,85\strut
\end{minipage}\tabularnewline
\begin{minipage}[t]{0.22\columnwidth}\raggedright
无穷阶非共振\strut
\end{minipage} & \begin{minipage}[t]{0.22\columnwidth}\raggedright
\begin{quote}
41
\end{quote}\strut
\end{minipage} & \begin{minipage}[t]{0.22\columnwidth}\raggedright
异宿轨\strut
\end{minipage} & \begin{minipage}[t]{0.22\columnwidth}\raggedright
10,23,114\strut
\end{minipage}\tabularnewline
\begin{minipage}[t]{0.22\columnwidth}\raggedright
无限C水平曲线\strut
\end{minipage} & \begin{minipage}[t]{0.22\columnwidth}\raggedright
\begin{quote}
159
\end{quote}\strut
\end{minipage} & \begin{minipage}[t]{0.22\columnwidth}\raggedright
异宿点\strut
\end{minipage} & \begin{minipage}[t]{0.22\columnwidth}\raggedright
168\strut
\end{minipage}\tabularnewline
\begin{minipage}[t]{0.22\columnwidth}\raggedright
无限C垂直曲线\strut
\end{minipage} & \begin{minipage}[t]{0.22\columnwidth}\raggedright
\begin{quote}
159
\end{quote}\strut
\end{minipage} & \begin{minipage}[t]{0.22\columnwidth}\raggedright
异宿分念\strut
\end{minipage} & \begin{minipage}[t]{0.22\columnwidth}\raggedright
146\strut
\end{minipage}\tabularnewline
\begin{minipage}[t]{0.22\columnwidth}\raggedright
开折\strut
\end{minipage} & \begin{minipage}[t]{0.22\columnwidth}\raggedright
\begin{quote}
44
\end{quote}\strut
\end{minipage} & \begin{minipage}[t]{0.22\columnwidth}\raggedright
共振\strut
\end{minipage} & \begin{minipage}[t]{0.22\columnwidth}\raggedright
35.40\strut
\end{minipage}\tabularnewline
\begin{minipage}[t]{0.22\columnwidth}\raggedright
不徳定流形\strut
\end{minipage} & \begin{minipage}[t]{0.22\columnwidth}\raggedright
\begin{quote}
7
\end{quote}\strut
\end{minipage} & \begin{minipage}[t]{0.22\columnwidth}\raggedright
共振多项式\strut
\end{minipage} & \begin{minipage}[t]{0.22\columnwidth}\raggedright
36,40\strut
\end{minipage}\tabularnewline
\begin{minipage}[t]{0.22\columnwidth}\raggedright
切向童\strut
\end{minipage} & \begin{minipage}[t]{0.22\columnwidth}\raggedright
\begin{quote}
290
\end{quote}\strut
\end{minipage} & \begin{minipage}[t]{0.22\columnwidth}\raggedright
闭轨分岔\strut
\end{minipage} & \begin{minipage}[t]{0.22\columnwidth}\raggedright
64\strut
\end{minipage}\tabularnewline
\begin{minipage}[t]{0.22\columnwidth}\raggedright
切空间\strut
\end{minipage} & \begin{minipage}[t]{0.22\columnwidth}\raggedright
\begin{quote}
290
\end{quote}\strut
\end{minipage} & \begin{minipage}[t]{0.22\columnwidth}\raggedright
曲线坐标\strut
\end{minipage} & \begin{minipage}[t]{0.22\columnwidth}\raggedright
66\strut
\end{minipage}\tabularnewline
\begin{minipage}[t]{0.22\columnwidth}\raggedright
切丛\strut
\end{minipage} & \begin{minipage}[t]{0.22\columnwidth}\raggedright
\begin{quote}
290
\end{quote}\strut
\end{minipage} & \begin{minipage}[t]{0.22\columnwidth}\raggedright
多重极限辞\strut
\end{minipage} & \begin{minipage}[t]{0.22\columnwidth}\raggedright
68\strut
\end{minipage}\tabularnewline
切映射 & 291,292 & 后继函数 & 67,76,95\tabularnewline
\bottomrule
\end{longtable}

\begin{longtable}[]{@{}llll@{}}
\toprule
\endhead
自由返回 & 254 & 典型族 & 228\tabularnewline
\begin{minipage}[t]{0.22\columnwidth}\raggedright
有界返回\strut
\end{minipage} & \begin{minipage}[t]{0.22\columnwidth}\raggedright
. 255\strut
\end{minipage} & \begin{minipage}[t]{0.22\columnwidth}\raggedright
\begin{quote}
九画
\end{quote}\strut
\end{minipage} & \begin{minipage}[t]{0.22\columnwidth}\raggedright
\strut
\end{minipage}\tabularnewline
全局中心藏形 & 24 & 结构稳定 & 6,8\tabularnewline
全局中心流形定理 & 23 & 矩阵表示法 & 38\tabularnewline
全局分岔 & 10 & 相交条件 & 160\tabularnewline
向量丛 & 288,289 & 复合映射的双曲性 & 178\tabularnewline
\begin{minipage}[t]{0.22\columnwidth}\raggedright
纤维型\strut
\end{minipage} & \begin{minipage}[t]{0.22\columnwidth}\raggedright
289\strut
\end{minipage} & \begin{minipage}[t]{0.22\columnwidth}\raggedright
\begin{quote}
十画
\end{quote}\strut
\end{minipage} & \begin{minipage}[t]{0.22\columnwidth}\raggedright
\strut
\end{minipage}\tabularnewline
\begin{minipage}[t]{0.22\columnwidth}\raggedright
\begin{quote}
七百
\end{quote}\strut
\end{minipage} & \begin{minipage}[t]{0.22\columnwidth}\raggedright
\strut
\end{minipage} & \begin{minipage}[t]{0.22\columnwidth}\raggedright
弱Hilbert第16问题\strut
\end{minipage} & \begin{minipage}[t]{0.22\columnwidth}\raggedright
98---100,109\strut
\end{minipage}\tabularnewline
局部分岔 & 10 & 弱等价 & 45\tabularnewline
局部中心流形 & 25 & 通有族 & \emph{47}\tabularnewline
局部中心流形定理 & 25 & 倍周期分岔 & 16\tabularnewline
局部族 & 44 & 高阶Melnikov函数 & \emph{97}\tabularnewline
局部表示 & 286 & 积检形 & \emph{287}\tabularnewline
更簪法 & 60 & 浸入 & 7.297\tabularnewline
投影 & 288,289 & 浸盖 & 49,297\tabularnewline
\begin{minipage}[t]{0.22\columnwidth}\raggedright
极限环\strut
\end{minipage} & \begin{minipage}[t]{0.22\columnwidth}\raggedright
13,67\strut
\end{minipage} & \begin{minipage}[t]{0.22\columnwidth}\raggedright
\begin{quote}
+\_画
\end{quote}\strut
\end{minipage} & \begin{minipage}[t]{0.22\columnwidth}\raggedright
\strut
\end{minipage}\tabularnewline
芽 & \emph{44} & &\tabularnewline
& & 移位映射 & 167\tabularnewline
坐标卡 & 284 & &\tabularnewline
\begin{minipage}[t]{0.22\columnwidth}\raggedright
\begin{quote}
八画
\end{quote}\strut
\end{minipage} & \begin{minipage}[t]{0.22\columnwidth}\raggedright
\strut
\end{minipage} & \begin{minipage}[t]{0.22\columnwidth}\raggedright
符号动力系统\strut
\end{minipage} & \begin{minipage}[t]{0.22\columnwidth}\raggedright
165\strut
\end{minipage}\tabularnewline
\begin{minipage}[t]{0.22\columnwidth}\raggedright
\strut
\end{minipage} & \begin{minipage}[t]{0.22\columnwidth}\raggedright
\strut
\end{minipage} & \begin{minipage}[t]{0.22\columnwidth}\raggedright
\begin{quote}
+二画
\end{quote}\strut
\end{minipage} & \begin{minipage}[t]{0.22\columnwidth}\raggedright
\strut
\end{minipage}\tabularnewline
周期点 & 4,168 & &\tabularnewline
周期執 & 168 & 普适开折 & 19.45\tabularnewline
拓扑執道等价 & 5 & 焦点量 & 78\tabularnewline
环 & 216 & 游荡环 & 217.227\tabularnewline
非游荡集 & 4 & 超稳定周期轨道 & 243\tabularnewline
非游荡环 & 217 & 遍历 & 271\tabularnewline
非共振 & 41 & 揉序列 & 276\tabularnewline
非本质自由返回 & 255 & 嵌入 & 300\tabularnewline
\begin{minipage}[t]{0.22\columnwidth}\raggedright
单重极限环\strut
\end{minipage} & \begin{minipage}[t]{0.22\columnwidth}\raggedright
68\strut
\end{minipage} & \begin{minipage}[t]{0.22\columnwidth}\raggedright
\begin{quote}
+三画
\end{quote}\strut
\end{minipage} & \begin{minipage}[t]{0.22\columnwidth}\raggedright
\strut
\end{minipage}\tabularnewline
单峰映射 & 243 & 临界元 & 4\tabularnewline
奇点分岔 & 58.64 & 强等价 & 45\tabularnewline
细焦点 & \emph{73} & 稠密轨道 & 169\tabularnewline
细鞍点 & \emph{93} & 辐角原理 & 124\tabularnewline
\bottomrule
\end{longtable}

\begin{longtable}[]{@{}llll@{}}
\toprule
\endhead
微分流形 & 285 & ,次正规形 & 34\tabularnewline
微分结构 & 285 & \emph{i-jet} & 49\tabularnewline
微分同胚 & 286 & \emph{i}阶非共振 & 41\tabularnewline
零截面 & 288,289 & 为参数开折 & 44\tabularnewline
\begin{minipage}[t]{0.22\columnwidth}\raggedright
\begin{quote}
十四画
\end{quote}\strut
\end{minipage} & \begin{minipage}[t]{0.22\columnwidth}\raggedright
\strut
\end{minipage} & \begin{minipage}[t]{0.22\columnwidth}\raggedright
人阶Hopf分岔\strut
\end{minipage} & \begin{minipage}[t]{0.22\columnwidth}\raggedright
73\strut
\end{minipage}\tabularnewline
稳定流形 & 7 & \&阶细焦点 & 73\tabularnewline
稳定周期轨道 & 243 & Lebesgue 测度 & \emph{242}\tabularnewline
端点 & 158 & Lebesgue正桐点 & 282\tabularnewline
\begin{minipage}[t]{0.22\columnwidth}\raggedright
\begin{quote}
十五画
\end{quote}\strut
\end{minipage} & \begin{minipage}[t]{0.22\columnwidth}\raggedright
\strut
\end{minipage} & \begin{minipage}[t]{0.22\columnwidth}\raggedright
Lebesgue全稠点\strut
\end{minipage} & \begin{minipage}[t]{0.22\columnwidth}\raggedright
282\strut
\end{minipage}\tabularnewline
鞍结点分岔 & 11 & Liapunov系数法 & 78.79\tabularnewline
鞍点量 & 190,200 & Liapunov-Schmidt 方法 & 60.62,63\tabularnewline
鞍焦点 & 43 & Maigrartge 定理 & 18\tabularnewline
鞍焦环 & 216 & Melnikov 函数 & 90,97,109\tabularnewline
横截 & 46,304 & Misiurewicz 条件 & 282\tabularnewline
& & Morse-Smale 向■场 & 8\tabularnewline
Abel 积分 99,101,106-109 & Pichfork 分岔 & 11 &\tabularnewline
Banach流形 & 284 & Picard-Fuchs 方程 & 111\tabularnewline
Bogdanov-Takens \emph{系统} & 47,109.130 & Pioneer\^{} 映射 &
\emph{5}\tabularnewline
Birkhoff-Smale 定理 & 184 & Pioncare 分岔 & 94\tabularnewline
C水平曲线 & 158 & Hiss约化原理 & 27\tabularnewline
C垂直曲线 & 158 & 序单峰映射 & 244\tabularnewline
C水平带域 & 159 & Smale马歸 & 170\tabularnewline
C垂直帯城 & 159 & Schwartz 导数 & 243\tabularnewline
6线性化 & 41,43 & Thom横检定理 & 305,306\tabularnewline
germ & 44 & (/\%,㈤)矩形 & 159\tabularnewline
Fuchs 31 方程 & 118 & (相■四)矩形的高 & 162,163\tabularnewline
Hartman-Grobman 定理 & 6 & \emph{g心}矩形的宽 & 162,163\tabularnewline
Hilbert第16问题 & 99 & (如代)锥形条件 & 159\tabularnewline
Hilbert-Arnold 问题 & 98 & 。爆炸 & 228\tabularnewline
Hopf分岔 & 13 & &\tabularnewline
\bottomrule
\end{longtable}

责任编辑杨芝馨 封面设计季思九 责任绘囲都林

版式设计李承治

责任印制王彦

(京\textbf{)112}号

全书分为六章,各章内容分别是:基本槪念和准备知识,常见的局部与菲局都分
岔,几类余维2的平面同曩场分岔,双曲不动点及马蹄存在定理,空间中双曲戦点的同
宿分岔,实二次单峰映射族的吸引子.在訶三章的每章之后,部配备了一定致量的习
题.

本书可作为高等学校數学专业高年级本科生的选修课敎材,或相关专业研究生的
基BB课教材;也可供希望了解分岔理佬这门学科的学生、教师或科技人员作为参考书.

图书在版编目\textbf{(CIP)}数据

向量场的分岔理论基础/张芷芬等编.-北京:高等教

育出版社,1997

ISBN 7-04-006216-X

I.向\ldots{} 口.张\ldots{}皿.矢量场 0413.3

中国版本图书馆CIP数据核字(97)第22634号

高等教育出版社出版\\
北京沙摊后街55号\\
邮政编码:100009 传真\textsubscript{:}64014048 电话:64054588

新华书店总店北京发行所发行

国防工业出版社印刷广印刷

*■

开 \^{} 850X 1168 1/32 印张 10.375 字敷 250 000

.1997年10月第I版 1997年10月第I次印刷

印数 0 001-1 715\\
定价10.20元

凡购买高等敎育出版社的图书,如有缺页、倒页、脱页等\\
质量问题者,请与当地图书箫售部门联系调换

版权所有.不得間印

\includegraphics[width=2.33333in,height=3.7in]{media/image90.jpeg}



