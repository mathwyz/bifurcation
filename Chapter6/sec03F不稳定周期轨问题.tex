\section{F(x,a)不存在稳定周期轨问题}

在本节里,我们要证明下面的定理:

定理3.1 存在具有正Lebesgue测度的集合 厶C (0,2),使 得对所有a £丄,映射F
, 屹1 一妃岸色J没有稳定的 題熱凱.

显然,F3,a),a £ (0,2),是S单峰映射.特别,由Singer理 论,当a 6
3,2)血靠近2时,F(z,a)在\emph{J(F)丄}中无稳定不动点
而在中至多有一条稳定周期轨;并且假如x = 0没有
被吸引到一条周期轨,那么F\textless{}x,a)没有稳定周期轨•于是,定理
3.1可以从下面的定理3. 2得到.

定理3.2 存在ZUU(0,2),丄具有正Lebesgue測度,以及
一个正整数蜘,使得对所有Q £丄,

\textbar{}为硏(1,。)\textbar{} 2exp(*3),
\textsubscript{v\textgreater{}yo},

证明这个定理的证明比较长.因此,我们将它分成下面三 个部分.

A我们根据\textgreater{}(0,«)不断返回到/•的特点,讨论返回的形式.
与此同时用自由返回概念,归纳地将参数区间4分类.

由引理2. 3,存在一个最小整数观J〉\emph{N)},使得

\begin{quote}
/1\textsuperscript{!} \emph{a}
\end{quote}

是4,到厂上的1 一 1映射.称旳是映射的首次自由返回的指标,
相应的参数首次分类为

厶1 = (o+l) U /\textsc{'「'(Ach).}

在下一步的讨论中,参数集合耳=4,/4不再考虑,从4中排除
呂是基于如下事实'因为F",(o,A)=尸,因此存在参数角G 4,,
使得0点是FU.oo)的周期为屿的超稳定周期点.为了保证定理
结论成立,将向连同發的一个邻域\&排除掉.请注意,实际上在参
数区间A中,我们全部排除了使兀可能有小于等于观I的稳定周 期轨的参数.

设B是第艮次分类的任意一个构成区间\#21.下面我们定义
第龙+1次自由返回以及相应的分类.

设=7''《•=[如切,\textbar{}``\textbar{} \textgreater{}
0.我们要讨论3在\emph{F} 下的进一步划分,定义非负整数\emph{P}
=少(3),它是使 聞S 一 \emph{Fg©} \textbar{} C鸟辭

成立的最大值.更确切地讲,对所有a £ ®和所有? G (0,c''\textgreater{}),

有伝)一*3,a)\textbar{} M 饗,顶=1,2,・``,夕. (3.1)

但是存在a£s和〃£ \textsc{(O,c``t),}

屁+危)---"+】(")\textbar{}\textgreater{} \sout{J 篇界,.} \textless{}3.
2)

应当注意,对某些\emph{P,\^{}\textsubscript{i+j}}怎)可能返回到厂,我们将它
们记为皿}如,并被称为有界返回.

令,'是最小整数\emph{i\textgreater{}p+l,}使得尸"*+'(0,3)PI广=丄*0.
下面要做的工作是,在任基础上,迸一步从d中排除一些参数集
合,使得在任的剩余集中,不含使\emph{f\textsubscript{a}}有小于等于\^{}\textsubscript{i+1}的稳定周期
轨的参数.不仅如此,相应于剩余集中的参数映射\emph{f\textsubscript{a}}至少在峋
时是指数扩张的.

令\emph{Si}是小区间\emph{ly}的指标集,满足(i)石U聂和(ii)
\textbar{}y\textbar{} M。+
\emph{k.}但是我们在s中排除掉满足条件\&,+:伝)£小,卩\textbar{}
\emph{\textgreater{}a + k +} 1的所有参数(记作瓦,y).

\begin{enumerate}
\def\labelenumi{(\alph{enumi})}
\item
  如果\& = 0,那么我们不做分类而是继续迭代下去.用 爆\textsubscript{1}
  =讯* + \emph{i}表示第一次非本质自由返回的指标,在独1之后,重复
  上述过程,令A表示可能的有界返回指标,小是最小的正整数使
  得它重新返回到\emph{I'}.此时伝)可能再一次出现.此时用\emph{n\textsubscript{ki}
  = \% + Px} + ?i表示第二次非本质自由返回的时刻,并继续这一过程.由
  后面证明中可以得到的映射的指数扩张性,这一过程将在有限步
  后停止(也参见{[}TTY{]}).在根.与昭之间的所有非本质自由返
  回我们用伝*}:=表示.
\item
  若§•尹。,令 f 表示3中映到\emph{1\textsubscript{7}}上的子区间\emph{,7}
  6 5,-
  那么R\textbar{}\textgreater{}«.此时,F"*+''(O,s)未必可以严格地表为区间\&的并.
  令
\end{enumerate}

K = F性+'(0,s)\textbackslash{}(( U 功 U (― c°+砰{]},d+``{]})).

K可能由0,1或2个区间构成.

\begin{enumerate}
\def\labelenumi{(\arabic{enumi})}
\item
  如果对某些由于每个刀毗邻于一个已选定
  的4,此时我们只要稍稍扩展相应的参数区间纳■,使得\^{}*\textsuperscript{+,}'(0,
  \emph{吋)}= ZrU K.这样的构造意味着对恰当的了和符号士有
\end{enumerate}

GUF"*(O,M,7)U0U\&±1・ (3.3)

由此,我们可以定义第\^{} + 1次自由返回.

\begin{enumerate}
\def\labelenumi{(\arabic{enumi})}
\setcounter{enumi}{1}
\item
  如果K中含有区间J,满足七,UJ,M\textbar{} =a,记F\^{}+,(0,
  ")=U糾并考虑它的进一步迭代直至与厂相交.在此基础
  上重复上面的构造.用此方法我们得到由U A*的一不分类已(%
  啲)=(«»:■,\textgreater{}\}■记线=U a(i,y 并定义为+1 为
\end{enumerate}

血+1 = U 练+1O).

a»W勺

附注3.3在构造分类R的过程中,每个参数的小区间

R,都与!'中的一个区间关联.这样的区间分成两类,一类区间是
丄,而另一类区间除包含形如七的区间外,它的端点落入丄毗邻
的区间内(见公式(3. 3)).为了避免引入复杂的记号,在下面的讨
论中,我们只就第一类区间展开讨论,而不考虑第二类区间存在的
可能性.读者可以仿照证明思想,对第二类区间得到同样的结论.

B我们用归纳方法证明,对第\&次自由返回和所有aU瓦,下 面的事实成立.

\begin{enumerate}
\def\labelenumi{(\roman{enumi})}
\item
  如果㈣=吗念)是第\emph{k}次返回的指标成\textgreater{} 1,那么
\end{enumerate}

\textbar{}3\textsubscript{I}F\textsuperscript{n,}4-\textsuperscript{I}(l,a)\textbar{}
2exp(2miP);

\begin{enumerate}
\def\labelenumi{(\roman{enumi})}
\setcounter{enumi}{1}
\item
  如果1 MiM皿(a),那么
\end{enumerate}

5(1,a) I 2expi"3;

(hi) \textgreater{}exp(--- \emph{•/\textasciitilde{}j} ), 1\textless{} j
C m*.

我们注意到对\& = 1,由引理2. 3和叫的定义,显然上面的
⑴,(ii)和(iii)都成立.

下面我们对应+ 1证明(i), (ii)和顷)成立.令3是瓦中一个
小参数区间,并令力=\emph{P3} 网和``如A中所定义.我们也假定旬
中的小区间``映到区间孔,以\textbar{} V产时,归纳步骤成立.下面我们首
先估计P的上界.由中值公式

\begin{quote}
弓(``)-''(\%\textless{}!)--- FPT(l,q) - "一1(1 -
\emph{\textsubscript{a}f,d)} =4矿\&砂一1(1一对2,\textless{}1), (3.4)
\end{quote}

其中 0
\textless{}7\textsuperscript{,}\textless{}7\textless{}c\textsubscript{p}\_\textsubscript{1},由(2.
3)和(3.1)对 我

们有

{庖糜1 一时,a)\textbar{}} 行 {成(1 一妙,4)\textbar{}}

\textbar{}a『(l,a)\textbar{} \textasciitilde{}丛
\textbar{}H(l,a)\textbar{} VW2,(3.5)

其中c = exp(\^{}£日.同理可证

{IM(1 ---硏。)丨}卜丄\\
l"(S `` "2,

容易看到,(3. 5)和(3. 6)对0〈矿〈〃也成立.因此,由(3. 1),(3.
4)和(3.6)我们得到

M亲旧怎)一州(孔©\textbar{} £ 謗V \emph{玄}烏exp(2 E W exp(2 g 只要j
---iWm''那么由(ii),exp(2 /?)Nexp(严)•但是由引理 2.
3和不等式(3.1),当S充分小和向充分靠近2时,我们有

' W 2\#(a + j»+V2¥(ai + 房)

W 157(靛1 + 犊Q W 15-牝 \textless{}". (3.7)

此式也说明,当j增加时不等式\emph{j} M 2故十在\emph{j =
m\textsubscript{k}}之前就破坏掉 了 .因此

PW2M*. (3.8)

令a = \emph{(a,bf.}我们讨论\emph{n =} 的长度的下界.

邛可以表为

\begin{quote}
I糾=\textbar{}E»+P+l(0,4)--- I .

\emph{={[}FX1(釦 s),a)---砂+1 苦% ⑶,占汁 (3.9)}
\end{quote}

因为

I\&*Q) 一扌叫0)\textbar{} =
\textbar{}。一石\textbar{}\textbar{}\%广*(0,廿)丨\\
N g ---们•佥expm¥\$, \emph{a\textless{}a' \textless{}t\textgreater{},\\
}并注意到M对纟和a的导数是同阶的,而从上式可见\emph{\textbackslash{}a-b\textbackslash{}}与\\
I\&Q) --- \&冲)\textbar{}比较是微不足道的(也可参考{[}TTY{]}).因此

\textbar{}糾濠■1\^{}031 \textbar{}\&0) - 轧3)\textbar{} \textbar{}为''0
\textbar{}

貝扣, l/\&F",a)\textbar{}, (3.10)

其中1 一播

由(3. 2)和(3.4),存在 7,r,0\textless{}r
\textless{}7\textless{}\^{}-1\textgreater{} 使得

"\textbar{}小%1 - 寸@) \textbar{} \textgreater{} 备讐暖, (3.11)
利用一致估计(3. 5)和(3. 6)我们有

\begin{quote}
\textbar{}西"(夕@)\textbar{} \textgreater{}-\^{}19\^{}(1
\end{quote}

将(3.11)和上式代入(3.10)

ici \textgreater{} 1 \emph{\textsuperscript{Cfi} \textbackslash{}1
\textbackslash{}} {丨\&+i(Q)丨} g ''、

冋法説右仔』3+安 ③⑵

我们要进一步证明\textbar{}fi\textbar{}\textgreater{}exp(-2z(\textsuperscript{s/8}).事实上,由假设(iii),
底+13)1 2 exP(--- \emph{-/pTi).}利用(3. 8)并注意到 *Na =
log\textsuperscript{J}{[}\textbar{}{]},那么当\$充分小,

\&\textbar{} =exp(---\emph{』p}。)一 exp(--- /V)--- ,

I钉+l(a)\textbar{} \textgreater{}exp{[}-( {]} + 2抒)可,\\
\sout{d =竺p} -厂\^{}\^{}),.

\textsc{\textsuperscript{c}a---}1 exp ' 2 \emph{J卩 \_ {]})}

\begin{quote}
\emph{\textbackslash{}0\textbackslash{} \textgreater{}---} Q +
2+】)・2expG(QN 厂弟,(3.13) 60 V\textgreater{}
\end{quote}

下面我们对\& + I证明⑴ 成立,首先证明对V a E血, la\^{}i+\^{}d.a)!
Nexp(2S* + Q* + 会\#*) • (3.14)

由微分链式法则

=否"厂1(1,侄)(一 \emph{2a(\textsubscript{m} (a)) •}

i

为FP黄叫+1@),a) \textbar{}2a氧(e)为网(紐+心),a) \textbar{}

\begin{quote}
\textgreater{} 2aexp(--- 序)• \&xp(2 \emph{JL} 1) • 彳备沖(-少+ 2
QT)•帰帯.
\end{quote}

因\emph{为''》為埒、}并且\textbar{}弓+1愆)\textbar{}
\textgreater{}exp(--- 上式可表为

\textbar{}球《5(京+心)優)丨鴻温可•

exp(2 --- 1 --- //r --- \emph{Jp} + 1)》exp(+/\textgreater{}*).

(3.15)

由(3. 7),砰》15出从而对所有a £ 3,\\
庖 E+P(l/)\textbar{} \textgreater{} exp{[} \emph{2mf + 十声*)\\
}濠 exp(2O*+ 见专 +

这是因为(戒? +会A (wi* + ",

由分类定义,设对f = h.\^{}+\^{}CO.a)与/•相交.那么

\textbar{}招冲+3+:L(l,a) \textbar{} \textgreater{}exp(2(m* \emph{+
p +} h)b.\\
事实上,由于 \textless{}\#,并且 \textbar{}F《%,a)\textbar{} \emph{R,j
=}

一 2,其中 % = F\^{}+\%1,@),所以由引理 2.1\\
\textbar{}粕-1(從)\textbar{} 2(L9)L.

于是

\begin{quote}
\textbar{}为砂,+Z\textgreater{}K\_l(l,a)\textbar{} \textgreater{}
(1.9/Texp(2S* + *)* + 备疗) N exp(2(m* + 力 + 标)4).
\end{quote}

如果是自由返回,那么记m*+i = m\textsubscript{t} + /\textgreater{} + ;i,
并且对\emph{k} + 1,我们证明了(i),如果是非本质自由返回,由(3.13),
时,+\#+七(0仲)与九,\textbar{}叫〈产相交.重复前面的讨论,并在下一次
返回时,我们有

\begin{quote}
'IV"L(l,a)\textbar{} 2exp(2''£).
\end{quote}

如此继续下去直至到吗时,返回是自由的时侯为止.自由返回在
有限次非本质自由返回后必会出现,这是因为s的象的长度,如同
我们在A中看到的一样,在返回时是非常快地增长的.我们完成了 «)'的证明.

当* + 1时关于W)的证明,我们只对给出证明. 而对时间段\emph{叫〈V 吼}
的证明是类似的.

\begin{quote}
J j+l
\end{quote}

如果i 一皿V内V皿,其中\emph{p = pk}如(3.1)和(3. 2)定义, 那么我们有

=区评-f (最+l(O,a) ,

》exp(2m;〃)\textsc{・2kP(-} 77)1
勿戸T-叫(氧+i(a),a)\textbar{}.

由一致估计(3. 5)和(3.6),归纳假设(ii)和m* \textgreater{}产,

Nexp(2m? --- ■/\}? + (1 --- 1 --- m*)\#) 2 exp\#.

如果 \emph{i\textgreater{}p\textsubscript{k} +
m\textsubscript{k},}由(3.14)

\textbar{}杏尸」1(1")\textbar{}

=\textbar{}\%E`+A(l,a) •布尸-\%---》厂1(8叫+a+{[}(a),a))\textbar{}

2 exp(2g + 力)* + 佥讨)l/FF-l(机+A+1MB 卜. 由引理23,我们得到

低不一叫"L(氧+p,+i(e),a)丨 \textgreater{} 5exp(r(t 一唯一勿一{]})).
注意到

\begin{quote}
2 12 1 \emph{2}
\end{quote}

\emph{2(m\textsubscript{t} + fit)\textsuperscript{3} +
---/\textgreater{}\textsuperscript{3} + 7(I --- 1 --- p\textsubscript{k}
--- mQ} --- log y \textgreater{} 技. 从而(ii)得显.

(iii)的证明.要证当 \emph{m\textsubscript{t} \textless{} j K}
m\textsubscript{i+1} 时\textgreater{} exp(--- /7).由引理2.
3,只要a。充分靠近2,总可适当选取 使得阳\textgreater{}
\emph{2d}注意到\emph{屿}1,我们有

况* 2 \emph{\textasciitilde{}2\^{}\textsuperscript{m}t +} 2 \& + 1 + a.

因此,当时

\textbar{}烏 \textbar{} 2 exp(--- Ja + 点 + 1) \textgreater{} exp (---
\emph{Vm\textsubscript{k})} \textgreater{} exp(--- \emph{R).}
C最后我们来估计,在构造血+1时,从旬中排除的参数集的
测度.更确切地讲,我们要证明

\textbar{}g\textbar{} 2 \textbar{}血丨(1 一9, (3.16)

\begin{quote}
8
\end{quote}

其中0Way 1且U(l-4)\textgreater{}0.如果(3.16)得以证明,那么由

\begin{quote}
*=1
\end{quote}

归纳过程我们有

\begin{quote}
n---1
\end{quote}

14.1 \textgreater{} \textbar{}1 一 J I U I》\ldots{}\textgreater{} U(1
--- G 心 \textbar{}.

注意到4+1 U\&,我们有

Al = I n4t\textbar{}\textgreater{}0.

*=1

并完成了定理3. 2的证明.在证明(3.16)之前,我们暂时假设对所 有%,個G
3UJ、存在与互无关的常数c,使得

\begin{quote}
V {\textbar{}*中(0,代)\textbar{}} 〜
\end{quote}

c ; \textbar{}"E(O,s)\textbar{} f \textsuperscript{C}'

令\textsc{q = u} F"妇(0,3)可以表示为

\textbar{}P«E(O*)\textbar{} = \textbar{}砂5(0\^{}) ---
砂+(0』)\textbar{}

=叫FT(氧+p+i(a)Q -

与推导(3.10)相似,我们得到

\textbar{}尸%1 (0仲)\textbar{}

\textgreater{}号旧『中-叫一1(") \textbar{}
\textbar{}命+小8)-上++(方)I , 其中H在%+a+i(a)和氧+*10)之间.因此得

(1) \textbar{}F'(8,\textless{}z) \textbar{} N 凯丿=0,1,2,•••,/»*+!
\emph{- m\textsubscript{k}- p- 2t}

\emph{(2) C3.}

利用引理2、1

\textbar{}砂中(0,也)丨

\textgreater{} y(l. 9)FF-"\%叫+*(a)-氧+宀0) h

9 j.

=号(1. 9)叫+厂叫一》一1\textbar{}糾\textgreater{}exp(- 2舟).
我们也有

\textbar{}E+i(0,a)\textbar{} = \textbar{}\%FF(0,次)1 I叫, 其中W
£a,于是,我们得到了 a长度的一个下界

\begin{quote}
即\emph{参}.{燮- 2満)}

\textsuperscript{1} ' - .
\end{quote}

另一方面,令刃表示a中可以包含于A+i的小区间全体.根 据四+1定义,记鼠+1 =
w\textbackslash{}a/,则

l\^{}™\textsuperscript{i+1}(O,wt\textsubscript{+1})J \textless{}exp(---
/a + A +1).

而

\textbar{}E+i(O,祐i)\textbar{} = I弟+11,

其中术£添+1,因此

{exp(--- /a + H)\\
}一\textbar{} 讣f(O,a〃)\textbar{} '

利用(3.17),我们可以估计从\&中排除的参数集的测度

\begin{longtable}[]{@{}ll@{}}
\toprule
\endhead
\begin{minipage}[t]{0.47\columnwidth}\raggedright
{丨如一\textbar{}«/ \textbar{}}

1刎 '

因此,有\strut
\end{minipage} & \begin{minipage}[t]{0.47\columnwidth}\raggedright
\strut
\end{minipage}\tabularnewline
\bottomrule
\end{longtable}

I血I - I4t+il ' ...\_exp(--- -/a+T+T)

nr{]} W const r- = «i

\begin{quote}
\textsuperscript{141} exp(-2``E)
\end{quote}

和

\begin{quote}
\textbar{}四+11 (1 --- a*) \textbar{}\textless{}i*
\end{quote}

因为I闵\textless{}a +如所以四£戲,其中

\emph{队}---const exp(2(a + 幻M --- ■/« -pA + 1).

8 8

容易验证IT (I f \&) \textgreater{} 0,因此U (1 - M \textgreater{} 0.

XI \emph{k --- \textbackslash{}}

附注3.4读者可以看到,在证明(3.16)时,我们仅考虑了从
,次自由返回到\emph{k +} 1次自由返回的过程中,没有非本质自由返
回.更明确地说,我们仅■考虑了"=吨+1情况,对于一般情况,读
者可仿照上述过程证明间样的结果.

下面证明(3、17)成立.我们将给出一个形式上更一般的结

论.

.引3 3.5设砂,(0,3)那么有常数q和\&,使

得对任意\emph{a,b£3}和丿 \textless{} 所》+1,下列不等式成立

(3.19)

证明 显然,我们只需证明(3.18)成立.(3,19)式的结果可 从(3.18)和引理2.
4推得.(3.18)式等价于

崗0)\textbar{}

\begin{quote}
l\^{})T \^{}\textsuperscript{C,}
\end{quote}

由归纳的结论⑴,得

22 \textbar{}ET(/)\textbar{} =

因此,利用引理2. 4,得

\textbar{}如 Wdexp(--- 2邳).

因为m》l,

囹'w (1 + 备卩 M (1 + "exp(\_ »d))y 2,\\
于是(W)'在对参数的一致性估计中是不重要的因素.为了证明\\
(3.18),我们只要估计]1 譲勢*因为

吝 \textbar{}£0)\textbar{} ' 丄修川)一\&sin

口吋1 =咬 g\textsuperscript{ln}\textbar{}\textsuperscript{1+}
\textbar{}\^{})\textbar{} ))

b 踞旧。)一\&(4)\textbar{}

W exp応一偽(a) I .

所以我们只要估计

„ I演)一")丨

3 -

令佑}蔼为自由返回或非本质自由返回的指标,并且专V知
如果f\textless{}j,那么S可分段表示为

\begin{quote}
簽芙``W)一膈)1簽V
\end{quote}

\textsc{\textsuperscript{5} = 2j 2j} ---旧商 =匕为.

\begin{quote}
\textgreater{}-1 v=(.
\textsuperscript{\textbar{}C}-\textsuperscript{WI} 斤1
\end{quote}

由有的定义,q e加,并且书=(句伝),句8))u\%.我们将,分 成两部分 ' '

„ I\&3) ---句。)\textbar{} 弓*` \textbar{}5怎)一\&。)\textbar{}

月=* ")1 +『謂 険)1 '

首先估计第一个和式,当卩= \textless{}\textgreater{}
时,\sout{吃:二;件\textsuperscript{1}} M 导;当勺

\emph{+ pj} 时,记 \emph{S = V ---}如显然/ = 1,2,
---,/\textgreater{}\textgreater{},此时

厲(仅)一 "3)丨=\textbar{}玲(匕 S)M)--- F;(\& 9),为)\textbar{}

\begin{quote}
\emph{i I}
\end{quote}

勺 \textbar{}囱\textbar{}.

同时

.\textbar{}\&(a)\textbar{}=\textbar{}F,怎怎)g)\textbar{}

\begin{quote}
\emph{i}
\end{quote}

\textsc{\textgreater{} \textbar{}F,(O,q)\textbar{}} ---
\textbar{}F,(q(Q),a) - FM0,4)L 记\&*)= 〃,则 .因此由(3.1)式,

0a) I \textgreater{} \textbar{}?,(«) I ---法I \textbar{}.

我们得到

成 {0)1}

\emph{q} I 蜘)1

(3. 20) 我们把不等式右边的和式分成两部分

\emph{\% f; f,}

月=力十\emph{i -}

\begin{quote}
5=1 1=1 \$---时+1
\end{quote}

其中\emph{pi} = I Vw.第一部分我们用基本估计

IMI",

\begin{quote}
\emph{.\textbackslash{}W \textbackslash{}} \textgreater{}e-
\end{quote}

第二部分我们利用(3.1),(3. 4)和(3. 5)推导出

\textbar{}割1(1," £ 詞\&(a) ---F,(")\textbar{} M 亦志 lg)\textbar{},
其中咋 g,糾\_\textbar{}).于是

\%\textbar{}丄」旧此一1(1,0)\textbar{} \textbar{}加丨

V戸.

因此

2 =象+套日4也\textsc{mi}吳"十华!{]}

\$=1 sT g + 1 I s=l \emph{气} 〃 + 1 ''

注意到\emph{Pi'} = j Vw和土丨\emph{Ui W*} 我们有 安'{l\%(a) ---4(6)1} V
``顽 {Wl}

因为当勺+ \emph{Pi \textless{}v\textless{}
t\textsubscript{j+1}}时,句(a) n /■ = 0以及

庵詩)一 \emph{\&揷)}I 2号杞此f-"`a) I I物)一晶(6) I,
其中0在5(4)与之间,我们有

\begin{quote}
\textbar{}\&(龄-£0)\textbar{} / 3 \emph{t}
lOl\^{}\textsuperscript{-1}* 驼+«)- \emph{\&揷)}I
一而可一0万\textbar{}的J 1句+,怎)1
\end{quote}

事实上,旧怎)\textbar{} \textgreater{}
\textbar{}弓+[伝)\textbar{},并由引理2.1

\textbar{}古尸;+厂"Sa)\textbar{} \textgreater{} (1.9)',+「",

于是

\textbar{}腿)一伊)I W专鼎``定詩)一鈴件)I.
因此,第二部分可以估计如下:

W {13) --- 60)1}

\emph{\textsubscript{\textless{}}1} 卩O\^{}+L {I句+\&)\_ 句+Q)I}

£如思+J司 呢孩)1

\textsc{\textless{}Ay(io1\textsuperscript{j}} {庇+S ---与璀I}

毛万4【1时 呢,(心

{竄3-队仞1\\
}I氣s -

\begin{quote}
'/+L
\end{quote}

\emph{s} \textless{} const 2 -7=学V

i V\^{}\textgreater{} \textsuperscript{p}'

为完成(3.18)的估计,我们只要估计上面不等式右边和式的
界.完全类似于(3.13)的估计,我们有

\begin{quote}
\textbar{}川+1(\%,次)\textbar{}》exp(\_ 2游), (3.21)
\end{quote}

其中囱£農,切,% =下面我们要证明陶+1! \textgreater{}
5囱I.为此,先证明在有限的时间段 \emph{KKP,}中,\sout{鴛:⑶糾} \emph{la/s
血)[\\
}的一致有界性,其中气,工合£七"3如6欢

\begin{quote}
1---1
\end{quote}

{\textbar{}\%刑(工1,绮)\textbar{}} =(" {巨成 愆",}

\begin{quote}
\emph{V=O}

\emph{I \textsuperscript{a}i\textbackslash{}'}」頌
{\textbar{}Fu(Zi,ai)---狄(丑,企)1}1
\end{quote}

w M丿 \textsuperscript{exp}l \emph{:}---g\&i r

由(3.1),当vW物时

\textbar{}F*"(H,a)---知(\textless{}z) I M \sout{":V 丨},z £ 弓,a 任
\emph{a\textgreater{},}

我们得到

I硏(跖,角)一卧(务血)\textbar{}

M IA(ai)I + \textbar{}\&伝2)+ \textbar{}\^{}(«1) --- \^{}(此)I,

和

\textbar{}殆(处,%)1 \textgreater{}
y\textbar{}\^{}(0,a\textsubscript{2})\textbar{}.

于是

{\textbar{}F"(;C1,勿)一户"(以,\%)丨}

\textbar{}尸6血)[\textsuperscript{\_}

由归纳的结论(ii),引理2. 4,(3. 5)和(3. 6)

\textbar{}"-1(瑚)\textbar{} Vxp(2 4),

IM广(l,a)l N«tP(2罗).

因此

\textbar{}\&(角)---f\textsubscript{M}(a\textsubscript{2}) \textbar{}
--- const\textbar{}\&F*-l(lM) \textbar{} \textbar{}向一此\textbar{}

\begin{quote}
W const \textbar{}9\textsubscript{X}F*-\textsuperscript{1}(1 \textbar{}
\textbar{}角---\emph{a\textsubscript{t} \textbar{}}

C const exp(2 \textsc{a/\^{})} \textbar{}鬲 一% \textbar{},
\end{quote}

以及

\begin{quote}
\textbar{}勺愆)一亀(6)丨
\end{quote}

论 一 K \sout{15-1(丄)\textbar{} -}淄眦 exp(\_ 2痒.\\
注意到\textbar{}嬴饥)\textbar{}濠exp(--- O 和立得\\
頌 屮(时。1)---欢《气血)1 ,

\begin{quote}
\textsc{2j} 77J7 Ti W const,

£ \textbar{}殆(以,角)\textbar{}
\end{quote}

因为囹'是一个有界数,所以

广二 r V const \textsubscript{f}

\begin{quote}
\textbar{}\%尸(工2\& I
\end{quote}

其中册 e \emph{Ip.,} a* a\textgreater{}.

由引理2.1和上述估计,我们得到

\textbar{}bj+"= \textbar{}P\textgreater{}+i(0»a) ---
\textsc{Pj+i(0,Z*)1}

\begin{quote}
=I尸心M,+3+l)(q+3+i(a),a)- F«,zM,+D+I)(q+p,+10)0)\textbar{}

Nconst呢+3+i(a) - q+D+i(6) I r {I"叫叫},

2 const jjT 1。打

\textgreater{}const
0\textbar{}exp(\textbackslash{}/Z)exp(\textasciitilde{} 2才).
(3.22)
\end{quote}

因为\$很小,所以(3. 22)意味着\textbar{}b,+``N5]巩.最后,我们估计 和式

要注意,某些円可能同时位于某个区间L中.当此情况发生时,由 于血+1N
5囱\textbar{},我们有血I \textless{} I''⑴I, J(s)表示这些\emph{。代L}的

最大下标,由此有

\includegraphics[width=0.91319in,height=0.45972in]{media/image81.png}\includegraphics[width=0.65347in,height=0.41319in]{media/image82.png}于是

-s s

\begin{quote}
/ VI 1 \textbar{}bj(\$)丨
\end{quote}

不等式(3. 24)表明和式亩的\emph{杓}不妨是不同的•将和式

V' {1 丨勺丨\\
}叨

分成两部分,令儿是使得

方冊5

的指标集,而儿是剩余指标集.那么

. QO

S 、後 M const,

即上式左端是一致有界的.如果ye则

由(3,22),如果 \emph{k\textgreater{}j}

\textbar{}ct*\textbar{} \textgreater{} const
\textbar{}q\textbar{}exp(y\^{}J) exp( --- )

\begin{itemize}
\item
  \begin{quote}
  const \emph{插}加 \textbar{}exp(V\^{}i exp( 一 2,)
  \end{quote}
\item
  \begin{quote}
  exp( --- 3力).
  \end{quote}
\end{itemize}

但唳U/七由加的定义立得件C 9\#.此不等式说明{的}心,有

限.因此

2声件、声\\
\emph{jWJ*} V \emph{fij \%} 丿€七 V 巧

是一致有界的.于是,我们证明了引理3. 5,从而定理3.2证毕.\textbar{}