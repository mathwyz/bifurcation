\section{动力系统及其结构稳定性}\label{1.1}
动力系统的概念和理论是从人们对常微分方程的研究中产生和发展起来的,
而且对常微分方程的研究,
至今任是动力系统理论的重要组成部分.

考虑\(R^n\) 中的自治微分方程
\begin{equation}
    \label{eq:1.1.1}
  \dxdt=f(x)
\end{equation}
其中\(f:\RR^n \to \RR^n\) 是 \(C^r\) 向量场,
\(r \geq 1\).
由常微分方程中熟知的结果,
\(\forall x_0 \in \RR^n\),
方程~(\ref{eq:1})以\(x(0)=x_0\) 为初值的解\(\alpha(t,x_0)\) 在包含\(t=0\) 的区间上存在.
如果f(x)满足适当条件(或在某种等价意义下对f(x)进行改造,见
%\cite{zdhd}
则解\(\alpha(t,x_0)\) 可以对一切\(t\in \RR\) 存在,
并且\(\alpha(t,x)\)满足:

\begin{enumerate}
\item\label{item:1.1.1}
  \(\alpha(0,x)=x,\forall x \in \RR^n\);
\item\label{item:1.1.2}
  \(\alpha(s+t,x)=\alpha(s,\alpha(t,x)),\forall s,t \in R ,x \in R^n\);
\item\label{item:1.1.3}
  \(\alpha(t,x)\)对\(t,x\)连续.
\end{enumerate}
我们把满足上述条件
\ref{item:1.1.1}-~\ref{item:1.1.3}
的映射:
\(\alpha:\RR \times \RR^n \to \RR^n\)
称为\(\RR^n\) 中的\textbf(动力系统),
或者称为方程\ref{zz}的\textbf{流},
并把点集
\[O_\alpha(x)=\{\alpha(t,x)|t\in \RR\}\subset \RR^n\]
称为流\(\alpha\) 过x的\emph{轨道}.


不难证明,对\(\forall x_1,x_2 \in \RR^n\),
\(O_\alpha(x_1)\) 和\(O_\alpha(x_2)\) 或者重合,
或者(对有限的时间t)不相交,
因此,\ref{zz}的轨道集合依f的不同而呈现不同的规律.


常微分方程定性理论(或者称为几何理论)的首要目标,
就是对于给定的\(f\in C^r(\RR^n,\RR^n),r \geq 1\),
研究方程\ref{zz}的轨道集合的结构
(轨道集合的拓扑结构图称为\textbf{相图},
通常在相图上用箭头标明对应时间t增大的轨道方向).
一般而言,
方程\ref{zz}不可能用初等函数的有限形式求解, 
因此研究它的相图是一个困难的任务.
例如,即使当\(n=2\)且f(x)为某些二次多项式这种最简单的非线性情形,
人们至今尚不清楚相图的确切结构.
参见
\cite{y12}
\cite{q}


现在把动力系统的概念加以推广.
设M是紧致的\(C^\infty\)\textbf{微分流形},
记\(Diff^r(M)\)为M上所有\textbf{\(C^r\)微分同胚}的集合(见附录A中定义A.8),
\(\mathcal{X}\)为M上所有\(C^r\)\textbf{向量场}的集合,
\(r \geq 1\).
在\(C^r\)拓扑下,
\(Diff^r(M)\)和\(\mathcal{X}\)均为完备的度量空间(见附录C中的定义C.1,定义C.2,附注C.4)
\(\forall X \in \mathcal{X}^r(M)\),
存在X中过\(p\in M\)的\textbf{极大流}\(alpha_X\)(见附录B中定义B.13,定义B.15,和定理B.16).
为了讨论方便,假设M是无边流形,从而使
\(\phi \in Diff^r(M)\)可以往正、负向无限延伸;
使\(X \in \mathcal{X}^r(M)\)的极大流在
\(-\infty,+\infty\)上存在
(否则要对微分同胚或向量场做适当处理)。


\(\forall \phi \in Diff^r(M)\),
考虑从整数集\(Z\)到\(Diff^r(M)\)的映射,
\begin{equation}
  Phi:Z \to Diff^r(M),\Phi(n)=\phi^n.
\end{equation}
对固定的\(\phi\),
\(\phi^n\)是M上的一个单参数变换群;
对固定的\(n\in Z\),
\(\phi^n\)给出了\(M \to M\)的微分同胚.


\(\forall \mathcal X\in \mathcal{X}^r(M)\),
考虑从实数集R到\(Diff^r(M)\)的映射
\begin{equation}
\alpha_{\mathcal{X}}:R\to Diff^r(M),t|\to \alpha_{\mathcal{X}}(t,p)
\end{equation}
其中\(\alpha_{\mathcal{X}}\)是相应于X的流。
对固定的X,\(\alpha_x(t,p)\)是M上的一个单参数变换群;
对固定的\(t\in R\)和所有的
\(p \in M\),
\(\alpha_x(t,p)\)是\(M \to M\)的一个微分同胚。


因此,我们可以把上面两种映射统一写成
\[\phi_t:M\to M\],
当\(t \in R\)时,
它是由\(\alpha_x\)导出的连续流,
当\(t \in Z\)时,
称它为连续流。


\(\phi_t\)满足:
\begin{enumerate}
\item\label{item:1.1.1}
  \(\phi_0=id_M\);
\item\label{item:1.1.2}
  \(\phi_t \circ \phi_s = \phi_{t+s}\);
\item\label{item:1.1.3}
  \(\phi_t(x)\)对一切t,x一并连续;
\end{enumerate}
有时也把\(\phi_t\)称为M上的\textbf{\(C^r\)微分动力系统},
特别,当\(t \in Z\),称为\textbf{离散动力系统}。
本书主要讨论连续流,
由于它与离散流有密切的关系(参见下面的定义1.6),
必要时也讨论离散流。

\begin{defination}
  设\(\phi_t:M\to M\)如上,
  M中的集合
  \[O_\phi(x)=\{\phi_t(t)|t \in R (或t \in Z)\}\subset M\]
  称为连续流(或离散流)\(\phi_t\)过x的\textbf{轨道}。
  如果把上式中的R(或Z)改为\(R_+\)(或\(Z_+\))或者\(R_-\)(或\(Z_-\)),
  则相应地得到过x的\textbf{正半轨道}或者\textbf{负半轨道},
  并分别记为\(O_\phi^+\)或者\(O_\phi^-(x)\).
\end{defination}

\begin{defination}
  过x的的正(或负)半轨道的极限点称为x的
  \textbf{\(\omega  \)或\(\alpha\)极限点}。
  x的全体\(\omega\)(或\(\alpha\))极限点组成的集合称为x的\textbf{\(\omega\)(或\(\alpha\))极限集},记为\(\omega(x)\)(或\(\alpha(x)\))。
  既
  ??????????????????????????????
  % \[\omega(x)=\{y \in M| \exists  t_i \to +\infty , \text{使} \phi_t_i\to y\}\]
  % \[\omega(x)=\{y \in M| \exists t_i \to -\infty , \text{使} \phi_t_i\to y\}\]
  显然,当M紧致时\(\omega(x)\neq \varnothing,\alpha(x)\neq \varnothing\).
  % 空集 \emptyset
 % \documentclass{article}
%\usepackage{amsmath,amssymb}
%\begin {document}

%which is kongji ?

%\o, \O, $\emptyset$, $\varnothing$

%\end {document}
\end{defination}

\begin{defination}
  设\(\phi_t:M\to M\)如上,\(p \in M\)称为\textbf{游荡点},如果存在p的领域\(U\subset M\),和某个正整数N,使\(\forall |t|>N\),有\(\phi_t(U)\cap U=\varnothing\);
  不是游荡点的点称为\textbf{非游荡点},\(\phi_t\)的所有非游荡点的集合称为\textbf{非游荡集},记为\(\omega(\phi)\).既
  \[
    \omega(\phi)=\{
    p \in M| \text{对p的任意领域} U,\forall t ,|t|>1,\text{使} \phi_t(U)
    \cap U =\emptyset
    \}.
  \]
显然,\(\omega(x)\subset \omega (\phi),\alpha(x)\subset\omega (\phi)\).
因此,当M紧致时,\(Omiga(\phi)\neq \emptyset\).
下面定义的临界元是\(\omega(\phi)\)的重要组成部分。
\end{defination}

\begin{defination}
  设\(\phi_t:M\to M\)如上,M的连通子集
  \[L=\{p|\phi_t(p)=p对某一t\neq 0成立\}\]
  称为一个\textbf{临界元}.
\end{defination}
给出临界元的定义是为了陈述上简洁。
有时要对临界元进行如下细致的区分。


在微分同胚的情形,如果存在\(k \in Z^+\),使得\(\phi^t(p)=p\),则p为临界元。
满足这个条件的最小数k称为p的\textbf{周期},并称p为k\textbf{周期点}。
特别地,若周期为1,则称p为\textbf{不动点}。
对于k周期点p,如果\(D\phi^t(p)\)的所有特征值的模均不为1,则称p是一\textbf{双曲不动点}(k=1)或(\textbf{双曲周期点})\((k>1)\).


在向量场的情形,临界元L有两种类型。
一种类型是,L由一个点p组成。
在这种情况下,\(\forall t,\phi_t(p)=p\)(或等价地\(X(p)=0\)),此时称p为想两场的一个\textbf{奇点}。
称p为一个\textbf{双曲齐点},如果\(\forall t \neq 0\),p是 \(\phi_t\)的一个双曲不动点(或等价地,\(DX(p)\)的所有特征根都有非零的实部)。
另一种类型是,L由向量场的闭轨\(\gamma\)组成。
此时\(\forall p \in \gamma , \exists t \neq 0\),使\(\phi_t ( p ) = p\).
这样最小的正数\(t=T\)称为闭轨\(\gamma\)的\textbf{周期}。
\(\gamma\)称为\textbf{双曲闭轨},如果对某个\(p \in \gamma \)(从而\( \forall p \in \gamma\)),\(D \varphi _ { T } ( p )\)的所有特征值的模除了一个以外都不等于1.
注意,此时\(\gamma\)上每一点都不是向量场的奇点,\(\forall p \in \gamma , X ( p ) \) 是\( D\varphi _ { r } ( p )\) 的以1为特征值的特征向量。


在研究向量场的轨道结构时,局部困难在奇点附近(就奇点本身而言,它无非是动力系统的一个\textbf{平衡点},但在它附近的轨道结构却可能千变万化);
而整体的困难在于非游荡点集的结构,它反映出动力系统的本质特征。

\begin{defination}
  M中集合\(Lambda\) 称为\(\varphi\) 的\textbf{不变集},如果\(\forall x \in \Lambda\),有$O _ { \varphi } ( x ) \subset \Lambda$。
\end{defination}

不难证明,$O _ { \varphi } ( x ) , \omega ( x ) , a ( x ) , \Omega ( \varphi )$都是\(\varphi\) 的不变集。


在研究闭轨的分岔时,建议如下定义的$Poincar\acute{e}$映射是一个很重要的手段。

\begin{defination}
  设\(\gamma\) 是\(C^\infty\) 流$\varphi _t:M \rightarrow M$的一条闭轨,\(p \in \gamma\).
  取M内包含p点的一个如此"小"的光滑余维1子流形U,使得流\(\varphi_t\)相应的向量场在U上每一点与U都是无切的。从而$\exists U ^ { \prime } , p \in U ^ { \prime } \subset U$,和在\(U^{\prime}\)上定义的\(C^r\)函数T,使得\(T(p)\)等于\(\gamma\)的周期,且$\forall p ^ { \prime } \in U ^ { \prime },\varphi _ { T \left( p ^ { \prime } \right) } \left( p ^ { \prime } \right) \in U.$由此可定义\textbf{$Poicare\acute{e}$映射}
  \[P : \quad U^{\prime} \rightarrow U,P \left( p ^ { \prime } \right) = \varphi_{T(p^\prime)}(p^\prime),\]
见图\ref{tu1-1}.
\end{defination}
?????????????????????????????????图片
% \begin{picture}
%   \includegraphics[]{picture.png}
%   \lable{tu1-1}
% \end{picture}

显然,$P ( p ) = p$,既p是映射P的不动点。
利用Poincare映射,可以把对\(C^r\) 向量场在闭轨\(\gamma\) 附近轨道结构的研究,转化为对\(C^r\) 微分同胚P在不动点p附近轨道结构的研究。


我们现在转向结构稳定性问题。
简言之,在"小扰动"下不改变其轨道结构的动力系统是结构稳定的。


\begin{defination}
  称两个向量场\(X_1\)与\(X_2\)\textbf{拓扑轨道等价},如果存在同胚$h: M \rightarrow M$,它把\(X_1\)的每条轨道保向地映到\(X_2\)的相应轨道。
  称两个微分同胚$ \varphi _ { 1 } , \varphi _ { 2} M \rightarrow M$\textbf{拓扑共轭},如果存在同胚$\boldsymbol { h } : M \rightarrow M$,使
  $\varphi _ { 2 } = h ^ { - 1 } \cdot \varphi_ { 1 } \cdot h _ { n }$。
  称$X \in \mathscr { X } ^ { r } ( M )$(或$\varphi \in \operatorname { Diff } ^ { r } ( M ) $)\textbf{\(C^k\)结构稳定}($k \leqslant r$),如果存在\(C^k\)拓扑中的领域$ U ,X \in U \subset \mathscr { X } ^ { r } ( M )$(或$\varphi \in U \subset \operatorname { Diff } ^ { r } ( M )$),使$\forall X ^ { \prime } \in U$拓扑轨道等价于X(或\(\forall \varphi^{\prime} \in U \subset D i f f ^ { r } ( M )\)拓扑共轭与\(\varphi\)).
  \label{gui}
\end{defination}



当同胚h还保持\(X_1\)与\(X_2\)相应轨道的时间对应时,称为\textbf{拓扑等价}。
关于\(C^r\)拓扑的定义,见附录C。通常考虑\(C^1\)结构稳定性。
此时常省略\("C^1"\),简称为结构稳定性。


在向量场的双曲奇点(或微分同胚的双曲不动点)附近,有下面的局部结果。
\begin{theorem}[Hartman-Grobman定理]
  设$U \subset \mathbf { R } ^ { n }$是包含O点的开集;
  向量场X以O为双曲奇点
  (或$\varphi : \quad U \rightarrow \mathbf { R } ^ { n }$以O为双曲`不动点),
  则存在O的开领域\(V\subset U\),使X与其他相应的线性场\(DX(O)\)在V上拓扑轨道等价(或\(\varphi\)与其相应的线性映射\(D\varphi(O)\)在V上拓扑共轭)。
  \end{theorem}

  \begin{theorem}[局部结构稳定性定理]
    设$X \in \mathscr { X } ^ { r } \left( \mathbf { R } ^ { n } \right)$,X以O为双曲奇点,则X在O点附近局部结构稳定,既存在X在$\mathscr { X } ^ { r } \left( \mathbf { R } ^ { \boldsymbol { n } } \right)$中的一个\(C^1\)领域U,$\forall Y \in U , Y$在O附近有唯一的双曲奇点p,且Y在p附近的某领域与X在O的某领域内拓扑轨道等价。
 \end{theorem}


 (对于离散的情形,也有平行的结果。)
 
 \begin{corollary}

   如果把定义\ref{gui}中的轨道等价从\(C^0\)加强到$C ^ { k } , k \geqslant 1$,既要求h为\(C^k\)微分同胚,则\(X_1\)与\(X_2\)在相应奇点处的线性系统的特征根有相同的比值(见[GH,p42).
   这就把等价关于限制过严,使得两个轨道结构相同的向量场也未必等价。
   例如,由定理\ref{dingli}可知,二维系统
   \[\dot { \boldsymbol { x } } = \boldsymbol { x } , \quad \dot { \boldsymbol { y } } = \boldsymbol { y }\]
   是(局部)结构稳定的,它与系统
   \[\dot { x } = x , \quad \dot { y } = y + \mu y\]
   $( u \neq 0 )$是\(C^0\)等价的。
   但他们不是\(C^1\)等价的。
   另一方面,如果把定义\ref{gui}中扰动从$C ^ { r } ( r \geqslant 1 )$加强到\(C^0\),则总是可以扰动处不同的轨道结构,因而无结构i稳定性可言。
   例如,由定义\ref{dingli1.9}可知,一维系统\(\dot{x}=x\)是局部结构稳定的,但它的\(C^0\)扰动系统$\dot { x } = x + \mu \sqrt { | x | }$与原系统在原点的任意小领域内都有不同的轨道结构,只要\(0 <|\mu| \ll 1\).
   因此,如无特别声明,下文中的等价都指\(C^0\)等价,而扰动都是\(C^1\)扰动。
  \end{corollary}
  在给出进一步的结果之前,我们需要下面的定义。

  \begin{defination}
    设\(U,\varphi\)同上,称集合
    \[W _ { \varphi } ^ { s } ( O ) = \left\{ x \in U \left| \varphi ^ { k } ( x ) \rightarrow O\right.当 k \rightarrow +\infty \right\}\]
    和
    \[W _ { \psi } ^ { u } ( O ) = \left\{ x \in U \left| \varphi ^ { - k } ( x ) \rightarrow O\right. ,当k \rightarrow + \infty\right\}\]
分别称为X在\(\gamma\)的\textbf{稳定流形}与\textbf{不稳定流形}.
  \end{defination}

  对于向量场的情形,有类似的定义。

  \begin{defination}
    设\(\gamma\)是向量场X的双曲奇点或双曲闭轨,集合
    \[W _ { X} ^ { s } ( \gamma ) = \{ x \in M | a _ { X } ( t ) x \rightarrow \gamma , 当 t \rightarrow + \infty \}\]
    和
    \[W _ { x } ^ { u } ( \gamma ) = \left\{ x \in M \left| \alpha _ { x } ( t ) x \rightarrow \gamma\right.,当 t \rightarrow - \infty \right\}\]
    分别称为X在\(\gamma\)的\textbf{稳定流形}和\textbf{不稳定流形}.
  \end{defination}
  利用Poincare映射,可以把向量场在双曲闭轨附近的研究化为在不动点附近的研究。
  由Hartman-Grobman定理,$W _ { X } ^ { s} ( O ) , W _ { X } ^ { u} ( O )$(或$W _ { \varphi } ^ { s } ( O ) , W _ { \varphi } ^ { u } ( O )$)可在O的小领域内分别与线性映射$A = D X ( O )$(或$D \varphi ( O )$)相应的集合$E ^ { \mathrm { s } } , E ^ { \mathrm { u } }$(它们是\(\mathbf{R}^n\)的线性子空间)建立同胚。
  因此$W _ { X } ^ { s } ( O )$与$W _ { X } ^ { \mathrm { k } } ( O )$(或$W _ { \varphi } ^ { s } ( O )$与$\mathrm { W } _ { \varphi } ^ { u } ( \mathrm { O } )$)在O点附近是M的子流形。
  进而可以证明,它们在O点分别与\(\mathrm{E}^s\)和\(\mathrm{E}^u\)相切;
  从整体上看$W _ { X } ^ { s} ( O )$与$W _ { X } ^ { n} ( O )$(或$W _ { \varphi } ^ { s} ( O )$与$W _ { \varphi } ^ { u} ( O )$)是M的\(C^r\)\textbf{浸入子流形}(但未必是M的子流形,见附录B中附注B.24),证明可以参考[ZQ,定理4.9和4.10].

  现在我们可以陈述M.M.Peixito在1962年证明的一个有关全局结构稳定性的结果(参见[ZDHD]).

  \begin{theorem}
    设M是紧致的二维光滑流形$X \in \mathscr { X} ^ { r } ( M )$.则X是\(C^1\)结构稳定的,当且仅当


    (1)X的非游荡集仅由临界元组成;

    (2)X的临界元(包括奇点和闭轨)个数有限,并且他们都是双曲的;

    (3)任何双曲临界元的稳定流形和任何双曲临界元的不稳定流形横截相交。
  \end{theorem}
  这里横截相交的定义见附录C中定义C.10.注意,不相交也算作横截。
  另一个有关结构稳定集的重要结果,是下面的
  \begin{theorem}
二维可定向紧致流形M上\(C^r\)结构稳定向量场的集合在(对)中是一个开稠集。
  \end{theorem}

  \begin{corollary}
    peixoto等人的上述结果,从60年代开始吸引了在国外以Smale为代表、在国内以廖山涛为代表的一批数学家,在微分动力系统方面他们做了大量的工作(参见[L1],[Zzs]等).人们把满足定理1.13中三个条件的向量场称为\textbf{Morse_Smale向量场}.(类似可定义M-S微分同胚)。
    自然要问:定理1.13和定理1.14在高纬流形上是否任然成立?
    Smale在60年代构造的称为“马蹄”的著名例子(见第四章),以及Newhouse随后对“马蹄”的改造,说明上述问题的答案都是否定的,既在高纬流形上的结构稳定向量场(或微分同胚)未必是M-S的,而全体结构稳定的向量场(或微分同胚)的集合在\(\mathscr{X}^r(M)\)(或\(Diff^r(M)\))中不一定是稠集。
    进一步的问题是:当\(dimM>2\)时,在\(\mathscr(X)^r(M)\)中(或当\(dimM>1\)时,在\(Diff^r(M)\)中)结构稳定的充要条件是是什么?
    对此Smale提出了比M-S条件更广的公理A条件,并且给出了两个结构稳定性的猜测:
    结构稳定\(\Leftrightarrow\)公理A+强横截条件;
    \(\omega\)稳定性\(Leftrightarrow\)公理A+无环条件。
    这两个猜测的充分性部分已为Smale本人和其他人所证明;
    必要性部分则于1987年分别由Mane和Palis对微分同胚的情形给出了证明。
    对向量场情形的第一个猜测,直到最近才由廖山涛、胡森、和Hayashi、文兰分别对三维和一般情形给出证明。
  \end{corollary}
