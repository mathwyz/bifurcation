\chapter{双曲不动点及马蹄存在性定理}

本章中我们将给出关于$\RR^2$中映射的几个定理。
这些定理在第五章中是我们研究空间$\RR^3$中鞍点同宿分岔的基础,同时这些定理也有其自身的重要价值。
在\ref{sec:0401}中我们将证明一个双曲不动点定理。
在\ref{sec:0402}中引进符号动力学的基本概念。
在\ref{sec:0403}中给出马蹄存在定理。
在\ref{sec:0404}中给出一个关于两个线性映射的复合映射的双曲性的判断引理。
在\ref{sec:0405}中作为\ref{sec:0402}~\ref{sec:0404}中诸结果的一个应用,我们将给出
$\RR^4$中$Birhoff-Smale$定理的证明。
\par
本章和下一章的主要定理在$Sil'nikov'$的文章[Sil1-3]和$Wiggins$的书[Wi2]中都能找到,但此处所有定理的证明都是独立给出的。
我们力图把几何直观与数学的严密性统一起来,并给予读者一套易于掌握的方法,用以解决高维空间中其他类似的问题。
\section{双曲不动点定理}
\subsection{定理的陈述}
首先我们引进一些基本概念。令
$$
\RR^{1} \times \RR^{1}=\left\{(x, y) | x \in \RR^{1}, y \in \RR^{1}\right\}
$$
\begin{defination}
  设$C>0$是一个常数,
  平面$\RR^1 \times \RR^1$n上的一条$C$ \textbf{水平曲线}指的是一个李氏常数为$C$的$x$的函数$y=f(x),x\in (a,b)$的图像。
  当$-\infty<a<b<+\infty$时,
  称点$(a,f(a)),(b,f(b))$为该曲线的\textbf{端点}。
  类似的,$\RR^1\times \RR^1$上一条$C$\textbf{垂直曲线} 指的是一个李氏常数为$C$的$y$的函数$x=g(y),y\in (a,b)$的图像。
  特别的,当$a=-\infty,b=+\infty$时,则称其为\textbf{无限$C$水平曲线}或\textbf{无限$C$垂直曲线}。
\end{defination}
\begin{defination}
  称平面$\RR^1\times \RR^1$上两条不交的无限$C$水平曲线所界区域为$C$\textbf{水平带域};
两条不交的无限$C$垂直曲线所界区域为$C$垂直带域。
\end{defination}
\par
以下我们总是假设$\mu_h,\mu_v$是两个正常数,满足$\mu_h\dot \mu_v <1$.
\begin{defination}
  平面$\RR^1 \times \RR^1$上的区域$D$称为$(\mu_h,\mu_v)$\textbf{矩形},
  若它是一个$\mu_h$水平带域与一个$\mu_v$垂直带域的交,
  那组$\mu_h$水平对边称为\textbf{边界的水平部分},记作$\partial_hD$;
  那组$\mu_v$垂直对边g称为\textbf{边界的垂直部分},记作$\partial_vD$.
\end{defination}
\par
对$p \in \RR^1\times\RR^1$,我们记$T_p(\RR^1\times\RR^1)$为平面$\RR^1\times\RR^1$过点$p$的切空间。令
$$K_{p}^{+}=\left\{\zeta_{p}=\left(\zeta_{p}^{-}, \zeta_{p}^{+}\right) \in T_{p}\left(\RR^{1} \times \RR^{1}\right)| | \zeta_{p}^{+}\left|\geqslant \mu_{0}^{-1}\right| \zeta_{p}^{-} |\right\},$$
$$K_{p}^{-}=\left\{\zeta_{p}=\left(\zeta_{p}, \zeta_{p}^{+}\right) \in T_{p}\left(\RR^{1} \times \RR^{1}\right)| | \zeta_{p}^{+}\left| \leqslant \mu_{h}\right| \zeta_{p} |\right\},$$
见图(\ref{pic:4-1}).
\begin{defination}
  令$D\subset\RR^1\times\RR^1$是一个区域.
  $f:D\to\RR^1\times\RR^1$是一个微分同胚。
  称$f$在$D$上满足$(\mu_h,\mu_v)$\textbf{锥形条件},
  如果有
  \begin{enumerate}
  \item $Df K_p^+ \subset K^+_f(p),\forall p \in D$;
  \item $Df^{-1}K^- \subset K^-f^{-1}(p),\forall p \in fD$;
    且存在常数$\lambda >1$,使得
  \item
$ {\left|\left(D f \zeta_{p}\right)^{+}\right| \geqslant \lambda\left|\zeta_{p}^{+}\right|} ,{\forall p \in D, \zeta_{p} \in K_{p}^{+}}$
\item $\left|\left(D f^{-1} \xi_{p}\right)^{-}\right| \geqslant \lambda\left|\zeta_{p}\right|, \quad \forall p \in f D, \xi_{p} \in K_{p}^{-}$
  \end{enumerate}
\end{defination}

直观地讲,
映射$f$满足$(\mu_h,\mu_V)$锥形条件指的是$f$在水平方向压缩和在垂直方向拉伸,
并且它将任意$\mu_0$垂直曲线变为$\mu_v$垂直曲线,
而其逆$f^{-1}$将任何$\mu_h$水平曲线变为$\mu_h$水平曲线。
\begin{theorem} [双曲不动点定理]
  令$D$是平面$\RR^1 \times \RR^1$上一个$(\mu_h,\mu_v)$矩形,
  $f:D\to \RR^1\times\RR^1$是一个微分同胚,
  满足$(\mu_h,\mu_v)$锥形条件,并且有
  \begin{enumerate}
  \item 相交条件成立:$f D \cap D \neq \varnothing$;
  \item  边界条件成列:$f D \cap \partial_{v} D=\varnothing;f \partial_{h} D \cap D=\varnothing$,
  \end{enumerate}
  则$f$在$D$中有唯一双曲不动点
$$
O=\prod_{i=-\infty}^{\infty} f^{i} D
$$
\end{theorem}
上述定理的几何直观见图\ref{pic:4.2.a}.
\subsection{证明的思路}
令
\begin{equation}
  \label{eq:4.1.1}
\begin{array}{ll}{D_{0}=D,} & {D_{i}=f\left(D_{i-1}\right) \cap D} \\ {D_{-i}=f^{-1}\left(D_{-i+1} \cap f D\right),} & {i \in \NN}.\end{array}
\end{equation}
第一步证明$D_{\pm i}$是$(\mu_h,\mu_V)$矩形,并满足条件
\begin{equation}
  \label{eq:4.1.2}
\begin{array}{ll}{\partial_{k} D_{i} \subset \partial_{h} D,} & {\partial_{v} D_{-i} \subset \partial_{v} D} \\ {D_{i} \subset D_{i-1},} & {D_{-i} \subset D_{-i+1}, \quad i \in \NN}.\end{array}
\end{equation}
第二步证明 $V=\cap\limit_{i=0}^{\infty}D_i$和$H=\cap\limit_{i=0}^{\infty}D_{-i}$分别是端点属于$\partial_hD$和$\partial_vD$的$\mu_v$垂直曲线和$\mu_h$水平曲线。
\par
第三步证明$H\cap V$是$f$的一个双曲不动点。
\par
见图\ref{pic:4.2.b}.
\subsection{几个引理}
\begin{corollary}
  \label{col:4.1.6}
平面上$\RR^1\times \RR^1$上一条$\mu_h$水平曲线与一条$\mu_v$垂直曲线最多有一个交点。
\end{corollary}

\begin{proof}
  令$h:x \mapsto y=h(x), \quad v: y \mapsto x-v(y)$是两个函数,它们的图像分别为$\mu_h$水平去曲线$H$及$\mu_v$垂直曲线$V$.再令$(x_1,y_1),(x_2,y_2)\in H \cap V,$则
  $\left|x_{1}-x_{2}\right|=\left|v \circ h\left(x_{1}\right)-v \circ h\left(x_{2}\right)\right|$
  $\leqslant \mu_{v}\left|h\left(x_{1}\right)-h\left(x_{2}\right)\right| \leqslant \mu_{h} \cdot \mu_{v}\left|x_{1}-x_{3}\right|$
  因为$\mu_h\mu_v<1$,故$|x_1-x_2|=0$.
  进一步$|y_1-y_2|=0$.
\end{proof}

\begin{corollary}
  \label{col:4.1.7}
  令$D\subset \RR^1 \times \RR^1$是一个$(\mu_h,\mu_v)$矩形,
  $H,V\subset D$  分别是端点集属于$\partial_vD$和$\partial_hD$的$\mu_h$水平曲线及$\mu_v$垂直曲线,
  则$H$与$V$相交于唯一一点。
\end{corollary}
\begin{proof}
交点的存在性可由连续性得到,而唯一性由引理\ref{col:4.1.6}保证。
\end{proof}

\begin{corollary}
  \label{col:4.1.8}
  令$D_1,D_2 \subset\RR^1\times \RR^1$是两个$(\mu_h,\mu_v)$矩形.
  设$f:D_1\to \RR^1 \times \RR^1$是一个微分同胚,
  并满足$(\mu_h,\mu_v)$锥形条件及
  \begin{enumerate}
  \item 相交条件:$fD_1 \cap D_2 \ne \varnothing $
  \item 边界条件:$f \partial_{h} D_{1} \cap D_{2}=\varnothing, \quad
    f D_{1} \cap \partial_{v} D_{2}=\varnothing$,
  \end{enumerate}
  则
  \begin{enumerate}
  \item $ fD_1 \cap D_2$是一个$(\mu_h,\mu_v)$矩形,满足
$$
\partial_{h}\left(f D_{1} \cap D_{2}\right) \subset \partial_{h} D_{2};
$$
\item $f^{-1} (D_2 \cap fD_1)$是一个$(\mu_h,\mu_v)$矩形,满足
$$
\partial_{v}\left(f^{-1}\left(D_{2} \cap f D_{1}\right)\right) \subset \partial_{v} D_{1}.
$$
\end{enumerate}
\end{corollary}
上述引理可参见图\ref{pic:4.4.3}.

\begin{proof}
  因为$f$满足$(\mu_h,\mu_v)$锥形条件,故$f$将$\mu_V$垂直曲线变为$\mu_V$垂直曲线,
  而$f^{-1}$将$\mu_h$水平曲线变为$\mu_h$水平曲线。
  因此$f$将区域$D_1$变成一个左右两边为$\mu_v$垂直曲线的曲边矩形$fD_1$.
  再由相交条件及边界条件,$fD_1$是上下穿过$D_2$的,
  也就是说$fD_1\cup D_2$ 形成一个十字形.
  因而$fD_1\cap D_2$也是一个曲边矩形,
  其上下两边属于$D_2$的边界的水平部分,
  而其左右两边属于$fD_1$的左右两边,
  因而$fD_1 \cap D_2$是一个$(\mu_h,\mu_v)$矩形,
  满足$\partial_h (fD_1 \cap D_2) \subset \partial_h D_2$.
  类似地可以证明另一个结论。
\end{proof}
\par
令$D_{\pm i},i \in \NN$,
是由\eqref{eq:4.1.1}定义的区域,
用数学归纳法及引理\ref{col:4.1.8}我们有
\begin{corollary}
  \label{cor:4.1.9}
  $D_{\pm i}$是$(\mu_h,\mu_v)$矩形,满足~\ref{eq:4.1.2}.
\end{corollary}

\subsection{(\mu_h,\mu_v)矩形的高和宽}
令
$$
\begin{array}{l}{K^{+}=\left\{(x, y) \in \mathbf{R}^{1} \times \mathbf{R}^{1}| | y\left|\geqslant \mu_{v}^{-1}\right| x |\right\}} \\ {K^{-}=\left\{(x, y) \in \mathbf{R}^{1} \times \mathbf{R}^{2}| | y\left| \leqslant \mu_{h}\right| x |\right\}}\end{array}
$$

给定$\RR^1\times \RR^1$上两点$p,q$,
我们记$\overrightarrow{p q} \in K^{\pm}$,
若连接$p,q$的向量属于$K^{\pm}$.
\par
令$D$是一个$(\mu_h,\mu_v)$矩形.
下面我们定义$D$的\textbf{高}$h(D)$及D的\textbf{宽}$w(D)$如下:
$$
h(D)=\sup _{p, q \in D \atop p q \in K^{+}} \operatorname{dist}(p, q)
$$

$$
w(D)=\sup _{p, q \in D \atop p q \in K^{-}} \operatorname{dist}(p, q)
$$

这里$\operatorname{dist}(\dot,\dot)$表示$\RR^1\times \RR^1$上两点之间的距离.

\begin{corollary}
  \label{cor:4.1.10}
令 $D \subset \RR^1 \times \RR^1$是一个闭区域,
$f:D \to \RR^1 \times \RR^1$是一个微分同胚,
满足$(\mu_h,\mu_v)$锥形条件.
设$\gamma \subset D$是一条$C^1$光滑的$\mu_v$垂直(或$\mu_h$水平)曲线.
若对某自然数$i \in \NN$,
$f^i$(或 f^{-i})在$\gamma$上有定义,
则
$ |f^i\gamma| \geqslant A_v \lambda^i |\gamma|$,
(或 $f^{-i} \gamma \geqslant A_h \lambda^i |\gamma|$ ).
这里$|\dot|$代表曲线的长度,
$A_v=\frac{1}{\sqrt{1+\mu_v^2}}$,$A_h=\frac{1}{\sqrt{1+\mu_h^2}}$,
而$\lambda>1$是定义 \ref{def:4.1.4}的常数.
\end{corollary}
\begin{proof}
我们只对$\gamma$是$\mu_v$垂直曲线情况证明引理,
另一种情况的证明可用类似方法得到.
\par
首先设$i=1$.
令曲线$\gamma$是一个映射$\gamma(t):[0,1] \to D$的象,
令$\zeta=(\zeta^-,\zeta^+)=\gamma(t)$,
则$\zeta \in K^+$.令
$$\eta(t) =(\eta^-,\eta^+)=Df\zeta, $$
则
\begin{equation}

\end{equation}

\end{proof}