
Page-29
第 一 章
E
B 园
E 河
E 沥
E

【
E
E

. 不 3
E
E
E

第 三 章
E
B
E

第 四 章
E
E

[

基 本 概 念 和 准 备 知 识

E 伟 莲

分 岔 与 分 岔 何 题 的 提 法
E

E

E
E

常 见 的 局 部 与 非 局 部 分 岔

奇 点 分 岗

c

Hopt 分岔 E
E

Poincart 分 岛 与 弘 Hilbert 第 16 何 题
关 于 Petrov 定 理 的 诛 明
E
E
E 沥 林 标 许 一
河
E
E

友 曲 不 动 点 及 马 蹄 存 在 定 理
E

E
Page-30

Page-31

Page-32

Page-33
[

述

江 万 人
沥

高 等 教 育 出 版 社
Page-34
史 3 “Smale 马 蹄 。 i
$ 4 线 恒 胥 射 的 复 吊 昼 尉 的 双 峡 何
5 “Birkhoff-Smale 定 理
第 五 章 空 间 中 双 曲 鞍 点 的 同 宿 分 岔
E t
2 n
医
邹 公 章 “ 实 二 次 单 峰 脓 射 旋 的 颂 引 子
E 河 玲
E 沥 沥 腾 3 浩 5
$ 3 PCzva) 不 存 在 稳 定 周 期 狸 闵 题
荃4 分布问题 ...... 2
耐′暴 E
附 录 A Denzct 浑 形 和 流 形 问 的 际 躬

E
附 录 C Thom 模 截 定 理 -

参 考 文 献
E 王
Page-35
[ 一

怀 沥 沥
E
突 砾 性 进 展 , 对 结 衿 不 穗 定 素 统 的 研 究 ( 即 分 岛 理 论 ) 便 取 到 超 来
超 多 的 关 注 , 分 岔 理 论 具 有 深 厚 的 实 际 胍 景 , 又 雷 借 助 于 现 代 数
E 吴 东
i 、

E
E

E
E e
医 不
E

既 焰 分 岛 现 象 智 速 地 存 在 于 自 熊 界 中 , 因 而 在 描 译 自 然 现 象
E 圭 政 东

E

s訾+侣一D莹+毓=&s岫'

E 一沙 仪 不 脱 政 沥
引 起 了 人 们 的 关 注 . 随 后 发 现 当 8 取 菜 些 特 定 值 时 , 系 统 有 非 通
怀 怡 不 2 河 林 东
沥
s
Page-36

\part{基本概念和准备知识}
第 一 章 基 本 概 念 和 准 备 知 识

作 为 全 书 的 准 备 , 我 们 在 $ 1 中 简 述 有 关 动 力 系 统 和 结 构 程
定 的 基 本 概 念 , 不 加 证 明 地 陈 述 一 些 重 要 结 果 ; 在 8 2 中 引 入 分 岑
E
心 流 形 定 理 和 正 规 形 理 论 , 在 研 究 分 岔 问 题 时 它 们 是 进 行 简 化 处
理 的 有 效 手 段 ! 最 后 , 在 8 5 中 介 绍 奇 异 向 量 场 的 普 适 开 折 和 分 岔
的 余 维 这 两 个 重 要 的 概 念 .

E 宇

动 力 系 统 的 概 念 和 理 论 是 从 人 们 对 常 微 分 方 程 的 研 究 中 产 生
和 发 展 起 来 的 , 而 丁 对 常 微 分 方 程 的 研 究 , 至 今 仍 是 动 力 系 统 理 论
a

E

董=/彻)】 0

其 中 广 R* 一 R“ 是 C 向 量 场 ,7 之 1 . 由 常 微 分 方 程 中 熟 知 的 缤
E
含 t 一 0 的 棠 区 间 上 存 在 . 如 果 /Cz) 满 足 适 当 条 件 ( 或 在 某 种 等
E 江 标 2 李 p 水 i
E

0 李 2

422 E 玖
E 河 3 技 2
Page-37
E 国 胡

又 如 ,60 年 代 从 气 象 学 研 究 中 提 出 的 Lorenz 方 程
巳
5 E

盖=叩_彻,

1雹=一zz+胤叭

E 伟扬 达
E

岔)值7】~13.926,7】~24'06和7a~z4.74附近时,相应系统的
目 玟 玖 江
玟
e
物 理 和 数 学 界 关 注 的 热 点 问 题 之 一 .

胡 如 , 从 生 态 学 中 提 出 的 虫 巳 差 分 模 塑

王 2 沥 E 水
E 租 训 2
E

英 用 2
0 途 缠 增 加 时 , 尕 (z》 不 断 出 现 借 周 期 分 岛 点 , 一 明 对 应 于 出 现 鹤
定 周 期 点 的 那 些 分 员 值 具 有 很 强 的 规 律 性 , 从 而 发 现 了 一 个 新 的
河
关 注 、

敏 学 上 作 为 研 究 分 岔 现 象 的 理 论 一 - 分 岔 理 论 主 要 研 究 三 类
匹 氓
甾 眺 艇 所 定 义 的 离 敬 动 力 系 统 的 分 岔 ; 函 数 方 程 的 零 解 随 参 数 炕
化 而 产 生 的 分 岔 . 前 两 类 分 岔 称 为 动 态 分 贪 , 一 第 三 粤 分 员 称 为
木 河
技
Page-38
E 河

E

e
E

Ed2oto

不 隼 证 明 , 对 V z,z: E R“, O。(z) 和 O。(za) 或 者 重 合 , 或
者 ( 对 有 限 的 时 间 : ) 不 相 交 . 因 此 ,(1. 1) 的 转 道 集 合 依 一 的 不 同
E

常 微 分 方 程 定 性 理 论 ( 或 称 为 儿 何 理 论 ) 的 首 要 目 标 , 就 是 对
E 林t d 2s
构 ( 扬 道 集 合 的 拓 扑 结 构 图 称 为 相 图 , 通 常 在 相 图 上 用 箭 头 标 明 对
D
E 肉
E [
倩 形 , 人 们 至 今 尚 不 清 楚 相 图 的 确 切 结 构 . 参 见 [YL,2],[Q]

玟
E 木
E 林口 e 一 c
玟
C1, 定 义 C.2, 附 注 C.4).Y 叉 E & “(M), 存 在 弋 过 8 E 趸 的 概
大 流 ar ( 见 附 录 B 中 定 义 B. 13, 定 义 B 15, 和 定 理 B.16), 为 了 讨
1
无 限 延 伸 ; 使 XX E &r “(M) 的 极 大 流 在 ( 一 co, 十 co) 上 存 在 ( 香
E

伟

0 颖 5

E 林 i
E

王
Page-39
国 B

动 态 分 岔 理 论 生 要 研 究 动 力 系 统 的 转 道 族 的 拓 扑 结 构 随 参 数
变 化 所 发 生 的 变 化 及 关 规 律 . 例 如 , 奇 点 ( 成 不 动 点 ) 的 汇 聚 与 分
E
( 或 环 的 形 成 与 破 梁 ! 以 及 一 些 更 复 机 的 动 力 学 行 为 ( 例 如 浑 沛
态 的 出 现 与 消 失 等 .

日 然 分 会 理 论 的 森 些 方 面 可 以 追 潼 到 Poineare 阡 代 , 但 坂 这
E 江江 吊 沥 林
一 , 大 郭 分 工 作 集 中 于 平 面 上 逼 化 程 度 不 禽 ( 即 余 维 二 2 的 分 岛 ,
包 揣 闷 宿 分 岔 和 异 宿 分 岛 问 题 等 . 分 岔 理 论 的 发 展 很 大 程 度 上 依
r
的 结 构 稳 定 性 有 较 完 境 的 结 昂 . 因 而 , 当 祖 空 间 维 数 增 大 或 系 统 的
遥 化 程 度 增 大 时 , 问 题 的 复 机 性 大 大 增 加 , 完 整 的 工 作 尚 属 少 见 。
此 外 , 最 初 人 们 席 望 在 分 岔 值 附 近 都 能 进 行 开 折 , 即 在 分 岔 值 附 近
E

应 结 构 龚 定 的 系 统 . 但 人 们 递 淅 认 识 到 , 在 不 少 情 况 下 分 岛 值 附
医 志 沥 c 沥
迷 , 本 节 的 第 五 章 $ 3 和 第 六 章 将 涉 及 这 一 问 题

E
E 沥 招
笔 , 第 六 章 由 郑 志 明 执 笔 , 附 录 凸 李 承 治 执 笔 , 最 后 经 集 体 讨 论 定
E 沥

下 面 简 要 介 绍 本 书 的 内 容 安 排 。

第 一 章 介 绍 基 木 概 念 和 准 备 知 识 。 我们假定读者具有 E
玟
只 作 了 简 略 的 介 绍 . 然 后 通 过 实 例 引 进 分 岔 的 概 念 及 分 岛 问 题 的
i
e
第 二 竟 介 绍 儿 类 平 面 向 量 场 的 最 典 型 的 分 岔 现 象 , 如 资 点 分 岛 , 闭
执 分 岔 、Hopf 分 呕 \ 同 窥 分 岛 等 , 以 及 研 究 这 些 闰 题 的 典 珩 方 法 ;
Page-40
E 骏 ss E

园 2 [

医 吊 5 志5
参 数 变 换 群 ; 对 固 定 +E R 和 所 有 的 E 聪 ,ex(t, 办 是 故 一 版 的
微 分 同 胚 .

[

2

氏 2 a

E 林

0

0

[

有 时 也 把 句 称 为 壮 上 的 Cr 徽 分 动 力 票 统 特 别 , 当 + E Z ,
称 为 穴 散 动 力 系 统 , 本 书 主 要 讨 论 连 续 流 、 由 于 它 与 离 散 流 有 密
一 王 沥 国 许 汀 江 士 江 梁 圆 国

E 怀 园
E 沥 述 林 2

Ed2opssssples 命 t
改 为 R+ ( 或 Z+) 或 者 R- ( 或 Z- , 则 相 应 地 得 到 过 z 的 正 半 软 道
或 者 负 半 尬 道 , 并 分 别 记 为 Oz+ (z) 或 者 O (z) .
定 义 1 2 过 x 的 正 ( 或 负 ) 半 扬 道 的 极 限 点 称 为 的 e( 戡 口
林 totesuto 沥 2
E 芸 c ,
[ 肉
E 人 c
E
E 主 阮 一 孙

的 邹 域 口 人 吊 , 和 菜 个 正 整 数 丶 , 恋 V | 丿 , 有 8CD) 口 一
必 y 不 是 游 荫 点 的 点 称 为 非 渡 茹 点 , # 的 所 有 非 游 茹 点 的 集 合 称
Page-41
B 园 `

迦 介 绍 了 臂 Hilbert 第 16 问 题 , 在 第 三 章 中 , 我 们 综 合 运 用 第 二
E 述
E 口 一 东 沥 u
集 (Smale 马 踵 ) 存 在 性 的 简 浩 而 严 格 的 判 别 方 法 . 逄 些 结 果 在 研
2
E
的 由 一 个 双 曲 鞋 点 和 一 个 双 曲 闽 轨 形 成 的 环 的 分 岛 、 第 六 章 介 绍
医
沥 途 一
E 怡 1 河 E 东 A 达
圣 吊
江 n
E
园 2
E
E 不 命 沥 命
E
上 , 但 不 可 邀 免 地 涉 及 到 一 些 离 散 动 力 系 统 的 情 形 . 我 们 力 求 在 迹
医 逊 不
长 的 结 果 戡 证 明 ; 在 着 力 于 可 读 性 的 同 时 , 尽 量 兼 顾 一 定 的 理 论
深 度 ; 并 疫 注 重 解 林 推 理 的 同 时 , 薛 顾 几 何 直 觅 . 本 习 的 大 部 分
园 t 吴
了 使 读 者 晟 于 接 受 , 我 们 对 这 些 材 料 做 了 整 理 和 加 工 , 例 如 , 第 三
沥
作 者 重 新 络 出 的 } 在 第 六 章 大 部 分 定 理 的 证 明 中 , 作 者 对 原 始 材 料
沥 扬 不 尘 c 胡 I 怡 c i 二
如 , 第 二 章 中 对 参 数 一 致 的 Hopf 分 岛 定 理 ; 对 Abel 积 分 霆 点 个 数
E
和 能 力 , 书 中 难 兖 有 不 奚 或 错 误 之 处 , 我 们 热 诚 放 迎 读 者 们 的 批 评
Page-42
a 河n

E t 「
E 怀 2
E

星 然 , wCz) 巳 Q(8),a(z) CC Q(8). 因 此 , 当 M 素 致 时 ,Q(9》
E t

E 不 伟 园 不 吊

E 医
称 为 一 个 临 界 元 .

给 出 临 界 元 的 定 义 是 为 了 陈 述 上 筒 洁 . 有 时 要 对 临 界 元 进 行
如 下 细 致 的 区 分 .

在 徽 分 间 狄 的 情 形 , 如 果 存 在 E 2+ , 恋 得 (8) 一 , 则 2 为
u l
E 吴 a 怀
t2ctstunate s 怀 沥
园 阡

在 向 坤 场 的 惊 形 , 临 界 元 工 有 两 种 类 型 . 一 种 类 型 是 , 工 由 一
E us t 刑 述 422 i 92
E c
夺 0 是 口 的 一 个 双 曲 不 动 点 ( 或 等 价 地 , DXCB) 的 所 有 特 征 根
E i 沥 余 p 规 林 d
E 红 东 L t 不 王 2 纳
E 八 水 吴 余 政 技
E
E
1 为 特 征 值 的 特 征 向 量

在 研 究 向 量 场 的 轨 道 结 构 时 , 局 部 的 困 集 在 奇 点 附 近 ( 就 奇
E 述
E 命河
Page-43
述 颠

e
E 化 坤 医 沥 李 p 25 林 丞
c
不 隼 证 明 , Oo(z), w(z), a(z), 0(9) 都 是 p 的 不 变 集 .
n
达
定 义 1.6 设 7 是
E 虑 a
E
E 沥 5
e 途 贤 水
E
向 量 场 在 口 上 每 一 点 与
E 怀 5 小
U,pE D CU, 和 在
【 坤 标
使 得 TC》 等 于 ? 的 周
E 吴 孙 亚
E 林 0 江

E 0 林9
E

国 玟 2 技 刑 东 口 技
射 , 可 以 把 对 C 向 量 场 在 闭 转 7 附 近 执 道 绪 构 的 研 究 , 转 化 为 对
E 述 25

我 们 现 在 转 向 结 构 稳 定 问 题 , 篓 言 之 , 在 “ 小 扰 动 “ 下 不 改 变
E

E 不 王
E 余 2
Page-44
E 江 沥

园 渡 沥 怀 吴 一 沥
党 生 的 基 硬 课 教 材 ! 也 可 供 相 关 学 科 学 生 或 科 技 人 员 当 作 参 考 书 。
明 沥 述
1990 年 “ 南 开 动 力 系 统 年 * 期 间 被 用 作 敏 材 , 根 据 我 们 的 经 验 , 对
轻 河 达 沥
E i 汀 孝 沥 一 一 一 芸 沥 河 江
的 \ 二 个 学 期 的 谜 稿 , 则 可 以 讲 授 第 一 章 到 第 五 章 的 全 部 内 容 , 第
E 汀
立 即 转 入 第 二 \ 三 章 , 而 第 一 章 的 其 它 各 节 可 在 适 当 时 候 再 学 , 这
园 邹 育 沥 人

E
武 、 五 竹 君 , 王 铉 、 高 素 志 , 唐 云 \ 张 伟 年 、 李 宝 殿 、 李 举 萱 , 肖 冬 梅 、
齐 东 文 . 曹 欢 罗 、 王 兰 孙 , 赵 尔 琴 , 彭 临 平 等 同 志 和 我 们 的 研 究 生 们
表 示 愚 谢 , 他 们 的 报 告 和 讨 论 使 我 们 曼 益 匪 浅 其 中 有 的 间 志 诈 帮
n
E
E 扬 兰 不 述 述
坦 的 吊 位 专 家 , 他 们 从 审 定 本 书 的 撞 写 计 划 到 审 说 书 稿 都 提 出 了
E 河 述 东 东
援 , 仪 们 不 仅 对 本 习 提 出 了 很 多 建 设 性 的 意 见 , 西 东 还 提 供 了 部 分
习 题 ;. 感 谢 张 装 庆 教 捷 , 他 在 百 忍 中 审 阅 了 本 书 的 附 录 , 并 提 出 了
宝 贵 的 惧 见 ; 感 谢 高 等 敬 育 出 版 社 的 杨 芸 馨 等 同 志 , 变 有 他 们 的
E 江江 t 东

圭 描 写 本 书 期 间 , 作 者 们 得 到 国 家 自 然 科 学 基 金 和 离 等 学 校
e
E

L
1995 年 8 月 于 北 京 大 学
Page-45
第 一 章
E
B 园
E 河
E 沥
E

【
E
E

. 不 3
E
E
E

第 三 章
E
B
E

第 四 章
E
E

[

基 本 概 念 和 准 备 知 识

E 伟 莲

分 岔 与 分 岔 何 题 的 提 法
E

E

E
E

常 见 的 局 部 与 非 局 部 分 岔

奇 点 分 岗

c

Hopt 分岔 E
E

Poincart 分 岛 与 弘 Hilbert 第 16 何 题
关 于 Petrov 定 理 的 诛 明
E
E
E 沥 林 标 许 一
河
E
E

友 曲 不 动 点 及 马 蹄 存 在 定 理
E

E
Page-46
9 E 浩

n
水
E
52 0 技 标 t 标 沥 伟
【 t 江 y 不

当 同 背 还 保 持 儿 与 X: 相 应 转 道 的 时 间 对 应 时 , 秒 为 拓 扑
转
此 时 常 省 略 “ C1 “, 简 称 为 结 构 稳 定 .

诊

面 的 局 部 结 果 . -
玟

的 开 集 ; 向 量 场 元 以 O 为 双 曲 奇 炉 ( 或 g: 口 一 RF 以 O 为 双 曲 不
EJRS8 沥 ri i 沥 俊 s 沥 p
E 述 绍 e

扑 共 软 , 〖

E 沥珑 d
为 双 曲 奇 点 , 则 一 在 O 点 附 近 局 部 绪 构 稳 定 , 即 存 在 丿 在 . “CRT )
中 的 一 个 C+ 邻 域 ,Y Z E , 在 0 附 近 有 唯 一 的 双 曲 奇 炉 , 万
E ppasEueoet 沥 怀 e

E

附 注 1 10 “ 如 果 把 定 义 1. 7 中 的 拓 扑 扔 道 等 价 从 C? 加 强 到
E 心 述n l
性 化 系 统 的 特 征 根 有 相 同 的 比 值 ( 见 [GH,p42]. 这 就 把 等 价 关 系
限 制 过 严 , 使 得 两 个 软 道 结 构 相 同 的 向 量 场 也 未 必 等 价 例 如 , 由
定 理 1.9 可 知 , 二 维 系 统

五 二 z 予 二 3
e
E 林逊
Page-47
史 3 “Smale 马 蹄 。 i
$ 4 线 恒 胥 射 的 复 吊 昼 尉 的 双 峡 何
5 “Birkhoff-Smale 定 理
第 五 章 空 间 中 双 曲 鞍 点 的 同 宿 分 岔
E t
2 n
医
邹 公 章 “ 实 二 次 单 峰 脓 射 旋 的 颂 引 子
E 河 玲
E 沥 沥 腾 3 浩 5
$ 3 PCzva) 不 存 在 稳 定 周 期 狸 闵 题
荃4 分布问题 ...... 2
耐′暴 E
附 录 A Denzct 浑 形 和 流 形 问 的 际 躬

E
附 录 C Thom 模 截 定 理 -

参 考 文 献
E 王
Page-48
E c 0

( 不 0) 是 C 等 价 的 . 但 它 们 不 是 C 等 价 的 , 另 一 方 西 , 如 果 把 定
u y
E
一 是 局 部 结 构 稳 定 的 , 但 它 的 C 扰 动 系 统 丿 一 z 十 x VTzT 与
玟
人 1 因 此 , 如 无 特 别 声 明 , 下 文 中 的 等 价 都 挡 C“ 等 价 , 而 扰 动 都
栗 C 抓 动 .
在 给 出 进 一 步 的 结 果 之 前 , 我 们 需 要 下 面 的 定 义 .
定 义 1. 11 设 口 ,p 咏 上 . 称 集 合
E 林
和
E 一 伟
分 别 为 p 在 双 曲 不 动 点 0 的 穗 定 流 渺 与 不 穗 定 流 形
对 二 削 量 场 的 情 形 , 有 类 似 的 定 义 .
E
林
E
A 孝 仪 )
分 别 称 为 又 在 ? 的 穗 定 流 形 和 不 稳 定 流 形 .
利 用 Poincart 晓 射 , 可 以 把 吊 量 场 在 双 曲 闭 轨 附 近 的 研 究 转
中 育 沥
A g d 沥 芸 E

以
E
(O)) 在 0 点 附 近 都 是 M 的 子 流 形 进 而 可 以 证 明 , 它 们 在 O
E 沥 沥 林 5 振
E 沥
Page-49
第 一 章 基 本 概 念 和 准 备 知 识

作 为 全 书 的 准 备 , 我 们 在 $ 1 中 简 述 有 关 动 力 系 统 和 结 构 程
定 的 基 本 概 念 , 不 加 证 明 地 陈 述 一 些 重 要 结 果 ; 在 8 2 中 引 入 分 岑
E
心 流 形 定 理 和 正 规 形 理 论 , 在 研 究 分 岔 问 题 时 它 们 是 进 行 简 化 处
理 的 有 效 手 段 ! 最 后 , 在 8 5 中 介 绍 奇 异 向 量 场 的 普 适 开 折 和 分 岔
的 余 维 这 两 个 重 要 的 概 念 .

E 宇

动 力 系 统 的 概 念 和 理 论 是 从 人 们 对 常 微 分 方 程 的 研 究 中 产 生
和 发 展 起 来 的 , 而 丁 对 常 微 分 方 程 的 研 究 , 至 今 仍 是 动 力 系 统 理 论
a

E

董=/彻)】 0

其 中 广 R* 一 R“ 是 C 向 量 场 ,7 之 1 . 由 常 微 分 方 程 中 熟 知 的 缤
E
含 t 一 0 的 棠 区 间 上 存 在 . 如 果 /Cz) 满 足 适 当 条 件 ( 或 在 某 种 等
E 江 标 2 李 p 水 i
E

0 李 2

422 E 玖
E 河 3 技 2
Page-50
B E

形 , 见 附 录 日 中 附 注 B, 24》, 证 明 可 参 考 [ZQ, 定 理 4. 9 和 4. 10].

E 芸 沥 闵 团 李 国 口
全 局 绪 构 稳 定 性 的 结 果 ( 参 见 [ZDHD]).

E 一
s

265 东

《2 成 的 临 界 元 ( 奇 点 或 闭 轨 ) 个 数 有 限 , 并 且 它 们 都 是 双 曲
的 j

[ 沥 5
流 形 槲 截 相 交 ,【

这 里 朱 截 性 的 定 义 见 附 录 C 中 定 义 C 10. 注 意 , 不 相 交 也 算
E

定 理 1. 14 “ 二 维 可 定 向 紧 敌 流 形 M 上 C“ 结 构 穗 定 向 量 场 的
E

附 注 1. 15 Peixoto 等 人 的 上 述 结 果 , 从 60 年 代 开 始 呆 引 了
E
E 河
把 满 足 定 理 1 13 中 三 个 条 件 的 向 量 场 称 为 Morse-Smale 向 量 场
E 沥
E 沥 沥 招 达
著 名 侧 子 ( 见 第 四 章 ), 以 及 Newhouse 随 后 对 “ 马 跨 “ 的 改 造 , 说 明
野 B 命河
( 或 微 分 同 胚 ) 未 必 是 M-S 的 , 百 全 体 结 构 的 稳 定 向 最 来 ( 或 徽 分
同 胚 ) 的 集 合 在 ““(M)( 或 Ditf “CM)) 中 不 一 定 是 稠 集 . 进 一 步
i
Eurseceouetenseoaeatss 述
M-S 条 件 更 广 的 公 理 A 条 件 , 并 东 给 出 两 个 结 构 穗 定 性 猜 滁 ; 结
E d
结 构 稳 定 33 公 理 A 十 无 环 条 件 . 这 两 个 猜 洪 的 充 分 性 部 分 已 为
Page-51
E 河

E

e
E

Ed2oto

不 隼 证 明 , 对 V z,z: E R“, O。(z) 和 O。(za) 或 者 重 合 , 或
者 ( 对 有 限 的 时 间 : ) 不 相 交 . 因 此 ,(1. 1) 的 转 道 集 合 依 一 的 不 同
E

常 微 分 方 程 定 性 理 论 ( 或 称 为 儿 何 理 论 ) 的 首 要 目 标 , 就 是 对
E 林t d 2s
构 ( 扬 道 集 合 的 拓 扑 结 构 图 称 为 相 图 , 通 常 在 相 图 上 用 箭 头 标 明 对
D
E 肉
E [
倩 形 , 人 们 至 今 尚 不 清 楚 相 图 的 确 切 结 构 . 参 见 [YL,2],[Q]

玟
E 木
E 林口 e 一 c
玟
C1, 定 义 C.2, 附 注 C.4).Y 叉 E & “(M), 存 在 弋 过 8 E 趸 的 概
大 流 ar ( 见 附 录 B 中 定 义 B. 13, 定 义 B 15, 和 定 理 B.16), 为 了 讨
1
无 限 延 伸 ; 使 XX E &r “(M) 的 极 大 流 在 ( 一 co, 十 co) 上 存 在 ( 香
E

伟

0 颖 5

E 林 i
E

王
Page-52
E 振 g 9

Smale 本 人 和 其 它 人 所 证 明 ; 必 要 性 部 分 则 于 1987 年 分 别 由
ucutoueotuteuooalotao s 招 应 5 兰 阮 l 汀
的 第 一 个 猪 测 , 直 到 最 近 才 由 廖 山 溥 0“ , 胡 森 “7, 和 Hayashi
s

$ 2 - 分 岔 与 分 岔 问 题 的 提 法

怡 故 江 林 林 余
a 沥
研 究 结 构 不 穗 定 向 量 场 在 “ 扰 动 “ 下 捉 道 结 构 的 变 化 规 律 , 是 分 岔
E

分 岔 的 概 念

E
e

E 沥 沥 2 0
则 称 ; 为 族 丿 (6) 的 分 岱 值 , 当 参 数 s 通 过 分 岔 值 时 , 在 相 空 间 中
sss

对 于 离 敬 动 力 系 统 , 也 可 给 出 类 似 的 定 义 .

[ 园 u
奇 点 必 是 非 双 曲 的 .

医 肖
E 1
或 者 它 的 奇 点 和 闭 轨 是 双 曲 的 , 伯 它 们 的 穗 定 流 形 和 不 稳 定 流 形
i
游 荣 集 含 有 无 限 多 个 临 界 元 .

E 林
E 不 b
Page-53
E 骏 ss E

园 2 [

医 吊 5 志5
参 数 变 换 群 ; 对 固 定 +E R 和 所 有 的 E 聪 ,ex(t, 办 是 故 一 版 的
微 分 同 胚 .

[

2

氏 2 a

E 林

0

0

[

有 时 也 把 句 称 为 壮 上 的 Cr 徽 分 动 力 票 统 特 别 , 当 + E Z ,
称 为 穴 散 动 力 系 统 , 本 书 主 要 讨 论 连 续 流 、 由 于 它 与 离 散 流 有 密
一 王 沥 国 许 汀 江 士 江 梁 圆 国

E 怀 园
E 沥 述 林 2

Ed2opssssples 命 t
改 为 R+ ( 或 Z+) 或 者 R- ( 或 Z- , 则 相 应 地 得 到 过 z 的 正 半 软 道
或 者 负 半 尬 道 , 并 分 别 记 为 Oz+ (z) 或 者 O (z) .
定 义 1 2 过 x 的 正 ( 或 负 ) 半 扬 道 的 极 限 点 称 为 的 e( 戡 口
林 totesuto 沥 2
E 芸 c ,
[ 肉
E 人 c
E
E 主 阮 一 孙

的 邹 域 口 人 吊 , 和 菜 个 正 整 数 丶 , 恋 V | 丿 , 有 8CD) 口 一
必 y 不 是 游 荫 点 的 点 称 为 非 渡 茹 点 , # 的 所 有 非 游 茹 点 的 集 合 称
Page-54
E E

E 仪 是这个开稠集的〖边界[但在高维流形的情形,

E 胡 3

E 怀 坂 国 圆 y
E 林
极 限 集 与 。 极 限 集 都 与 这 奇 点 ( 或 闭 辅 ) 一 致 - 相 轨 线 称 为 异 宿
Cheterolinie) 扬 , 如 果 它 的 e 极 限 集 和 。 极 限 集 是 不 同 的 契 点 或
沥

定 义 2.4 发 生 在 奇 炉 ( 或 闭 转 ) 的 小 邻 域 内 , 并 且 与 它 的 双
E 不
i
E

一 面 的 定 义 2.1 一 2.4 叟 自 [AAIS]. 我 们 将 在 后 面 看 到 , 在 研
野 不
也 可 能 伴 随 出 现 全 局 分 岔 .

附 注 2.58 “ 如 上 所 述 , 分 岔 集 心 (M) 一 4 “CMJNSrCM) 是

t 述 i 述
E d 不 育 1s 浩 6 儿 4 河 浩
仁 GM 中 具 有 更 复 杂 分 岔 现 象 的 闭 子 集 . 我 们 可 以 继 绪 对

47.QK) 进 行 这 种 剔 分 , 得 到 各 种 层 次 的 分 岔 集 .
E 技

盖=焯一鲈=瞄″】磺沈 0

E
0
e
E i i
是 双 幼 的 , 图 1-2 中 给 出 (2. 1 皇 奇 点 分 布 对 参 数 x 的 依 赖 关 系 .
E
Page-55
a 河n

E t 「
E 怀 2
E

星 然 , wCz) 巳 Q(8),a(z) CC Q(8). 因 此 , 当 M 素 致 时 ,Q(9》
E t

E 不 伟 园 不 吊

E 医
称 为 一 个 临 界 元 .

给 出 临 界 元 的 定 义 是 为 了 陈 述 上 筒 洁 . 有 时 要 对 临 界 元 进 行
如 下 细 致 的 区 分 .

在 徽 分 间 狄 的 情 形 , 如 果 存 在 E 2+ , 恋 得 (8) 一 , 则 2 为
u l
E 吴 a 怀
t2ctstunate s 怀 沥
园 阡

在 向 坤 场 的 惊 形 , 临 界 元 工 有 两 种 类 型 . 一 种 类 型 是 , 工 由 一
E us t 刑 述 422 i 92
E c
夺 0 是 口 的 一 个 双 曲 不 动 点 ( 或 等 价 地 , DXCB) 的 所 有 特 征 根
E i 沥 余 p 规 林 d
E 红 东 L t 不 王 2 纳
E 八 水 吴 余 政 技
E
E
1 为 特 征 值 的 特 征 向 量

在 研 究 向 量 场 的 轨 道 结 构 时 , 局 部 的 困 集 在 奇 点 附 近 ( 就 奇
E 述
E 命河
Page-56
E 源 e

一

E E

|
薯
|

丨 D
薯
E E

E 怡 玟 应 38 in 诊 沥 才 仪
i
E 水 5
刑 (2.1 的 转 道 拓 扑 结 构 所 发 生 的 变 化 . 显 然 , 一 0 是 唯 一 的
E

例 2.7 “ 考 虑 R 上 的 向 量 场

蔷=#_z=, E

与 上 例 同 样 的 分 析 可 知 ,k 一 0 是 (2 2) 的 唯 一 分 岔 值 ; 一 0 时 向
量 场 (2. 2) 无 奇 点 ; 当 p>0 时 ,(2. 2) 有 二 个 双 曲 奇 炎 当 一 0 时
[ 应 沥
(sadale-uode) , 并 把 这 种 分 岗 现 象 称 为 鞍 结 点 分 岔 , 与 上 例 相 似 ,
可 分 别 作 出 图 1-4 与 图 1-5. 从 下 面 的 例 中 , 我 们 可 以 对 这 一 名 称
E '

例 2.8 “ 二 维 的 鞍 结 点 分 岔 . 考 虑 R 上 的 向 量 地

{ 0
F '

国 d O是唯一的分岔值,并可作出轨道拓扑分类图
Page-57
述 颠

e
E 化 坤 医 沥 李 p 25 林 丞
c
不 隼 证 明 , Oo(z), w(z), a(z), 0(9) 都 是 p 的 不 变 集 .
n
达
定 义 1.6 设 7 是
E 虑 a
E
E 沥 5
e 途 贤 水
E
向 量 场 在 口 上 每 一 点 与
E 怀 5 小
U,pE D CU, 和 在
【 坤 标
使 得 TC》 等 于 ? 的 周
E 吴 孙 亚
E 林 0 江

E 0 林9
E

国 玟 2 技 刑 东 口 技
射 , 可 以 把 对 C 向 量 场 在 闭 转 7 附 近 执 道 绪 构 的 研 究 , 转 化 为 对
E 述 25

我 们 现 在 转 向 结 构 稳 定 问 题 , 篓 言 之 , 在 “ 小 扰 动 “ 下 不 改 变
E

E 不 王
E 余 2
Page-58
E 途

云 素 轼
五 M

E
E
例 2.9 “ 考 虑 R 上 (0,0) 点 附 近 的 单 参 数 系 统 族
i
【 y
E 沥

它 的 线 伯 部 分 短 阵 以 x 土 i 为 特 征 值 . 在 极 坐 标 变 换 下 , 方 程 (2. 4
变 形 为 「
Page-59
E b E

当 口 0 时 , 原 点 是 租 定 的 焦 点 (x 一 0 时 非 双 曲 ) 当 x 丿 0 时 , 原
点 为 不 稳 定 焦 点 , 并 有 唯 一 闭 轨 r 一 VA , 它 是 租 定 的 极 限 环 T.,
E 不
捉 道 以 它 为 极 限 集 , 因 此 称 它 为 稳 定 的 极 限 环 . 注 意 , 当 x 一 0
(

图 1-7

E t
E
E

例 2. 19 ([DL]》 首 先 考 虑 一 个 平 面 系 统

0

E
E

F(z,汊)亭鲈十如z_藁氢=c` E

其 中 C 为 任 意 常 数 . 为 了 下 文 中 的 方 便 , 记

巳 招 「
C_27 E 0
Page-60
E

E
则 方 程 (2. 5 的 扬 道 依 x 的 不 同 选 取 , 有 如 下 分 布 ,

Ep N
E

EC2 t e 皇 东
所 国 有 界 区 域 之 外 和 Z 的 左 仁 当 ~> 0- 时 ,2x 收 缉 到 P U
E 所

E
T 它 位 于 e 所 围 有 界 区 域 的 内 部 ; 另 一 条 ( 无 界 ) 转 道 x 在 H
u n

令 函 数

E (龛 5 ″〕 ,

E i 技tn 沥
D
于
5

一 3

1
Page-61
9 E 浩

n
水
E
52 0 技 标 t 标 沥 伟
【 t 江 y 不

当 同 背 还 保 持 儿 与 X: 相 应 转 道 的 时 间 对 应 时 , 秒 为 拓 扑
转
此 时 常 省 略 “ C1 “, 简 称 为 结 构 稳 定 .

诊

面 的 局 部 结 果 . -
玟

的 开 集 ; 向 量 场 元 以 O 为 双 曲 奇 炉 ( 或 g: 口 一 RF 以 O 为 双 曲 不
EJRS8 沥 ri i 沥 俊 s 沥 p
E 述 绍 e

扑 共 软 , 〖

E 沥珑 d
为 双 曲 奇 点 , 则 一 在 O 点 附 近 局 部 绪 构 稳 定 , 即 存 在 丿 在 . “CRT )
中 的 一 个 C+ 邻 域 ,Y Z E , 在 0 附 近 有 唯 一 的 双 曲 奇 炉 , 万
E ppasEueoet 沥 怀 e

E

附 注 1 10 “ 如 果 把 定 义 1. 7 中 的 拓 扑 扔 道 等 价 从 C? 加 强 到
E 心 述n l
性 化 系 统 的 特 征 根 有 相 同 的 比 值 ( 见 [GH,p42]. 这 就 把 等 价 关 系
限 制 过 严 , 使 得 两 个 软 道 结 构 相 同 的 向 量 场 也 未 必 等 价 例 如 , 由
定 理 1.9 可 知 , 二 维 系 统

五 二 z 予 二 3
e
E 林逊
Page-62
弧

E 堇虹_ 国
紫 | , - 媒 訾 + 要 刑 | -arceouoy,

由 此 可 知 , 当 |x| 心 1 时 , 系 统 (2.8) 的 轨 道 分 布 如 图 1-9 所 示 . ( 所

Q

E E 《o p>0

国
用 的 论 探 , 类 似 于 用 Liapunov 函 数 判 断 奇 点 的 穗 定 性 ). 并 东 容 易
2
E
[
E
玟
种 半 局 部 分 会 .
y 沥 5
0 沥
【n a 汀 E
其 中 a七 0. 在 乙 一 0 附 近 的 一 个 邰 域 内 , 晏 射 尹 以 z 一 0 为 唯 一
不 动 点 . 注 意 , 由 隐 函 数 定 理 可 知 ,(2. 9) 的 任 一 扰 动 系 统 在 z 一 0
附 近 仍 有 唯 一 的 不 动 点 , 所 以 我 们 不 妙 取 它 的 扰 动 系 统 保 持 z 一 0
为 不 助 点 , 东 具 有 下 面 的 形 式
E 应2n
E 中 (
性 Fx 的 两 次 境 代 晓 射 , 得 到
Page-63
E c 0

( 不 0) 是 C 等 价 的 . 但 它 们 不 是 C 等 价 的 , 另 一 方 西 , 如 果 把 定
u y
E
一 是 局 部 结 构 稳 定 的 , 但 它 的 C 扰 动 系 统 丿 一 z 十 x VTzT 与
玟
人 1 因 此 , 如 无 特 别 声 明 , 下 文 中 的 等 价 都 挡 C“ 等 价 , 而 扰 动 都
栗 C 抓 动 .
在 给 出 进 一 步 的 结 果 之 前 , 我 们 需 要 下 面 的 定 义 .
定 义 1. 11 设 口 ,p 咏 上 . 称 集 合
E 林
和
E 一 伟
分 别 为 p 在 双 曲 不 动 点 0 的 穗 定 流 渺 与 不 穗 定 流 形
对 二 削 量 场 的 情 形 , 有 类 似 的 定 义 .
E
林
E
A 孝 仪 )
分 别 称 为 又 在 ? 的 穗 定 流 形 和 不 稳 定 流 形 .
利 用 Poincart 晓 射 , 可 以 把 吊 量 场 在 双 曲 闭 轨 附 近 的 研 究 转
中 育 沥
A g d 沥 芸 E

以
E
(O)) 在 0 点 附 近 都 是 M 的 子 流 形 进 而 可 以 证 明 , 它 们 在 O
E 沥 沥 林 5 振
E 沥
Page-64
E 国 a 振 s

p 胡2 仪 4
从 面 硼 (z 一 z 可 表 示 为
z[x(2+ 闪 十 O(UAJz~- (2+0CO)Jazz+ O(|z[-
E 不
E 河 河芬河 沥 扬
点 ) 之 外 , 又 有 两 个 新 的 不 动 点 , 它 们 是 F, 的 2 周 期 点 ( 图 1-100a)

- 训
E
E
e @ A
\
E
史 E E

0

E
E
的 Poincare 映 射 , 则 当 x 的 值 从 负 到 正 的 瞬 间 ( 设 a>> 0), 原 有 的 稚
E
儿 乎 两 倍 于 原 周 期 的 闭 轨 兰 , 这 种 分 岔 现 象 称 为 借 周 期 分 岑 , 它 发
E 浩 东 达 应2
这 带 的 边 界

分 岔 问 题 的 提 法
E
寄 x [

E 吴 技 l 一 江 4
Page-65
B E

形 , 见 附 录 日 中 附 注 B, 24》, 证 明 可 参 考 [ZQ, 定 理 4. 9 和 4. 10].

E 芸 沥 闵 团 李 国 口
全 局 绪 构 稳 定 性 的 结 果 ( 参 见 [ZDHD]).

E 一
s

265 东

《2 成 的 临 界 元 ( 奇 点 或 闭 轨 ) 个 数 有 限 , 并 且 它 们 都 是 双 曲
的 j

[ 沥 5
流 形 槲 截 相 交 ,【

这 里 朱 截 性 的 定 义 见 附 录 C 中 定 义 C 10. 注 意 , 不 相 交 也 算
E

定 理 1. 14 “ 二 维 可 定 向 紧 敌 流 形 M 上 C“ 结 构 穗 定 向 量 场 的
E

附 注 1. 15 Peixoto 等 人 的 上 述 结 果 , 从 60 年 代 开 始 呆 引 了
E
E 河
把 满 足 定 理 1 13 中 三 个 条 件 的 向 量 场 称 为 Morse-Smale 向 量 场
E 沥
E 沥 沥 招 达
著 名 侧 子 ( 见 第 四 章 ), 以 及 Newhouse 随 后 对 “ 马 跨 “ 的 改 造 , 说 明
野 B 命河
( 或 微 分 同 胚 ) 未 必 是 M-S 的 , 百 全 体 结 构 的 稳 定 向 最 来 ( 或 徽 分
同 胚 ) 的 集 合 在 ““(M)( 或 Ditf “CM)) 中 不 一 定 是 稠 集 . 进 一 步
i
Eurseceouetenseoaeatss 述
M-S 条 件 更 广 的 公 理 A 条 件 , 并 东 给 出 两 个 结 构 穗 定 性 猜 滁 ; 结
E d
结 构 稳 定 33 公 理 A 十 无 环 条 件 . 这 两 个 猜 洪 的 充 分 性 部 分 已 为
Page-66
E ss

蛀 E 0

是 结 构 不 稳 定 的 , 当 川 人 1 时 , 常 把 (2. 10) 称 寺 (2. 11 的 一 个
E u 1

[ i
7 中 变 动 时 , 弄 清 系 统 (2. 10 的 转 道 结 构 如 何 变 化 ?

E 余
们 对 应 (2. 10 的 轨 道 拓 扑 结 构 的 不 同 等 价 类 ? 在 例 2.6 中 , 一 0
E d 东
0

进 一 歪 的 问 题 是 。

间 题 B “ 能 否 找 到 (2. 11) 的 开 折 , 它 “ 包 含 * 了 (2. 113 的 任 一
开 折 所 能 出 现 的 转 道 结 构 ?

[
d

对 上 述 问 题 中 一 些 名 词 的 确 切 含 义 进 行 澄 清 , 将 会 引 导 到 “ 普
适 开 扳 “ 和 “ 分 岔 的 余 维 “ 这 样 一 些 深 刻 的 概 念 , 我 们 将 在 8 5 中
介 绍 , 本 节 先 对 这 些 概 念 给 出 直 观 的 描 述 , 这 里 需 要 指 出 的 是 , 由
E
Ba d c
E 浩 d 人 2
E

E

茁 E 0

i
折 . 我 们 要 证 明 ,(2. 1 不是满足间题B要求的那种开折 E
感 (2.127 的任意一个C阗 E

茁 A E
Page-67
E 振 g 9

Smale 本 人 和 其 它 人 所 证 明 ; 必 要 性 部 分 则 于 1987 年 分 别 由
ucutoueotuteuooalotao s 招 应 5 兰 阮 l 汀
的 第 一 个 猪 测 , 直 到 最 近 才 由 廖 山 溥 0“ , 胡 森 “7, 和 Hayashi
s

$ 2 - 分 岔 与 分 岔 问 题 的 提 法

怡 故 江 林 林 余
a 沥
研 究 结 构 不 穗 定 向 量 场 在 “ 扰 动 “ 下 捉 道 结 构 的 变 化 规 律 , 是 分 岔
E

分 岔 的 概 念

E
e

E 沥 沥 2 0
则 称 ; 为 族 丿 (6) 的 分 岱 值 , 当 参 数 s 通 过 分 岔 值 时 , 在 相 空 间 中
sss

对 于 离 敬 动 力 系 统 , 也 可 给 出 类 似 的 定 义 .

[ 园 u
奇 点 必 是 非 双 曲 的 .

医 肖
E 1
或 者 它 的 奇 点 和 闭 轨 是 双 曲 的 , 伯 它 们 的 穗 定 流 形 和 不 稳 定 流 形
i
游 荣 集 含 有 无 限 多 个 临 界 元 .

E 林
E 不 b
Page-68
E E

E d 许
d
E

林
开 集 , 丁 E C“(U,R) 并 且 漪 足 ft,0) 一 tg( , 其 中 &E 2+,g 在
t 才
仪 8 i
E 河
E 人 2

E ( 红 , (om E 人

E
到 有 限 光 滑 性 . 由 上 面 的 定 理 可 知 , 对 于 开 折 ( 13) 存 在 R““, 中
t a
E (z E
E

E
E 05 仪
玟

黯 E 林 达 A 225

E 应
E

c
+2C08Z, xto = c 林

2 改 写 为 z 则 系 统 (2.15) 变 为
盖: E 怡 n
星 然 ,(2. 16) 所 能 出 现 的 转 道 拓 扑 类 型 不 趣 出 系 统
Page-69
E

莹 沥 [
E p p b
们 证 明 了 (2. 12) 的 任 一 开 折 ( 可 以 含 有 任 意 多 个 参 数 ) 所 能 出 现
E 标 沥
c
c d
园 沥 吊 d 训

集 合 为 原 点 O(0,0) 和 由
伟 a 人

玲
3

余 吴

Eotuteegcutot 芸 a p

图 -11 E
岔 图 1-11. 相 应 于 不 同 的 , 开 折 (2, 17》 的 5 种 转 道 拓 扑 分 类 中 有
E
E 匹 才
E 浩 育 一
Page-70
E E

E 仪 是这个开稠集的〖边界[但在高维流形的情形,

E 胡 3

E 怀 坂 国 圆 y
E 林
极 限 集 与 。 极 限 集 都 与 这 奇 点 ( 或 闭 辅 ) 一 致 - 相 轨 线 称 为 异 宿
Cheterolinie) 扬 , 如 果 它 的 e 极 限 集 和 。 极 限 集 是 不 同 的 契 点 或
沥

定 义 2.4 发 生 在 奇 炉 ( 或 闭 转 ) 的 小 邻 域 内 , 并 且 与 它 的 双
E 不
i
E

一 面 的 定 义 2.1 一 2.4 叟 自 [AAIS]. 我 们 将 在 后 面 看 到 , 在 研
野 不
也 可 能 伴 随 出 现 全 局 分 岔 .

附 注 2.58 “ 如 上 所 述 , 分 岔 集 心 (M) 一 4 “CMJNSrCM) 是

t 述 i 述
E d 不 育 1s 浩 6 儿 4 河 浩
仁 GM 中 具 有 更 复 杂 分 岔 现 象 的 闭 子 集 . 我 们 可 以 继 绪 对

47.QK) 进 行 这 种 剔 分 , 得 到 各 种 层 次 的 分 岔 集 .
E 技

盖=焯一鲈=瞄″】磺沈 0

E
0
e
E i i
是 双 幼 的 , 图 1-2 中 给 出 (2. 1 皇 奇 点 分 布 对 参 数 x 的 依 赖 关 系 .
E
Page-71
E 源 e

一

E E

|
薯
|

丨 D
薯
E E

E 怡 玟 应 38 in 诊 沥 才 仪
i
E 水 5
刑 (2.1 的 转 道 拓 扑 结 构 所 发 生 的 变 化 . 显 然 , 一 0 是 唯 一 的
E

例 2.7 “ 考 虑 R 上 的 向 量 场

蔷=#_z=, E

与 上 例 同 样 的 分 析 可 知 ,k 一 0 是 (2 2) 的 唯 一 分 岔 值 ; 一 0 时 向
量 场 (2. 2) 无 奇 点 ; 当 p>0 时 ,(2. 2) 有 二 个 双 曲 奇 炎 当 一 0 时
[ 应 沥
(sadale-uode) , 并 把 这 种 分 岗 现 象 称 为 鞍 结 点 分 岔 , 与 上 例 相 似 ,
可 分 别 作 出 图 1-4 与 图 1-5. 从 下 面 的 例 中 , 我 们 可 以 对 这 一 名 称
E '

例 2.8 “ 二 维 的 鞍 结 点 分 岔 . 考 虑 R 上 的 向 量 地

{ 0
F '

国 d O是唯一的分岔值,并可作出轨道拓扑分类图
Page-72
20 E 渡 s

E 沥
E 伟 江 医 一
E 一
妮

E 沥 医 不
E 述 E
引 出 的 分 岔 为 标 维 2 的 事 实 上 , 由 于 (2. 12 的 任 一 开 折 都 可 等
E 木
l 刑
E 达 才
E
tC 八 月 3 相 肖 匹 77 技 2江 许 河 a E 办 水
E 一

2 颖 兄 二 莲 沥

沥 江
E
河
E 仪
绍 的 曲 面 ( 相 应 于
[ 胡
开 折 ) 才 能 与 4 模
朐
1-13, 而 (2. 12) 的
E
的 工 中 的 曲 线 蛎 然 闯 命
E
相 交 , 但 在 任 意 小 的 批 动 下 , 它 都 可 以 与 口 分 离 . 换 男 话 说 , 至 少
E 一
【
Page-73
E 途

云 素 轼
五 M

E
E
例 2.9 “ 考 虑 R 上 (0,0) 点 附 近 的 单 参 数 系 统 族
i
【 y
E 沥

它 的 线 伯 部 分 短 阵 以 x 土 i 为 特 征 值 . 在 极 坐 标 变 换 下 , 方 程 (2. 4
变 形 为 「
Page-74
E b E

当 口 0 时 , 原 点 是 租 定 的 焦 点 (x 一 0 时 非 双 曲 ) 当 x 丿 0 时 , 原
点 为 不 稳 定 焦 点 , 并 有 唯 一 闭 轨 r 一 VA , 它 是 租 定 的 极 限 环 T.,
E 不
捉 道 以 它 为 极 限 集 , 因 此 称 它 为 稳 定 的 极 限 环 . 注 意 , 当 x 一 0
(

图 1-7

E t
E
E

例 2. 19 ([DL]》 首 先 考 虑 一 个 平 面 系 统

0

E
E

F(z,汊)亭鲈十如z_藁氢=c` E

其 中 C 为 任 意 常 数 . 为 了 下 文 中 的 方 便 , 记

巳 招 「
C_27 E 0
Page-75
0 国

纳
E

水
E 木 东 木

0
E 朋

E 沥 圆 河 丿

s
u ss 述 i 江
E y 沥 朐 述 i 一
E 不
E 河 i 月
把 结 果 推 广 到 微 分 同 狐 的 情 形 ( 例 如 , 参 见 [Wil]).

观 察 上 节 图 1-6 可 以 发 现 , 二 维 空 间 上 的 分 岔 现 象 其 实 主 要
E 水
图 1-5), 而 在 这 不 变 流 形 之 外 的 执 线 , 无 非 是 向 这 不 变 流 形 的 压
绘 . 出 现 这 种 规 律 并 不 是 健 然 的 , 系 统 (2. 8 在 (0,0 点 的 线 性 部
分 短 阵 为

[
1

E
E l
[
, 近 限 制 在 树 一 个 m 维 的 不 变 流 形 上 , 从 而 使 和 题 的 隼 度 得 以 降
[
Page-76
E

E
则 方 程 (2. 5 的 扬 道 依 x 的 不 同 选 取 , 有 如 下 分 布 ,

Ep N
E

EC2 t e 皇 东
所 国 有 界 区 域 之 外 和 Z 的 左 仁 当 ~> 0- 时 ,2x 收 缉 到 P U
E 所

E
T 它 位 于 e 所 围 有 界 区 域 的 内 部 ; 另 一 条 ( 无 界 ) 转 道 x 在 H
u n

令 函 数

E (龛 5 ″〕 ,

E i 技tn 沥
D
于
5

一 3

1
Page-77
第 一 章 药 本 李 伊 和 淅 备 知 订

线 性 情 形
先 考 虑 线 性 方 程
蕃 E 医

玟

5 [
的 性 态 完 全 被 矩 阵 4 的 特 征 值 的 性 质 所 决 定 .

E p
E 育 7 河
E
E 吴 F 刑
0

E

记 口 为 R 中 相 应 于 2E c 的 那 些 特 征 值 的 广 义 特 征 向 量 所 张 成
E
E 沥 八 达 述

邵 相 应 的 投 影
动 t 酥 一 , i 职 一 酥 t 联 一 阮
E
ker(m) 二 E′“鼻E翻 E
ker(xu) 二 丘v…鼻Ec 申 吴 ,
E E^矗鼻E翼 日 .
E 仪

医
的 , 而 在 B 中 则 是 指 数 型 “ 增 长 “ 的 (t 一 一 eo 时 的 情 况 相 反 . 所
Page-78
弧

E 堇虹_ 国
紫 | , - 媒 訾 + 要 刑 | -arceouoy,

由 此 可 知 , 当 |x| 心 1 时 , 系 统 (2.8) 的 轨 道 分 布 如 图 1-9 所 示 . ( 所

Q

E E 《o p>0

国
用 的 论 探 , 类 似 于 用 Liapunov 函 数 判 断 奇 点 的 穗 定 性 ). 并 东 容 易
2
E
[
E
玟
种 半 局 部 分 会 .
y 沥 5
0 沥
【n a 汀 E
其 中 a七 0. 在 乙 一 0 附 近 的 一 个 邰 域 内 , 晏 射 尹 以 z 一 0 为 唯 一
不 动 点 . 注 意 , 由 隐 函 数 定 理 可 知 ,(2. 9) 的 任 一 扰 动 系 统 在 z 一 0
附 近 仍 有 唯 一 的 不 动 点 , 所 以 我 们 不 妙 取 它 的 扰 动 系 统 保 持 z 一 0
为 不 助 点 , 东 具 有 下 面 的 形 式
E 应2n
E 中 (
性 Fx 的 两 次 境 代 晓 射 , 得 到
Page-79
E E

有 对 f ~ 士 c 有 界 的 执 道 ( 特 别 地 , 所 有 契 点 , 闭 轨 ) 都 停 国 在 砂
医 s 技 不 园 2
E s u
E 渡 c
E

非 线 性 情 形
E
需=Az十八z兆 【

E 技 0
程 (3. 5 的 转 道 结 构 是 否 仍 然 具 有 方 程 (3. 1) 的 上 述 规 律 ? 下 面
E d 沥 i
0 浩 g 途 途 木

常 实 用 的 还 是 在 契 点 z 一 0 的 局 部 , 那 些 “ 复 杂 现 象 “( 特 别 地 , 所
有 奇 点 . 闭 转 . 同 宿 转 、 异 宪 执 等 ) 都 发 生 在 W 上 ; 在 一 定 条 件 下 ,
国 O 述 沥
过 对 E 上 诱 导 的 方 程 的 研 究 而 得 到 . 本 节 的 内 容 主 要 参 考 了 [V]
E

我 们 先 陈 述 整 佛 的 结 果 .

不 伟 沥 园 育 2 0 技 人

[
E
林
一 一 一
E
C
0
Page-80
E 国 a 振 s

p 胡2 仪 4
从 面 硼 (z 一 z 可 表 示 为
z[x(2+ 闪 十 O(UAJz~- (2+0CO)Jazz+ O(|z[-
E 不
E 河 河芬河 沥 扬
点 ) 之 外 , 又 有 两 个 新 的 不 动 点 , 它 们 是 F, 的 2 周 期 点 ( 图 1-100a)

- 训
E
E
e @ A
\
E
史 E E

0

E
E
的 Poincare 映 射 , 则 当 x 的 值 从 负 到 正 的 瞬 间 ( 设 a>> 0), 原 有 的 稚
E
儿 乎 两 倍 于 原 周 期 的 闭 轨 兰 , 这 种 分 岔 现 象 称 为 借 周 期 分 岑 , 它 发
E 浩 东 达 应2
这 带 的 边 界

分 岔 问 题 的 提 法
E
寄 x [

E 吴 技 l 一 江 4
Page-81
第 一 章 荣 本 林伊 和 醉 吉 知 订

E 噩副嘲…0解‖ d 医 力

E 2
E

E 吴 c 居2 E 述
[
E 述 2 园 医
是 (3.5 的 不 变 集 , 则 M 一 印 , 且 一 一 9 !

[ 林 述
箐=^跷+骗八跷十俨〈蹄)), E 沥 E 达

定 义 3.3 “ 定 理 3. 2 中 的 不 变 集 卫 * 称 为 (3.5) 的 全 局 中 心 流
E

[ 诊 玟 明 述 怀 述 sss
指 出 , 眙 Pr 在 (3.5) 的 流 下 不 变 , 则 p E C8CB,) 是 唯 一 确 定
E 唐 u 沥

E
很 难 应 用 . 由 于 0) 一 0,DfK(0) = 0 故 在 奇 库 < 一 0 附 近 这 个 条
件 却 是 自 然 成 立 的 , 因 此 , 利 用 截 断 (cut-oftt 函 数 从 定 理 3. 2 得 出
i
E

1, 当 肥 奶 1,

k {0y E
E

E /(z)刁[ 熹〕 E (3.9)

此 时 , 为 了 研 究 方 程 (3. 5) 在 x 一 0 附 近 的 中 心 流 形 , 我 们 可 以 考
乐 方 程

塔 <4z+ foCam, [
Page-82
E ss

蛀 E 0

是 结 构 不 稳 定 的 , 当 川 人 1 时 , 常 把 (2. 10) 称 寺 (2. 11 的 一 个
E u 1

[ i
7 中 变 动 时 , 弄 清 系 统 (2. 10 的 转 道 结 构 如 何 变 化 ?

E 余
们 对 应 (2. 10 的 轨 道 拓 扑 结 构 的 不 同 等 价 类 ? 在 例 2.6 中 , 一 0
E d 东
0

进 一 歪 的 问 题 是 。

间 题 B “ 能 否 找 到 (2. 11) 的 开 折 , 它 “ 包 含 * 了 (2. 113 的 任 一
开 折 所 能 出 现 的 转 道 结 构 ?

[
d

对 上 述 问 题 中 一 些 名 词 的 确 切 含 义 进 行 澄 清 , 将 会 引 导 到 “ 普
适 开 扳 “ 和 “ 分 岔 的 余 维 “ 这 样 一 些 深 刻 的 概 念 , 我 们 将 在 8 5 中
介 绍 , 本 节 先 对 这 些 概 念 给 出 直 观 的 描 述 , 这 里 需 要 指 出 的 是 , 由
E
Ba d c
E 浩 d 人 2
E

E

茁 E 0

i
折 . 我 们 要 证 明 ,(2. 1 不是满足间题B要求的那种开折 E
感 (2.127 的任意一个C阗 E

茁 A E
Page-83
E E

E d 许
d
E

林
开 集 , 丁 E C“(U,R) 并 且 漪 足 ft,0) 一 tg( , 其 中 &E 2+,g 在
t 才
仪 8 i
E 河
E 人 2

E ( 红 , (om E 人

E
到 有 限 光 滑 性 . 由 上 面 的 定 理 可 知 , 对 于 开 折 ( 13) 存 在 R““, 中
t a
E (z E
E

E
E 05 仪
玟

黯 E 林 达 A 225

E 应
E

c
+2C08Z, xto = c 林

2 改 写 为 z 则 系 统 (2.15) 变 为
盖: E 怡 n
星 然 ,(2. 16) 所 能 出 现 的 转 道 拓 扑 类 型 不 趣 出 系 统
Page-84
E

E 沥 25 李 训 0 河
u y C3. 10)
E 东 沥 明 不 一 圆 育 2标 吴
ACo 二 0,D/(0) =0, 剧 3 pE CKCR,Bh) 和 z 一 0 在 R 中 的 开
E
CD 流 形
E 52 着 育 2
对 (3, 5) 的 流 局 部 不 变 , 印
E 仪 一 A 人 3
E 沥 c 林g 5 加
E 朋
【C2E X 河A 月
[ 沥 招 川
E 国 育 林EE i 日 技 振 才 f 技 川 [
E 朋
沥
E y
男 一 方 面 , 由 (3. 9) 可 知 , 英 故 - (z E R 国 i 川
E 技 0

E 述 人 育 沥
现 设 zE U,Ju(z) 一 R. 则 Y xE R,2(tyz) 一 t,a) 口 ,

从而着彗…I″^…〈hz)| 一 co, 由 (3.6) 式 ,z E 佛 r, 限 制 在 乙 内 , 也 就
E 宏 吴
匹 出 东 河 ] 如果婢仨C^(E(,互*)浓蓼L渊m E 兴 X00 河 玟
A 林目 园 技 孙 芸 不
国 42
[ 王 a 沥n 挂 u 口
Page-85
E

莹 沥 [
E p p b
们 证 明 了 (2. 12) 的 任 一 开 折 ( 可 以 含 有 任 意 多 个 参 数 ) 所 能 出 现
E 标 沥
c
c d
园 沥 吊 d 训

集 合 为 原 点 O(0,0) 和 由
伟 a 人

玲
3

余 吴

Eotuteegcutot 芸 a p

图 -11 E
岔 图 1-11. 相 应 于 不 同 的 , 开 折 (2, 17》 的 5 种 转 道 拓 扑 分 类 中 有
E
E 匹 才
E 浩 育 一
Page-86
E E

同 的 局 部 中 心 流 形 ( 尽 管 对 每 一 个 截 断 函 数 而 言 ,(3. 10 的 全 局
中 心 流 形 是 唯 一 的 . 例 如 , 在 图 !-6(x 一 0) 中 的 原 点 附 近 , 取 有
0
i
持 在 U 内 的 任 何 有 界 执 道 ( 包 括 奇 点 . 周 期 转 、 同 宿 转 , 异 宿 轨 等 )
都 出 现 在 (3. 5) 的 任 一 局 部 中 心 流 形 上 . 因 此 , 对 于 研 究 分 岔 现 象
00 沥
蚺 然 了 的 C 光 滢 性 保 证 了 吐 的 C4 光 滑 性 , 一 舫 来 说 , 了 的 C“ 光
玟
i
IDAal 一 8# 一 舫 来 说 , 当 & 一 eo 时 ,5x 一 0, 这 可 能 导 致 p 一 0

现 在 我 们 把 局 部 中 心 流 形 、 稳 定 、 不 稳 定 流 形 的 结 果 合 写 成 下
面 的 定 理 .

E 王 5 东 沥 英 育 5222 技
DAC0) 一 0; 相 对 于 4, 有 如 上 所 述 的 子 空 间 史 ,BY 和 Y, 则 在 R“
中 乙 一 0 附 近 存 在 开 邻 域 [, 和 口 中 的 Ct 流 形 WPr,Wv 和 伟 , 它 们
的 维 数 分 别 与 这 三 个 子 空 间 相 同 , 在 z = 0 点 分 别 与 吴 , 砂 和 砂
相 切 , 并 且 在 口 内 是 方 程 (3. 5 的 不 变 流 形 iW7“ 和 P“ 有 定 义 1. 12
野
E 命 巳 (

[
E 沥 e 洁
到 , 系 统 在 契 点 附 近 的 “ 复 杂 现 象 “ 发 生 在 它 的 任 一 局 部 中 心 流 形
上 . 下 面 的 两 个 定 理 说 月 了 中 心 流 形 的 其 它 重 要 作 用 、

定 理 3.9 ( 渐 近 性 质 定 理 〉 设 了 E C(R“,R“),f(0) 一 0,
D
i
EA22S5ac2 明 l 伟 一 东 人
Page-87
20 E 渡 s

E 沥
E 伟 江 医 一
E 一
妮

E 沥 医 不
E 述 E
引 出 的 分 岔 为 标 维 2 的 事 实 上 , 由 于 (2. 12 的 任 一 开 折 都 可 等
E 木
l 刑
E 达 才
E
tC 八 月 3 相 肖 匹 77 技 2江 许 河 a E 办 水
E 一

2 颖 兄 二 莲 沥

沥 江
E
河
E 仪
绍 的 曲 面 ( 相 应 于
[ 胡
开 折 ) 才 能 与 4 模
朐
1-13, 而 (2. 12) 的
E
的 工 中 的 曲 线 蛎 然 闯 命
E
相 交 , 但 在 任 意 小 的 批 动 下 , 它 都 可 以 与 口 分 离 . 换 男 话 说 , 至 少
E 一
【
Page-88
E

玖
E 东 述 l 国
0 s i 逊 t 浩 仪
E 5 沥
E
E 吴 育 沥 政
明 , 在 一 定 条 件 下 ,
E
纳
i
型 地 趋 于 中 心 流 形
上 的 某 一 解 ( 当 4
一 十 co 如 果 ou 一
【nice
E 砂
定 理 3. 10 说 明 , 在
类 似 条 件 下 , 为 了
得 到 局 部 中 心 流 形 上 0 点 附 近 的 执 道 结 构 , 只 需 要 对 它 在 线 性 子
E 水
0ooxudtu i 仁 e
E 述 2 河
E
E 东 0 月 0

E 吴 见 0 丞 A
| z 五 )} 是 (3. 5 的 一 个 局 部 中 心 流 形 , 当 且 仅 当 存 在 B 中 原

点 的 开 邰 域 0, 侩 得 V z E , 有
E 述
Page-89
0 国

纳
E

水
E 木 东 木

0
E 朋

E 沥 圆 河 丿

s
u ss 述 i 江
E y 沥 朐 述 i 一
E 不
E 河 i 月
把 结 果 推 广 到 微 分 同 狐 的 情 形 ( 例 如 , 参 见 [Wil]).

观 察 上 节 图 1-6 可 以 发 现 , 二 维 空 间 上 的 分 岔 现 象 其 实 主 要
E 水
图 1-5), 而 在 这 不 变 流 形 之 外 的 执 线 , 无 非 是 向 这 不 变 流 形 的 压
绘 . 出 现 这 种 规 律 并 不 是 健 然 的 , 系 统 (2. 8 在 (0,0 点 的 线 性 部
分 短 阵 为

[
1

E
E l
[
, 近 限 制 在 树 一 个 m 维 的 不 变 流 形 上 , 从 而 使 和 题 的 隼 度 得 以 降
[
Page-90
第 一 章 焦 本 概 伊 和 淅 备 知 识 -

E 河 [

E
利 用 (3. 14) 式 算 出 它 的 Taylor 展 式 的 前 儿 项 -
为 了 简 单 , 先 把 (3.5) 化 成 如 下 的 标 准 形 式 :

E

4 [
E

医 林 技 人 5
D
门 善 以 0]'
0 C

上 面 的 C 与 C: 的 特 征 根 实 部 分 别 为 负 数 与 正 数 , 图 此 ,B 一 [(z,
C 2 沥
刑 技 水

u 林
E

u 仁
[
e
E

利 用 待 定 系 数 法 , 可 逐 项 计 算 p (z2.

例 3. 12 考 虑 二 维 方 程

血`
E 汀
E

月
经 过 扰 动 在 奇 点 (0,0) 附 近 可 能 发 生 的 分 岔 现 象 , 其 中 8 么 0 注

意(3′17)在(0,0)点的线性部分矩阵为〔彗 熹〕y pudseatuu
Page-91
第 一 章 药 本 李 伊 和 淅 备 知 订

线 性 情 形
先 考 虑 线 性 方 程
蕃 E 医

玟

5 [
的 性 态 完 全 被 矩 阵 4 的 特 征 值 的 性 质 所 决 定 .

E p
E 育 7 河
E
E 吴 F 刑
0

E

记 口 为 R 中 相 应 于 2E c 的 那 些 特 征 值 的 广 义 特 征 向 量 所 张 成
E
E 沥 八 达 述

邵 相 应 的 投 影
动 t 酥 一 , i 职 一 酥 t 联 一 阮
E
ker(m) 二 E′“鼻E翻 E
ker(xu) 二 丘v…鼻Ec 申 吴 ,
E E^矗鼻E翼 日 .
E 仪

医
的 , 而 在 B 中 则 是 指 数 型 “ 增 长 “ 的 (t 一 一 eo 时 的 情 况 相 反 . 所
Page-92
述 E

国 我 们 首 先 设 法 找 出 方
E 巳p 中 水
E 2 理2 振 3 招 5

令

圆 目
[ 沥
盯一一岸她+州 音(″+讪v,

【 、 《C3.18)
茁 E 萨(鹰 吴 小 霄(″ E
Eosothe tn 技 5
E .
“ E 沥 水 0 河 逊
E
E

E 痨“z a

E 的第一个方程 u

导 的 方 程 为
豇一_卑″z+ 目 世 0.

E 沥 一 伟
E atpt b 河 沥
E 诊 L 一 广

E [

E
E 吊 E 水 江 2
Page-93
E E

有 对 f ~ 士 c 有 界 的 执 道 ( 特 别 地 , 所 有 契 点 , 闭 轨 ) 都 停 国 在 砂
医 s 技 不 园 2
E s u
E 渡 c
E

非 线 性 情 形
E
需=Az十八z兆 【

E 技 0
程 (3. 5 的 转 道 结 构 是 否 仍 然 具 有 方 程 (3. 1) 的 上 述 规 律 ? 下 面
E d 沥 i
0 浩 g 途 途 木

常 实 用 的 还 是 在 契 点 z 一 0 的 局 部 , 那 些 “ 复 杂 现 象 “( 特 别 地 , 所
有 奇 点 . 闭 转 . 同 宿 转 、 异 宪 执 等 ) 都 发 生 在 W 上 ; 在 一 定 条 件 下 ,
国 O 述 沥
过 对 E 上 诱 导 的 方 程 的 研 究 而 得 到 . 本 节 的 内 容 主 要 参 考 了 [V]
E

我 们 先 陈 述 整 佛 的 结 果 .

不 伟 沥 园 育 2 0 技 人

[
E
林
一 一 一
E
C
0
Page-94
E 第 一 章 基 本 梓 伊 和 淅 吊 知 识

E 林202 月
医 东 技 国
E

E 园 王
E 团

门 吴 0
E

i
0

附 注 3.14 “ 为 了 研 究 (3. 19) 的 中 心 洪 形 , 我 们 把 x 也 视 作
[

巳 0 国
沥 (3.21》

玟
而 中 心 子 空 间 为 B X Re. 故 (3. 21) 的 局 部 稳 定 流 形 与 不 稳 定 流
形 印 与 WP* 的 结 构 与 x 三 0 时 类 似 . 此 时 中 心 流 形 印 一 {(zr, 当
水

砂 . 由 (3. 21) 的 第 二 个 方 程 易 知 ,(Cz,471x 一 常 数 } 是 (3. 21)
的 不 变 集 , 从 而 对 固 定 的 x,W<| 心 希 数 是 (3. 21) 第 一 个 方 程 的 不
E 浩 00 圭
对 于 不 同 的 sW“ |,e- 帕 数 上 的 辐 道 结 构 可 能 不 同 , 见 下 例 -

例 3. 15 “ 设 z,y,x E R, 考 虑 光 滢 系 统

振 一

[ 蟒 一 厂 一 口 +JCaoyot,

【 。
雹言 E 技 林班35江 秀

E
E
Page-95
第 一 章 荣 本 林伊 和 醉 吉 知 订

E 噩副嘲…0解‖ d 医 力

E 2
E

E 吴 c 居2 E 述
[
E 述 2 园 医
是 (3.5 的 不 变 集 , 则 M 一 印 , 且 一 一 9 !

[ 林 述
箐=^跷+骗八跷十俨〈蹄)), E 沥 E 达

定 义 3.3 “ 定 理 3. 2 中 的 不 变 集 卫 * 称 为 (3.5) 的 全 局 中 心 流
E

[ 诊 玟 明 述 怀 述 sss
指 出 , 眙 Pr 在 (3.5) 的 流 下 不 变 , 则 p E C8CB,) 是 唯 一 确 定
E 唐 u 沥

E
很 难 应 用 . 由 于 0) 一 0,DfK(0) = 0 故 在 奇 库 < 一 0 附 近 这 个 条
件 却 是 自 然 成 立 的 , 因 此 , 利 用 截 断 (cut-oftt 函 数 从 定 理 3. 2 得 出
i
E

1, 当 肥 奶 1,

k {0y E
E

E /(z)刁[ 熹〕 E (3.9)

此 时 , 为 了 研 究 方 程 (3. 5) 在 x 一 0 附 近 的 中 心 流 形 , 我 们 可 以 考
乐 方 程

塔 <4z+ foCam, [
Page-96
E 人
3.14 和 例 2. 6 可 知 , 在 (z,y,4) 一 (0,0,0) 的 小 邻 域 内 , 中 心 流 形
e 沥 一 诊
E 一
E

c
E 沥
Elaeiopk 5 丶 吴 纳 才 a

在 下 文 对 所 都 分 岖 的 讨 论 中 , 我 们 大 都 假 定 已 经 把 闭 题 化 归

到 它 的 中 心 流 形 上 , 即 对 所 论 方 程 的 线 性 部 分 短 阵 A 一 言 ,o(40
E ′

$4 正 规 形

E
Eoposeeosueuetiaiog g g 月 加 芸 s 沥 余
尽 可 能 简 单 的 形 式 , 以 便 于 研 究 . 这 是 源 于 Poincarg 时 代 的 一 个 课
Page-97
E

E 沥 25 李 训 0 河
u y C3. 10)
E 东 沥 明 不 一 圆 育 2标 吴
ACo 二 0,D/(0) =0, 剧 3 pE CKCR,Bh) 和 z 一 0 在 R 中 的 开
E
CD 流 形
E 52 着 育 2
对 (3, 5) 的 流 局 部 不 变 , 印
E 仪 一 A 人 3
E 沥 c 林g 5 加
E 朋
【C2E X 河A 月
[ 沥 招 川
E 国 育 林EE i 日 技 振 才 f 技 川 [
E 朋
沥
E y
男 一 方 面 , 由 (3. 9) 可 知 , 英 故 - (z E R 国 i 川
E 技 0

E 述 人 育 沥
现 设 zE U,Ju(z) 一 R. 则 Y xE R,2(tyz) 一 t,a) 口 ,

从而着彗…I″^…〈hz)| 一 co, 由 (3.6) 式 ,z E 佛 r, 限 制 在 乙 内 , 也 就
E 宏 吴
匹 出 东 河 ] 如果婢仨C^(E(,互*)浓蓼L渊m E 兴 X00 河 玟
A 林目 园 技 孙 芸 不
国 42
[ 王 a 沥n 挂 u 口
Page-98
32 E 招

莲
新 引 起 人 们 对 它 的 重 视 , 并 得 出 若 干 计 算 正 规 形 的 新 方 法 .
众 所 周 知 , 经 非 退 化 线 性 变 换 z 一 7 3, 线 性 微 分 方 程

气亩7C=^薹
变 换 为

蛊翼 E 林

54 沥 沥 芒 春 沥 坂 步 招 a 命 口 P
线 性 系 统 的 辅 道 结 构 时 , 我 们 无 妙 假 设 4 为 Jordan 标 准 形 , 除 了
E
对 非 线 性 部 分 是 否 可 以 作 类 似 的 简 化 ? 在 一 定 意 义 下 , 答 案 是 肯 定
e
a 沥

微 分 方 程 在 奇 点 附 近 的 正 规 形

考 感 以 乙 一 0 为 奇 点 的 C 微 分 方 程 Cr 之 3 它 在 < 一 0 附 近

可 表 示 为

蔫 E 河 玟 伟 02 沥 沥 2 人

E 沥
维 & 次 半 次 向 量 多 项 式 所 成 的 空 间 心 一 2,...,7 一 1.
先 进 行 变 摄
E 关 志 玟 5 0

E 2 怀 ag
换 64 2) 代 入 (4 1, 并 注 意

[ c
阮
O(ly1), 而 (ly|5 表 示 X % 短 阵 , 它 的 每 一 元 素 都 是 O(110,
由 此 把 方 程 (4 1 化 为
Page-99
84 正 规 形 33

许

的 李 玟 4 志 人 北 小 6 国
其 中 户 与 (4. 1) 中 的 相 同 , 而 共 是 经 过 运 算 得 到 的 新 的 次 齐 次
多 项 式
引 入 算 于 ad: 技 一 ,
E dn
E

沥

…+弄叶嫩)+O‖川′), 0
记 毋 ? 为 算 子 at& 在 H 中 的 值 域 , 而 罗 “ 是 命 * 在 H8 中 的 一 个 补
唐

E 伟

E 林 3 朐 ]
n

E 5 发 扬 [
这 祥 , 我 们 把 (4. 1 化 为

林林 伟

0
E
E
E 河 或 技 0
E 恩沥 浩 规 i 江 a 育 t d

E

蔫 E 李 沥 汀 林 述 0

E 园 8
Page-100
E E

同 的 局 部 中 心 流 形 ( 尽 管 对 每 一 个 截 断 函 数 而 言 ,(3. 10 的 全 局
中 心 流 形 是 唯 一 的 . 例 如 , 在 图 !-6(x 一 0) 中 的 原 点 附 近 , 取 有
0
i
持 在 U 内 的 任 何 有 界 执 道 ( 包 括 奇 点 . 周 期 转 、 同 宿 转 , 异 宿 轨 等 )
都 出 现 在 (3. 5) 的 任 一 局 部 中 心 流 形 上 . 因 此 , 对 于 研 究 分 岔 现 象
00 沥
蚺 然 了 的 C 光 滢 性 保 证 了 吐 的 C4 光 滑 性 , 一 舫 来 说 , 了 的 C“ 光
玟
i
IDAal 一 8# 一 舫 来 说 , 当 & 一 eo 时 ,5x 一 0, 这 可 能 导 致 p 一 0

现 在 我 们 把 局 部 中 心 流 形 、 稳 定 、 不 稳 定 流 形 的 结 果 合 写 成 下
面 的 定 理 .

E 王 5 东 沥 英 育 5222 技
DAC0) 一 0; 相 对 于 4, 有 如 上 所 述 的 子 空 间 史 ,BY 和 Y, 则 在 R“
中 乙 一 0 附 近 存 在 开 邻 域 [, 和 口 中 的 Ct 流 形 WPr,Wv 和 伟 , 它 们
的 维 数 分 别 与 这 三 个 子 空 间 相 同 , 在 z = 0 点 分 别 与 吴 , 砂 和 砂
相 切 , 并 且 在 口 内 是 方 程 (3. 5 的 不 变 流 形 iW7“ 和 P“ 有 定 义 1. 12
野
E 命 巳 (

[
E 沥 e 洁
到 , 系 统 在 契 点 附 近 的 “ 复 杂 现 象 “ 发 生 在 它 的 任 一 局 部 中 心 流 形
上 . 下 面 的 两 个 定 理 说 月 了 中 心 流 形 的 其 它 重 要 作 用 、

定 理 3.9 ( 渐 近 性 质 定 理 〉 设 了 E C(R“,R“),f(0) 一 0,
D
i
EA22S5ac2 明 l 伟 一 东 人
Page-101
0 E 河

其 中 g 户 与 (4. 7 中 的 相 同 , 而 i
洁

E p z 国 c
E 技 u P 梁 i
E6 芸
则 当 交 s) E 2 时 , 存 在 危 ( , 使 得 经 变 换 4 8 可 消 去 [4. 7) 中
的 三 次 项 x 否 败 , 只 能 找 到 ( E 8, 使 (4. 7》 变 为

E 林 林

E d

E 吊 0 沥 河 2n 永

玟 二
一 《, 并 且 乃 有 表 达 式 (4. 1) , 则 在 原 点 附 近 的 邻 域 内 存 在 一 系 列
E

E 沥 连 42 兰n 【

其 中 伟 ( E f, 经 过 这 一 系 列 变 换 ( 每 次 变 挨 后 把 y 换 回 z) , 可
把 (4. 1 变 成 如 下 形 式

蔷 E 沥 沥 02 沥 2 芸
E 沥 林0 东 l
E 亚 江
E 2 技t 2
E
E 林 江
玟

E E
E 吴 育
Page-102
E

玖
E 东 述 l 国
0 s i 逊 t 浩 仪
E 5 沥
E
E 吴 育 沥 政
明 , 在 一 定 条 件 下 ,
E
纳
i
型 地 趋 于 中 心 流 形
上 的 某 一 解 ( 当 4
一 十 co 如 果 ou 一
【nice
E 砂
定 理 3. 10 说 明 , 在
类 似 条 件 下 , 为 了
得 到 局 部 中 心 流 形 上 0 点 附 近 的 执 道 结 构 , 只 需 要 对 它 在 线 性 子
E 水
0ooxudtu i 仁 e
E 述 2 河
E
E 东 0 月 0

E 吴 见 0 丞 A
| z 五 )} 是 (3. 5 的 一 个 局 部 中 心 流 形 , 当 且 仅 当 存 在 B 中 原

点 的 开 邰 域 0, 侩 得 V z E , 有
E 述
Page-103
$4 正 规 形 E

E 沥 国 沥 c sg t 江 a 振
成 收 敛 的 席 级 数 时 , 这 种 步 骤 原 则 上 可 以 无 限 地 进 行 下 去 , 闭 题 在
玟
绪 论 是 胤 定 的 , 这 就 是 Poincare-Dulac 定 理 , 见 [A1], 或 [CLW].

共 振 与 非 共 振

E 沥 林
0
巳 化 成 它 的 Jordan 标 准 形 , 并 引 入 共 振 的 概 念 .

E 东 园 2
E a e

4 达
吴 D
E

用 沥
丶 二 Cm, 加 一 一 DPmai- )
5

c

E 肉 唐 河 为 育
E 仪 a

考 察 (4. 11) 中 晤 些 gt(z) 不 出 现 , 就 是 要 考 察 同 伦 方 程 ˇ 一

E 2 73 一 3 沥 3 途 命 一

E

设 是 对 角 矩 阵 , 特 征 值 % 互 不 相 同 ,e 是 相 应 于 丶 的 特 征
D
E 仪

藁”粤_z护…z;"' d E
就 是 A 中 元 素 菪 一 分 量 中 的 最 简 形 式 .
医
s
Page-104
第 一 章 焦 本 概 伊 和 淅 备 知 识 -

E 河 [

E
利 用 (3. 14) 式 算 出 它 的 Taylor 展 式 的 前 儿 项 -
为 了 简 单 , 先 把 (3.5) 化 成 如 下 的 标 准 形 式 :

E

4 [
E

医 林 技 人 5
D
门 善 以 0]'
0 C

上 面 的 C 与 C: 的 特 征 根 实 部 分 别 为 负 数 与 正 数 , 图 此 ,B 一 [(z,
C 2 沥
刑 技 水

u 林
E

u 仁
[
e
E

利 用 待 定 系 数 法 , 可 逐 项 计 算 p (z2.

例 3. 12 考 虑 二 维 方 程

血`
E 汀
E

月
经 过 扰 动 在 奇 点 (0,0) 附 近 可 能 发 生 的 分 岔 现 象 , 其 中 8 么 0 注

意(3′17)在(0,0)点的线性部分矩阵为〔彗 熹〕y pudseatuu
Page-105
E

羞′4′~ [ 唐 巳 ′z_ E 盂"

E 发 人 沥
古 一 方 面 , 由 于 e 是 4 的 相 应 于 % 的 特 征 向 量 , 因 此
5 ^箕"馨禽 E 入jz_′】-
把 上 面 的 结 果 代 入 (4. 15》 的 左 端 , 得 到
E 吴 利 0
这 说 明 ai$ 也 是 对 角 的 , 并 且 它 的 特 征 值 具 有 [Cm,0 一 A] 的 形
式 . 由 此 可 知 , 当 A 的 特 征 值 非 共 振 时 ,adt 的 所 有 特 征 值 均 非 零 ,
E 怡 述
E a
ad 也 有 相 应 的 Jordan 坡 , 并 且 ad$ 的 特 征 值 仍 具 有 [Cmo 一 ]
的 形 式 .
定 义 4. 5 “ 向 量 值 多 项 式 z“e, 称 为 共 摄 多 项 式 , 如 果
0 圆 [ 不
E 述
技
林技 园 育 沥 技
换 (4, 10), 使 (4 11 有 端 的 诺 gi(z) 仅 由 共 振 多 项 式 组 成 , 【

E
[ 木

吴 国
佳 /
E 河 t

仪
Page-106
述 E

国 我 们 首 先 设 法 找 出 方
E 巳p 中 水
E 2 理2 振 3 招 5

令

圆 目
[ 沥
盯一一岸她+州 音(″+讪v,

【 、 《C3.18)
茁 E 萨(鹰 吴 小 霄(″ E
Eosothe tn 技 5
E .
“ E 沥 水 0 河 逊
E
E

E 痨“z a

E 的第一个方程 u

导 的 方 程 为
豇一_卑″z+ 目 世 0.

E 沥 一 伟
E atpt b 河 沥
E 诊 L 一 广

E [

E
E 吊 E 水 江 2
Page-107
设 a 在这组塞下的矩阵为 乙至,即
u

0
2
0
E
E
0 D
现 将 空 间 H 与 RI 等 同 ; 一 e, 其 中 e,..,ex 为 R 中 的 标 准
D
E 吊 d 一
E 绍 0 吴 圭 国 技 一
医 纳 2 沥

E

由 此 得 到

E
其 中 罗 心 Spantst, 6 十 25j, 由 此 得 出 二 次 正 规 形 为

人

或 化 成 等 价 形 式

d需=箕z十肋蔬,

菩罡 E 招 玟 兄 江

E
E 20
Page-108
E 葛
园 朐 园 E
M 李 H
E 沥 仪 d

沥

蔷 E 河 玟

E
d

[

砦 E 沥 公 2 述 人 2

2
E 圭n i
dzu

F 5

砦 E 育 不 玟

E 乐 芸
E 医 玟 应 5 en 命
E 达 g 应 不

空 间 后 , 正 规 形 中 的 系 数 就 唯 一 确 定 了 . 在 例 4 7 中 , 求 正 规 形 的

方 法 称 为 矩 阵 表 示 法 , 由 于 dimF 或 一 n ″十霹_l) 匹

E
E 庞
形 的 共 树 算 子 法 和 群 表 示 论 法 等 , 见 王 铐 的 综 述 文 章 [Wd] 及 其
所 引 的 文 献 .

例 4 9 考 虑 复 方 程

d 53
林 罚

0

[

E
Page-109
E 第 一 章 基 本 梓 伊 和 淅 吊 知 识

E 林202 月
医 东 技 国
E

E 园 王
E 团

门 吴 0
E

i
0

附 注 3.14 “ 为 了 研 究 (3. 19) 的 中 心 洪 形 , 我 们 把 x 也 视 作
[

巳 0 国
沥 (3.21》

玟
而 中 心 子 空 间 为 B X Re. 故 (3. 21) 的 局 部 稳 定 流 形 与 不 稳 定 流
形 印 与 WP* 的 结 构 与 x 三 0 时 类 似 . 此 时 中 心 流 形 印 一 {(zr, 当
水

砂 . 由 (3. 21) 的 第 二 个 方 程 易 知 ,(Cz,471x 一 常 数 } 是 (3. 21)
的 不 变 集 , 从 而 对 固 定 的 x,W<| 心 希 数 是 (3. 21) 第 一 个 方 程 的 不
E 浩 00 圭
对 于 不 同 的 sW“ |,e- 帕 数 上 的 辐 道 结 构 可 能 不 同 , 见 下 例 -

例 3. 15 “ 设 z,y,x E R, 考 虑 光 滢 系 统

振 一

[ 蟒 一 厂 一 口 +JCaoyot,

【 。
雹言 E 技 林班35江 秀

E
E
Page-110
E 人
3.14 和 例 2. 6 可 知 , 在 (z,y,4) 一 (0,0,0) 的 小 邻 域 内 , 中 心 流 形
e 沥 一 诊
E 一
E

c
E 沥
Elaeiopk 5 丶 吴 纳 才 a

在 下 文 对 所 都 分 岖 的 讨 论 中 , 我 们 大 都 假 定 已 经 把 闭 题 化 归

到 它 的 中 心 流 形 上 , 即 对 所 论 方 程 的 线 性 部 分 短 阵 A 一 言 ,o(40
E ′

$4 正 规 形

E
Eoposeeosueuetiaiog g g 月 加 芸 s 沥 余
尽 可 能 简 单 的 形 式 , 以 便 于 研 究 . 这 是 源 于 Poincarg 时 代 的 一 个 课
Page-111
其 中

E

四

0
才

〖 g
即 A 有 一 对 共 软 纸 虚 特 征 根 , 求 它 的 ( 形 式 正 规 形
解 “ 我 们 用 共 振 原 理 求 解 - 记 一 讨 ,b 一 一 ia, 则 共 振 条 件
[
E 育 小
E 规
由 定 理 4. 6 可 知 , 复 正 规 形 为

羞=i跑z十唰z尸z十-"+…z…z十… [
E

ag

考 意 以 z 一 0 为 不 助 点 的 C 眺 射 F 丿 3), 它 在 z 一 0 附 近
可 表 示 为
E 玟 仪 0 河 沥 2 胡 述 仁 人
其 中 < E R“( 或 CD,4 是 线 性 映 射 ( 我 们 把 它 在 某 组 基 下 的 矩 阵
E 林 沥
E
E
E 沥 刑 [
E 吴 林述 d 沥 a 刑 川 [ 颖 丿 5 月 诊 3
逆 变 换
3 圆
令
E 沥 李
则 可 把 (4. 23) 化 为
Page-112
E E

E E 林 2
[ 命 园 许
兰
E 园 一 育
(4 23), 则 在 原 点 附 近 的 邻 域 内 存 在 一 系 列 变 换
E 水 0 刑 一 [
其 中 任 ( E f 砺 , 经 过 这 一 系 列 变 换 ( 每 次 变 换 后 把 换 回 z), 可
把 (4. 23) 变 成 如 下 形 式
E 沥 沥 0 河 2 沥 6
玟
E
5 邹 c 沥 育 仪 7
E 0
E 园 沥 0 园
E
E 坂 国 沥 5 才 一
E 二 一 吴

之 0, 展′″展祟 B
l

0 ″鼻净…净. 0
正 数 |m| 称 作 共 振 的 阶 .

E 怀 c
E 林
E 志 s d 河
EEuuioosyt

深 220 步 武 s 林述
Page-113
32 E 招

莲
新 引 起 人 们 对 它 的 重 视 , 并 得 出 若 干 计 算 正 规 形 的 新 方 法 .
众 所 周 知 , 经 非 退 化 线 性 变 换 z 一 7 3, 线 性 微 分 方 程

气亩7C=^薹
变 换 为

蛊翼 E 林

54 沥 沥 芒 春 沥 坂 步 招 a 命 口 P
线 性 系 统 的 辅 道 结 构 时 , 我 们 无 妙 假 设 4 为 Jordan 标 准 形 , 除 了
E
对 非 线 性 部 分 是 否 可 以 作 类 似 的 简 化 ? 在 一 定 意 义 下 , 答 案 是 肯 定
e
a 沥

微 分 方 程 在 奇 点 附 近 的 正 规 形

考 感 以 乙 一 0 为 奇 点 的 C 微 分 方 程 Cr 之 3 它 在 < 一 0 附 近

可 表 示 为

蔫 E 河 玟 伟 02 沥 沥 2 人

E 沥
维 & 次 半 次 向 量 多 项 式 所 成 的 空 间 心 一 2,...,7 一 1.
先 进 行 变 摄
E 关 志 玟 5 0

E 2 怀 ag
换 64 2) 代 入 (4 1, 并 注 意

[ c
阮
O(ly1), 而 (ly|5 表 示 X % 短 阵 , 它 的 每 一 元 素 都 是 O(110,
由 此 把 方 程 (4 1 化 为
Page-114
日 4 正 规 形 朱

E 东 沥
E
E 一 园 一 圆 肖 水 2标 23 述 n
刊 六 次 正 规 形
t 标
E 月 园 二 一 切

E
[ 吴
E
【 2 庞 河 c 浩
其 中 an6 为 常 数

光 滑 线 性 化

E 不 伟河 许 不 伟 2
胚 ) 的 双 曲 奇 点 ( 或 双 曲 不 动 点 ) 为 万 阶 非 共 振 的 , 如 果 它 的 特 征
根 不 满 足 所 有 二 & 阶 的 共 振 关 系 , 如 果 一 个 奇 点 ( 或 不 动 点 ) 是 任
意 有 限 阶 非 共 振 的 , 则 称 它 为 无 穷 阶 非 共 振 的 , 或 简 称 非 共 振

从 前 面 的 讨 论 可 以 看 出 , 一 个 & 阶 非 共 搬 奇 点 ( 或 不 动 点 ) 的 &
次 正 规 形 是 线 性 的 , 换 句 语 说 , 在 奇 点 ( 或 不 动 点 的 邻 域 里 可 以
招 到 一 个 多 项 式 的 坐 标 变 换 , 使 得 在 新 坐 标 系 下 系 统 可 以 表 示 为
e 不
通 过 什 么 样 的 坂 标 变 换 能 把 这 个 & 阶 小 量 去 掉 -

E 余
坤 场 ( 或 徽 分 同 胚 ) 在 它 的 奇 点 ( 或 不 动 点 ) O 处 可 以 C# 线 性 化 , 如
果 存 在 点 0 的 邻 域 口 和 Ct 徽 分 同 胚 一 Rr,H(O) 一 , 使 得
E

定 理 4.18 [[IY]) 设 是 一 个 自 然 数 或 二 co,4 是 一 个 n
E
Page-115
84 正 规 形 33

许

的 李 玟 4 志 人 北 小 6 国
其 中 户 与 (4. 1) 中 的 相 同 , 而 共 是 经 过 运 算 得 到 的 新 的 次 齐 次
多 项 式
引 入 算 于 ad: 技 一 ,
E dn
E

沥

…+弄叶嫩)+O‖川′), 0
记 毋 ? 为 算 子 at& 在 H 中 的 值 域 , 而 罗 “ 是 命 * 在 H8 中 的 一 个 补
唐

E 伟

E 林 3 朐 ]
n

E 5 发 扬 [
这 祥 , 我 们 把 (4. 1 化 为

林林 伟

0
E
E
E 河 或 技 0
E 恩沥 浩 规 i 江 a 育 t d

E

蔫 E 李 沥 汀 林 述 0

E 园 8
Page-116
0 E 河

其 中 g 户 与 (4. 7 中 的 相 同 , 而 i
洁

E p z 国 c
E 技 u P 梁 i
E6 芸
则 当 交 s) E 2 时 , 存 在 危 ( , 使 得 经 变 换 4 8 可 消 去 [4. 7) 中
的 三 次 项 x 否 败 , 只 能 找 到 ( E 8, 使 (4. 7》 变 为

E 林 林

E d

E 吊 0 沥 河 2n 永

玟 二
一 《, 并 且 乃 有 表 达 式 (4. 1) , 则 在 原 点 附 近 的 邻 域 内 存 在 一 系 列
E

E 沥 连 42 兰n 【

其 中 伟 ( E f, 经 过 这 一 系 列 变 换 ( 每 次 变 挨 后 把 y 换 回 z) , 可
把 (4. 1 变 成 如 下 形 式

蔷 E 沥 沥 02 沥 2 芸
E 沥 林0 东 l
E 亚 江
E 2 技t 2
E
E 林 江
玟

E E
E 吴 育
Page-117
E

伟 , 当蜱<瞻`
E E
使 得 如 果 原 点 是 C“ 微 分 方 程

盖=^罩十… E 吴 0

【

或 微 分 同 胚

E a 园 育 孙 2 2
的 心 阶 非 共 振 双 曲 奇 点 ( 或 非 共 振 双 曲 不 动 点 ), 则 系 统 (4.
E 吴
[ 此处没有
E
E 一

羞=咖十…` E

或 微 分 同 朊
E 沥 招 、

因 为 待 征 根 2 一 a( 或 2 一 4 不 满 足 任 意 阶 共 振 关 系 , 故 由 定
2
微 分 同 胚 ) 在 它 们 的 双 曲 奇 点 ( 或 双 曲 不 动 点 ) 处 可 以 C“ 线 性 化

例 4. 21 “ 由 于 上 的 向 量 场 在 双 曲 焦 点 的 特 征 根 为 2 土 io,
2
它 的 双 曲 焦 炉 处 可 以 C“ 线 性 化

在 考 虑 分 岔 问 顶 时 , 我 们 常 常 只 需 要 C“ 线 性 化 . 对 此 , 有 下 面
E 河

E 育 E 述 0
胚 ) 吊 的 双 曲 奇 点 ( 或 双 曲 不 动 点 ) 如 果 口 在 点 0 的 线 伯 部 分 算
i 命 沥

E 一 月 园 一 月

车 健 江 一 t
Page-118
$4 正 规 形 E

E 沥 国 沥 c sg t 江 a 振
成 收 敛 的 席 级 数 时 , 这 种 步 骤 原 则 上 可 以 无 限 地 进 行 下 去 , 闭 题 在
玟
绪 论 是 胤 定 的 , 这 就 是 Poincare-Dulac 定 理 , 见 [A1], 或 [CLW].

共 振 与 非 共 振

E 沥 林
0
巳 化 成 它 的 Jordan 标 准 形 , 并 引 入 共 振 的 概 念 .

E 东 园 2
E a e

4 达
吴 D
E

用 沥
丶 二 Cm, 加 一 一 DPmai- )
5

c

E 肉 唐 河 为 育
E 仪 a

考 察 (4. 11) 中 晤 些 gt(z) 不 出 现 , 就 是 要 考 察 同 伦 方 程 ˇ 一

E 2 73 一 3 沥 3 途 命 一

E

设 是 对 角 矩 阵 , 特 征 值 % 互 不 相 同 ,e 是 相 应 于 丶 的 特 征
D
E 仪

藁”粤_z护…z;"' d E
就 是 A 中 元 素 菪 一 分 量 中 的 最 简 形 式 .
医
s
Page-119
E 克 8

园

给 出 .
例 4 23 “ 平 面 上 的 C“ 光 滢 向 量 场 ( 或 徽 分 同 胚 ) 在 它 的 双 曲
e

E 一 a
根 2 土 泗 和 一 个 实 根 A 满 足 w 丿 0,A 一 0 则 称 奇 点 为 鞍 焦 点 . 鞍
焦 点 的 特 征 根 显 然 满 足 (4. 303. 故 R 中 C“ 向 量 场 在 它 的 鞍 焦 点
处 可 以 C 线 性 化

E 东
【

【

则 它 在 该 点 可 以 C! 线 性 化

在 第 五 章 讨 论 非 局 部 分 岔 时 , 迹 到 的 向 量 场 都 是 依 赖 于 参 数
的 . 因 此 , 下 面 我 们 讨 论 带 参 数 的 向 量 场 或 晔 射 的 线 性 化 问 题

E 标 邦 沥 2 江 d 训 s
圆
i
以 C 线 性 化 , 如 果 存 在 参 数 空 间 中 e 一 e 的 邺 域 77 相 空 间 R“ 中
点 0 的 邻 域 口 , 以 及 一 个 C4 晃 射 上 ; U X 耳 一 R 源 趸

0 达

(2) 对 每 一 个 参 数 5 E , 一 (.,: 口 一 R“ 是 一 个 徽 分 同 胚 ,
使 得 通 过 依 赖 于 参 数 e 的 坐 标 变 换 z ++ 上 (z,s 后 , 系 统 在 0 点 的
邻 域 变 成 一 个 线 性 系 统

E 林政 04 浩 圆 d
医 ) 族 , 且 s 一 日 时 点 0 是 系 统 又 s 的 非 共 振 双 曲 奇 点 ( 或 双 曲 不 动

5s
[ 沥 玟 技 江
t
Page-120
E

羞′4′~ [ 唐 巳 ′z_ E 盂"

E 发 人 沥
古 一 方 面 , 由 于 e 是 4 的 相 应 于 % 的 特 征 向 量 , 因 此
5 ^箕"馨禽 E 入jz_′】-
把 上 面 的 结 果 代 入 (4. 15》 的 左 端 , 得 到
E 吴 利 0
这 说 明 ai$ 也 是 对 角 的 , 并 且 它 的 特 征 值 具 有 [Cm,0 一 A] 的 形
式 . 由 此 可 知 , 当 A 的 特 征 值 非 共 振 时 ,adt 的 所 有 特 征 值 均 非 零 ,
E 怡 述
E a
ad 也 有 相 应 的 Jordan 坡 , 并 且 ad$ 的 特 征 值 仍 具 有 [Cmo 一 ]
的 形 式 .
定 义 4. 5 “ 向 量 值 多 项 式 z“e, 称 为 共 摄 多 项 式 , 如 果
0 圆 [ 不
E 述
技
林技 园 育 沥 技
换 (4, 10), 使 (4 11 有 端 的 诺 gi(z) 仅 由 共 振 多 项 式 组 成 , 【

E
[ 木

吴 国
佳 /
E 河 t

仪
Page-121
E E 摄 c

E 圭 i 政
本 小 节 给 出 的 定 理 是 我 们 在 第 四 章 和 第 五 章 中 讨 论 问 题 的 基
E

E 河 河 水 t

E 振沙
接 触 分 岔 理 论 的 读 者 可 以 晓 过 本 节 的 内 容 , 只 需 承 认 定 理 5. 13 的
结 果 , 而 不 影 响 对 随 后 章 节 的 学 习 . 、

普 适 开 折 的 定 义

E 伟 d
E 李 a
E 东 e d

E 玟 圭 t 沥 e
E 河 河 ( 吊 一
示 . 在 考 慈 局 部 间 题 时 , 利 用 苓 的 说 法 可 使 陈 述 简 明 , 附 录 C 中 定
dospetdl d 林 t 2
给 出 .

现 在 考 虑 向 量 场 族 叉 E < “CM). 在 局 部 情 形 下 , 无 妨 设
E t t 生 二 ; e
L 述 达

盖′ E 八

园 t 林命 y 胡 i
E

东 伟 伟 明 c 吊
为 从 参 数 空 间 E RK 在 原 点 的 小 邻 域 到 向 量 场 空 间 的 春 射 时 , 我
Ebakeotoooke p u 沥 s
Page-122
设 a 在这组塞下的矩阵为 乙至,即
u

0
2
0
E
E
0 D
现 将 空 间 H 与 RI 等 同 ; 一 e, 其 中 e,..,ex 为 R 中 的 标 准
D
E 吊 d 一
E 绍 0 吴 圭 国 技 一
医 纳 2 沥

E

由 此 得 到

E
其 中 罗 心 Spantst, 6 十 25j, 由 此 得 出 二 次 正 规 形 为

人

或 化 成 等 价 形 式

d需=箕z十肋蔬,

菩罡 E 招 玟 兄 江

E
E 20
Page-123
E 葛
园 朐 园 E
M 李 H
E 沥 仪 d

沥

蔷 E 河 玟

E
d

[

砦 E 沥 公 2 述 人 2

2
E 圭n i
dzu

F 5

砦 E 育 不 玟

E 乐 芸
E 医 玟 应 5 en 命
E 达 g 应 不

空 间 后 , 正 规 形 中 的 系 数 就 唯 一 确 定 了 . 在 例 4 7 中 , 求 正 规 形 的

方 法 称 为 矩 阵 表 示 法 , 由 于 dimF 或 一 n ″十霹_l) 匹

E
E 庞
形 的 共 树 算 子 法 和 群 表 示 论 法 等 , 见 王 铐 的 综 述 文 章 [Wd] 及 其
所 引 的 文 献 .

例 4 9 考 虑 复 方 程

d 53
林 罚

0

[

E
Page-124
不 5 萍 适 扎 抚 与 切 盅 的 余 绮 45

E
道 2 一 技 e

E 凶 孝 坂 t河
E
E
Eaectel ll

E 东 全 技 腾
户 ) 导 出 的 , 如 果 存 在 连 续 晔 射 g , 一 p (e), 在 e 的 映 射 芽 , 使 得
E 明

E 穿 坂 明 刑 5 振 l
t
何 一 个 包 告 的 局 部 族 都 与 (ozzo,jo) 的 一 个 导 出 族 等 价 ,

附 注 5.7 “ 注 意 两 个 向 量 场 族 的 等 价 性 要 求 它 们 含 有 相 同 维
数 的 参 数 , 而 导 出 族 的 引 迹 使 得 同 一 个 退 化 向 量 场 的 普 适 开 折 可
以 含 有 不 同 维 数 的 参 数 , 从 而 可 以 迹 一 步 考 虑 含 参 数 最 少 的 普 适
开 折 .
. 附 注 5.8 定 义 5.4 一 5.6 都 取 自 Arnold 的 书 [Al,p267] 定
形 林
a
A 逊 续 , 则 称 这 种 等 价 为 强 等 价 , 并 可 得 到 强 等 价 意 义 下 的 普 适
E 江 余
国 江

E 王 坂 河 entseimtie d 河
量 场 ( 芽 )oo, 它 的 普 适 开 折 的 存 在 伯 并 不 是 明 品 的 , 只 有 在 周 密 的
讨 论 之 后 , 才 能 得 出 结 论 , 参 见 第 三 章 $ 1

读 者 可 以 用 本 节 的 观 点 重 新 考 察 $ 2 的 讨 论 , 在 那 里 利 用
Malgrange 定 理 证 明 了 , 奇 异 向 量 场 (2. 12) 任 耿 的 开 折 (2, 13) 都
2
焦 是 (2. 17) 的 一 个 导 出 族 , 按 定 义 5.6,(2. 17) 是 (2. 12) 的 一 个
Page-125
其 中

E

四

0
才

〖 g
即 A 有 一 对 共 软 纸 虚 特 征 根 , 求 它 的 ( 形 式 正 规 形
解 “ 我 们 用 共 振 原 理 求 解 - 记 一 讨 ,b 一 一 ia, 则 共 振 条 件
[
E 育 小
E 规
由 定 理 4. 6 可 知 , 复 正 规 形 为

羞=i跑z十唰z尸z十-"+…z…z十… [
E

ag

考 意 以 z 一 0 为 不 助 点 的 C 眺 射 F 丿 3), 它 在 z 一 0 附 近
可 表 示 为
E 玟 仪 0 河 沥 2 胡 述 仁 人
其 中 < E R“( 或 CD,4 是 线 性 映 射 ( 我 们 把 它 在 某 组 基 下 的 矩 阵
E 林 沥
E
E
E 沥 刑 [
E 吴 林述 d 沥 a 刑 川 [ 颖 丿 5 月 诊 3
逆 变 换
3 圆
令
E 沥 李
则 可 把 (4. 23) 化 为
Page-126
0 E 源 s
啧 透 开 折 、 从 这 里 可 以 看 出 导 出 族 的 作 用 .

分 岔 的 余 维 , 几 何 考 虑

我 们 现 在 对 向 量 场 局 部 族 的 分 岔 问 题 考 虑 它 的 余 维 , 在 全 体
向 量 场 所 成 的 空 间 .&“ 中 , 奇 异 ( 即 结 构 不 稳 定 ) 向 量 场 e 表 示 一
E 余
具 有 非 常 复 杂 的 结 构 . 但 如 果 限 于 考 虑 么 中 z 邻 近 的 点 , 它 具 有 与
zv“ 完 全 相 同 的 奇 异 性 “, 则 这 样 的 点 集 可 能 具 有 规 则 的 结 构 . 例
E 芸
E 医
E 胡应 肖 15
2 连 e
相 交 于 书 或 关 近 旁 的 a ( 见 图 1-17). 换 句 话 说 ,K 所 具 有 的 o 这

22
晋 然 也 能 在 8 中 的 z 点 与 心 相 交 , 但 在 小 扰 动 下 , 它 就 可 能 与 乃
E
0ttoeo t
Page-127
E E

E E 林 2
[ 命 园 许
兰
E 园 一 育
(4 23), 则 在 原 点 附 近 的 邻 域 内 存 在 一 系 列 变 换
E 水 0 刑 一 [
其 中 任 ( E f 砺 , 经 过 这 一 系 列 变 换 ( 每 次 变 换 后 把 换 回 z), 可
把 (4. 23) 变 成 如 下 形 式
E 沥 沥 0 河 2 沥 6
玟
E
5 邹 c 沥 育 仪 7
E 0
E 园 沥 0 园
E
E 坂 国 沥 5 才 一
E 二 一 吴

之 0, 展′″展祟 B
l

0 ″鼻净…净. 0
正 数 |m| 称 作 共 振 的 阶 .

E 怀 c
E 林
E 志 s d 河
EEuuioosyt

深 220 步 武 s 林述
Page-128
5 普 适 开 扯 与 分 吴 的 伟 维 5

Thom 机 截 定 理 ( 见 附 录 C 中 定 理 C. 15, 满 足 横 截 条 件 的 向 量 场
玟
一 放 为 一 个 通 有 (generie) 族 , 或 一 舫 族 . 因 此 , 可 以 粗 略 地 说 , 那
种 在 至 少 x 参 数 通 有 族 中 “ 不 可 去 “ 的 分 岔 现 象 是 余 维 * 的 . 当 然 ,
在 空 间 “ 中 的 奶 点 附 近 , 除 了 4 之 外 , 还 可 能 有 余 维 佗 于 A 的 奇
玟
玟
逃 择 参 数 族 K, 使 得 它 在 r 点 与 各 层 次 的 4 都 横 截 , 则 这 个 又
就 是 一 个 普 适 开 折 . 此 阡 , 如 果 把 定 义 开 折 K 的 晓 射 记 为 @yRt 一
幻 , 则 一 与 各 分 岔 曲 面 的 截 痕 在 8 下 的 象 , 就 形 成 了 参 数 空 间

Rt 中 的 分 岔 图 ( 见 图 1-183. 这 种 儿 何 的 考 虑 有 时 是 方 便 的 .

医 沥 林 中 林 不 育 沥 沥

考 虔 一 个 芽 , 限 制 在 奇 点 处 的 中 心 流 形 上 , 其 线 性 部 分
E

[

0

w

团
0

这 是 一 个 有 二 重 零 特 征 根 的 奇 异 向 量 场 . 我 们 关 心 的 阿 题 是 :
它 在 z 一 0 附 近 是 否 存 在 普 适 开 折 分 岔 的 余 维 是 多 少 ? 它 的 分 岔
图 如 何 ? 其 开 拙 的 拓 扑 结 构 有 骈 些 不 同 的 类 型 ? 它 们 怎 祥 随 参 数 的
变 动 从 一 种 类 型 变 成古 一 种 类 型 ? 这 些 问 题 的 解 冶 不 是 轻 而 易 举
b 标 d
玲
E 3 2标 成 河 当 ( 沥
字 , 借 以 介 绍 向 量 场 分 岔 的 一 些 基 本 理 论 与 方 法 本 节 主 要 研 究

]' 0 沥 沥 吊 0

E
Page-129
日 4 正 规 形 朱

E 东 沥
E
E 一 园 一 圆 肖 水 2标 23 述 n
刊 六 次 正 规 形
t 标
E 月 园 二 一 切

E
[ 吴
E
【 2 庞 河 c 浩
其 中 an6 为 常 数

光 滑 线 性 化

E 不 伟河 许 不 伟 2
胚 ) 的 双 曲 奇 点 ( 或 双 曲 不 动 点 ) 为 万 阶 非 共 振 的 , 如 果 它 的 特 征
根 不 满 足 所 有 二 & 阶 的 共 振 关 系 , 如 果 一 个 奇 点 ( 或 不 动 点 ) 是 任
意 有 限 阶 非 共 振 的 , 则 称 它 为 无 穷 阶 非 共 振 的 , 或 简 称 非 共 振

从 前 面 的 讨 论 可 以 看 出 , 一 个 & 阶 非 共 搬 奇 点 ( 或 不 动 点 ) 的 &
次 正 规 形 是 线 性 的 , 换 句 语 说 , 在 奇 点 ( 或 不 动 点 的 邻 域 里 可 以
招 到 一 个 多 项 式 的 坐 标 变 换 , 使 得 在 新 坐 标 系 下 系 统 可 以 表 示 为
e 不
通 过 什 么 样 的 坂 标 变 换 能 把 这 个 & 阶 小 量 去 掉 -

E 余
坤 场 ( 或 徽 分 同 胚 ) 在 它 的 奇 点 ( 或 不 动 点 ) O 处 可 以 C# 线 性 化 , 如
果 存 在 点 0 的 邻 域 口 和 Ct 徽 分 同 胚 一 Rr,H(O) 一 , 使 得
E

定 理 4.18 [[IY]) 设 是 一 个 自 然 数 或 二 co,4 是 一 个 n
E
Page-130
05 E

这 个 奇 异 问 量 场 的 余 维
E 仪 c
E 述
{ E 沥
为 了 对 它 的 奇 异 性 加 以 限 制 , 这 里 假 设 c5 万 0.
E 志 辽
E 不 传
医 招 技 余
医 2 i

败 (,o 可 以 表 示 为

盖 E 水 (5.2)

因 此 5. 1 可 以 篓 单 表 示 为 (O,z). 我 们 称 ($,o) 具 有 与 (0vz) 相
[ 述

E

E 2 沥 t

E

伟 /
E 怀 2 孝
现 在 可 以 把 与 (5. 1 有 相 同 奇 异 性 的 向 量 场 表 示 成 。

沥 如 江 c 林育 2 不 0
E 仪
E

[ 兵 王 技 u 水 一 莲 i 芸 c
c

量 场 . 这 符 合 流 形 上 向 量 场 的 一 舫 定 义 ( 见 附 录 B 中 附 注 B. 14,
由 于 在 R 中 的 钗 一 点 , 切 空 间 就 是 R 自 身 , 所 以 常 把 向 量 来 与 其
主 部 等 同 ( 见 附 录 B 中 附 注 B 173. 我 们 此 处 的 取 法 , 对 描 述 上 面
E 沥 标
Page-131
E

伟 , 当蜱<瞻`
E E
使 得 如 果 原 点 是 C“ 微 分 方 程

盖=^罩十… E 吴 0

【

或 微 分 同 胚

E a 园 育 孙 2 2
的 心 阶 非 共 振 双 曲 奇 点 ( 或 非 共 振 双 曲 不 动 点 ), 则 系 统 (4.
E 吴
[ 此处没有
E
E 一

羞=咖十…` E

或 微 分 同 朊
E 沥 招 、

因 为 待 征 根 2 一 a( 或 2 一 4 不 满 足 任 意 阶 共 振 关 系 , 故 由 定
2
微 分 同 胚 ) 在 它 们 的 双 曲 奇 点 ( 或 双 曲 不 动 点 ) 处 可 以 C“ 线 性 化

例 4. 21 “ 由 于 上 的 向 量 场 在 双 曲 焦 点 的 特 征 根 为 2 土 io,
2
它 的 双 曲 焦 炉 处 可 以 C“ 线 性 化

在 考 虑 分 岔 问 顶 时 , 我 们 常 常 只 需 要 C“ 线 性 化 . 对 此 , 有 下 面
E 河

E 育 E 述 0
胚 ) 吊 的 双 曲 奇 点 ( 或 双 曲 不 动 点 ) 如 果 口 在 点 0 的 线 伯 部 分 算
i 命 沥

E 一 月 园 一 月

车 健 江 一 t
Page-132
E 5

E 仪 技
数 空 间 的 乘 积 空 间 中 考 虑 ), 要 注 意 的 是 , 如 果 5 在 “ 中 构 成 余
园 志 应 不 述
t en a 江
E

0
子 流 形 . 记

E 人 沥
E 扬 八

2 着 李 5
E 心

p 途 [

E 怡 招 才 训
3
E c 一 3
的 Hesse 矩 阵 ( 它 是 6 维 的 .

E 东 圆 河 不 江 沥 河浩 才
E 述 标 5
子 流 形 .

E 圆 圆 八 育

E 兰 动 5 刑 仁 沥 林 招 t 仪 江 李
医 林22
E
E 林 兰 5 胡

国
27,x5*(mi3) 是 兰 中 的 光 滢 子 流 形 ; 再 由 定 理 C 16,x5「Cm5) 在 公
Page-133
E 克 8

园

给 出 .
例 4 23 “ 平 面 上 的 C“ 光 滢 向 量 场 ( 或 徽 分 同 胚 ) 在 它 的 双 曲
e

E 一 a
根 2 土 泗 和 一 个 实 根 A 满 足 w 丿 0,A 一 0 则 称 奇 点 为 鞍 焦 点 . 鞍
焦 点 的 特 征 根 显 然 满 足 (4. 303. 故 R 中 C“ 向 量 场 在 它 的 鞍 焦 点
处 可 以 C 线 性 化

E 东
【

【

则 它 在 该 点 可 以 C! 线 性 化

在 第 五 章 讨 论 非 局 部 分 岔 时 , 迹 到 的 向 量 场 都 是 依 赖 于 参 数
的 . 因 此 , 下 面 我 们 讨 论 带 参 数 的 向 量 场 或 晔 射 的 线 性 化 问 题

E 标 邦 沥 2 江 d 训 s
圆
i
以 C 线 性 化 , 如 果 存 在 参 数 空 间 中 e 一 e 的 邺 域 77 相 空 间 R“ 中
点 0 的 邻 域 口 , 以 及 一 个 C4 晃 射 上 ; U X 耳 一 R 源 趸

0 达

(2) 对 每 一 个 参 数 5 E , 一 (.,: 口 一 R“ 是 一 个 徽 分 同 胚 ,
使 得 通 过 依 赖 于 参 数 e 的 坐 标 变 换 z ++ 上 (z,s 后 , 系 统 在 0 点 的
邻 域 变 成 一 个 线 性 系 统

E 林政 04 浩 圆 d
医 ) 族 , 且 s 一 日 时 点 0 是 系 统 又 s 的 非 共 振 双 曲 奇 点 ( 或 双 曲 不 动

5s
[ 沥 玟 技 江
t
Page-134
E E 蒙 s

E 河定 诊 伟

E 技 5 伟

E 国 人 育 沥

E alutyyipy a 仪

[aposey e 述
7 玟

E 标s lspsiteis: 沥 命 5 命 |
现 在 设

0 沥 [ 沥

是 心 中 的 向 量 场 族 ( 参 见 附 注 5. 10) , 且 m 就 显 原 来 的 奇 异 向 量
E 芽 . 因 此 , 又 把 (5. 4) 称 为 (O,oo) 的 一 个
开 折 ; 与 它 相 应 的 微 分 方 程 是

羞 [ 河 人 [

E 出 人 沥 6 林 日P i 河 23 英

到 一 的 映 射
【 2

E 训
E

n

定 理 5. 13

02 扬

E 沥 50 技 技
E t 林 壮 才 才 沥 3
E
力 一 259Couo6),
E 水 怀 达 述 人 沥 林 八

E
Page-135
E E 摄 c

E 圭 i 政
本 小 节 给 出 的 定 理 是 我 们 在 第 四 章 和 第 五 章 中 讨 论 问 题 的 基
E

E 河 河 水 t

E 振沙
接 触 分 岔 理 论 的 读 者 可 以 晓 过 本 节 的 内 容 , 只 需 承 认 定 理 5. 13 的
结 果 , 而 不 影 响 对 随 后 章 节 的 学 习 . 、

普 适 开 折 的 定 义

E 伟 d
E 李 a
E 东 e d

E 玟 圭 t 沥 e
E 河 河 ( 吊 一
示 . 在 考 慈 局 部 间 题 时 , 利 用 苓 的 说 法 可 使 陈 述 简 明 , 附 录 C 中 定
dospetdl d 林 t 2
给 出 .

现 在 考 虑 向 量 场 族 叉 E < “CM). 在 局 部 情 形 下 , 无 妨 设
E t t 生 二 ; e
L 述 达

盖′ E 八

园 t 林命 y 胡 i
E

东 伟 伟 明 c 吊
为 从 参 数 空 间 E RK 在 原 点 的 小 邻 域 到 向 量 场 空 间 的 春 射 时 , 我
Ebakeotoooke p u 沥 s
Page-136
E

E

E 伟 0 口
【C3 扬

E 述

河

乏】 E
壬 E 述 河 水 3 朋
[ 九
玟
【 沥 发 0 罚 沥 1

E
{ E E
1 沥 动
其 中 @ 吴 滔 足 与 Q,@ 相 同 的 条 件 .

定 理 的 结 论 (1) 是 定 义 5. 12 和 定 理 C, 15(Thom 定 理 的 jet 形
d
E dn 林
[

0

E
1 吴
3 一 几 十 tay 十 丁 十 zy G.1

Eotuetrttet e 当 育

[ 沥
0 0
的 一 个 普 造 开 折 , 并 且 是 余 维 2 的 , .

在 作 第 ( 一 ) 步 讨 论 时 , 除 了 运 用 徽 分 方 程 定 性 理 论 的 知 识 和
n
Page-137
E s

E 河 U
题 .

定 理 5. 13 中 结 论 (2) 的 证 明

E 木
552 AC 2eidp 3 技 1
{ 萼=伽 0
E 庞 玟 不 2 河 玟
E

邦 E
E E E E
l 匹 页 匹 人 2 刑 匹 刀 2 CC0.0 刑

E el i 吊 王 招
项 系 数
E 国 园 e 河

E

{ E 班2

E 沥 玟 心 沥 应 玟 s2
E
林
E

E 沥

E

E
{ 一 一 口 【
E 技 02 庞 玟 人 53 命 命 林 3

其中云逼'耿仨C阊,并且
~ E E
E 刑 罡
2n 5 咤 0 59

|
央

利用上面的条件,可由隐函效方程己@醴)腥)=0碗定C阗函
Page-138
不 5 萍 适 扎 抚 与 切 盅 的 余 绮 45

E
道 2 一 技 e

E 凶 孝 坂 t河
E
E
Eaectel ll

E 东 全 技 腾
户 ) 导 出 的 , 如 果 存 在 连 续 晔 射 g , 一 p (e), 在 e 的 映 射 芽 , 使 得
E 明

E 穿 坂 明 刑 5 振 l
t
何 一 个 包 告 的 局 部 族 都 与 (ozzo,jo) 的 一 个 导 出 族 等 价 ,

附 注 5.7 “ 注 意 两 个 向 量 场 族 的 等 价 性 要 求 它 们 含 有 相 同 维
数 的 参 数 , 而 导 出 族 的 引 迹 使 得 同 一 个 退 化 向 量 场 的 普 适 开 折 可
以 含 有 不 同 维 数 的 参 数 , 从 而 可 以 迹 一 步 考 虑 含 参 数 最 少 的 普 适
开 折 .
. 附 注 5.8 定 义 5.4 一 5.6 都 取 自 Arnold 的 书 [Al,p267] 定
形 林
a
A 逊 续 , 则 称 这 种 等 价 为 强 等 价 , 并 可 得 到 强 等 价 意 义 下 的 普 适
E 江 余
国 江

E 王 坂 河 entseimtie d 河
量 场 ( 芽 )oo, 它 的 普 适 开 折 的 存 在 伯 并 不 是 明 品 的 , 只 有 在 周 密 的
讨 论 之 后 , 才 能 得 出 结 论 , 参 见 第 三 章 $ 1

读 者 可 以 用 本 节 的 观 点 重 新 考 察 $ 2 的 讨 论 , 在 那 里 利 用
Malgrange 定 理 证 明 了 , 奇 异 向 量 场 (2. 12) 任 耿 的 开 折 (2, 13) 都
2
焦 是 (2. 17) 的 一 个 导 出 族 , 按 定 义 5.6,(2. 17) 是 (2. 12) 的 一 个
Page-139
E

数 e 一 e(e) , 再 经 过 变 捣
E

把 方 程 (5. 13) 变 为 5. 11) 的 形 式 , 并 漾 尸 引 理 的 要 求 。 【

E 林 技 一 途 育 223 江2 技
i 沥 c

E 园 厉
E 林

E 志班 玟 仪

0
E

E
余 G , , 里 Gue
{抢 一 p + ea+ 明 + 吊 yos+ 黛 0 吊 圭 enoe.

2
E

列 (5.157 转 化 成

[ 怀 人 不
E {P(E〉+钩(e)″+″Z+

E
7 命 唐

国 此 ,(5. 16) 可 以 写 成

E

E 吴 2 述 2 沥 2 朋
E
Page-140
0 E 源 s
啧 透 开 折 、 从 这 里 可 以 看 出 导 出 族 的 作 用 .

分 岔 的 余 维 , 几 何 考 虑

我 们 现 在 对 向 量 场 局 部 族 的 分 岔 问 题 考 虑 它 的 余 维 , 在 全 体
向 量 场 所 成 的 空 间 .&“ 中 , 奇 异 ( 即 结 构 不 稳 定 ) 向 量 场 e 表 示 一
E 余
具 有 非 常 复 杂 的 结 构 . 但 如 果 限 于 考 虑 么 中 z 邻 近 的 点 , 它 具 有 与
zv“ 完 全 相 同 的 奇 异 性 “, 则 这 样 的 点 集 可 能 具 有 规 则 的 结 构 . 例
E 芸
E 医
E 胡应 肖 15
2 连 e
相 交 于 书 或 关 近 旁 的 a ( 见 图 1-17). 换 句 话 说 ,K 所 具 有 的 o 这

22
晋 然 也 能 在 8 中 的 z 点 与 心 相 交 , 但 在 小 扰 动 下 , 它 就 可 能 与 乃
E
0ttoeo t
Page-141
E 第 一 章 基札 概 伊 和 雍 备 知 订

E 河 p

d
E
下 面 再 迹 行 一 次 变 换 , 把 (5. 17) 第 二 式 中 的 (sJz 变 成 (so
E

则 (5. 17 化 奶
E zz歹 [25

E 吴 林李 心
1
E

E 吵(s),s),
E T游傍),

P 告帷川 E

C 音矽(鬓)挎〉 一 告哟(辜)芗〈藁‖e〉,

E 鲁砂〈E), E 吴
E0 703
应 许

把 ,9,,@ 和 雨 换 同 6,p,,Q 和 申 的 形 式 后 , 方 程 (5. 18) 成 为
[ n

附 注 5. 16 “ 显 然 , 局 部 族 (5. 6)( 在 (y,6) 一 (0,0) 附 近 的 一
i
定 义 5.4 中 的 映 射 9 与 A 都 可 取 为 相 应 空 间 中 的 恒 同 映 射 , 在 这 个
Page-142
5 普 适 开 扯 与 分 吴 的 伟 维 5

Thom 机 截 定 理 ( 见 附 录 C 中 定 理 C. 15, 满 足 横 截 条 件 的 向 量 场
玟
一 放 为 一 个 通 有 (generie) 族 , 或 一 舫 族 . 因 此 , 可 以 粗 略 地 说 , 那
种 在 至 少 x 参 数 通 有 族 中 “ 不 可 去 “ 的 分 岔 现 象 是 余 维 * 的 . 当 然 ,
在 空 间 “ 中 的 奶 点 附 近 , 除 了 4 之 外 , 还 可 能 有 余 维 佗 于 A 的 奇
玟
玟
逃 择 参 数 族 K, 使 得 它 在 r 点 与 各 层 次 的 4 都 横 截 , 则 这 个 又
就 是 一 个 普 适 开 折 . 此 阡 , 如 果 把 定 义 开 折 K 的 晓 射 记 为 @yRt 一
幻 , 则 一 与 各 分 岔 曲 面 的 截 痕 在 8 下 的 象 , 就 形 成 了 参 数 空 间

Rt 中 的 分 岔 图 ( 见 图 1-183. 这 种 儿 何 的 考 虑 有 时 是 方 便 的 .

医 沥 林 中 林 不 育 沥 沥

考 虔 一 个 芽 , 限 制 在 奇 点 处 的 中 心 流 形 上 , 其 线 性 部 分
E

[

0

w

团
0

这 是 一 个 有 二 重 零 特 征 根 的 奇 异 向 量 场 . 我 们 关 心 的 阿 题 是 :
它 在 z 一 0 附 近 是 否 存 在 普 适 开 折 分 岔 的 余 维 是 多 少 ? 它 的 分 岔
图 如 何 ? 其 开 拙 的 拓 扑 结 构 有 骈 些 不 同 的 类 型 ? 它 们 怎 祥 随 参 数 的
变 动 从 一 种 类 型 变 成古 一 种 类 型 ? 这 些 问 题 的 解 冶 不 是 轻 而 易 举
b 标 d
玲
E 3 2标 成 河 当 ( 沥
字 , 借 以 介 绍 向 量 场 分 岔 的 一 些 基 本 理 论 与 方 法 本 节 主 要 研 究

]' 0 沥 沥 吊 0

E
Page-143
E

t
E 2 述
E 河 王 ′
E
2CSv6J) |s-e -
工
E p 35 nn
E
E
[ [
E
E
E
[ 吴 3
E
陶 t ss e
E 沥 t e
E 河 林t
2 述
E
劝 一 方 面 , 在 厂 中 (0,oo 点 附 近 z:S 可 申 如 下 方 程 确 定 ( 见
条 件 CH,) 和 CH。)》
E c

E 吴 吴 国 莲 工
E 一 0 匕 52

国 t
价 于

a E

E

E
Page-144
E 源

0 【
邹 匹
3 E
河
医
53 E

E

|

引 理 5.18 “ 设 (5. 4) 是 非 退 化 的 开 折 , 则 存 在 C“ 变 换 一
ACe),p(0) 一 0, 它 在 5 一 0 附 近 非 退 化 , 并 把 (5. 8) 变 到 (5. 9.

0
式 给 出 。

肉 二 卯 ( 日 , 肌 二 日 , 皂 一 口 , 一 Eni

则 由 条 件 (5. 7 知 AC0) 一 0 又 由 于 (5. 8) 是 非 退 化 的 , 则 由 引 理
5. 17 可 知 , 一 xC6) 在 e 一 0 是 非 退 化 的 , 引 理 得 证 . 方 程 (5.9) 中
E E
E i 技 | ,

e

习 题 与 怡 考 题 一

1.1 考 虔 R 上 的 动 力 系 统 , 设 它 皑 辅 线 分 别 具 有 图 1-6, 图 1-8 或 图 1-9
的 7 种 分 布 , 对 铁 一 种 分 布 , 取 平 面 上 不 吾 区 域 的 点 z, 议 论 极 眼 集 atz) 和
z). 莲 研 究 系 统 的 汀 游 茹 集 D(9). 对 郧 一 种 分 布 , 你 可 以 断 言 系 统 不 显 结
构 穗 定 的 3

1.2 利 用 对 (2.13) 的 讨 论 方 法 , 证 明 R 上 的 系 绕 (2.2) 是 C“ 系 统

羞=_zz+0(恤忏)

0
匹 :
英

2

国

E

1 3 对 下 列 系 统 , 求 出 与 中 心 诚 形 相 应 的 诱 导 方 程 (3.8》, 并 由 此 作 出 源
玟

八
Page-145
05 E

这 个 奇 异 问 量 场 的 余 维
E 仪 c
E 述
{ E 沥
为 了 对 它 的 奇 异 性 加 以 限 制 , 这 里 假 设 c5 万 0.
E 志 辽
E 不 传
医 招 技 余
医 2 i

败 (,o 可 以 表 示 为

盖 E 水 (5.2)

因 此 5. 1 可 以 篓 单 表 示 为 (O,z). 我 们 称 ($,o) 具 有 与 (0vz) 相
[ 述

E

E 2 沥 t

E

伟 /
E 怀 2 孝
现 在 可 以 把 与 (5. 1 有 相 同 奇 异 性 的 向 量 场 表 示 成 。

沥 如 江 c 林育 2 不 0
E 仪
E

[ 兵 王 技 u 水 一 莲 i 芸 c
c

量 场 . 这 符 合 流 形 上 向 量 场 的 一 舫 定 义 ( 见 附 录 B 中 附 注 B. 14,
由 于 在 R 中 的 钗 一 点 , 切 空 间 就 是 R 自 身 , 所 以 常 把 向 量 来 与 其
主 部 等 同 ( 见 附 录 B 中 附 注 B 173. 我 们 此 处 的 取 法 , 对 描 述 上 面
E 沥 标
Page-146
E

E 伟
0
c

E 述 河 正 规 形

1.5 设 R 上 的 和 量 杨 以 (0,0) 为 帛 炭 , 其 线 伯 鄂 分 在 (0,0) 的 短 阵 具 有
二重零特征根'而且向量场在翼转角震誓下保持不变,其中疃为正整敷,并且
E t 林553 加

盖 E 为 河 水 河浩

其 中 vepA 为 复 数 ,m 一
E 达
1 6
0 医 北
[ 颖
玟
' 1.7 考 虑 C“ 眶 射 P 一 R 它 以 一 1 十 5 为 特 征 根 ,|e| 人 1. 证 明 对
给 定 的 正 个 数 A 存 在 8 > 0, 使 当 ls| 一 日 时 , 七 可 以 经 过 C“ 变 换 化 为
E 沥 辽 人 江 达
E 河
利 用 此 结 果 对 例 2. 11 的 结 论 给 出 简 单 的 证 明
n
E 林政 c 林 达
在 小 扰 动 下 的 分 盆 蜀 律 .
Page-147
E 5

E 仪 技
数 空 间 的 乘 积 空 间 中 考 虑 ), 要 注 意 的 是 , 如 果 5 在 “ 中 构 成 余
园 志 应 不 述
t en a 江
E

0
子 流 形 . 记

E 人 沥
E 扬 八

2 着 李 5
E 心

p 途 [

E 怡 招 才 训
3
E c 一 3
的 Hesse 矩 阵 ( 它 是 6 维 的 .

E 东 圆 河 不 江 沥 河浩 才
E 述 标 5
子 流 形 .

E 圆 圆 八 育

E 兰 动 5 刑 仁 沥 林 招 t 仪 江 李
医 林22
E
E 林 兰 5 胡

国
27,x5*(mi3) 是 兰 中 的 光 滢 子 流 形 ; 再 由 定 理 C 16,x5「Cm5) 在 公
Page-148
第 二 章 常 见 的 局 部 与 非 局 部 分 岔

E 东
岔 ,Hopf 分 岑 . 同 宿 分 岔 .Poincare 分 岔 等 , 其 中 前 三 种 为 局 部 分 岔
闭 题 , 后 两 种 分 别 为 半 局 部 分 岔 和 全 局 分 岔 问 题 , 除 了 奇 点 分 岔
外 , 本 章 的 大 部 分 讨 论 都 限 制 在 相 空 间 为 一 维 的 情 形 .

E

E
e
分 岔 现 象 称 为 奇 点 分 岖

一 舱 理 论

E 怀 园 s 木 5
点 的 线 性 郯 分 算 子 是 非 奇 异 的 , 即 它 的 所 有 特 征 根 均 非 零 . 否 则
u

E 东 育 参 数 的 向 量 场 , 如 果 它 的 奇
点 是 非 退 化 的 , 则 奇 点 本 身 也 光 滔 地 依 赖 于 参 数

证 明 ˇ 设 向 量 场 由 微 分 方 程

羞=矾婀… 许
巴 沥 2 一 诚 扬 史 刑
二 乙 为 (1. 1) 的 非 退 化 奇 点 , 卵 o(zoopa 一 0, 一 英 一

奇 异 . 由 隐 函 数 定 理 , 在 (zu, 附 近 存 在 光 潘 函 数 z 一 YC0 , 使
Page-149
E E 蒙 s

E 河定 诊 伟

E 技 5 伟

E 国 人 育 沥

E alutyyipy a 仪

[aposey e 述
7 玟

E 标s lspsiteis: 沥 命 5 命 |
现 在 设

0 沥 [ 沥

是 心 中 的 向 量 场 族 ( 参 见 附 注 5. 10) , 且 m 就 显 原 来 的 奇 异 向 量
E 芽 . 因 此 , 又 把 (5. 4) 称 为 (O,oo) 的 一 个
开 折 ; 与 它 相 应 的 微 分 方 程 是

羞 [ 河 人 [

E 出 人 沥 6 林 日P i 河 23 英

到 一 的 映 射
【 2

E 训
E

n

定 理 5. 13

02 扬

E 沥 50 技 技
E t 林 壮 才 才 沥 3
E
力 一 259Couo6),
E 水 怀 达 述 人 沥 林 八

E
Page-150
E u

【dCtoc l tdto g 沥 浩 |

E 玟 园 怀 园 东 不
数 的 微 小 变 化 下 不 变 , 它 的 位 置 也 光 滑 地 依 赖 于 参 数 的 变 化 . 需
要 注 意 , 奇 点 的 非 退 化 性 与 双 曲 性 是 不 同 的 概 念 . 例 如 , 一 个 们
E
是 非 退 化 的 , 但 它 是 非 双 曲 的 . 此 时 在 扰 动 下 , 蚀 然 奇 点 个 数 ( 在
小 邹 域 内 ) 不 发 生 变 化 , 但 其 附 近 的 扬 道 结 构 可 能 变 化 , 出 现 Hopf
E

E 江 刑
退 化 奇 点 ( 它 们 必 是 孤 立 奇 点 ) 或 无 奇 点 的 向 量 场 集 合 形 成 一 个
E

[ 国 圆 八 沥 标 亚 一 者 吊 622

E

E 河 述
讨 论 ; 考 虑 投 影
E 仪

其 中 了 - /C6). 具 有 契 炉 的 向 量 场 集 合 在 空 间 JoCM ,M) 中 有 表
E

E
d
/〔z)在辜点非退化,即噩 园 l

玟
C, 15, 得 知 仅 有 非 退 化 夹 点 或 无 奇 点 的 向 量 场 在 <g““CM ) 中 形 成
开 稽 子 集 ,

这 个 定 理 说 明 , 向 量 场 的 一 个 退 化 奇 点 可 以 经 过 任 意 小 的 扶
E 吊 皋 尿 应
Eouolok a i 江
Page-151
E

E

E 伟 0 口
【C3 扬

E 述

河

乏】 E
壬 E 述 河 水 3 朋
[ 九
玟
【 沥 发 0 罚 沥 1

E
{ E E
1 沥 动
其 中 @ 吴 滔 足 与 Q,@ 相 同 的 条 件 .

定 理 的 结 论 (1) 是 定 义 5. 12 和 定 理 C, 15(Thom 定 理 的 jet 形
d
E dn 林
[

0

E
1 吴
3 一 几 十 tay 十 丁 十 zy G.1

Eotuetrttet e 当 育

[ 沥
0 0
的 一 个 普 造 开 折 , 并 且 是 余 维 2 的 , .

在 作 第 ( 一 ) 步 讨 论 时 , 除 了 运 用 徽 分 方 程 定 性 理 论 的 知 识 和
n
Page-152
E E

stouyzyltisyogp 志 E
tuskts p 志 途 p
奇 点 . 对 一 个 具 体 的 奇 点 分 盆 问 题 , 通 常 有 两 种 处 理 方 法 : 一 种 是
利 用 中 心 流 形 定 理 , 把 问 题 归 结 到 中 心 流 形 上 , 见 第 一 章 例 3. 121 ˇ
8n 河 目 J 师 0 沥
method), 为 了 说 明 这 个 方 法 的 基 本 性 想 , 我 们 先 看 一 种 特 殊 情
E 沥 2 e 一

沥

其 中 《 的 特 征 根 均 为 零 , 而 旦 的 特 征 根 均 不 为 零 ! 人 g E Cr , 之
E
E 伟

E 玟 玲
i
a

Ecae n t 余

E

d
E 标 逊 6 E
0

一 {Cy 史 EDICGOoo 主 二 口 , 马 二 5 门 仁 二 加 ),
、

则 对 不 同 的 A,|2| 心 1,8, 结 构 的 变 化 反 晖 了 奇 点 个 数 的 变 化 规
E 河 2
闰 维 数 得 到 降 低 通 常 称 (1. 4) 为 方 程 (1. 2) 的 分 岑 函 数 .

为 了 应 用 上 的 便 利 , 下 面 在 更 一 舱 的 框 架 下 讨 论 这 个 间 题 .

Liapunov-Schmidt 方 法

设 又 ,Z 和 A 为 实 Banach 空 间 ,D 和 砌 分 别 为 又 和 A 中 零 点
[ 一 e 人
Page-153
E s

E 河 U
题 .

定 理 5. 13 中 结 论 (2) 的 证 明

E 木
552 AC 2eidp 3 技 1
{ 萼=伽 0
E 庞 玟 不 2 河 玟
E

邦 E
E E E E
l 匹 页 匹 人 2 刑 匹 刀 2 CC0.0 刑

E el i 吊 王 招
项 系 数
E 国 园 e 河

E

{ E 班2

E 沥 玟 心 沥 应 玟 s2
E
林
E

E 沥

E

E
{ 一 一 口 【
E 技 02 庞 玟 人 53 命 命 林 3

其中云逼'耿仨C阊,并且
~ E E
E 刑 罡
2n 5 咤 0 59

|
央

利用上面的条件,可由隐函效方程己@醴)腥)=0碗定C阗函
Page-154
E

数 e 一 e(e) , 再 经 过 变 捣
E

把 方 程 (5. 13) 变 为 5. 11) 的 形 式 , 并 漾 尸 引 理 的 要 求 。 【

E 林 技 一 途 育 223 江2 技
i 沥 c

E 园 厉
E 林

E 志班 玟 仪

0
E

E
余 G , , 里 Gue
{抢 一 p + ea+ 明 + 吊 yos+ 黛 0 吊 圭 enoe.

2
E

列 (5.157 转 化 成

[ 怀 人 不
E {P(E〉+钩(e)″+″Z+

E
7 命 唐

国 此 ,(5. 16) 可 以 写 成

E

E 吴 2 述 2 沥 2 朋
E
Page-155
园 国 u

E
胡 (zi0 一 0 0
t t 朋
0 i
E
(H) -4 (4) 在 万 中 存 在 补 空 间 ; 4 (4) 是 Z 中 的 闭 集 , 并 且
在 中 存 在 补 空 间 . ( 当 A 为 Fxedholm 算 子 时 , 这 个 假 设 总 是 成 立
的 . 在 下 文 的 应 用 中 , 经 常 是 这 种 情 形 . ) , .
[ 林匹 5 技
E 林577 医 3
【 志 东 n 不 沥 河 一 明 沥 林 河
E 林 0 沥 不 沥 李 E
投 影 九 和 一 乙 的 值 域 , 显 然 , 方 程 (L 5) 等 价 于 ,
E 河 辽 0

[ 沥 .7b)
E 吴 技

E 诊
E 吴 5 [ 阮 口
的 同 构 . 由 隐 函 数 定 理 , 存 在 X 在 原 点 的 邺 域 U。, 友 _ 在 原 点 的
邹 域 ,A 在 原 点 的 邻 域 印 , 以 及 C+ 酯 射 史 。 D X W 一 V,, 侍
吴 e

b 述 ,
E 耿 d L d 扬
c
EeC y 江江 ) [
[ 技 7
E ( 5 李 厂 沥 沥 刑 7 伟 c 顺 5 唐 人
Page-156
E 第 一 章 基札 概 伊 和 雍 备 知 订

E 河 p

d
E
下 面 再 迹 行 一 次 变 换 , 把 (5. 17) 第 二 式 中 的 (sJz 变 成 (so
E

则 (5. 17 化 奶
E zz歹 [25

E 吴 林李 心
1
E

E 吵(s),s),
E T游傍),

P 告帷川 E

C 音矽(鬓)挎〉 一 告哟(辜)芗〈藁‖e〉,

E 鲁砂〈E), E 吴
E0 703
应 许

把 ,9,,@ 和 雨 换 同 6,p,,Q 和 申 的 形 式 后 , 方 程 (5. 18) 成 为
[ n

附 注 5. 16 “ 显 然 , 局 部 族 (5. 6)( 在 (y,6) 一 (0,0) 附 近 的 一
i
定 义 5.4 中 的 映 射 9 与 A 都 可 取 为 相 应 空 间 中 的 恒 同 映 射 , 在 这 个
Page-157
E E

U
总 结 上 面 的 讨 论 , 我 们 有 下 面 的 结 果 .
E 玟玟 育 s 沥 0 浩
[ t
沥 y
2
其 中 加 与 G 分 别 由 C1. 8) 和 ( 9 定 义 , 【
i
E 一 5 沥 c 沥 人 5 林亚 亚 小 途 国
d 团 hn 命
域 及 值 域 都 作 了 显 著 的 约 化 这 就 是 Liapunov-Schmidt 方 法 的 核
E
E
s

巳
F c 0

E 吊 沥 八汀
考 廊 奇 点 分 岗 阿 题 , 就 是 要 在 R* X R 的 原 点 附 近 考 察 方 程

E 5 林) 沥 口 CL 0
0
【
E 吴 i 吴
人 河

医 纳 t

野
[ 2 沥 0
b

[
Page-158
E

t
E 2 述
E 河 王 ′
E
2CSv6J) |s-e -
工
E p 35 nn
E
E
[ [
E
E
E
[ 吴 3
E
陶 t ss e
E 沥 t e
E 河 林t
2 述
E
劝 一 方 面 , 在 厂 中 (0,oo 点 附 近 z:S 可 申 如 下 方 程 确 定 ( 见
条 件 CH,) 和 CH。)》
E c

E 吴 吴 国 莲 工
E 一 0 匕 52

国 t
价 于

a E

E

E
Page-159
E B

E 人 5

o“ (a,0 旦 Ca, 加 满 足 分 岔 方 程
&(a, 加 二 0,
E 不 沥 一 林招 浩 。
&(aoas 一 匹 一 @3F(am 十 讨 (a, 加 , 加 . 9 玟 )
玟

我 们 考 虑 R 中 一 类 更 广 泛 的 微 分 方 程

吊
门 。 2
FF E 逊 小 胡 沥 2 87

E 2 伟

分 岔 问 题 , 此 时 线 性 部 分 短 阵 为 4 一 彗 真), E 东 技

n

- 园 w 园 - 似

纳 an 2 沥
觉 (4 ) 一 纠 (Q) 一 Spantaoj.
E 绍 t [ ] E 2

0
E
E 伟 江
0 述 2 河 扬 吊
E 沥 引
E

匹 E
相 (】_Q)'(0〉= (′_Q)〔

′ 鲫z) E 翼27′JD.
p
Page-160
E 源

0 【
邹 匹
3 E
河
医
53 E

E

|

引 理 5.18 “ 设 (5. 4) 是 非 退 化 的 开 折 , 则 存 在 C“ 变 换 一
ACe),p(0) 一 0, 它 在 5 一 0 附 近 非 退 化 , 并 把 (5. 8) 变 到 (5. 9.

0
式 给 出 。

肉 二 卯 ( 日 , 肌 二 日 , 皂 一 口 , 一 Eni

则 由 条 件 (5. 7 知 AC0) 一 0 又 由 于 (5. 8) 是 非 退 化 的 , 则 由 引 理
5. 17 可 知 , 一 xC6) 在 e 一 0 是 非 退 化 的 , 引 理 得 证 . 方 程 (5.9) 中
E E
E i 技 | ,

e

习 题 与 怡 考 题 一

1.1 考 虔 R 上 的 动 力 系 统 , 设 它 皑 辅 线 分 别 具 有 图 1-6, 图 1-8 或 图 1-9
的 7 种 分 布 , 对 铁 一 种 分 布 , 取 平 面 上 不 吾 区 域 的 点 z, 议 论 极 眼 集 atz) 和
z). 莲 研 究 系 统 的 汀 游 茹 集 D(9). 对 郧 一 种 分 布 , 你 可 以 断 言 系 统 不 显 结
构 穗 定 的 3

1.2 利 用 对 (2.13) 的 讨 论 方 法 , 证 明 R 上 的 系 绕 (2.2) 是 C“ 系 统

羞=_zz+0(恤忏)

0
匹 :
英

2

国

E

1 3 对 下 列 系 统 , 求 出 与 中 心 诚 形 相 应 的 诱 导 方 程 (3.8》, 并 由 此 作 出 源
玟

八
Page-161
E

E 伟
0
c

E 述 河 正 规 形

1.5 设 R 上 的 和 量 杨 以 (0,0) 为 帛 炭 , 其 线 伯 鄂 分 在 (0,0) 的 短 阵 具 有
二重零特征根'而且向量场在翼转角震誓下保持不变,其中疃为正整敷,并且
E t 林553 加

盖 E 为 河 水 河浩

其 中 vepA 为 复 数 ,m 一
E 达
1 6
0 医 北
[ 颖
玟
' 1.7 考 虑 C“ 眶 射 P 一 R 它 以 一 1 十 5 为 特 征 根 ,|e| 人 1. 证 明 对
给 定 的 正 个 数 A 存 在 8 > 0, 使 当 ls| 一 日 时 , 七 可 以 经 过 C“ 变 换 化 为
E 沥 辽 人 江 达
E 河
利 用 此 结 果 对 例 2. 11 的 结 论 给 出 简 单 的 证 明
n
E 林政 c 林 达
在 小 扰 动 下 的 分 盆 蜀 律 .
Page-162
64 第 二 章 常 见 的 局 部 与 非 尸 郡 分 岗

加 果 我 们 考 虑 方 程 (1. 13) 的 C+ 扰 动 , 拥 动 参 数 为 , 则 扰 动 后
玟
玟
E 沥 伟
E 史 I 不 胡 弘
E 林 57 5 列
E
加 一 0 当 ACDDI(0,0) 一 0 时 有 两 个 零 点 , 当 p(D(0,0) 丿 0 时
E
玟
时 , 最 后 的 讨 论 可 从 第 一 章 定 理 2. 12 直 接 得 到 .

E 河 玟 育 余 t
E
化 的 同 时 ( 甚 至 在 奇 点 个 数 不 变 时 , 见 附 注 1.3), 执 道 结 构 还 可 能
玟
i

$ 2 “ 闭 轨 分 岔

考 忠 徽 分 方 程 族
E 黯=训z汹% E

E 林 命 砺 扬
【 s t
有 儿 条 闭 转 ? 这 就 是 闭 转 分 岔 问 题 , 当 ? 为 双 曲 闭 转 时 , 问 题 是 平
凡 的 ( 见 第 一 章 $ 13. 因 此 , 我 们 要 找 到 一 些 方 法 , 来 判 别 7 的 双
E 伟n 东
My 沥
E
Page-163
第 二 章 常 见 的 局 部 与 非 局 部 分 岔

E 东
岔 ,Hopf 分 岑 . 同 宿 分 岔 .Poincare 分 岔 等 , 其 中 前 三 种 为 局 部 分 岔
闭 题 , 后 两 种 分 别 为 半 局 部 分 岔 和 全 局 分 岔 问 题 , 除 了 奇 点 分 岔
外 , 本 章 的 大 部 分 讨 论 都 限 制 在 相 空 间 为 一 维 的 情 形 .

E

E
e
分 岔 现 象 称 为 奇 点 分 岖

一 舱 理 论

E 怀 园 s 木 5
点 的 线 性 郯 分 算 子 是 非 奇 异 的 , 即 它 的 所 有 特 征 根 均 非 零 . 否 则
u

E 东 育 参 数 的 向 量 场 , 如 果 它 的 奇
点 是 非 退 化 的 , 则 奇 点 本 身 也 光 滔 地 依 赖 于 参 数

证 明 ˇ 设 向 量 场 由 微 分 方 程

羞=矾婀… 许
巴 沥 2 一 诚 扬 史 刑
二 乙 为 (1. 1) 的 非 退 化 奇 点 , 卵 o(zoopa 一 0, 一 英 一

奇 异 . 由 隐 函 数 定 理 , 在 (zu, 附 近 存 在 光 潘 函 数 z 一 YC0 , 使
Page-164
E 65

s
E 东 颖 沥
伟
E i 一
E 李 22 林林 E 水 月 河 耿 认
E a 沥
片 (0,0) 二 0 的 条 件 .

在 解 决 具 体 问 题 时 , 图 雅 在 于 如 何 实 施 上 述 原 则 . 下 面 , 我 们
河 tn s h u
苗 些 重 要 结 论

E

[ [

E 述 林0 运 仪 不 0
E

2 z=棘C)=[

E
瞎0)]′
E 扬 2
[ 纳
1 b

50 = E 2
E 述 医 志 刑 E 林学 万 王 东 绍

E 沥 ( 一 E
其 中 (.,“) 表 示 R 中 的 内 积 , 联 坐 标 变 捣

E 沥 人 仪 庞 述 2 0

其 中 z 在 7 附 近 0 如 s 人 T,|a| 人 1. 坐 标 (om3 可 以 这 样 理 解 : 从
玟
t
E 不

1 余
Page-165
63 E

E

E
25
'
孙
E
0
图 2-1

我 们 先 把 方 程 (2. 2) 转 挨 成 曲 线 坐 标 系 下 的 方 程 , 然 后 建 立
E 沥 i 6 朱沥

E25AC 加 述 22 黯 E 伟 人 应 逊 2 盖 E 途 盖′

E
Eoculuoc t t 水 林芸
盖_ E

E 标 命 (XC 朋 L 东 107 扬 0

E grroyarD7
E
E 罡 c 班2 口 二 00 颂 子 1
d醴 CX7EXXZORNOZND

E 20 沥 沥
E 达
E 江

E 江2 口
Page-166
E u

【dCtoc l tdto g 沥 浩 |

E 玟 园 怀 园 东 不
数 的 微 小 变 化 下 不 变 , 它 的 位 置 也 光 滑 地 依 赖 于 参 数 的 变 化 . 需
要 注 意 , 奇 点 的 非 退 化 性 与 双 曲 性 是 不 同 的 概 念 . 例 如 , 一 个 们
E
是 非 退 化 的 , 但 它 是 非 双 曲 的 . 此 时 在 扰 动 下 , 蚀 然 奇 点 个 数 ( 在
小 邹 域 内 ) 不 发 生 变 化 , 但 其 附 近 的 扬 道 结 构 可 能 变 化 , 出 现 Hopf
E

E 江 刑
退 化 奇 点 ( 它 们 必 是 孤 立 奇 点 ) 或 无 奇 点 的 向 量 场 集 合 形 成 一 个
E

[ 国 圆 八 沥 标 亚 一 者 吊 622

E

E 河 述
讨 论 ; 考 虑 投 影
E 仪

其 中 了 - /C6). 具 有 契 炉 的 向 量 场 集 合 在 空 间 JoCM ,M) 中 有 表
E

E
d
/〔z)在辜点非退化,即噩 园 l

玟
C, 15, 得 知 仅 有 非 退 化 夹 点 或 无 奇 点 的 向 量 场 在 <g““CM ) 中 形 成
开 稽 子 集 ,

这 个 定 理 说 明 , 向 量 场 的 一 个 退 化 奇 点 可 以 经 过 任 意 小 的 扶
E 吊 皋 尿 应
Eouolok a i 江
Page-167
E E

stouyzyltisyogp 志 E
tuskts p 志 途 p
奇 点 . 对 一 个 具 体 的 奇 点 分 盆 问 题 , 通 常 有 两 种 处 理 方 法 : 一 种 是
利 用 中 心 流 形 定 理 , 把 问 题 归 结 到 中 心 流 形 上 , 见 第 一 章 例 3. 121 ˇ
8n 河 目 J 师 0 沥
method), 为 了 说 明 这 个 方 法 的 基 本 性 想 , 我 们 先 看 一 种 特 殊 情
E 沥 2 e 一

沥

其 中 《 的 特 征 根 均 为 零 , 而 旦 的 特 征 根 均 不 为 零 ! 人 g E Cr , 之
E
E 伟

E 玟 玲
i
a

Ecae n t 余

E

d
E 标 逊 6 E
0

一 {Cy 史 EDICGOoo 主 二 口 , 马 二 5 门 仁 二 加 ),
、

则 对 不 同 的 A,|2| 心 1,8, 结 构 的 变 化 反 晖 了 奇 点 个 数 的 变 化 规
E 河 2
闰 维 数 得 到 降 低 通 常 称 (1. 4) 为 方 程 (1. 2) 的 分 岑 函 数 .

为 了 应 用 上 的 便 利 , 下 面 在 更 一 舱 的 框 架 下 讨 论 这 个 间 题 .

Liapunov-Schmidt 方 法

设 又 ,Z 和 A 为 实 Banach 空 间 ,D 和 砌 分 别 为 又 和 A 中 零 点
[ 一 e 人
Page-168
E

不 COvs,0 二 0,
河 (

E
0 s

从 而 (2. 7 可 写 成

盖 应 0

E 一 一
E

RCasd, 加 一 膘〔epr工〔/氮(震) 00
5

0
E 0 泊
E 标
E 沥 0 一 不
E 吴
E 1253
E 林 仪
i 沥 许 东 许 不 0 林 一 达
E 许 e 林t E 一
定 义 2.1 若 存 在 e 井 0, 使 V a E (0,e) 都 有 G(a,b 一 0
0 一 一 不
E 林 仁 技 c 人 河 红 2 一 王
E 技
E s
从 上 述 定 义 可 知 , 稳 定 ( 不 稳 定 极 限 环 必 为 颈 立 闭 执 . 下 面
E 东
E 出 孙孙 招 肖 2 伟 扬 医 [ 吴 ( 吊 355 朋
E 河 绍 c 万 一 t 河
E i 述 万 一
Page-169
E E 渡
b

类 似 可 定 义 内 侧 复 型 极 限 环 与 内 侧 周 期 环 域

E 沥

E 人 万 儿 3 一 仁 浩 中 仁 刑
c eletsasogaresssa 吴 s 相 河
伟 东 逊n 圆 a 沥
端 函 数 F(rvs,2 解 析 , 从而c她从)解析 E
渡 八 才 2 振

东 朋 李 口
有

[ 林门 [Ey
E u t t 林 ag 一 0
时 称 为 多 重 极 限 环 .

显 然 , 当 & 为 奇 数 时 ,ct 一 0 表 明 7 为 稳 定 的 极 限 环 , 而 c 一 0
E 沥 纳 e不 江 技
注 意 , 这 里 说 的 穗 定 性 为 转 道 穗 定 性 , 而 不 是 结 构 穗 定 性 , 事 实
玟
E 河泊
重 的 , 我 们 记

E I工tr 寄z(棘 (,0)d5 Et3)

E 梁 育 d 芸
E 人
Eod 国 图 M 9 4253

E exp]′二〔矛/(盒) E 1).
E d

吴 D OO
E

4

E exp丘′矛z(J)dj 0
E
Page-170
园 国 u

E
胡 (zi0 一 0 0
t t 朋
0 i
E
(H) -4 (4) 在 万 中 存 在 补 空 间 ; 4 (4) 是 Z 中 的 闭 集 , 并 且
在 中 存 在 补 空 间 . ( 当 A 为 Fxedholm 算 子 时 , 这 个 假 设 总 是 成 立
的 . 在 下 文 的 应 用 中 , 经 常 是 这 种 情 形 . ) , .
[ 林匹 5 技
E 林577 医 3
【 志 东 n 不 沥 河 一 明 沥 林 河
E 林 0 沥 不 沥 李 E
投 影 九 和 一 乙 的 值 域 , 显 然 , 方 程 (L 5) 等 价 于 ,
E 河 辽 0

[ 沥 .7b)
E 吴 技

E 诊
E 吴 5 [ 阮 口
的 同 构 . 由 隐 函 数 定 理 , 存 在 X 在 原 点 的 邺 域 U。, 友 _ 在 原 点 的
邹 域 ,A 在 原 点 的 邻 域 印 , 以 及 C+ 酯 射 史 。 D X W 一 V,, 侍
吴 e

b 述 ,
E 耿 d L d 扬
c
EeC y 江江 ) [
[ 技 7
E ( 5 李 厂 沥 沥 刑 7 伟 c 顺 5 唐 人
Page-171
E 69

劝 一 方 面 , 把 (2. 9) 式 的 内 积 按 分 量 展 开 , 并 利 用 (2, 3),(2. 8) 可
5

E 〈芗(占), (P(s),0)芗(s)〉
=t【盂`(孵(裹)】o) 一 _铲′(s)_z迈

巳 河 E /
江 7

0
E
1
注意到7以T为周期 E
′「r蟹艾 i 伟 E
。 【 L辽 F5 仪

023

0
[

医 a 技 5
它 的 稳 定 性 与 0 的 符 号 之 间 的 关 系 是 显 然 的 , 定 理 得 证 , 〖

E 沥 沥 沥 仪 红
近 为 周 期 环 域 时 , 必 有 v 一 0.

E 北 余 育 2

【2 怡 chizle 沥 i t 刑
E

0 坂 坂 林 广 二 招 e
E
为 余 数 时 , 上 述 结 论 可 扩 充 至 一 0.

EC2s e 河纺 el 育 动 0 前团 国 吴 二 4
E
Page-172
E E

U
总 结 上 面 的 讨 论 , 我 们 有 下 面 的 结 果 .
E 玟玟 育 s 沥 0 浩
[ t
沥 y
2
其 中 加 与 G 分 别 由 C1. 8) 和 ( 9 定 义 , 【
i
E 一 5 沥 c 沥 人 5 林亚 亚 小 途 国
d 团 hn 命
域 及 值 域 都 作 了 显 著 的 约 化 这 就 是 Liapunov-Schmidt 方 法 的 核
E
E
s

巳
F c 0

E 吊 沥 八汀
考 廊 奇 点 分 岗 阿 题 , 就 是 要 在 R* X R 的 原 点 附 近 考 察 方 程

E 5 林) 沥 口 CL 0
0
【
E 吴 i 吴
人 河

医 纳 t

野
[ 2 沥 0
b

[
Page-173
E B

E 人 5

o“ (a,0 旦 Ca, 加 满 足 分 岔 方 程
&(a, 加 二 0,
E 不 沥 一 林招 浩 。
&(aoas 一 匹 一 @3F(am 十 讨 (a, 加 , 加 . 9 玟 )
玟

我 们 考 虑 R 中 一 类 更 广 泛 的 微 分 方 程

吊
门 。 2
FF E 逊 小 胡 沥 2 87

E 2 伟

分 岔 问 题 , 此 时 线 性 部 分 短 阵 为 4 一 彗 真), E 东 技

n

- 园 w 园 - 似

纳 an 2 沥
觉 (4 ) 一 纠 (Q) 一 Spantaoj.
E 绍 t [ ] E 2

0
E
E 伟 江
0 述 2 河 扬 吊
E 沥 引
E

匹 E
相 (】_Q)'(0〉= (′_Q)〔

′ 鲫z) E 翼27′JD.
p
Page-174
E 招

Dn 国 4
E
E

E

E
E 述 a 木
蘑_Ir订 吊 `d囊_「(入 E 吴 2 林
沥 “

这 里 利 用 系 统 的 第 一 个 方 程 得 到 2zydt 一 2zdz 心 d( 丿 . 因 此 , 这
0 |

[
0 技 门
E
13) 式 中 第 一 个 不 为 零 的 ct, 以 便 接 定 义 2.4 判 断 7 的 重 次 . 这 时
t

3 Hopf 分 岛

当 向 量 场 在 奇 点 的 线 性 部 分 短 阵 有 一 对 复 特 征 根 , 井 东 随 参
数 变 化 而 穿 越 虚 轴 时 , 在 奇 点 附 近 的 一 个 二 维 中 心 流 形 上 , 奇 点 的
稳 定 性 发 生 翻 转 , 从 而 在 奇 点 附 近 产 生 闭 转 的 现 象 , 称 为 Hopf 分
岔 , 第 一 章 例 2.9 就 是 一 个 典 型 的 实 例 . 既 然 Hopt 分 岗 发 生 在 二
医

经 典 的 Hopf 分 岔 定 理
g
0 蕃 E 河 吴 公
Page-175
64 第 二 章 常 见 的 局 部 与 非 尸 郡 分 岗

加 果 我 们 考 虑 方 程 (1. 13) 的 C+ 扰 动 , 拥 动 参 数 为 , 则 扰 动 后
玟
玟
E 沥 伟
E 史 I 不 胡 弘
E 林 57 5 列
E
加 一 0 当 ACDDI(0,0) 一 0 时 有 两 个 零 点 , 当 p(D(0,0) 丿 0 时
E
玟
时 , 最 后 的 讨 论 可 从 第 一 章 定 理 2. 12 直 接 得 到 .

E 河 玟 育 余 t
E
化 的 同 时 ( 甚 至 在 奇 点 个 数 不 变 时 , 见 附 注 1.3), 执 道 结 构 还 可 能
玟
i

$ 2 “ 闭 轨 分 岔

考 忠 徽 分 方 程 族
E 黯=训z汹% E

E 林 命 砺 扬
【 s t
有 儿 条 闭 转 ? 这 就 是 闭 转 分 岔 问 题 , 当 ? 为 双 曲 闭 转 时 , 问 题 是 平
凡 的 ( 见 第 一 章 $ 13. 因 此 , 我 们 要 找 到 一 些 方 法 , 来 判 别 7 的 双
E 伟n 东
My 沥
E
Page-176
E 理 0

E c 江 林 d 小 0
线 伯 部 分 矩 阵 4 ( 有 特 征 值 a(4 士 识 ( , 满 足 条 件

【 2

0 怀 纳 0

0 明

其 中 a 为 向 量 场 X。 的 如 下 复 正 规 形 中 的 系 数 ( 见 第 一 章
一 97。 “

蓥″ E 伟

【6H
E 林 l 吊 述
E u 前
E ss 命 d
E 标 d 李
[ 沥
道 是 它 的 唯 一 闭 软 . 当 Rec 一 0 时 , 它 是 稳 家 的 ; 当 Reci > 0 时 ,
a
1
时 ,Cmo 一 0.
E 沥 医 n l 沥n
E 医 沥 扬 一 s
E 吴 一
E d 标 刑
E 水
立[z〕= [0 ^姚] 目 国
一 E E 2 叉y E “
E 人 小 2 发n 吴 c
E

E 林 李 仪 沥 )
Page-177
E 65

s
E 东 颖 沥
伟
E i 一
E 李 22 林林 E 水 月 河 耿 认
E a 沥
片 (0,0) 二 0 的 条 件 .

在 解 决 具 体 问 题 时 , 图 雅 在 于 如 何 实 施 上 述 原 则 . 下 面 , 我 们
河 tn s h u
苗 些 重 要 结 论

E

[ [

E 述 林0 运 仪 不 0
E

2 z=棘C)=[

E
瞎0)]′
E 扬 2
[ 纳
1 b

50 = E 2
E 述 医 志 刑 E 林学 万 王 东 绍

E 沥 ( 一 E
其 中 (.,“) 表 示 R 中 的 内 积 , 联 坐 标 变 捣

E 沥 人 仪 庞 述 2 0

其 中 z 在 7 附 近 0 如 s 人 T,|a| 人 1. 坐 标 (om3 可 以 这 样 理 解 : 从
玟
t
E 不

1 余
Page-178
E

1 林)

E 沥 江 林沥 闵 吴

退 化 Hopf 分 岖 定 理
E 人

E
E

人

Eodii 48 沥

引 理 3.4 设 在 奇 点 (z,) 一 (0,0) 系 绕 (3. 4 的 线 性 鄂 分 矩
阵 有 一 对 复 特 征 根 a(4 土 讨 (49 , 满 足 条 件 (H, 则 对 任 意 自 然 数
E 述
e

0

十 C′(#)″/+】歹/ E 0( |槽/|″十…) ,

E 吴 0 佳 河 不 芸 水
e 技 2

E 沥 t 0
[ 沥 王 一
E 技

[ p

由 条 件 CH) 知 , 力 (0) 二 (m,A(03) 给 出 一 2 十 2 阶 的 共 摄 条 件 ,

巳 一 许
E 途 202 志 y 万 一 4 途 育 小 应25
0

E
Page-179
63 E

E

E
25
'
孙
E
0
图 2-1

我 们 先 把 方 程 (2. 2) 转 挨 成 曲 线 坐 标 系 下 的 方 程 , 然 后 建 立
E 沥 i 6 朱沥

E25AC 加 述 22 黯 E 伟 人 应 逊 2 盖 E 途 盖′

E
Eoculuoc t t 水 林芸
盖_ E

E 标 命 (XC 朋 L 东 107 扬 0

E grroyarD7
E
E 罡 c 班2 口 二 00 颂 子 1
d醴 CX7EXXZORNOZND

E 20 沥 沥
E 达
E 江

E 江2 口
Page-180
E 瑶 汀 15 0

E 沥 沥 t a |

E 出 东 仪 y 水
E

E 医 述

定 理 3.6 “ 设 闯 量 场 X 以 (0,03 点 为 & 阶 细 焦 点 ; 则 X 在 拼
E 水

0
的 邻 域 口 , 使 得 当 |x| 一 a 时 , 在 乙 内 至 多 有 8 个 极 限 环 !

E 沥 林
(z,3 一 (0,0) 的 任 意 邹 域 口 “ 人 口 , 存 在 一 个 开 折 系 绕 ; , 使 得
恒 在 口 “ 内 恰 有 了 个 极 限 环 , 其 中 |x| 一 .

E 沥 ( 沥 俊 圆 沥 t
E 达 技 一 技 p
意 口 一 z5, e2pi 二 i, 可 得 到

{ E 朐
E 命

0

E l 沥

盖 g 吴 达

[
办 许
E

t - 仪 ,
2 e昙?亘畲))+矾 E

e(缥 (
0 Z室 E 沥 口

沥 政 沥
Page-181
E

不 COvs,0 二 0,
河 (

E
0 s

从 而 (2. 7 可 写 成

盖 应 0

E 一 一
E

RCasd, 加 一 膘〔epr工〔/氮(震) 00
5

0
E 0 泊
E 标
E 沥 0 一 不
E 吴
E 1253
E 林 仪
i 沥 许 东 许 不 0 林 一 达
E 许 e 林t E 一
定 义 2.1 若 存 在 e 井 0, 使 V a E (0,e) 都 有 G(a,b 一 0
0 一 一 不
E 林 仁 技 c 人 河 红 2 一 王
E 技
E s
从 上 述 定 义 可 知 , 稳 定 ( 不 稳 定 极 限 环 必 为 颈 立 闭 执 . 下 面
E 东
E 出 孙孙 招 肖 2 伟 扬 医 [ 吴 ( 吊 355 朋
E 河 绍 c 万 一 t 河
E i 述 万 一
Page-182
国 第 二 章 常 见 的 局 部 与 非 局 部 分 芸

这 里 乃 j 二 办 j(Py 史 是 pE [0,2z] 和 p( 在 0 附 近 ) 的 光 滑 函 数
E
霉… 二 噜 [ C3.9)
在 乙 轴 上 建 立 方 程 (3.8) 的 Poincare 晔 射 PCz:,4 , 并 令
ELC 命 52 河 洁 [
显 然 ,
【 A 水 《C3. 11》
【 怀 i ss 26 志 752 浩 胡 诊
E
医 江 玖
E
E 人 不
E
E
E

E 英 E 余 国
F `二〕___】((),0) E a′(0,2‖) E

0

[ 达
E 朱 吴 吴 gr8
E

E 医 述 2 八 2 技 1
「' 行 日 d 诊
E

3 扎 国

0 〔(2发+1)丨〕蒂…' E
。

y

0

0 E 明 d 理

萨鲜J(′-'2‖) i 35 一 Re乙`鏖
e

《3. 13)
Page-183
E E 渡
b

类 似 可 定 义 内 侧 复 型 极 限 环 与 内 侧 周 期 环 域

E 沥

E 人 万 儿 3 一 仁 浩 中 仁 刑
c eletsasogaresssa 吴 s 相 河
伟 东 逊n 圆 a 沥
端 函 数 F(rvs,2 解 析 , 从而c她从)解析 E
渡 八 才 2 振

东 朋 李 口
有

[ 林门 [Ey
E u t t 林 ag 一 0
时 称 为 多 重 极 限 环 .

显 然 , 当 & 为 奇 数 时 ,ct 一 0 表 明 7 为 稳 定 的 极 限 环 , 而 c 一 0
E 沥 纳 e不 江 技
注 意 , 这 里 说 的 穗 定 性 为 转 道 穗 定 性 , 而 不 是 结 构 穗 定 性 , 事 实
玟
E 河泊
重 的 , 我 们 记

E I工tr 寄z(棘 (,0)d5 Et3)

E 梁 育 d 芸
E 人
Eod 国 图 M 9 4253

E exp]′二〔矛/(盒) E 1).
E d

吴 D OO
E

4

E exp丘′矛z(J)dj 0
E
Page-184
E 瑞水 5

i 技
壬0, 当1<m<触+L
〈

2″〔(2桑十1)【〕 婶" [ 李 玲

0

注意方程(3 E 东
E
i
林
s 吴 玟 不 e 朋
园

PCru,40 二 @(zruDh(zru,19.

古 一 方 面 ,Y(0,) 二 0, 且 (ru,40 对 心 的 正 根 与 负 根 成 对 出 现
0 i
uoapyeeoyoyt u 不 a 河A 招
对 至 多 有 & 个 正 根 , 定 理 的 结 论 (1) 得 证

为 了 证 明 结 论 (2), 我 们 假 设 Xu 以 z 一 0 为 阶 细 焦 点 , 即 它
具 有 如 下 的 正 规 琪

E 八 林 许 有 述 门 Rect 奂 0.
取 它 的 扰 动 系 统
E 不
[ 沥 5
E 土 贝 训 招 中
E
红 林述
9
t

E 东 述 林
命
Page-185
E 69

劝 一 方 面 , 把 (2. 9) 式 的 内 积 按 分 量 展 开 , 并 利 用 (2, 3),(2. 8) 可
5

E 〈芗(占), (P(s),0)芗(s)〉
=t【盂`(孵(裹)】o) 一 _铲′(s)_z迈

巳 河 E /
江 7

0
E
1
注意到7以T为周期 E
′「r蟹艾 i 伟 E
。 【 L辽 F5 仪

023

0
[

医 a 技 5
它 的 稳 定 性 与 0 的 符 号 之 间 的 关 系 是 显 然 的 , 定 理 得 证 , 〖

E 沥 沥 沥 仪 红
近 为 周 期 环 域 时 , 必 有 v 一 0.

E 北 余 育 2

【2 怡 chizle 沥 i t 刑
E

0 坂 坂 林 广 二 招 e
E
为 余 数 时 , 上 述 结 论 可 扩 充 至 一 0.

EC2s e 河纺 el 育 动 0 前团 国 吴 二 4
E
Page-186
0 第 二 犊 常 见 的 局 部 与 非 尾 部 分 岑

53 吊 2 里
E 兰 吊 东
E d 月 月 汀 7 租 ( 顶 荣

【 7 河 砺 7 月 园 0 园 7
Estyuupuy4 ,#′′一j,r′一j,1吏得 E E
E 胺
[ 余 A

[ 李 前 = [

[ 沥 7 口
由 Poincare-Bendixson 环 域 定 理 , 扰 动 系 统 (3. 15) 至 少 存 在 个 极
限 环 .

E 才 [
E
个 极 限 环 . 若 不 然 , 则 对 任 意 的 a> 0 和 7 一 0 的 邺 域 7,(3. 15) 在
口 内 有 多 于 } 个 极 限 环 , 则 我 们 可 以 仿 照 上 面 的 方 法 选 取 _ 一 ,
rk-j-Ly “ 皋 叶 以 及 eroy 屋 而 在 口 内 再 获 得 古 外 的 & 一 3 个 极 限
E
0 |

定 理 3. 1 的 证 明 “ 当 一 1 时 , 可 从 定 理 3.6 得 到 定 理 3. 1.
E 3 s 林浩 日 ( 仪 2 招
一 阶 细 焦 点 , 因 此 定 理 3, 1 的 前 一 部 分 结 论 成 立 , 再 设 条 件 (Ho)
E 命河 应图 生 刑

Ed5 沥 多 庞 z_T〉 【e2525 [ 途
其 中

0 【 〔exp E 羞座〕 5 1〕十 d 述

冉
[
再 由 (3. 14 可 知
Page-187
E 招

Dn 国 4
E
E

E

E
E 述 a 木
蘑_Ir订 吊 `d囊_「(入 E 吴 2 林
沥 “

这 里 利 用 系 统 的 第 一 个 方 程 得 到 2zydt 一 2zdz 心 d( 丿 . 因 此 , 这
0 |

[
0 技 门
E
13) 式 中 第 一 个 不 为 零 的 ct, 以 便 接 定 义 2.4 判 断 7 的 重 次 . 这 时
t

3 Hopf 分 岛

当 向 量 场 在 奇 点 的 线 性 部 分 短 阵 有 一 对 复 特 征 根 , 井 东 随 参
数 变 化 而 穿 越 虚 轴 时 , 在 奇 点 附 近 的 一 个 二 维 中 心 流 形 上 , 奇 点 的
稳 定 性 发 生 翻 转 , 从 而 在 奇 点 附 近 产 生 闭 转 的 现 象 , 称 为 Hopf 分
岔 , 第 一 章 例 2.9 就 是 一 个 典 型 的 实 例 . 既 然 Hopt 分 岗 发 生 在 二
医

经 典 的 Hopf 分 岔 定 理
g
0 蕃 E 河 吴 公
Page-188
医 脱 u

E

E 命 7 1 E 述 c

由 (3. 17) 和 条 件 (Ha) 可 知

E

蠢(O'O) 园 湃(O)挝(0) Eal t2E
E 存在疃〉0和在0<zl<疃定义的光滑函数″_
[

E [ E

至 此 , 定 理 3 1 的 结 论 (1) 得 证 . 为 了 证 明 结 论 (2, 从 (3. 20) 求 导 、
E

E

E
F +萜烬(z】) 二

E
E

[
E 及条件(H】) 可得

, 把 上 面 的 结 果 及 (3. 19) 代 人 (8. 21) 得 到

国 国 E
E r

冉 此 得 定 理 的 结 论 (22, 【
E 招 医
E 人
一 0 的 邻 域 口 , 使 得
E
个 极 限 环 , 当 Ree 一 0 (> 0 时 , 它 是 穗 定 ( 不 稳 定 ) 的 并 且 当 A
s 口
0
Page-189
E 理 0

E c 江 林 d 小 0
线 伯 部 分 矩 阵 4 ( 有 特 征 值 a(4 士 识 ( , 满 足 条 件

【 2

0 怀 纳 0

0 明

其 中 a 为 向 量 场 X。 的 如 下 复 正 规 形 中 的 系 数 ( 见 第 一 章
一 97。 “

蓥″ E 伟

【6H
E 林 l 吊 述
E u 前
E ss 命 d
E 标 d 李
[ 沥
道 是 它 的 唯 一 闭 软 . 当 Rec 一 0 时 , 它 是 稳 家 的 ; 当 Reci > 0 时 ,
a
1
时 ,Cmo 一 0.
E 沥 医 n l 沥n
E 医 沥 扬 一 s
E 吴 一
E d 标 刑
E 水
立[z〕= [0 ^姚] 目 国
一 E E 2 叉y E “
E 人 小 2 发n 吴 c
E

E 林 李 仪 沥 )
Page-190
. 78 第 二 章 常 见 的 尿 部 与 非 局 部 分 责

E 浩 ]

附 注 3.8 “ 在 应 用 定 理 3.6 时 , 霁 要 首 先 判 斯 未 扰 动 系 统 X
以 0 为 细 焦 点 的 阶 数 , 也 就 是 确 定 满 足 条 件 (3.5 的 & 在 实 际 计
算 时 , 经 常 应 用 下 面 介 绍 的 Liapunov 系 数 法 , 细 节 请 见 [ZDHD]
医

〈 E 之 圭 玟
E

E 伟 d
医 达

E 兽位2 吴 成 应 吴 人 画 江

医 述 2

E
蔷 = 针…掳 沥 江 园 [ 沥
E

E

漾 足 上 式 的 (V,} 称 为 (3. 16) 的 iapunov 系 数 . 在 下 面 定 理 的 意 义
e

E 八 北 沥 朐 圆 一 仁 , 一 0,8. 丿 0( 或 一 0), 当 且 仅 当
E 唐 5

定 理 的 证 明 见 [BL], 下 文 中 , 我 们 把 (} 或 (Re(e,} 称 为 系
统 的 焦 点 量 .

应 用

例 3. 10 “ 考 虑 二 维 系 统
壬 园 【
林 沥 吴 玟 江志 玟 口 招 玟 技 招 途 仪
E 振 人 i 林人
( 一 0 附 近 发 生 Hopf 分 岔 的 可 能 性 . 令 一 乙 十 1, 系 统 (3. 24
. 变 为
Page-191
E

1 林)

E 沥 江 林沥 闵 吴

退 化 Hopf 分 岖 定 理
E 人

E
E

人

Eodii 48 沥

引 理 3.4 设 在 奇 点 (z,) 一 (0,0) 系 绕 (3. 4 的 线 性 鄂 分 矩
阵 有 一 对 复 特 征 根 a(4 土 讨 (49 , 满 足 条 件 (H, 则 对 任 意 自 然 数
E 述
e

0

十 C′(#)″/+】歹/ E 0( |槽/|″十…) ,

E 吴 0 佳 河 不 芸 水
e 技 2

E 沥 t 0
[ 沥 王 一
E 技

[ p

由 条 件 CH) 知 , 力 (0) 二 (m,A(03) 给 出 一 2 十 2 阶 的 共 摄 条 件 ,

巳 一 许
E 途 202 志 y 万 一 4 途 育 小 应25
0

E
Page-192
E
E

红 2
E 江 达 0 述
根 沥 5 技 水 沥 逊 T 医
这 个 系 统 在 (0,0) 的 线 性 都 分 短 阵 有 一 对 纯 虚 特 征 根 的 条 件 为
浩 胡理 1 沥
医 林 d 渡 仪 一 仪 s 途 1 莲 2 李 浩7
d

d 沥 E
″ 75 河 玟

E 河 理 2 吴
E 沥 江

E 圭(的 玟

1
E
国 育 5

v…=訾仰 E
由 正 应 用 定 理 3.6, 可 得 下 列 结 论 :
E 李 u [ 林史 王 吊 不 不 吊
盆 , 并 东 系 统 (3. 24》 在 原 点 附 近 存 在 唯 一 极 限 环 的 参 数 区 域 是
E 月
0 化 2 儿 沥 沥 不 7 林丿 史 吊 不 c 永
Hopt 分 岔 ,(3,. 24) 在 原 点 附 近 存 在 二 个 极 限 环 的 参 数 区 域 是
E 朱应
D 厂 才 尸

0 坂 i d

[ E
E 昔#u [ 河 浩 二 伟 口 河

D 林ss

t 河
Page-193
E 瑶 汀 15 0

E 沥 沥 t a |

E 出 东 仪 y 水
E

E 医 述

定 理 3.6 “ 设 闯 量 场 X 以 (0,03 点 为 & 阶 细 焦 点 ; 则 X 在 拼
E 水

0
的 邻 域 口 , 使 得 当 |x| 一 a 时 , 在 乙 内 至 多 有 8 个 极 限 环 !

E 沥 林
(z,3 一 (0,0) 的 任 意 邹 域 口 “ 人 口 , 存 在 一 个 开 折 系 绕 ; , 使 得
恒 在 口 “ 内 恰 有 了 个 极 限 环 , 其 中 |x| 一 .

E 沥 ( 沥 俊 圆 沥 t
E 达 技 一 技 p
意 口 一 z5, e2pi 二 i, 可 得 到

{ E 朐
E 命

0

E l 沥

盖 g 吴 达

[
办 许
E

t - 仪 ,
2 e昙?亘畲))+矾 E

e(缥 (
0 Z室 E 沥 口

沥 政 沥
Page-194
第 二 章 常 见 的 局 郭 与 非 尾 部 分 岔

EXcu Y
E +陲) 命
c 八沙
E
E
的 一 种 标 准 形 式 导 出 了 著 名 的 焦 点 量 公 式 ( 见 [Ba], 并 证 明 了 二
技
数 值 上 都 有 误 , 在 [QL], 及 [FLLL] 中 得 到 纠 正 . 下 面 介 绍 的 结
E 沥
E 技
壬 2 E 缥舶翼z 十 anzy 十 ″叩燮z'
E 河 园 / 园
E
E
【 757 有 厂 木 272
E 述
则 ( 不 计 正 数 图 子 》
卯 一 ha 一 B,
【 01
[ E 口 2 振 EAgs
E 沥 王 孙 芒 u 刑
(3. 28)
在 最 后 一 种 情 形 下 , 系 统 可 积 ,(0,03 为 中 心 点
附 注 3. 12 从 原 则 上 说 , 当 X 以 0 为 细 焦 点 时 , 总 可 以 通 过
有 限 步 运 算 确 定 细 焦 点 的 阶 数 &. 但 是 当 较 大 时 , 对 一 舫 系 统 息

用 (3. 28) 的 方 式 来 表 达 焦 点 重 公 式 , 计 算 重 非 常 大 . 例 如 , 当 把 方
Easetoakypsp p

阮
Page-195
国 第 二 章 常 见 的 局 部 与 非 局 部 分 芸

这 里 乃 j 二 办 j(Py 史 是 pE [0,2z] 和 p( 在 0 附 近 ) 的 光 滑 函 数
E
霉… 二 噜 [ C3.9)
在 乙 轴 上 建 立 方 程 (3.8) 的 Poincare 晔 射 PCz:,4 , 并 令
ELC 命 52 河 洁 [
显 然 ,
【 A 水 《C3. 11》
【 怀 i ss 26 志 752 浩 胡 诊
E
医 江 玖
E
E 人 不
E
E
E

E 英 E 余 国
F `二〕___】((),0) E a′(0,2‖) E

0

[ 达
E 朱 吴 吴 gr8
E

E 医 述 2 八 2 技 1
「' 行 日 d 诊
E

3 扎 国

0 〔(2发+1)丨〕蒂…' E
。

y

0

0 E 明 d 理

萨鲜J(′-'2‖) i 35 一 Re乙`鏖
e

《3. 13)
Page-196
[ 颖 水 2 5

的 焦 点 量 公 式 十 分 复 杂 ; 即 使 利 用 计 算 机 , 目 前 也 仅 推 导 出 前 儿 个
E
点 这 个 固 难 阿 题 紧 密 相 关 , 在 这 里 我 们 仅 列 举 我 国 学 者 在 这 方 面
E 吊 t 招
中心和焦点的较]般方法;杜乃林、曾宪武圃〕给出了计算焦点童
E 沥 沥
沥
黄 文 灵 “ 证 明 , 当 非 线 性 方 程 零 点 的 拓 扑 度 变 名 时 , 会 产 生 述 通
的 分 岔 曲 线 , 等 等

E 些

【 xta 2 怀 e 李 d 吊 2
其 中 口 是 R: 中 原 点 的 一 个 开 集 , 则 当 条 件 (H),CH) 和 (He) 成
立 时 , 或 者 系 统 (3.1) 当 一 0 时 以 原 点 为 中 心 , 或 者 当 AE ( 一 ,
E
d
E i 沥 t 东 p 浩
E

例 3. 14

羞 国 八 述 0 吴

【
0

s 标 班2 述 月

[ 朋

【
E 罚 0

a u 林s 一 国
u
d′_

L 匹
F E 浩 河 玟 薯) g
E

医 沥
Page-197
E 瑞水 5

i 技
壬0, 当1<m<触+L
〈

2″〔(2桑十1)【〕 婶" [ 李 玲

0

注意方程(3 E 东
E
i
林
s 吴 玟 不 e 朋
园

PCru,40 二 @(zruDh(zru,19.

古 一 方 面 ,Y(0,) 二 0, 且 (ru,40 对 心 的 正 根 与 负 根 成 对 出 现
0 i
uoapyeeoyoyt u 不 a 河A 招
对 至 多 有 & 个 正 根 , 定 理 的 结 论 (1) 得 证

为 了 证 明 结 论 (2), 我 们 假 设 Xu 以 z 一 0 为 阶 细 焦 点 , 即 它
具 有 如 下 的 正 规 琪

E 八 林 许 有 述 门 Rect 奂 0.
取 它 的 扰 动 系 统
E 不
[ 沥 5
E 土 贝 训 招 中
E
红 林述
9
t

E 东 述 林
命
Page-198
5 第 二 章 帝 见 的 尸 部 与 非 尿 部 分 盅

其 中 0 一 r 一 1. 故 原 点 是 (3. 29)。 的 渐 近 稳 定 焦 点
E

1
【 江 sinr一菩)'

图 2-2

t 林 e一睾与 E 3e一丢之间(见图
E 由于

巳 技 E 一
园 言c鹏r 詹

i
汀 东 t河 才 年 9
0 3 技 [ d′【(″〉 E 取 巳 刑

E 29〉″在原点附近都至少有两条闭轨′因

E ss 途 伟 沥
请 凄 者 验 算 , 此 例 中 对 一 切 正 整 数 A, 都 有 Recx 二 0. 团 此 , 对

【omestsrs 2 吊 e 怡

对 参 数 一 致 的 Hopf 分 岔 定 理
E
Page-199
E 脱 u

巳 河 国 匹 ,
E 沥

@ 张
E 林 丞

其 中 函 数 双 一 友 (z,y7, 参 数 3,xE R「, 丁 $ 为 小 参 数 . 设 系 统 有
d
E d
【5) E

则 在 一 定 的 附 加 条 件 下 , 当 一 x(8, 8 人 (0,o 时 , 系 统 (3. 32》
E t 沥
3.7 可 知 , 存 在 (8) 丿 0, 使 得 当 |x 一 A(8)| 一 e(8 且 人 丿
E
E 沥 0 吊 e
s, 使 得 对 所 有 的 E (0,8), 都 有 e(8 之 e 这 就 是 所 谓 对 参 数
g

[ 沥

E 林 振 芸 s 怀

目 国 园 伟 芒 97 扬
[ 国 P

国
Page-200
0 第 二 犊 常 见 的 局 部 与 非 尾 部 分 岑

53 吊 2 里
E 兰 吊 东
E d 月 月 汀 7 租 ( 顶 荣

【 7 河 砺 7 月 园 0 园 7
Estyuupuy4 ,#′′一j,r′一j,1吏得 E E
E 胺
[ 余 A

[ 李 前 = [

[ 沥 7 口
由 Poincare-Bendixson 环 域 定 理 , 扰 动 系 统 (3. 15) 至 少 存 在 个 极
限 环 .

E 才 [
E
个 极 限 环 . 若 不 然 , 则 对 任 意 的 a> 0 和 7 一 0 的 邺 域 7,(3. 15) 在
口 内 有 多 于 } 个 极 限 环 , 则 我 们 可 以 仿 照 上 面 的 方 法 选 取 _ 一 ,
rk-j-Ly “ 皋 叶 以 及 eroy 屋 而 在 口 内 再 获 得 古 外 的 & 一 3 个 极 限
E
0 |

定 理 3. 1 的 证 明 “ 当 一 1 时 , 可 从 定 理 3.6 得 到 定 理 3. 1.
E 3 s 林浩 日 ( 仪 2 招
一 阶 细 焦 点 , 因 此 定 理 3, 1 的 前 一 部 分 结 论 成 立 , 再 设 条 件 (Ho)
E 命河 应图 生 刑

Ed5 沥 多 庞 z_T〉 【e2525 [ 途
其 中

0 【 〔exp E 羞座〕 5 1〕十 d 述

冉
[
再 由 (3. 14 可 知
Page-201
医 脱 u

E

E 命 7 1 E 述 c

由 (3. 17) 和 条 件 (Ha) 可 知

E

蠢(O'O) 园 湃(O)挝(0) Eal t2E
E 存在疃〉0和在0<zl<疃定义的光滑函数″_
[

E [ E

至 此 , 定 理 3 1 的 结 论 (1) 得 证 . 为 了 证 明 结 论 (2, 从 (3. 20) 求 导 、
E

E

E
F +萜烬(z】) 二

E
E

[
E 及条件(H】) 可得

, 把 上 面 的 结 果 及 (3. 19) 代 人 (8. 21) 得 到

国 国 E
E r

冉 此 得 定 理 的 结 论 (22, 【
E 招 医
E 人
一 0 的 邻 域 口 , 使 得
E
个 极 限 环 , 当 Ree 一 0 (> 0 时 , 它 是 穗 定 ( 不 稳 定 ) 的 并 且 当 A
s 口
0
Page-202
第 二 辅 常 见 的 尸 郭 与 非 尿 部 分 盅

E 二 沥 沥 圆 胡 沥
e
[ 沥 p 4 李
E 加
E 吊 p

068 林 2 仪
E 怡 述
5 尘

【63 i 仪 2 羞彻戚)>伪 当 a“ci 丿 0 阡 , 羞慨删
E

0 圆 技
EaLep a 不
E 仪 73 技 10 27 2 述
[ 张 水 育 2 河 z 育 技
医 途 园 应 江 [

玟
e
蜇 萱鲨z'(0)0,0)荠0.

E 李
定 理 即 可 . 其 它 推 理 与 定 理 3. 1 的 证 明 相 同 。 〖

E 技 李
i

( 当 0 一 8 心 品 ,|K 一 KK81 一 a 且 aroi (K 一 AC8D) 一 0
E
定 ( 不 薹 定 ) 的 极 限 环

02 林
E

y
Page-203
. 78 第 二 章 常 见 的 尿 部 与 非 局 部 分 责

E 浩 ]

附 注 3.8 “ 在 应 用 定 理 3.6 时 , 霁 要 首 先 判 斯 未 扰 动 系 统 X
以 0 为 细 焦 点 的 阶 数 , 也 就 是 确 定 满 足 条 件 (3.5 的 & 在 实 际 计
算 时 , 经 常 应 用 下 面 介 绍 的 Liapunov 系 数 法 , 细 节 请 见 [ZDHD]
医

〈 E 之 圭 玟
E

E 伟 d
医 达

E 兽位2 吴 成 应 吴 人 画 江

医 述 2

E
蔷 = 针…掳 沥 江 园 [ 沥
E

E

漾 足 上 式 的 (V,} 称 为 (3. 16) 的 iapunov 系 数 . 在 下 面 定 理 的 意 义
e

E 八 北 沥 朐 圆 一 仁 , 一 0,8. 丿 0( 或 一 0), 当 且 仅 当
E 唐 5

定 理 的 证 明 见 [BL], 下 文 中 , 我 们 把 (} 或 (Re(e,} 称 为 系
统 的 焦 点 量 .

应 用

例 3. 10 “ 考 虑 二 维 系 统
壬 园 【
林 沥 吴 玟 江志 玟 口 招 玟 技 招 途 仪
E 振 人 i 林人
( 一 0 附 近 发 生 Hopf 分 岔 的 可 能 性 . 令 一 乙 十 1, 系 统 (3. 24
. 变 为
Page-204
E 85

在 第 三 章 $ 1 中 , 我 们 将 看 到 这 种 对 参 数 一 致 的 Hopf 分 岐 定
E

E 浩二 D

E 沥 a n 不
医 技 el 芸 认 水 国
上 * 这 种 连 接 鞍 点 的 扬 线 在 拼 动 下 可 能 破 裂 , 从 而 改 变 系 统 的 拓 扑
E
en

医

E 江 泊

盖=训z跚鹑 E

其中″eC镳mzXR,R2).设X。的轨线结构如第一章图 L9(b) 所
示 ! 它 有 一 条 初 等 鞍 点 的 间 宿 捉 , 内 部 是 穗 定 焦 点 的 吸 引 域
技 技
洛 形 , 见 图 1-9 的 (a) 与 Cc).

显 然 , 在 图 1-9(c) 的 情 形 , 分 界 线 破 瑞 的 方 向 与 砾 褐 前 了 的 穗
E 伟 王 刹
东
辅 线 . 因 此 , 我 们 可 以 认 为 闭 软 是 从 二 经 扰 动 破 裂 而 产 生 的 ( 或 反
B 人 l 运 标 t 技
atetueuupiakspott 刑

E

[ [
E 伟 c 日 顺 7 林林 扬
C
Page-205
E
E

红 2
E 江 达 0 述
根 沥 5 技 水 沥 逊 T 医
这 个 系 统 在 (0,0) 的 线 性 都 分 短 阵 有 一 对 纯 虚 特 征 根 的 条 件 为
浩 胡理 1 沥
医 林 d 渡 仪 一 仪 s 途 1 莲 2 李 浩7
d

d 沥 E
″ 75 河 玟

E 河 理 2 吴
E 沥 江

E 圭(的 玟

1
E
国 育 5

v…=訾仰 E
由 正 应 用 定 理 3.6, 可 得 下 列 结 论 :
E 李 u [ 林史 王 吊 不 不 吊
盆 , 并 东 系 统 (3. 24》 在 原 点 附 近 存 在 唯 一 极 限 环 的 参 数 区 域 是
E 月
0 化 2 儿 沥 沥 不 7 林丿 史 吊 不 c 永
Hopt 分 岔 ,(3,. 24) 在 原 点 附 近 存 在 二 个 极 限 环 的 参 数 区 域 是
E 朱应
D 厂 才 尸

0 坂 i d

[ E
E 昔#u [ 河 浩 二 伟 口 河

D 林ss

t 河
Page-206
E 第 二 辅 常 见 的 尸 郭 与 非 局 部 分 盅

形 (b) 可 类 似 讨 论 2.

1 吴 c

0

离 2-4 图 2-5

e
是 重 要 的 。

n
例 2. 10 中 , 这 是 利 用 7 内 的 焦 点 的 稳 定 性 得 出 来 的 . 我 们 帝 望 从

向 量 场 在 鞍 点 和 7 的 特 性 来 获 得 这 个 借 息 .

(3) 如 何 判 断 X, 的 稳 定 流 形 与 不 穗 定 流 形 的 相 互 位 置 ?

为 了 解 决 间 题 (1), 我 们 先 对 X 在 T 内 侧 引 入 Poincare 跚 射
[ 余 园 小 林 t
( 向 内 为 正 3. 则 在 吉 附 近 存 在 一 个 邻 域 , 使 得 V E 口 史 , 从
丶 出 发 的 辅 线 9 , 经 过 + 一 T 办 将 再 次 三 I 交 于 一 点 PCD) 一
E28t42 志 2 沥 俊 d 丞
0

丶 一 助 十 史 m, 0 河
河 t
E 逊 [
E 沥 园 水 技
E 河 仁
定 义 4. 1 乃 , 的 同 宿 转 T 称 为 是 渐 近 稳 定 ( 或 不 稳 定 ) 的 , 如
Page-207
第 二 章 常 见 的 局 郭 与 非 尾 部 分 岔

EXcu Y
E +陲) 命
c 八沙
E
E
的 一 种 标 准 形 式 导 出 了 著 名 的 焦 点 量 公 式 ( 见 [Ba], 并 证 明 了 二
技
数 值 上 都 有 误 , 在 [QL], 及 [FLLL] 中 得 到 纠 正 . 下 面 介 绍 的 结
E 沥
E 技
壬 2 E 缥舶翼z 十 anzy 十 ″叩燮z'
E 河 园 / 园
E
E
【 757 有 厂 木 272
E 述
则 ( 不 计 正 数 图 子 》
卯 一 ha 一 B,
【 01
[ E 口 2 振 EAgs
E 沥 王 孙 芒 u 刑
(3. 28)
在 最 后 一 种 情 形 下 , 系 统 可 积 ,(0,03 为 中 心 点
附 注 3. 12 从 原 则 上 说 , 当 X 以 0 为 细 焦 点 时 , 总 可 以 通 过
有 限 步 运 算 确 定 细 焦 点 的 阶 数 &. 但 是 当 较 大 时 , 对 一 舫 系 统 息

用 (3. 28) 的 方 式 来 表 达 焦 点 重 公 式 , 计 算 重 非 常 大 . 例 如 , 当 把 方
Easetoakypsp p

阮
Page-208
E E

沥
0 英

恤n捌 [

P

所决定:当H辖耀 E t 怡 a 月 也就是b鹫障 E 吊 李 市 东

[
E

E 噩(0'O)l

E 东 2 t tn
EuEs b p ag 芸 i
E g 河
Poincare 映射
P: 厂 研 一 加 , PCB) 二 8(T(B, 切 ,
E 朋
E

e

园 伟 7 朋 d 5
/′ 7 ″。_卜a棘 【_『(′】”0, 5

E
E
E

d
f 口 - 《 疫 一

如 果 记
羞 林

E

E 沥 人 仪
Page-209
第 一 竟 腐 见 的 尿 部 与 非 屉 部 分 盆

E E
D
E
E 命 玲 2 2
其 中 a 息 一 0 当 a 0. 男 一 方 面 , 由 于
E 人 河 玟 w
技
E

欣 国
E 【_TU)/″ 交 国

E

园
ag :

才

[l E 河 玟 薯]
E i

E
deta'> ro 一 胡 玟 江 人 D

其 次 , 我 们 来 计 算 上 步 左 端 的 行 列 式 ( 它 与 坐 标 系 的 选 联 无 关 ). 注
意 怡 F(t,) 是 变 分 方 程
琶 5 垩(P E0222D
a 东
E 亘P 〔妻y/>)〕 二 exp I二n 暴/砷 【 沥 d

E
由 C4. 7),C4 11) 和 (4 12》 最 后 得 出
Page-210
[ 颖 水 2 5

的 焦 点 量 公 式 十 分 复 杂 ; 即 使 利 用 计 算 机 , 目 前 也 仅 推 导 出 前 儿 个
E
点 这 个 固 难 阿 题 紧 密 相 关 , 在 这 里 我 们 仅 列 举 我 国 学 者 在 这 方 面
E 吊 t 招
中心和焦点的较]般方法;杜乃林、曾宪武圃〕给出了计算焦点童
E 沥 沥
沥
黄 文 灵 “ 证 明 , 当 非 线 性 方 程 零 点 的 拓 扑 度 变 名 时 , 会 产 生 述 通
的 分 岔 曲 线 , 等 等

E 些

【 xta 2 怀 e 李 d 吊 2
其 中 口 是 R: 中 原 点 的 一 个 开 集 , 则 当 条 件 (H),CH) 和 (He) 成
立 时 , 或 者 系 统 (3.1) 当 一 0 时 以 原 点 为 中 心 , 或 者 当 AE ( 一 ,
E
d
E i 沥 t 东 p 浩
E

例 3. 14

羞 国 八 述 0 吴

【
0

s 标 班2 述 月

[ 朋

【
E 罚 0

a u 林s 一 国
u
d′_

L 匹
F E 浩 河 玟 薯) g
E

医 沥
Page-211
5 第 二 章 帝 见 的 尸 部 与 非 尿 部 分 盅

其 中 0 一 r 一 1. 故 原 点 是 (3. 29)。 的 渐 近 稳 定 焦 点
E

1
【 江 sinr一菩)'

图 2-2

t 林 e一睾与 E 3e一丢之间(见图
E 由于

巳 技 E 一
园 言c鹏r 詹

i
汀 东 t河 才 年 9
0 3 技 [ d′【(″〉 E 取 巳 刑

E 29〉″在原点附近都至少有两条闭轨′因

E ss 途 伟 沥
请 凄 者 验 算 , 此 例 中 对 一 切 正 整 数 A, 都 有 Recx 二 0. 团 此 , 对

【omestsrs 2 吊 e 怡

对 参 数 一 致 的 Hopf 分 岔 定 理
E
Page-212
E i E

3
8「Cm = 卜 Eexp 〔鲈槽圣八似岫刑池 0

E 晕八趴m b en

E

E 沥 0 5
E
E 一 厂 芒
E 不
由 附 注 4 2 立 得 定 理 的 结 论 , 〖
2

怀

u 一 唐 排
E 一 技
闻 ′
E 个
门
E
E 人 伟 切
E c 一
Poinc墓【e 映射P″= 。
E 吊 cn y 页 c 刀 用
技 口
一
武
、 0 当 o 一 0,
E
E 逊
Page-213
E 脱 u

巳 河 国 匹 ,
E 沥

@ 张
E 林 丞

其 中 函 数 双 一 友 (z,y7, 参 数 3,xE R「, 丁 $ 为 小 参 数 . 设 系 统 有
d
E d
【5) E

则 在 一 定 的 附 加 条 件 下 , 当 一 x(8, 8 人 (0,o 时 , 系 统 (3. 32》
E t 沥
3.7 可 知 , 存 在 (8) 丿 0, 使 得 当 |x 一 A(8)| 一 e(8 且 人 丿
E
E 沥 0 吊 e
s, 使 得 对 所 有 的 E (0,8), 都 有 e(8 之 e 这 就 是 所 谓 对 参 数
g

[ 沥

E 林 振 芸 s 怀

目 国 园 伟 芒 97 扬
[ 国 P

国
Page-214
筝 二 章 常 见 的 尾 部 与 非 尾 部 分 盆

古 -- 方 面 , 两 个 具 有 相 同 稳 定 性 的 闭 技 不 可 能 并 列 共 存 , 因 此
E 不 朔 ]

现 在 , 茵 下 要 解 决 的 就 是 我 们 在 前 面 所 提 的 间 题 (2) , 即 X 的
Dk 棣
形 X 的 相 对 位 置 ? Melnikov 函 数 就 是 用 以 措 述 P; 和 x 之 间
a 才

E

d
[ 光 D

E 伟 伟 XR瓜饥r>2' E 东
E
E E 仪 月 国 余 人 c

E 庞 t [″`] 二 [重_],定义

E
[
〉 a

容 易 验 证 , 对 于 任 意 二 阶 方 阵 4 , 有
L 述
过 卯 上 的 (0) 点 取 截 线 , 使 它 沿 法 向 a(03 一 (p「(0))
E 2 林 挂 水 年 江 招
y 浩
a 林
E 怀 l 林 2
t en
EzuaRspaeusltey 河 a a e 间 * 缴
E
E 5 园
6
Page-215
4 平 国 上 的 合 宿 分 盅

五 (p(6) 8tCPCC),0)
E 仪
E

E 技 2

E 丘钛 寰′〈霄 【 E

E 东 河 技 纳 玖 芸 一 水 水 水
E 沥 不

502 河 人 干 一 胡 0
E

E 亡岗D(霉)e一′“〉d′, 0

[ 李 1 朐
E 国 胡

E
5

E
52
l 志 1
E

注 意 印 ;(4) 是 (4.14) 的 解 ( 丿 0), 把 它 代 人 (4. 143, 对 求 导 后 取

E
林

[

目
E 林

E

E 32 沥 钨/「(矽(壹〉〉
Page-216
第 二 辅 常 见 的 尸 郭 与 非 尿 部 分 盅

E 二 沥 沥 圆 胡 沥
e
[ 沥 p 4 李
E 加
E 吊 p

068 林 2 仪
E 怡 述
5 尘

【63 i 仪 2 羞彻戚)>伪 当 a“ci 丿 0 阡 , 羞慨删
E

0 圆 技
EaLep a 不
E 仪 73 技 10 27 2 述
[ 张 水 育 2 河 z 育 技
医 途 园 应 江 [

玟
e
蜇 萱鲨z'(0)0,0)荠0.

E 李
定 理 即 可 . 其 它 推 理 与 定 理 3. 1 的 证 明 相 同 。 〖

E 技 李
i

( 当 0 一 8 心 品 ,|K 一 KK81 一 a 且 aroi (K 一 AC8D) 一 0
E
定 ( 不 薹 定 ) 的 极 限 环

02 林
E

y
Page-217
第 二 章 常 见 的 局 部 与 非 尿 部 分 会

E [tr 巫胖)4…(Z) 45X8 吊 0 标 62
E
E e′(′)[〈廖(0〉 E ]′二e′′(薹) E 林2
0
男 一 方 面 , 因 为 t 一 co 时 ( 史 一 zo, 所 以 p (D) 一 廿 zo) 一 0,
E 林
E
界 线 兆 滑 依 赖 于 参 数 的 定 理 可 知 , 当 f 芸 0 时 办 ( 有 界 . 从 而 由
[ 垩位山则易知
林
E

E

从 而 由 (4. 23) 式 及 上 式 得 到
E
E in I" E

[

E E

5

队 (4. 167,C4 20) 那 (4 21) 易 知 : Q(0) 因 0
E 沥 2 圆 |

由 定 理 4 8, 定 理 4. 4 和 定 理 4 5 立 即 得 到

定 理 4.6 设 X, 由 (4- 14) 给 定 ,X 以 z 为 双 曲 鞍 点 , 昆 有 顺
E
E

( 若 j 丿 0 ( 或 一 03, 则 乙 , 在 P 的 了 邻 域 内 恰 有 一 个
从 卯 分 盆 出 的 极 限 环 . 当 r 一 0 时 , 它 是 稳 定 的 ! 当 o 之 0 时 , 它
是 不 稳 的 .
Page-218
E 85

在 第 三 章 $ 1 中 , 我 们 将 看 到 这 种 对 参 数 一 致 的 Hopf 分 岐 定
E

E 浩二 D

E 沥 a n 不
医 技 el 芸 认 水 国
上 * 这 种 连 接 鞍 点 的 扬 线 在 拼 动 下 可 能 破 裂 , 从 而 改 变 系 统 的 拓 扑
E
en

医

E 江 泊

盖=训z跚鹑 E

其中″eC镳mzXR,R2).设X。的轨线结构如第一章图 L9(b) 所
示 ! 它 有 一 条 初 等 鞍 点 的 间 宿 捉 , 内 部 是 穗 定 焦 点 的 吸 引 域
技 技
洛 形 , 见 图 1-9 的 (a) 与 Cc).

显 然 , 在 图 1-9(c) 的 情 形 , 分 界 线 破 瑞 的 方 向 与 砾 褐 前 了 的 穗
E 伟 王 刹
东
辅 线 . 因 此 , 我 们 可 以 认 为 闭 软 是 从 二 经 扰 动 破 裂 而 产 生 的 ( 或 反
B 人 l 运 标 t 技
atetueuupiakspott 刑

E

[ [
E 伟 c 日 顺 7 林林 扬
C
Page-219
E 5

E 坤 c 沥 2 en 志
E
n
Eacesuesl 招 是定号的 E
l 兰
E 江
【 E 述 达 2
3 一 QCzo37,

以原点为双曲鞍点y有顺时针定向的同宿轨风,并且姚=茎m胁

E 鬟m涮 E
{ 0
小 一 @Q(z,3) 十 户 PCzo9)
在 巩 的 小 邻 域 内 恰 有 一 个 极 限 环 ( 其 稳 定 性 由 oo 的 符 导 决 定 ) 而

当 jo 一 0 时 ,(4 24) 在 P 附 近 没 有 极 限 环 .
事 实 上 ,

国
. D(灞)=de【〔Q 申 〕=(尸z十Q【〉」=70)>0'

阮 余 因此4>0 利 用 定 理 4. 6, 上 而 的 结 .
论 立 即 可 得
E

E
分 盅 . RoussariePl 和 Joyal00 分 别 讨 论 了 从 ( 逾 化 的 同 宿 执 分 岔
出 多 个 闭 软 的 闰 题 . 他 们 的 基 本 怡 想 是 在 奇 点 附 近 利 用 鞍 点 性 质 。
与 大 范 图 的 微 分 同 胚 相 结 合 , 得 出 Poineark 映 射 的 表 达 式 , 从 而 在
逃 化 程 度 较 高 时 , 可 以 经 过 逛 次 适 当 的 扰 动 , 反 复 改 变 同 宿 轨 内 侧
bnoeaet dn 沥
D 沥 d 述 2 河
E t
Page-220
E 第 二 辅 常 见 的 尸 郭 与 非 局 部 分 盅

形 (b) 可 类 似 讨 论 2.

1 吴 c

0

离 2-4 图 2-5

e
是 重 要 的 。

n
例 2. 10 中 , 这 是 利 用 7 内 的 焦 点 的 稳 定 性 得 出 来 的 . 我 们 帝 望 从

向 量 场 在 鞍 点 和 7 的 特 性 来 获 得 这 个 借 息 .

(3) 如 何 判 断 X, 的 稳 定 流 形 与 不 穗 定 流 形 的 相 互 位 置 ?

为 了 解 决 间 题 (1), 我 们 先 对 X 在 T 内 侧 引 入 Poincare 跚 射
[ 余 园 小 林 t
( 向 内 为 正 3. 则 在 吉 附 近 存 在 一 个 邻 域 , 使 得 V E 口 史 , 从
丶 出 发 的 辅 线 9 , 经 过 + 一 T 办 将 再 次 三 I 交 于 一 点 PCD) 一
E28t42 志 2 沥 俊 d 丞
0

丶 一 助 十 史 m, 0 河
河 t
E 逊 [
E 沥 园 水 技
E 河 仁
定 义 4. 1 乃 , 的 同 宿 转 T 称 为 是 渐 近 稳 定 ( 或 不 稳 定 ) 的 , 如
Page-221
E E tn

d 河
E

E c
厂
叶 口 得 出 了 在 临 界 情 形 下 判 别 同 宿 转 或 异 宿 扬 的 穗 定 性 的 方 法 ;
Mourtada““ 对 含 两 个 鞍 点 的 异 宿 环 的 分 岔 问 题 进 行 了 深 入 的 研
E ,

对 参 数 一 致 的 同 宿 分 岔

类 似 于 对 参 数 一 致 的 Hopt 分 岔 问 题 , 现 在 考 虑 含 双 参 数 8
E

医 E
盂r 5 2

4 E
亩萱 国 诊 E 述

Eaai i 不
E
E
的 同 宿 技 7s. 则 在 适 当 的 条 件 下 , 对 每 一 个 国 定 的 $ > 0, 存 在
s(8 友 0, 例 当 |x 一 p(8D| 人 e(8) 时 , 疚 「 的 邻 域 内 有 定 理 4. 6
的 丽 条 结 论 , 我 们 关 心 的 是 : 当 8->0 时 , 如 何 保 证 s(8) 不 趋 于 零 ,
E

我 们 不 在 止 给 出 一 舫 的 定 理 , 只 在 第 三 章 引 理 1.6 中 对 一 类
特 殊 的 系 统 介 绍 这 种 对 参 数 一 致 的 同 宿 分 岗 的 结 果 .

0

5 e 梁

河 述

[ 儿
Page-222
E E

沥
0 英

恤n捌 [

P

所决定:当H辖耀 E t 怡 a 月 也就是b鹫障 E 吊 李 市 东

[
E

E 噩(0'O)l

E 东 2 t tn
EuEs b p ag 芸 i
E g 河
Poincare 映射
P: 厂 研 一 加 , PCB) 二 8(T(B, 切 ,
E 朋
E

e

园 伟 7 朋 d 5
/′ 7 ″。_卜a棘 【_『(′】”0, 5

E
E
E

d
f 口 - 《 疫 一

如 果 记
羞 林

E

E 沥 人 仪
Page-223
第 一 竟 腐 见 的 尿 部 与 非 屉 部 分 盆

E E
D
E
E 命 玲 2 2
其 中 a 息 一 0 当 a 0. 男 一 方 面 , 由 于
E 人 河 玟 w
技
E

欣 国
E 【_TU)/″ 交 国

E

园
ag :

才

[l E 河 玟 薯]
E i

E
deta'> ro 一 胡 玟 江 人 D

其 次 , 我 们 来 计 算 上 步 左 端 的 行 列 式 ( 它 与 坐 标 系 的 选 联 无 关 ). 注
意 怡 F(t,) 是 变 分 方 程
琶 5 垩(P E0222D
a 东
E 亘P 〔妻y/>)〕 二 exp I二n 暴/砷 【 沥 d

E
由 C4. 7),C4 11) 和 (4 12》 最 后 得 出
Page-224
0 颜 E

E 伟 t 沥 砂 沥 t 沥 技 水 E
环 基 , 即 存 在 一 系 列 闭 辅

c 不 有
E 一
s 不
t 一 芸
Poincare 分 岔 问 题 .

E

河 沥
列 ( 当 X 为 Hamilton 系 统 , 且 一 为 相 应 的 Hamilton 函 数 时 , 这 个
E 人 林c 一
化 . 设
E 仪 06 沥
其 中 T 是 7 的 周 期 为 了 考 察 当 x 一 0 时 丶 , 过 p (0,4 的 解 能
E i 沥 t
E 水
户 内 一 8 (0,40- 设 此 解 当 一 T 时 再 次 与 工 相 交 , 则 由 微 分
E
( 命 07 二 日 ( 工 (,0) 二 T 0
定 义 后 继 函 数
02 一 3
E 沥
u 江
且 G E Cr
江 d 林 莲 租 5
期 , 可 以 从 (5. 4 得 到
E 月 0
Page-225
E i E

3
8「Cm = 卜 Eexp 〔鲈槽圣八似岫刑池 0

E 晕八趴m b en

E

E 沥 0 5
E
E 一 厂 芒
E 不
由 附 注 4 2 立 得 定 理 的 结 论 , 〖
2

怀

u 一 唐 排
E 一 技
闻 ′
E 个
门
E
E 人 伟 切
E c 一
Poinc墓【e 映射P″= 。
E 吊 cn y 页 c 刀 用
技 口
一
武
、 0 当 o 一 0,
E
E 逊
Page-226
E 第 二 章 常 见 的 居 部 与 非 局 部 分 岔

[ 寸 胡 刑
[ (5.5)

E 东 2
E

[62 怀 汀 2 才 s标

0 东 沥
Eurza elugts 命 p 王
玖
ECLE E 4 不

E 国 08 怀 2 p 谅
[ 技
胡 园 [ p 一 发 口 5 一 育 i

[ 芸 2 述 林前
E 4 林i 训 育
E
E e
n
吴

E 发 东 江 刑 昙?(廖my#m) 二 0.

沥

为 了 实 际 应 闵 的 方 便 , 下 面 的 定 理 给 出 从 7 的 表 达 式 与 原 方
技

定 理 5.2 “ 对 于 方 程 (5. 1) 和 7「 的 表 达 式 (5. 2), 有

0
E L t 沥 林 2 车
E

『
c 二 L r 数 ( Gp)d
Page-227
E 沥 颖 3 江

E 圆 育 2
E 且G伪 卢)|

″【。

〈[az窃r(″,″)

E 国 上
珈)L动 E

E

解 , 利 用 (5. 3 及 T 为 T 的 周 期 可 得

医 E 2 林 2
E
颜 3

国 沥 E 达
0
着 令
2 〈噩亿加腻 交 Gp 0,m7》

一 噩化加m A ACp 0
园
E y 林 3 T 月 [
与 $ 中 (4. 23) 式 的 推 导 相 类 似 , 可 得 A(,4) 的 表 达 式 如 下

e′【“″)[4〈o,^) 十I二e一″“'″〉g(哽 【 22

E
E
[

E 余 北 沥 朐 胡 沥
[
E 为喜阶Md墟嶂。v函量,从它的零真分布可研究扰动
E 怀 水 江
Page-228
筝 二 章 常 见 的 尾 部 与 非 尾 部 分 盆

古 -- 方 面 , 两 个 具 有 相 同 稳 定 性 的 闭 技 不 可 能 并 列 共 存 , 因 此
E 不 朔 ]

现 在 , 茵 下 要 解 决 的 就 是 我 们 在 前 面 所 提 的 间 题 (2) , 即 X 的
Dk 棣
形 X 的 相 对 位 置 ? Melnikov 函 数 就 是 用 以 措 述 P; 和 x 之 间
a 才

E

d
[ 光 D

E 伟 伟 XR瓜饥r>2' E 东
E
E E 仪 月 国 余 人 c

E 庞 t [″`] 二 [重_],定义

E
[
〉 a

容 易 验 证 , 对 于 任 意 二 阶 方 阵 4 , 有
L 述
过 卯 上 的 (0) 点 取 截 线 , 使 它 沿 法 向 a(03 一 (p「(0))
E 2 林 挂 水 年 江 招
y 浩
a 林
E 怀 l 林 2
t en
EzuaRspaeusltey 河 a a e 间 * 缴
E
E 5 园
6
Page-229
4 平 国 上 的 合 宿 分 盅

五 (p(6) 8tCPCC),0)
E 仪
E

E 技 2

E 丘钛 寰′〈霄 【 E

E 东 河 技 纳 玖 芸 一 水 水 水
E 沥 不

502 河 人 干 一 胡 0
E

E 亡岗D(霉)e一′“〉d′, 0

[ 李 1 朐
E 国 胡

E
5

E
52
l 志 1
E

注 意 印 ;(4) 是 (4.14) 的 解 ( 丿 0), 把 它 代 人 (4. 143, 对 求 导 后 取

E
林

[

目
E 林

E

E 32 沥 钨/「(矽(壹〉〉
Page-230
E 第 二 章 常 见 的 局 部 不 非 局 郭 分 岔

E 河 述 述
医
i 怡 连 5

E 林 5 李 E 才 沥

垂 实 际 应 用 中 , 经 常 出 现 (5. 1) 的 一 种 特 殊 形 式 , 即 Hamitton
D 仪

巳 李
E

4 E
董 国 芊 E 达

口
“ 之 2. 设 当 仁 一 0 财 , 未 扰 动 系 统 有 闭 软 族 , 并 有 表 达 式
EHC5 1 5 玖 有 河 c 月
E 吴 李 途
E 圣r^[菩亘P E 箐Q [ [ 沥

0

注 意 当 x 二 0 时 , 波 有
E 丞
E 沥 7 林沥
图 此 , 可 把 (5. 11) 改 写 为

E 姿′ QCzoyvG, 吊 dz 一 PCr,y8,bQy,《S.12》
^

玟
E 江 不

L 八 东 坂 国 c 江 沥 河 芸 一 吴
王
的 零 点 个 数 间 题 称 为 骏 Hilbert 第 16 问 题 . 由 于 这 个 问 题 是 V. 工
阮
Page-231
第 二 章 常 见 的 局 部 与 非 尿 部 分 会

E [tr 巫胖)4…(Z) 45X8 吊 0 标 62
E
E e′(′)[〈廖(0〉 E ]′二e′′(薹) E 林2
0
男 一 方 面 , 因 为 t 一 co 时 ( 史 一 zo, 所 以 p (D) 一 廿 zo) 一 0,
E 林
E
界 线 兆 滑 依 赖 于 参 数 的 定 理 可 知 , 当 f 芸 0 时 办 ( 有 界 . 从 而 由
[ 垩位山则易知
林
E

E

从 而 由 (4. 23) 式 及 上 式 得 到
E
E in I" E

[

E E

5

队 (4. 167,C4 20) 那 (4 21) 易 知 : Q(0) 因 0
E 沥 2 圆 |

由 定 理 4 8, 定 理 4. 4 和 定 理 4 5 立 即 得 到

定 理 4.6 设 X, 由 (4- 14) 给 定 ,X 以 z 为 双 曲 鞍 点 , 昆 有 顺
E
E

( 若 j 丿 0 ( 或 一 03, 则 乙 , 在 P 的 了 邻 域 内 恰 有 一 个
从 卯 分 盆 出 的 极 限 环 . 当 r 一 0 时 , 它 是 稳 定 的 ! 当 o 之 0 时 , 它
是 不 稳 的 .
Page-232
E 沥 额 X 公 训 异 河 沥 大 5

年 ,D, Hilbert00 在 第 二 居 国 际 数 学 家 大 会 上 提 出 了 23 个 数 学 间
E t
极 限 环 的 最 小 上 界 HCm 是 多 少 ? 可 能 出 现 的 极 限 环 相 对 位 置 如
何 ? 近 一 个 世 纪 以 来 * 特 别 是 最 近 儿 个 年 来 , 出 现 了 大 量 的 工 作 .
例 如 , 史 松 龄 ““ 和 陈 兰 草 , 王 明 淑 “w 最 先 分 别 举 例 证 明 HC2) 之
的 李 继 彬 \ 黄 其 明 0 举 例 证 明 HC3) 丿 11; 叶 彦 谦 、 陈 兰 茹 和 杨 信
玟
证 明 二 次 多 项 式 系 统 中 按 叶 彦 谦 分 类 的 ( ! ) 类 方 程 至 多 有 一 个
E 李 育 刑
光 的 综 述 文 章 [Cs] 和 [CZJ, 马 知 恩 的 专 著 [Mz], 梁 肇 军 的 专 著
[Lz], 以 及 Dumortier 等 人 的 系 列 文 章 [DRR1,2] 和 [DER] 中 , 读
者 可 发 现 大 量 有 趣 的 绪 果 、 经 过 IIyashenkom 和 Eealle02 修 补 证
明 后 的 Dulaem 有 限 性 定 理 指 出 : 一 个 绘 定 的 n 许 多 项 式 系 统 的 极
限 环 个 数 有 限 . 但 是 , 对 全 体 n 次 多 项 式 系 统 而 言 , 其 极 限 环 个 数
的 一 致 上 界 如 何 佼 计 ( 郅 怕 是 否 有 限 , 即 使 对 n 一 2 这 种 最 简 单 的
非 线 性 情 形 , 仍 是 一 个 未 知 的 问 题 . S.Smalerem 认 为 , 对 HGo 的
e
作 者 的 一 个 重 大 拼 战

由 (5. 5) 式 可 知 ,Abel 积 分 (5. 12) 是 系 统 (5. 10) 在 7 附 近 后
E 圆 2
的 极 限 环 个 数 密 切 相 关 . 当 又 是 n 十 ! 次 多 项 式 P 和 Q 为 次 多
项 式 时 ,(5. 10) 是 一 类 特 殊 的 n 汀 系 统 , 即 Hamilton 向 量 场 的 扰 动
E
E t ss
E a
的 形 式 . 此 阡 及 ,P,@ 可 能 不 再 是 多 项 式 . 习 惯 上 仍 把 (5. 12) 称 为
E

例 5. 5 考 虑 van der Pol 方 程
Page-233
E 5

E 坤 c 沥 2 en 志
E
n
Eacesuesl 招 是定号的 E
l 兰
E 江
【 E 述 达 2
3 一 QCzo37,

以原点为双曲鞍点y有顺时针定向的同宿轨风,并且姚=茎m胁

E 鬟m涮 E
{ 0
小 一 @Q(z,3) 十 户 PCzo9)
在 巩 的 小 邻 域 内 恰 有 一 个 极 限 环 ( 其 稳 定 性 由 oo 的 符 导 决 定 ) 而

当 jo 一 0 时 ,(4 24) 在 P 附 近 没 有 极 限 环 .
事 实 上 ,

国
. D(灞)=de【〔Q 申 〕=(尸z十Q【〉」=70)>0'

阮 余 因此4>0 利 用 定 理 4. 6, 上 而 的 结 .
论 立 即 可 得
E

E
分 盅 . RoussariePl 和 Joyal00 分 别 讨 论 了 从 ( 逾 化 的 同 宿 执 分 岔
出 多 个 闭 软 的 闰 题 . 他 们 的 基 本 怡 想 是 在 奇 点 附 近 利 用 鞍 点 性 质 。
与 大 范 图 的 微 分 同 胚 相 结 合 , 得 出 Poineark 映 射 的 表 达 式 , 从 而 在
逃 化 程 度 较 高 时 , 可 以 经 过 逛 次 适 当 的 扰 动 , 反 复 改 变 同 宿 轨 内 侧
bnoeaet dn 沥
D 沥 d 述 2 河
E t
Page-234
英

其 中 0 一 |x| 奶 1. 可 把 它 改 写 为 如 下 的 等 价 形 式

[
FPgd

蒜=_z十粼1一菠)丛
述

T e E 北
或 写 成 参 数 方 程

0

【 招
G

国 吴

E I。 E z肋z〔 E T)′

显 然 & 一 2 是 它 的 唯 - 正 零 点 , 并 东 是 简 单 零 点 . 由 定 理 5.1, 当

r 洁

l
E

但 在 多 数 情 况 下 问 题 并 不 如 此 轻 而 易 举 , 请 看 下 例 .
0 一

E
【

E 月

E 坂 标 林
Hamilton 系 统 , 它 有 首 次 积 分

3
″(雾黜)=音+Z_言 0 [

t
E 育

E
Page-235
E E tn

d 河
E

E c
厂
叶 口 得 出 了 在 临 界 情 形 下 判 别 同 宿 转 或 异 宿 扬 的 穗 定 性 的 方 法 ;
Mourtada““ 对 含 两 个 鞍 点 的 异 宿 环 的 分 岔 问 题 进 行 了 深 入 的 研
E ,

对 参 数 一 致 的 同 宿 分 岔

类 似 于 对 参 数 一 致 的 Hopt 分 岔 问 题 , 现 在 考 虑 含 双 参 数 8
E

医 E
盂r 5 2

4 E
亩萱 国 诊 E 述

Eaai i 不
E
E
的 同 宿 技 7s. 则 在 适 当 的 条 件 下 , 对 每 一 个 国 定 的 $ > 0, 存 在
s(8 友 0, 例 当 |x 一 p(8D| 人 e(8) 时 , 疚 「 的 邻 域 内 有 定 理 4. 6
的 丽 条 结 论 , 我 们 关 心 的 是 : 当 8->0 时 , 如 何 保 证 s(8) 不 趋 于 零 ,
E

我 们 不 在 止 给 出 一 舫 的 定 理 , 只 在 第 三 章 引 理 1.6 中 对 一 类
特 殊 的 系 统 介 绍 这 种 对 参 数 一 致 的 同 宿 分 岗 的 结 果 .

0

5 e 梁

河 述

[ 儿
Page-236
[ 圆

为 边 界 . 闭 软 族 为
【 河 {(工'汊)I″(z,汊) E 昙 <矗<鲁}'

E

利 用 定 理 5.1, 定 理 5.2 和 (5. 12), 为 了 研 究 当 0 一 14| 丿 1 时
n

E 招 蛋r [ 命 2 【

E 一
E

n

Picard-Fuchs 方 程 法 ( 例 如 , 见 [CS]), 和 我 们 最 近 得 到 的 直 接 方
E

Pieard-Fuehs 方 程 法

i
E 标 玑 0
E
Page-237
第 二 章 常 见 的 尾 邢 与 非 尾 部 分 盅

E 林 诊 河 3 人
E 沥 沥

故 可 定 义
2

E
E
T
E 诊 晓 5SCD 为 工 所 图 成 的 紧 区 域 的 面

一 (h) 二

0 述
E 2
定 理 5.7 (1〉当一兽〈颢〈兽时,亭〈P仇)〈h且

^苎瑗填P(克) 河

沥 E 国 一 着 5
152 3 E 3 E 版 政 史 且^翌珥'…P 0
D

许
E 「
E 丧 育 水
医 移 u p 林 n
引 理 5.8 P( 满 足 如 下 的 Riceati 方 程
n 【
E

E
Page-238
0 颜 E

E 伟 t 沥 砂 沥 t 沥 技 水 E
环 基 , 即 存 在 一 系 列 闭 辅

c 不 有
E 一
s 不
t 一 芸
Poincare 分 岔 问 题 .

E

河 沥
列 ( 当 X 为 Hamilton 系 统 , 且 一 为 相 应 的 Hamilton 函 数 时 , 这 个
E 人 林c 一
化 . 设
E 仪 06 沥
其 中 T 是 7 的 周 期 为 了 考 察 当 x 一 0 时 丶 , 过 p (0,4 的 解 能
E i 沥 t
E 水
户 内 一 8 (0,40- 设 此 解 当 一 T 时 再 次 与 工 相 交 , 则 由 微 分
E
( 命 07 二 日 ( 工 (,0) 二 T 0
定 义 后 继 函 数
02 一 3
E 沥
u 江
且 G E Cr
江 d 林 莲 租 5
期 , 可 以 从 (5. 4 得 到
E 月 0
Page-239
[ 颜 25 招 水 2

0
E E
4605 2

E

R 才 沥 仪 朋
林林
【

古 -- 方 面 , 利 用 分 部 积 分 可 得
口

【 不 水
[ 月 ( 1

E

E

{ 伟河 2
[

E 河 标 述 0
E
t 朱3
E

[ 訾岫 E 绍
[C 7 7 誓盯「

把 (5. 27) 代 入 P 二 _』, 即 得 (5.22), 』
定 理 5.7 的 证 明 把 (5. 22) 改 写 为 系 统
F 【 沥 人 沥 盖=一 9 十 4 《5. 28)

E s paEE 相 g 葛 口 y 才 刑
P仇〉咔h当力令一鲁因此汨=P仇〉的图形是从系统佤2盼的

l 一 朐
Page-240
E

的 水 平 等 斜 线 由 方 程 7P: 十 3RP 一 5 一 0 给 出 , 它 的 图 形 为 双 曲
医 政 述

E 贵〔′ E 林 0

显 然 P 一 P(4) 的 图 形 与 乜 二 &( 的 国 形 除 了 两 奇 点 外 不 可 能 相
E

巳 吴 c 半,
E E

0
L 一 一
的 图 形 上 方 , 从 而 沿 P 一 P 有 P (D 一 0 lian 万 ( 一 一 “ 可
E

直 接 方 法

为 了 判 断 Abel 积 分 的 零 点 个 数 , 目 前 使 用 的 方 法 大 都 要 经 过
曲 折 的 推 导 . 下 面 介 绍 一 个 可 在 一 定 条 件 下 从 原 方 程 判 别 的 直 接
E

04 颖 形 i u 半 2 着 ) 沥
Page-241
E 第 二 章 常 见 的 居 部 与 非 局 部 分 岔

[ 寸 胡 刑
[ (5.5)

E 东 2
E

[62 怀 汀 2 才 s标

0 东 沥
Eurza elugts 命 p 王
玖
ECLE E 4 不

E 国 08 怀 2 p 谅
[ 技
胡 园 [ p 一 发 口 5 一 育 i

[ 芸 2 述 林前
E 4 林i 训 育
E
E e
n
吴

E 发 东 江 刑 昙?(廖my#m) 二 0.

沥

为 了 实 际 应 闵 的 方 便 , 下 面 的 定 理 给 出 从 7 的 表 达 式 与 原 方
技

定 理 5.2 “ 对 于 方 程 (5. 1) 和 7「 的 表 达 式 (5. 2), 有

0
E L t 沥 林 2 车
E

『
c 二 L r 数 ( Gp)d
Page-242
E 沥 颖 3 江

E 圆 育 2
E 且G伪 卢)|

″【。

〈[az窃r(″,″)

E 国 上
珈)L动 E

E

解 , 利 用 (5. 3 及 T 为 T 的 周 期 可 得

医 E 2 林 2
E
颜 3

国 沥 E 达
0
着 令
2 〈噩亿加腻 交 Gp 0,m7》

一 噩化加m A ACp 0
园
E y 林 3 T 月 [
与 $ 中 (4. 23) 式 的 推 导 相 类 似 , 可 得 A(,4) 的 表 达 式 如 下

e′【“″)[4〈o,^) 十I二e一″“'″〉g(哽 【 22

E
E
[

E 余 北 沥 朐 胡 沥
[
E 为喜阶Md墟嶂。v函量,从它的零真分布可研究扰动
E 怀 水 江
Page-243
[ 圆 2 异 河 八 105

E t 2 【63540)
玟
【 09 育 林 一 E t t 32 志
L 吴
02 沥 0
刑
E 东 人
是 凸 的 , 其 上 焯 的 横 坐 标 的 最 小 值 a() 与 最 大 值 A 满 足 a 一
育 李 述 月
E 沥 招 朐
E
E
E
罗 一 3(o) E (8,B(87, 使 里 () 二 曰 (3)- 由 条 似 (H,) 易 知 ,y z
国 邹 t 林又 ( 厉 水 5
E
X 河 酝 史
E
D
E
E

E

4

E 出 林0 林 诚 仪 河 小 招
【 5 河 c 志 仪 2
[ 林沥
园 达 厂

Ae3i 吴 扬 3 伟 3

5 E c 沥 c
Page-244
E 第 二 章 常 见 的 局 部 不 非 局 郭 分 岔

E 河 述 述
医
i 怡 连 5

E 林 5 李 E 才 沥

垂 实 际 应 用 中 , 经 常 出 现 (5. 1) 的 一 种 特 殊 形 式 , 即 Hamitton
D 仪

巳 李
E

4 E
董 国 芊 E 达

口
“ 之 2. 设 当 仁 一 0 财 , 未 扰 动 系 统 有 闭 软 族 , 并 有 表 达 式
EHC5 1 5 玖 有 河 c 月
E 吴 李 途
E 圣r^[菩亘P E 箐Q [ [ 沥

0

注 意 当 x 二 0 时 , 波 有
E 丞
E 沥 7 林沥
图 此 , 可 把 (5. 11) 改 写 为

E 姿′ QCzoyvG, 吊 dz 一 PCr,y8,bQy,《S.12》
^

玟
E 江 不

L 八 东 坂 国 c 江 沥 河 芸 一 吴
王
的 零 点 个 数 间 题 称 为 骏 Hilbert 第 16 问 题 . 由 于 这 个 问 题 是 V. 工
阮
Page-245
E

国 沥 265
c 力 Z 1

E 吊 2 林8 技 ( 技

定 理 5.9 ([LZJ〉 设 玑 (z,y) 具 有 (5. 30) 的 形 式 , 并 且 条 件
【
E d 坂 2 技 亚 102

有 时 所 考 虑 的 Hamilton 函 数 具 有 如 下 形 式

E 人 伟 2 【

E
考 虑 Abel 积 分 之 比

0

t E

阮 2

E 刑 50
0 仪 不
E 咤 志 王
邹 c 林 d
E 52 河A 示 EC2Eox e
[
E 林 4 国
E 65
t 林 技 圆

例 5.6 的 第 二 种 解 法

C 音鹧彻)=′一告孰

E 沥 沥 05 c 伟
E 八 2 标2
成 立 . 代 入 (5.37) 得 `
Page-246
E

E 江 知盖<0,且由图 2-6 知 乙 一 一 1 心 守 心 1 从 而

答′(冀)=(鞭阜…)z (1一…z)十(1一灞z)羞 E

E
D 二 浩河 p 2
E
E 命 仁 ( 河 途 2

E =踹,而且川脯〉伪当

0
E

E 庞 玟 述
其 中 a,b,c 为 常 数 , 一 A 心 加 , 面 且 当 5 丿 血 时 Jo(t) 丿 0, 则 可
把 7( 政 写 为

E t

医 0 一
E 沥

E a k 吴 沥 a
E 述
E ll 达 a
所 以 当 曲 线 @ 一 真 P) 无 变 曲 点 ( 或 有 & 个 变 曲 点 3 时 ,T(8) 在 ,
E
D
分 相 关 , 一 些 作 者 研 究 平 面 上 与 闭 轨 族 相 应 的 周 期 函 数 的 单 调 性
E 述 s

$ 6 “ 关 于 Petrov 定 理 的 证 明
l 贾 沥
Page-247
E 沥 额 X 公 训 异 河 沥 大 5

年 ,D, Hilbert00 在 第 二 居 国 际 数 学 家 大 会 上 提 出 了 23 个 数 学 间
E t
极 限 环 的 最 小 上 界 HCm 是 多 少 ? 可 能 出 现 的 极 限 环 相 对 位 置 如
何 ? 近 一 个 世 纪 以 来 * 特 别 是 最 近 儿 个 年 来 , 出 现 了 大 量 的 工 作 .
例 如 , 史 松 龄 ““ 和 陈 兰 草 , 王 明 淑 “w 最 先 分 别 举 例 证 明 HC2) 之
的 李 继 彬 \ 黄 其 明 0 举 例 证 明 HC3) 丿 11; 叶 彦 谦 、 陈 兰 茹 和 杨 信
玟
证 明 二 次 多 项 式 系 统 中 按 叶 彦 谦 分 类 的 ( ! ) 类 方 程 至 多 有 一 个
E 李 育 刑
光 的 综 述 文 章 [Cs] 和 [CZJ, 马 知 恩 的 专 著 [Mz], 梁 肇 军 的 专 著
[Lz], 以 及 Dumortier 等 人 的 系 列 文 章 [DRR1,2] 和 [DER] 中 , 读
者 可 发 现 大 量 有 趣 的 绪 果 、 经 过 IIyashenkom 和 Eealle02 修 补 证
明 后 的 Dulaem 有 限 性 定 理 指 出 : 一 个 绘 定 的 n 许 多 项 式 系 统 的 极
限 环 个 数 有 限 . 但 是 , 对 全 体 n 次 多 项 式 系 统 而 言 , 其 极 限 环 个 数
的 一 致 上 界 如 何 佼 计 ( 郅 怕 是 否 有 限 , 即 使 对 n 一 2 这 种 最 简 单 的
非 线 性 情 形 , 仍 是 一 个 未 知 的 问 题 . S.Smalerem 认 为 , 对 HGo 的
e
作 者 的 一 个 重 大 拼 战

由 (5. 5) 式 可 知 ,Abel 积 分 (5. 12) 是 系 统 (5. 10) 在 7 附 近 后
E 圆 2
的 极 限 环 个 数 密 切 相 关 . 当 又 是 n 十 ! 次 多 项 式 P 和 Q 为 次 多
项 式 时 ,(5. 10) 是 一 类 特 殊 的 n 汀 系 统 , 即 Hamilton 向 量 场 的 扰 动
E
E t ss
E a
的 形 式 . 此 阡 及 ,P,@ 可 能 不 再 是 多 项 式 . 习 惯 上 仍 把 (5. 12) 称 为
E

例 5. 5 考 虑 van der Pol 方 程
Page-248
英

其 中 0 一 |x| 奶 1. 可 把 它 改 写 为 如 下 的 等 价 形 式

[
FPgd

蒜=_z十粼1一菠)丛
述

T e E 北
或 写 成 参 数 方 程

0

【 招
G

国 吴

E I。 E z肋z〔 E T)′

显 然 & 一 2 是 它 的 唯 - 正 零 点 , 并 东 是 简 单 零 点 . 由 定 理 5.1, 当

r 洁

l
E

但 在 多 数 情 况 下 问 题 并 不 如 此 轻 而 易 举 , 请 看 下 例 .
0 一

E
【

E 月

E 坂 标 林
Hamilton 系 统 , 它 有 首 次 积 分

3
″(雾黜)=音+Z_言 0 [

t
E 育

E
Page-249
E 笺 二 章 常 见 的 局 部 不 非 属 鄯 分 贫

E 途 2s 沥
原 理 估 算 零 点 的 个 数 , 男 一 方 面 , 为 了 研 究 Hamitton 向 重 来 的 批
动 系 统 所 县 有 的 极 限 环 个 数 的 上 界 , 我 们 霜 要 把 相 应 的 Abet 积 分
零 点 个 数 的 佼 计 , 与 Hopf 分 岔 、 同 宿 分 岔 . 异 宿 分 岔 等 所 能 出 现 的
E 这是本节将戛介绍的另一个问题

考 怀 Hamilton 向 量 场 的 扰 动 系 统

盂一一一十册位拟十呱炒′

林

E
E 沥 诊 yE
设 Harrilton 向 髯 场 (6. 1)。 有 闭 转 族
F3 沥 9许 中 不 有
吴 江 才 罚 江
软 . 设 口 是 奇 闭 转 一 二 a 所 园 成 的 区 域 的 紧 致 邻 域 , 则 当 x 充 分
32 才 芸 3 水
面 Abel 积 分 ( 也 吸 一 阶 Melniktov 函 数 , 参 看 (5. 12) 式 )

E L“)Q(罩,汊)d工 E xta72o5 2

E 心 的 应

[ 形 芸 江
1 沥 5
E
「 【C3ie 技 3 志5 许5 途 一 罚 江 厂 步 沥 一 禅 5 许 灵 i
E

[ 技 c 沥 阮
E 河 吴 e
团 孝 i 9 河 E 逊 2
Page-250
[ 圆

为 边 界 . 闭 软 族 为
【 河 {(工'汊)I″(z,汊) E 昙 <矗<鲁}'

E

利 用 定 理 5.1, 定 理 5.2 和 (5. 12), 为 了 研 究 当 0 一 14| 丿 1 时
n

E 招 蛋r [ 命 2 【

E 一
E

n

Picard-Fuchs 方 程 法 ( 例 如 , 见 [CS]), 和 我 们 最 近 得 到 的 直 接 方
E

Pieard-Fuehs 方 程 法

i
E 标 玑 0
E
Page-251
[ 浩河 E

E
个 别 情 形 外 , 迄 今 未 得 到 其 表 达 式 , 对 一 些 特 殊 的 二 次 或 三 次
Hamilton 向 量 扬 在 a 一 2,3 或 4 的 情 况 下 来 佶 算 BCm,a 和 ZCm,
s 沥
使 寺 于 m 一 一 2 的 一 舱 情 形 , 问 题 也 远 没 有 後 底 解 欧

E P 河 唐 吾 时 , 卵 对 Bogdanov-Takens 系 统 ,
E 05 晓 7 切
E 招
幻 一 3, P MardesiePW0 证 明 ,BC2,m 一 A 一 1} 李 宝 毅 \ 张 芷 芬 PAm
E
2
的 研 究 中 少 有 的 完 整 结 果 . 本 节 主 要 介 绍 Petrov 和 Mardesic 的 感
E 故 东 5 河
辅 角 厝 理 来 直 接 伙 算 B(2,m)( 见 [LbZ])- 这 种 手 法 可 用 来 研 究 其
E

Abei 积 分 的 构 造

s 怀 一 坂 2 渡 5 应 招
刊 霆 点 个 数 , 必 须 先 研 究 它 的 构 造 , 由 Gzeen 公 式 , 有

E i GQudz,
E 命 E
其 中 Q 也 是 < 和 y 的 多 项 式 .

医 t 达 i
i

I「〈^)髓】 一 Z`(′I)'′′z′
雌式成立的一个充分条件是
林 河 一 朐
其中麒和 旦 都 是 z 和 y 的 多 项 式 , 显 焊 “~-“ 是 等 价 关 系 , 令
Page-252
第 二 章 常 见 的 尾 邢 与 非 尾 部 分 盅

E 林 诊 河 3 人
E 沥 沥

故 可 定 义
2

E
E
T
E 诊 晓 5SCD 为 工 所 图 成 的 紧 区 域 的 面

一 (h) 二

0 述
E 2
定 理 5.7 (1〉当一兽〈颢〈兽时,亭〈P仇)〈h且

^苎瑗填P(克) 河

沥 E 国 一 着 5
152 3 E 3 E 版 政 史 且^翌珥'…P 0
D

许
E 「
E 丧 育 水
医 移 u p 林 n
引 理 5.8 P( 满 足 如 下 的 Riceati 方 程
n 【
E

E
Page-253
c

E 沥 0
E
Dn
而 言 是 一 交 换 群 ; 男 外 还 有
了 0DCan 二 [FCEHJa0u
即

/仇〉I E 国 2 多 史
45 E

其 中 / 是 R 的 多 项 式 , 即 丿 可 在 多 项 式 环 火 上 进 行 乘 法 运 算 , 一
E 人 0 技 园 江
0 p a 0 胡 江3 10 月
【 应2 E 育 达 02
0 (
[ 标 途
E 李 n
E 训
E
E
E
定 理 6. 1 (LP1]) 考 虑 Bogdanov-Takens 系 统

吴
[

羞′=_一覃十zz E 吴 e 庞 运

E
E

E 仪 ( 02 仪 一 人 音, E
E I「伽五汊d艾j 2 t ( 技 江 03
Page-254
[ 颜 25 招 水 2

0
E E
4605 2

E

R 才 沥 仪 朋
林林
【

古 -- 方 面 , 利 用 分 部 积 分 可 得
口

【 不 水
[ 月 ( 1

E

E

{ 伟河 2
[

E 河 标 述 0
E
t 朱3
E

[ 訾岫 E 绍
[C 7 7 誓盯「

把 (5. 27) 代 入 P 二 _』, 即 得 (5.22), 』
定 理 5.7 的 证 明 把 (5. 22) 改 写 为 系 统
F 【 沥 人 沥 盖=一 9 十 4 《5. 28)

E s paEE 相 g 葛 口 y 才 刑
P仇〉咔h当力令一鲁因此汨=P仇〉的图形是从系统佤2盼的

l 一 朐
Page-255
E

的 水 平 等 斜 线 由 方 程 7P: 十 3RP 一 5 一 0 给 出 , 它 的 图 形 为 双 曲
医 政 述

E 贵〔′ E 林 0

显 然 P 一 P(4) 的 图 形 与 乜 二 &( 的 国 形 除 了 两 奇 点 外 不 可 能 相
E

巳 吴 c 半,
E E

0
L 一 一
的 图 形 上 方 , 从 而 沿 P 一 P 有 P (D 一 0 lian 万 ( 一 一 “ 可
E

直 接 方 法

为 了 判 断 Abel 积 分 的 零 点 个 数 , 目 前 使 用 的 方 法 大 都 要 经 过
曲 折 的 推 导 . 下 面 介 绍 一 个 可 在 一 定 条 件 下 从 原 方 程 判 别 的 直 接
E

04 颖 形 i u 半 2 着 ) 沥
Page-256
圆 肉 33

〕'dEg/】(凡) H 〔音〕一
我 们 将 证 明 作 为 习 题 留 给 读 者 . 只 要 对 做 归 纳 法 即 可 迦 得

Picard-Fuchs 方 程

E 沥

E 人

玟
满 足 以 下 的 Picard-Fuchs 方 程

4
0
哚 6>d′′Ll]一 【

目
【 7 ]y0<凡<6'

E 2
6凡

曹力

0

E 圆 225

在 5 巾 我 们 已 经 对 Bogdanov-Takens 系 统 的 男 一 种 等 价 形 式

E a 河 仁 02 育 仪 3 沥 2

《5.27). 由 于 Pieard-Fuchs 方 程 在 研 究 弼 Hilbert 夙 16 问 题 中 的 重
要 性 , 此 处 我 们 将 介 绍 古 一 种 递 推 公 式 “0““ 由 于

工矗
E 玄十玄 5 ′L,()〈′z<曹
E

E
E 河浩 刀

E
E 鬟 E 沥 算z弓 E 沥
利 门 分 部 积 分 得
B 一 I n 兰z″_3汊m十zdz E Ir仇)藁"一1汊mdz 一

E E
Page-257
常 见 的 局 部 与 非 尾 部 分 盆

0

「(^) E

一dz 03
TQ5 小

由 此 得 到

E 汀
Page-258
[ 圆 2 异 河 八 105

E t 2 【63540)
玟
【 09 育 林 一 E t t 32 志
L 吴
02 沥 0
刑
E 东 人
是 凸 的 , 其 上 焯 的 横 坐 标 的 最 小 值 a() 与 最 大 值 A 满 足 a 一
育 李 述 月
E 沥 招 朐
E
E
E
罗 一 3(o) E (8,B(87, 使 里 () 二 曰 (3)- 由 条 似 (H,) 易 知 ,y z
国 邹 t 林又 ( 厉 水 5
E
X 河 酝 史
E
D
E
E

E

4

E 出 林0 林 诚 仪 河 小 招
【 5 河 c 志 仪 2
[ 林沥
园 达 厂

Ae3i 吴 扬 3 伟 3

5 E c 沥 c
Page-259
不 6 关 于 Pewow 岱 理 的 证 明

4
E [ E
团 E

罡 I E [I
「(^) E 国 ′`(炫) 2汊 EX03 3y

刑 言 4dr [
E 「【^)

E 林 氓 招
达 吴 6 d′′ r(″)z帅

由 此 得 到
[ 暑茹伍) E 告/翼(颢)7

其中0<^<告. E

& 技 E
′l[′!_畜)′。′(庞〉_ 〔 6互 6〕′”(庞) 十36z】(庞)' 0 河

E 河 河 0 ]

i
医 c 水
u 刑

E
我 们 需 要 引 入 下 面 的 定 理 . 为 了 遮 免 行 文 冗 长 , 我 们 不 引 证 明 , 有
E
E a 一
E y 吴
c c
E
i

匹
t 沥 i 沥 仁 伟
Page-260
E

国 沥 265
c 力 Z 1

E 吊 2 林8 技 ( 技

定 理 5.9 ([LZJ〉 设 玑 (z,y) 具 有 (5. 30) 的 形 式 , 并 且 条 件
【
E d 坂 2 技 亚 102

有 时 所 考 虑 的 Hamilton 函 数 具 有 如 下 形 式

E 人 伟 2 【

E
考 虑 Abel 积 分 之 比

0

t E

阮 2

E 刑 50
0 仪 不
E 咤 志 王
邹 c 林 d
E 52 河A 示 EC2Eox e
[
E 林 4 国
E 65
t 林 技 圆

例 5.6 的 第 二 种 解 法

C 音鹧彻)=′一告孰

E 沥 沥 05 c 伟
E 八 2 标2
成 立 . 代 入 (5.37) 得 `
Page-261
第 二 章 常 见 的 尸 部 与 非 局 部 分 盆

E d 一 一

d

韩 茂 安 和 朱 德 明 在 专 著 [HZ] 中 对 非 Hamiton 系 统 的 线 性 中
心 也 证 得 与 定 理 6.3 完 全 类 似 的 结 论 .

E t 东 沥 ( 圆 e 朋
E 口 沥 园 妙 如
E J2n
函 数 p ( 有 以 下 展 式

00 2 人 林 运

沥
E
2

t

仿=呈俨…)L , 摹>十蔫〕d薹,当窿】=0y
其 中 积 分 式 前 符 叶 的 选 择 , 取 决 于 闭 转 TCD 的 定 向 和 当 增 加 时
E 园

E 王 刑 刀 一
E
E

定 理 6.4 是 研 究 同 宿 分 岔 的 主 要 依 搬 . 对 于 画 宿 转 而 言 , 韩
E 沥
E
指 敏 一 分 性 和 三 分 性 理 论 , 深 化 了 不 变 流 形 的 表 示 理 论 , 用 统 一 的
方 法 研 究 了 各 类 系 统 ( 高 维 或 低 维 , 可 积 或 不 可 积 ) 的 同 宿 扬 、 异
E .

E 林 一 一
林招 一 许
Page-262
E 浩 D E

日
夕 D 一 〔告 5 凡〕 0 告'

其 deg00 一 [252]-t deg0 一 [ 孝 - k
这 个 定 理 的 直 观 含 义 是 , 如 果 当 ~ 0 时 , 有 ?& 十 1 或 28 个 极
cn 沥 明
E
0

:
″(冀,汊)=誓十言`看=告一归o〈z〈告.

E 「(告 0 不 告〉 对 应 系 统 (6. 2), 的 闭 执 族 , 而
n
E

俨(告 g 2 标 述 友 标 d 林玟 仪

其 中 0 一 1 L,
E 东 述

E I……汊dz E 河 或 兄 人 标 逊 标

E I天“)箕仪d迈 河 政 儿 人 标 述 A

氓 吊 沥 仪 林 肖 5 林 孕 河 沥 孕
E 圆 0
Page-263
E

E 江 知盖<0,且由图 2-6 知 乙 一 一 1 心 守 心 1 从 而

答′(冀)=(鞭阜…)z (1一…z)十(1一灞z)羞 E

E
D 二 浩河 p 2
E
E 命 仁 ( 河 途 2

E =踹,而且川脯〉伪当

0
E

E 庞 玟 述
其 中 a,b,c 为 常 数 , 一 A 心 加 , 面 且 当 5 丿 血 时 Jo(t) 丿 0, 则 可
把 7( 政 写 为

E t

医 0 一
E 沥

E a k 吴 沥 a
E 述
E ll 达 a
所 以 当 曲 线 @ 一 真 P) 无 变 曲 点 ( 或 有 & 个 变 曲 点 3 时 ,T(8) 在 ,
E
D
分 相 关 , 一 些 作 者 研 究 平 面 上 与 闭 轨 族 相 应 的 周 期 函 数 的 单 调 性
E 述 s

$ 6 “ 关 于 Petrov 定 理 的 证 明
l 贾 沥
Page-264
第 二 章 常 见 的 局 部 与 非 局 部 分 盅

。 1 一 一
一羞z(z_l)(宁十玄)妾dz_ 沥

已 林圆 4 2
[ 〔_ E 国
“ 口 ( 亚 十 I
月 n aQ) 337
江 沥 标 二 E
1 [az 一 E E 1

直 接 运 算 便 可 证 明 引 理 .
E pn

E /。(告 2 仪 达

E /】(告 E 吴 伟

[ 达 人 木

g 人
E 一 仁
t
E
由 等 式 (6. 10) 立 刻 得 min(m,ma 丿 心
第 二 种 情 形 : zm 一 m:.
0 招 州 沥
2 吴 沥 兄 朋
由 引 理 6.6, 有
【 儿 林
【 技 丶 技
玟

E 熹俪肠 E 胡
国 o(Z鬓翼, E
Page-265
E 笺 二 章 常 见 的 局 部 不 非 属 鄯 分 贫

E 途 2s 沥
原 理 估 算 零 点 的 个 数 , 男 一 方 面 , 为 了 研 究 Hamitton 向 重 来 的 批
动 系 统 所 县 有 的 极 限 环 个 数 的 上 界 , 我 们 霜 要 把 相 应 的 Abet 积 分
零 点 个 数 的 佼 计 , 与 Hopf 分 岔 、 同 宿 分 岔 . 异 宿 分 岔 等 所 能 出 现 的
E 这是本节将戛介绍的另一个问题

考 怀 Hamilton 向 量 场 的 扰 动 系 统

盂一一一十册位拟十呱炒′

林

E
E 沥 诊 yE
设 Harrilton 向 髯 场 (6. 1)。 有 闭 转 族
F3 沥 9许 中 不 有
吴 江 才 罚 江
软 . 设 口 是 奇 闭 转 一 二 a 所 园 成 的 区 域 的 紧 致 邻 域 , 则 当 x 充 分
32 才 芸 3 水
面 Abel 积 分 ( 也 吸 一 阶 Melniktov 函 数 , 参 看 (5. 12) 式 )

E L“)Q(罩,汊)d工 E xta72o5 2

E 心 的 应

[ 形 芸 江
1 沥 5
E
「 【C3ie 技 3 志5 许5 途 一 罚 江 厂 步 沥 一 禅 5 许 灵 i
E

[ 技 c 沥 阮
E 河 吴 e
团 孝 i 9 河 E 逊 2
Page-266
日 6 关 于 Petow 定 理 的 证 晓

E
E 沥 y 吴 林 y 挂
E 林命 怡 妙 b i 李 |

E 达 东

扩 充 到 复 域

E 坂 东 颖 刑 伟
到 复 域

G=C\仙仨疃帆彦告〉 ,

并 且 具 有 以 下 性 质 。

E ~九昙, E 一 屋 , E 林

《23 7o(07 二 0, 卷 (0) 奂 0z1C03 一 卯 (03 一 0, 00 式 03
E 仪 东 亚 江

E 1 3 5
(3 Imn 0 E 林 不 蒽(沿上下边界无

定 理 6.8 “ 考 虑 方 程
E 命 A 23 3402 河 a 0
8 一 j 是 正 则 奇 点 的 亮 玖 条 件 是 灰 一 b 分 别 是 ( 和 9 的 次
转

02 振 ) E E
03 阊Z 12 砂 一 ″”)z/(′′) E 沥 沥 5

E

对 无 穷 远 奇 点 作 变 挨 文 一 二 则 有

d 国 E 团
丽一_拐_ , 腩-″「十胭茁 ,
[ 途
Page-267
118 第 二 章 胺 见 的 尿 部 与 非 局 部 分 皿

其中m)`′(翼〕 如果P[ 〕~″z,疃[半]~陟 则 由 定 理 6. 8,
一 0 是 上 面 方 程 的 正 则 奇 炙 , 即 无 穷 远 奇 点 是 方 程 (6. 14 的 正 则

定 义 6.9 “ 方 程 (6. 14) 称 为 Fuchs 型 方 程 , 如 果 它 只 有 正 则
E

E ( 东
技
E 兰

p 光 [ 沥 )

E

E 吴 沥 异 ( 达 沥

E

E 口 河师 e 笑 沥
E 林 达 20
Rep 乐 Repsoas 一 1 令 加 一 f 一 f, 期 果 办 奂 0, 则 功 一 1. 仪
【pssilapo i y
一 0. 其 它 ax 和 仁 的 确 定 可 参 看 高 维 新 的 解 析 理 论 讲 义 “1.
E 林才

E 暑岫′ 05 一 不 00 ,

国 认 吴 吴
E
【

于 Q E 吴 220 标
E

Edt3 河 王 丨6″已 [ 最)z。(″) 团

国
E ( H - 磊忡伽 一 0
Page-268
[ 浩河 E

E
个 别 情 形 外 , 迄 今 未 得 到 其 表 达 式 , 对 一 些 特 殊 的 二 次 或 三 次
Hamilton 向 量 扬 在 a 一 2,3 或 4 的 情 况 下 来 佶 算 BCm,a 和 ZCm,
s 沥
使 寺 于 m 一 一 2 的 一 舱 情 形 , 问 题 也 远 没 有 後 底 解 欧

E P 河 唐 吾 时 , 卵 对 Bogdanov-Takens 系 统 ,
E 05 晓 7 切
E 招
幻 一 3, P MardesiePW0 证 明 ,BC2,m 一 A 一 1} 李 宝 毅 \ 张 芷 芬 PAm
E
2
的 研 究 中 少 有 的 完 整 结 果 . 本 节 主 要 介 绍 Petrov 和 Mardesic 的 感
E 故 东 5 河
辅 角 厝 理 来 直 接 伙 算 B(2,m)( 见 [LbZ])- 这 种 手 法 可 用 来 研 究 其
E

Abei 积 分 的 构 造

s 怀 一 坂 2 渡 5 应 招
刊 霆 点 个 数 , 必 须 先 研 究 它 的 构 造 , 由 Gzeen 公 式 , 有

E i GQudz,
E 命 E
其 中 Q 也 是 < 和 y 的 多 项 式 .

医 t 达 i
i

I「〈^)髓】 一 Z`(′I)'′′z′
雌式成立的一个充分条件是
林 河 一 朐
其中麒和 旦 都 是 z 和 y 的 多 项 式 , 显 焊 “~-“ 是 等 价 关 系 , 令
Page-269
E E

由 定 理 6.8 =0和互=告都是m′20)和腑′2D 的 正 炊 奇 点 . 作

E
dz00〉

十歼[弼十 j′。(囊) 0 沥 )

dzz】〔灞) d童】(Z〉

p 标班 7 河 〔蕊十 j乙O)=0, 0

其中姚0)=跖〔鲁], 】(:)=/`〔鲁〕′ E 东

E
(6.zl)的芷则奇点,由定义s.9,佤20)和(6.21)鄙是Fuchs型方
E
下 面 我 们 来 证 明 定 理 6.7. 为 此 先 利 用 定 理 6.10 来 讨 论
E 仪 余 t
n
E 述 0
0
E

5 一 命
榨(/〕一1)十z懈+蕊_洋+严十3s 0,
玖

E 述 沥 仁 E 或打吾,当/咔山 即 一 D ~″吾或
吊 技 认 江
2

[ 巳 国

E +3严十蕊一弼十鲈十豁_仇
i 认
0 动
Page-270
c

E 沥 0
E
Dn
而 言 是 一 交 换 群 ; 男 外 还 有
了 0DCan 二 [FCEHJa0u
即

/仇〉I E 国 2 多 史
45 E

其 中 / 是 R 的 多 项 式 , 即 丿 可 在 多 项 式 环 火 上 进 行 乘 法 运 算 , 一
E 人 0 技 园 江
0 p a 0 胡 江3 10 月
【 应2 E 育 达 02
0 (
[ 标 途
E 李 n
E 训
E
E
E
定 理 6. 1 (LP1]) 考 虑 Bogdanov-Takens 系 统

吴
[

羞′=_一覃十zz E 吴 e 庞 运

E
E

E 仪 ( 02 仪 一 人 音, E
E I「伽五汊d艾j 2 t ( 技 江 03
Page-271
圆 肉 33

〕'dEg/】(凡) H 〔音〕一
我 们 将 证 明 作 为 习 题 留 给 读 者 . 只 要 对 做 归 纳 法 即 可 迦 得

Picard-Fuchs 方 程

E 沥

E 人

玟
满 足 以 下 的 Picard-Fuchs 方 程

4
0
哚 6>d′′Ll]一 【

目
【 7 ]y0<凡<6'

E 2
6凡

曹力

0

E 圆 225

在 5 巾 我 们 已 经 对 Bogdanov-Takens 系 统 的 男 一 种 等 价 形 式

E a 河 仁 02 育 仪 3 沥 2

《5.27). 由 于 Pieard-Fuchs 方 程 在 研 究 弼 Hilbert 夙 16 问 题 中 的 重
要 性 , 此 处 我 们 将 介 绍 古 一 种 递 推 公 式 “0““ 由 于

工矗
E 玄十玄 5 ′L,()〈′z<曹
E

E
E 河浩 刀

E
E 鬟 E 沥 算z弓 E 沥
利 门 分 部 积 分 得
B 一 I n 兰z″_3汊m十zdz E Ir仇)藁"一1汊mdz 一

E E
Page-272
常 见 的 局 部 与 非 尾 部 分 盆

0

「(^) E

一dz 03
TQ5 小

由 此 得 到

E 汀
Page-273
120 E

炀吾,当 E

国 F553633 才 2 标 3 E E ″吾,当 方 一
E 沥 a

为 求 ( 在 & 一 0 附 近 的 展 式 , 先 对 方 程 (6. 20 求 奇 点 一
0 的 狞 定 方 程 p(p 一 17 一 0 的 根 , 得 p 一 1 和 p 一 0. 因 办 一 一
E A d
《6. 24) 中 的 C 东 0, 伦 有 7(03 一 Cm(0 十 Coos(0) 二 Cabo 式 0,

E L伽 应 E 张 沥 四 志
【 O 7 0 东 3
[

地 扩 宁 到 G.
i

0 前 判 定 方 稚 p(p 一 1 一 p 一 友 一 26 一 0 的 框 , 得 一 2, 一

E 才 5 万 E

55 江 沥 诊 仪 0 沥

0 木 L侧 E 扬
0 0 一
【 辽

其 它 有 穷 奇 点 , 故 工 ( 可 单 值 解 析 地 扩 充 到 G-
0 仪

E 沥 ^>告〉 E
近 的 水 乎 边 界 , 我 们 断 言 下 列 不 等 式 成 立 ;

u 07 医 )

E

E 林口
Page-274
E
E 达

1 [ 人 人
″[^」畜〕 国 十〔言″+畜〕P一蒽允 【
考 虑 (6, 26) 的 等 价 系 统
目

亚_、_z 巳 吴
36′〕 十〔3九+ 6)P 6尤'

巳 网 国

E

6.2
【 E

E
系 统 (6. 27) 的 相 国 如 图 2-8 所 示 . 其 中 点 5: (&,P) 二 【告j]是

荣 点 , 由 方 程
E 林河 玟 李 2 十 1

【
筠 会 定 的 双 曲 线 是 过 点 $ 的 水 平 等 倾 线 - 利 用 7(h) 和 一 (8) 在 h

E
Page-275
不 6 关 于 Pewow 岱 理 的 证 明

4
E [ E
团 E

罡 I E [I
「(^) E 国 ′`(炫) 2汊 EX03 3y

刑 言 4dr [
E 「【^)

E 林 氓 招
达 吴 6 d′′ r(″)z帅

由 此 得 到
[ 暑茹伍) E 告/翼(颢)7

其中0<^<告. E

& 技 E
′l[′!_畜)′。′(庞〉_ 〔 6互 6〕′”(庞) 十36z】(庞)' 0 河

E 河 河 0 ]

i
医 c 水
u 刑

E
我 们 需 要 引 入 下 面 的 定 理 . 为 了 遮 免 行 文 冗 长 , 我 们 不 引 证 明 , 有
E
E a 一
E y 吴
c c
E
i

匹
t 沥 i 沥 仁 伟
Page-276
E E

一 不 陆 I 的 渐 近 展 式 , 不 雅 算 出 当 R E * 时
[ E 租 命 切
E 朐
E 达
[
E 东
p
E 一 述
E
质 ( , 不 难 用 辐 角 原 理 来 证 明 Ja(8)T,( 不 0, 当 E G 古 夺 0.
E
E 河 河 a 兔璞彗荠0 当^仨魍』>方
用 反 证 法 . 设

m/侦 0
E

二 工

园

ReCo( 胡 33 Im( 二 ( 77 一 Re(JCf“ )3Imto(f「)3 二 0.
即 向 量 Gmo(h) ,lmC3) 积 向 最 (ReJo(D,Re,(KD7 在 丁 一 不
E 胡
(6.18) 的 解 , 如 果 它 们 在 一 点 一 f“ 时 成 比 侧 , 则 它 们 将 在 一 切
沥

途

lm[艾丽) E

其 中 c 为 非 零 常 数 , 即 虚 郯 不 为 零 回到原变量^=告_Z,此结论

5
Page-277
第 二 章 常 见 的 尸 部 与 非 局 部 分 盆

E d 一 一

d

韩 茂 安 和 朱 德 明 在 专 著 [HZ] 中 对 非 Hamiton 系 统 的 线 性 中
心 也 证 得 与 定 理 6.3 完 全 类 似 的 结 论 .

E t 东 沥 ( 圆 e 朋
E 口 沥 园 妙 如
E J2n
函 数 p ( 有 以 下 展 式

00 2 人 林 运

沥
E
2

t

仿=呈俨…)L , 摹>十蔫〕d薹,当窿】=0y
其 中 积 分 式 前 符 叶 的 选 择 , 取 决 于 闭 转 TCD 的 定 向 和 当 增 加 时
E 园

E 王 刑 刀 一
E
E

定 理 6.4 是 研 究 同 宿 分 岔 的 主 要 依 搬 . 对 于 画 宿 转 而 言 , 韩
E 沥
E
指 敏 一 分 性 和 三 分 性 理 论 , 深 化 了 不 变 流 形 的 表 示 理 论 , 用 统 一 的
方 法 研 究 了 各 类 系 统 ( 高 维 或 低 维 , 可 积 或 不 可 积 ) 的 同 宿 扬 、 异
E .

E 林 一 一
林招 一 许
Page-278
6 关 于 Perov 定 理 的 证 明
E ea 0 |

E 江
E 东 ( 明 河 朋

E

域 的 任 意 紧 致 邻 垣 , 存 在 古 > 0, 当 || 一 么 时 , 中 极 限 环 个 数
E `

Eoao 国 胡 I 才 祖 标 2
[

[ 江 告, 巳
3 乙″'″_>F(告)` E
由 定 理 6. 1,
E 仪 0 703 告'

E 怀 0 0 技 〔″ 矽

E 〔音〕[ L

由 定 理 6.7,75(8D 和 于 (&) 可 单 值 解 析 地 扩 充 刻 复 城 G, 故
E
E

n

其中deg/嬴(″)<〔″ 〕[飚`deg/宙 仇)<〔昔〕一1_尤 E
0 沥 东 育 2

佶 _^门…们 十腊号/"M)) 园 0
Page-279
E 浩 D E

日
夕 D 一 〔告 5 凡〕 0 告'

其 deg00 一 [252]-t deg0 一 [ 孝 - k
这 个 定 理 的 直 观 含 义 是 , 如 果 当 ~ 0 时 , 有 ?& 十 1 或 28 个 极
cn 沥 明
E
0

:
″(冀,汊)=誓十言`看=告一归o〈z〈告.

E 「(告 0 不 告〉 对 应 系 统 (6. 2), 的 闭 执 族 , 而
n
E

俨(告 g 2 标 述 友 标 d 林玟 仪

其 中 0 一 1 L,
E 东 述

E I……汊dz E 河 或 兄 人 标 逊 标

E I天“)箕仪d迈 河 政 儿 人 标 述 A

氓 吊 沥 仪 林 肖 5 林 孕 河 沥 孕
E 圆 0
Page-280
E E 芸 e

由 定 理 6. 7 的 性 质 (23,FCD 在 G 上 解 析 . 取 区 域 口 人 G, 其 边 界
E
E 7 水

E
一

[ 林 河 恤_告_=n0<r《‖,

E 告十r<力<倒′
0
E 述 许 主 达 刑

E 一
E 技 沥 标江 日 阮
的 每 一 部 分 移 劲 时 , 相 应 的 pCF(7) 的 联 值 , 则

E 告. 0 浩 3
E 技 技
E 2 1
raoe <Sml[e] [ 扎 - 制 + 一
E 园
E 匹 不

[ 咤 吴
E
E 许

E 招 3

E
由 糯 角 原 理 ,F(8) 在 卫 中 至 多 有 a 一 24 一 1 个 零 点 , 从 而 F(R) 在
0 王 一 河
Page-281
第 二 章 常 见 的 局 部 与 非 局 部 分 盅

。 1 一 一
一羞z(z_l)(宁十玄)妾dz_ 沥

已 林圆 4 2
[ 〔_ E 国
“ 口 ( 亚 十 I
月 n aQ) 337
江 沥 标 二 E
1 [az 一 E E 1

直 接 运 算 便 可 证 明 引 理 .
E pn

E /。(告 2 仪 达

E /】(告 E 吴 伟

[ 达 人 木

g 人
E 一 仁
t
E
由 等 式 (6. 10) 立 刻 得 min(m,ma 丿 心
第 二 种 情 形 : zm 一 m:.
0 招 州 沥
2 吴 沥 兄 朋
由 引 理 6.6, 有
【 儿 林
【 技 丶 技
玟

E 熹俪肠 E 胡
国 o(Z鬓翼, E
Page-282
E 浩 125

[ 技 辽 一 刑

达 一 一 一 林敌
一
趴 一 吴 一 1 个 (2) 类 环 ,

由 定 理 6.3, 系 统 至 多 有 力 个 (1) 类 环

国 P
c

E
欢 十 工 个 (3) 类 环 . 此 时 由 I 的 展 式 (6. 13), 可 知 不 等 式 (6. 28)
E

E 着'

E 吴
述

E

图 皇…P们 E i
E

/y(/)=I E 技 0<z<玉.
E E

述
李
足 等 式 (6. 3 中 的 一 切 多 项 式 fo(8) 和 亿 (4) 都 是 可 以 实 现 的

[ 暑佯几U) E 水 江1 口
E

E 仪 水 仁 0 0 沥 水

E
Page-283
日 6 关 于 Petow 定 理 的 证 晓

E
E 沥 y 吴 林 y 挂
E 林命 怡 妙 b i 李 |

E 达 东

扩 充 到 复 域

E 坂 东 颖 刑 伟
到 复 域

G=C\仙仨疃帆彦告〉 ,

并 且 具 有 以 下 性 质 。

E ~九昙, E 一 屋 , E 林

《23 7o(07 二 0, 卷 (0) 奂 0z1C03 一 卯 (03 一 0, 00 式 03
E 仪 东 亚 江

E 1 3 5
(3 Imn 0 E 林 不 蒽(沿上下边界无

定 理 6.8 “ 考 虑 方 程
E 命 A 23 3402 河 a 0
8 一 j 是 正 则 奇 点 的 亮 玖 条 件 是 灰 一 b 分 别 是 ( 和 9 的 次
转

02 振 ) E E
03 阊Z 12 砂 一 ″”)z/(′′) E 沥 沥 5

E

对 无 穷 远 奇 点 作 变 挨 文 一 二 则 有

d 国 E 团
丽一_拐_ , 腩-″「十胭茁 ,
[ 途
Page-284
第 二 章 常 见 的 尾 部 与 非 尾 部 分 岔

E E 吊 刑
0 朐 逊 ZCo ,
扬
E E
0
E
不 0, 当 0 人 1 人 工 心 1 [ 沥 )
E 李 2 林 水 E
和 吉 是 乙 ( 的 简 单 零 点 , 我 们 将 此 作 为 习 题 留 给 读 者 .
E

4
二 2 l扁`&震zz煌一1 一 oCl)、
将 它 和 (6. 31) 右 侧 相 比 较 , 由 引 理 6.6 便 知 农 ~~0 时 , 一 0 . 同
E 林

[ 仪
国 t
), 有

0

【

E 7 明 u 芸 八 万 李 一 诊
E 江 江 一 4 伟 吊 人 厂
E 江 洁 水 才 仪 述 余 浩 夕″`它在(0'告〉 园 河

E 一 刀

E

【 沥 木 口 沥 林 沥 仪
圆 t 达 a
Page-285
118 第 二 章 胺 见 的 尿 部 与 非 局 部 分 皿

其中m)`′(翼〕 如果P[ 〕~″z,疃[半]~陟 则 由 定 理 6. 8,
一 0 是 上 面 方 程 的 正 则 奇 炙 , 即 无 穷 远 奇 点 是 方 程 (6. 14 的 正 则

定 义 6.9 “ 方 程 (6. 14) 称 为 Fuchs 型 方 程 , 如 果 它 只 有 正 则
E

E ( 东
技
E 兰

p 光 [ 沥 )

E

E 吴 沥 异 ( 达 沥

E

E 口 河师 e 笑 沥
E 林 达 20
Rep 乐 Repsoas 一 1 令 加 一 f 一 f, 期 果 办 奂 0, 则 功 一 1. 仪
【pssilapo i y
一 0. 其 它 ax 和 仁 的 确 定 可 参 看 高 维 新 的 解 析 理 论 讲 义 “1.
E 林才

E 暑岫′ 05 一 不 00 ,

国 认 吴 吴
E
【

于 Q E 吴 220 标
E

Edt3 河 王 丨6″已 [ 最)z。(″) 团

国
E ( H - 磊忡伽 一 0
Page-286
司 题 与 怡 考 颜 二 E

0
E 吊 0 沥 切

a 怦。 E '苛〕Z 】n/〕+ E

E c 李 仪 8 扬
E t 沥

构造沟几它在仙告〉 a d /(Z)在(0'音)
E 圭 标 刑 y 回到原变量丸=告_^则

一 一

0
E

医

2. 1 哲 量 场 契 点 的 双 曲 性 与 非 迹 化 性 有 什 么 联 系 , 有 什 么 区 别 ? 光 潘 向
基 场 的 双 曲 奇 点 或 非 追 化 奇 点 在 吊 量 炼 的 Cr(r 2 1) 扰 动 下 有 何 变 化 规 律 ?

2.2 证 明 当 | 心 1 时 , 例 5.5 中 的 wan der Pol 方 程 (5.14) 的 呐 一 闭 轨
E 1

2.3 眙 (2. 27 式 中 的 啶 量 场 X 是 n 次 多 项 式 系 统 「 试 问 定 理 2.7 的 绪 论
[ 渡 d

2.4 考 虑 方 程 组

E 江 d 江江 李

E 国 技 胡 许
4 一 6 时 的 情 形 来 谋 明 条 休 (Hl) 不 能 娆 少 , 否 则 4,0) 可 既 不 是 方 程 组 的 中
E

E

05 E 沥 玟 吴

E 沥 东 一 一 伟
Page-287
E E

由 定 理 6.8 =0和互=告都是m′20)和腑′2D 的 正 炊 奇 点 . 作

E
dz00〉

十歼[弼十 j′。(囊) 0 沥 )

dzz】〔灞) d童】(Z〉

p 标班 7 河 〔蕊十 j乙O)=0, 0

其中姚0)=跖〔鲁], 】(:)=/`〔鲁〕′ E 东

E
(6.zl)的芷则奇点,由定义s.9,佤20)和(6.21)鄙是Fuchs型方
E
下 面 我 们 来 证 明 定 理 6.7. 为 此 先 利 用 定 理 6.10 来 讨 论
E 仪 余 t
n
E 述 0
0
E

5 一 命
榨(/〕一1)十z懈+蕊_洋+严十3s 0,
玖

E 述 沥 仁 E 或打吾,当/咔山 即 一 D ~″吾或
吊 技 认 江
2

[ 巳 国

E +3严十蕊一弼十鲈十豁_仇
i 认
0 动
Page-288
二 章 帝 见 的 局 郭 与 非 尿 部 分 岑

s 沥
| 一 aiz 一 aay 十 5
D 朋
江 E
[
2.7 证 明 定 理 .15.
E 芸 沥
功 , 服 有 ( 一 0 试 问 显 咏 必 有 8「( 旭 不 09
2.9 试 用 定 理 6.3 木 求 方 程 组
乏_一j=一z一′】+灼z+炮z仪十灼鹦狮
E 李 沥
ctutuot a
d
2.11 试 证 (6, 30) 式 中 的 丿 和 双 都 是 公 式 (6. 29 中 丁 ( 的 简 单
E 伟 江 口

2.13 对 5.15), 试 用 $ 6 中 的 增 推 公 式 , 导 出 a(8) 和 于 ( 所 源 足
E

[ v昙 十 z最' E
E
E 向量场为)圣 E 沥 的 沥 招 沥 北 最汉 E

0 I「…】 E 兰 吴 浩 沥 a

出 万 D G 心 013? 所 满 足 的 Pleard-Fuchs 方 程
2.16 考 E 逊 55 朐 5 才

B
E 了zZ 园 巳 招 水E 招 玟

其 中 为 小 参 数 . 停 设 由 Absl 积 分 (5.12 定 义 的 一 阶 Mettikov 函 数 不 佩 为
雳 , 东其 问 宿 分 岔 的 最 高 阶 数 为 2, 利 用 定 理 6.4 水 其 一 阶 和 二 防 同 寇 分 岑
曲 线 在 ( 平 面 上 的 方 程 ( 即 对 8 的 一 阶 近 伴 方 程 ).
Page-289
120 E

炀吾,当 E

国 F553633 才 2 标 3 E E ″吾,当 方 一
E 沥 a

为 求 ( 在 & 一 0 附 近 的 展 式 , 先 对 方 程 (6. 20 求 奇 点 一
0 的 狞 定 方 程 p(p 一 17 一 0 的 根 , 得 p 一 1 和 p 一 0. 因 办 一 一
E A d
《6. 24) 中 的 C 东 0, 伦 有 7(03 一 Cm(0 十 Coos(0) 二 Cabo 式 0,

E L伽 应 E 张 沥 四 志
【 O 7 0 东 3
[

地 扩 宁 到 G.
i

0 前 判 定 方 稚 p(p 一 1 一 p 一 友 一 26 一 0 的 框 , 得 一 2, 一

E 才 5 万 E

55 江 沥 诊 仪 0 沥

0 木 L侧 E 扬
0 0 一
【 辽

其 它 有 穷 奇 点 , 故 工 ( 可 单 值 解 析 地 扩 充 到 G-
0 仪

E 沥 ^>告〉 E
近 的 水 乎 边 界 , 我 们 断 言 下 列 不 等 式 成 立 ;

u 07 医 )

E

E 林口
Page-290
E
E 达

1 [ 人 人
″[^」畜〕 国 十〔言″+畜〕P一蒽允 【
考 虑 (6, 26) 的 等 价 系 统
目

亚_、_z 巳 吴
36′〕 十〔3九+ 6)P 6尤'

巳 网 国

E

6.2
【 E

E
系 统 (6. 27) 的 相 国 如 图 2-8 所 示 . 其 中 点 5: (&,P) 二 【告j]是

荣 点 , 由 方 程
E 林河 玟 李 2 十 1

【
筠 会 定 的 双 曲 线 是 过 点 $ 的 水 平 等 倾 线 - 利 用 7(h) 和 一 (8) 在 h

E
Page-291
第 三 章 “ 几 类 余 维 2 的 平 面 向 量 场 分 岔

在 本 章 中 , 我 们 将 综 合 运 用 第 二 章 所 介 绍 的 儿 种 典 型 的 向 量
E 河
类 余 维 2 分 岔 .

E 坂 d 仪 [

CKo: 坤 =<fCao,

E 逊
玟
玟
玟

^一′0 1〕 ^_[o O)
】一〔00' 林
,

|
{
|
E 技 育 李 招
若 平 面 呐 重 场 在 奇 点 处 的 线 性 部 分 矩 阵 有 二 重 零 特 征 根 , 并

E
E 沥 园
E 沥
一 一
E 刀 一

化 条 件 下 , 扰 动 系 统 X 具 有 双 参 数 的 普 适 开 折 、 主 耐 参 考 文 献
Page-292
E E

一 不 陆 I 的 渐 近 展 式 , 不 雅 算 出 当 R E * 时
[ E 租 命 切
E 朐
E 达
[
E 东
p
E 一 述
E
质 ( , 不 难 用 辐 角 原 理 来 证 明 Ja(8)T,( 不 0, 当 E G 古 夺 0.
E
E 河 河 a 兔璞彗荠0 当^仨魍』>方
用 反 证 法 . 设

m/侦 0
E

二 工

园

ReCo( 胡 33 Im( 二 ( 77 一 Re(JCf“ )3Imto(f「)3 二 0.
即 向 量 Gmo(h) ,lmC3) 积 向 最 (ReJo(D,Re,(KD7 在 丁 一 不
E 胡
(6.18) 的 解 , 如 果 它 们 在 一 点 一 f“ 时 成 比 侧 , 则 它 们 将 在 一 切
沥

途

lm[艾丽) E

其 中 c 为 非 零 常 数 , 即 虚 郯 不 为 零 回到原变量^=告_Z,此结论

5
Page-293
130 第 么 章 儿 类 仁 维 2 的 平 国 向 塔 堤 分 岑

为 :e = 1 的 余 维 2 分 岔 见 [Bol,2] 和 [T], 余 维 3.4 的 讨 论 分 别 见
LDRSL,2] 和 [LR1];4 一 2 和 8 一 3 的 余 维 分 岔 见 [Ho], e 一 2 的
E 如 江 E 述 才
E
有 情 形 的 详 细 介 绍 .

$ 1 二 重 零 特 征 根 : Bogdanov-Takens 系 统
E 描
i
gn 沥
[

Qz 十 bry
当 史 一 0 时 , 由 第 一 章 定 理 5. 13, 系 统 (1, 2) 的 任 一 非 退 化 开 折 可
转 化 为
dz'
05

E 玑

′羞昔 E 河 玟 2 招 沥 2 庞 a

0
其 中 Q,88E C“,Q(0,0) 二 土 1 一 sgn(a5),xE R,m 之 2. 为 确 定
起 见 , 取 Q(0,0) 一 1 Q(0,0) 一 一 1 的 情 况 可 类 似 讨 论 .

分 岑 图 , 轨 线 的 拓 扑 分 类

t
E 技

0 河 t
Page-294
园 国

E 一
E
0 簧赚 E 标 〉0}`

其 中 SN*,H,HL 分 别 为 歌 结 点 分 岔 曲 线 ,Hopf 分 岔 曲 线 和 同 宿
E

E
Eouuaopusetsst 河 b 育
为 了 证 明 定 理 1.1, 注 意 当 h 之 0 时 ,(1. 3) 在 原 点 附 近 无 奇
E 东 园 沥
p

人 人
E 不 许 e 沥 沥 3 沥 人 医 莲 3 扬
Page-295
6 关 于 Perov 定 理 的 证 明
E ea 0 |

E 江
E 东 ( 明 河 朋

E

域 的 任 意 紧 致 邻 垣 , 存 在 古 > 0, 当 || 一 么 时 , 中 极 限 环 个 数
E `

Eoao 国 胡 I 才 祖 标 2
[

[ 江 告, 巳
3 乙″'″_>F(告)` E
由 定 理 6. 1,
E 仪 0 703 告'

E 怀 0 0 技 〔″ 矽

E 〔音〕[ L

由 定 理 6.7,75(8D 和 于 (&) 可 单 值 解 析 地 扩 充 刻 复 城 G, 故
E
E

n

其中deg/嬴(″)<〔″ 〕[飚`deg/宙 仇)<〔昔〕一1_尤 E
0 沥 东 育 2

佶 _^门…们 十腊号/"M)) 园 0
Page-296
e

一
E 沥 刑

园 沥 标 人 阮

E
我 们 可 以 把 (1. 5 看 成 (1. 5)。 的 扰 动 系 统 , 后 者 为 Hamilon 系
统 , 它 有 鞍 点 4(1,0) 的 同 宿 辅 , 以 及 该 同 宿 技 所 园 的 以 点 B( 一 1,
0) 为 中 心 的 周 期 环 域 ( 见 第 二 章 例 5. 6 及 图 2-6) , 周 期 环 域 中 的 闭
E

E {(z,汊)_H(渥'仅)=″,_鲁<″<鲁, 0
E
| …z唧)=誓十z_誓. 医 林 办
当凡令_吾十0时'厂^缩向奇点B【当凡令音~6时,厂″趋于同

E

注 意 对 任 意 的 8,(1. 5)s 都 以 4,B 为 奇 点 , 且 A 为 鞍 点 ,B 为
指 标 十 1 的 奇 点 . 图 此 , 若 (1. 5)3 存 在 闭 辅 , 它 必 定 与 线 段 工 一
0 男 一 方 面 , 由 于 「 与 工 的 交
E 人 逊 国

0

E 阮 圆
正 向 及 负 向 延 续 分 别 与 z 辐 ( 第 一 次 ) 交 于 点 Q, 与 @i. 记 7(&,8,5
E 2 林

E 东 林
]

F仇峨痊)…I E 人
E
E
Page-297
E E 芸 e

由 定 理 6. 7 的 性 质 (23,FCD 在 G 上 解 析 . 取 区 域 口 人 G, 其 边 界
E
E 7 水

E
一

[ 林 河 恤_告_=n0<r《‖,

E 告十r<力<倒′
0
E 述 许 主 达 刑

E 一
E 技 沥 标江 日 阮
的 每 一 部 分 移 劲 时 , 相 应 的 pCF(7) 的 联 值 , 则

E 告. 0 浩 3
E 技 技
E 2 1
raoe <Sml[e] [ 扎 - 制 + 一
E 园
E 匹 不

[ 咤 吴
E
E 许

E 招 3

E
由 糯 角 原 理 ,F(8) 在 卫 中 至 多 有 a 一 24 一 1 个 零 点 , 从 而 F(R) 在
0 王 一 河
Page-298
渡

一

E
u e 一 切

E 一
p

林 林 水

(】.5)′
E 吊 3 河 沥 沥 沥 沥 9
E
2
E I 钨`捡 E 秉 沥 9
沥

0
E 圆 |

我 们 可 以 用 取 极 限 的 方 法 把 F(f,3,5) 的 定 义 域 扩 大 到 集 合
i {(^,贪,玄)_、鲁<″<熹瓜<艘<鳃瓢<芗<鲢卜

E 沥 沥 ( d 李 不
E 林匹 中 吊 林 训
E

【 {(′z,贯,玄)_`鲁<凇<暑』<朦<虬瓢<鳖<鲢}
Page-299
E 浩 125

[ 技 辽 一 刑

达 一 一 一 林敌
一
趴 一 吴 一 1 个 (2) 类 环 ,

由 定 理 6.3, 系 统 至 多 有 力 个 (1) 类 环

国 P
c

E
欢 十 工 个 (3) 类 环 . 此 时 由 I 的 展 式 (6. 13), 可 知 不 等 式 (6. 28)
E

E 着'

E 吴
述

E

图 皇…P们 E i
E

/y(/)=I E 技 0<z<玉.
E E

述
李
足 等 式 (6. 3 中 的 一 切 多 项 式 fo(8) 和 亿 (4) 都 是 可 以 实 现 的

[ 暑佯几U) E 水 江1 口
E

E 仪 水 仁 0 0 沥 水

E
Page-300
E E 浩 usssgenlel s

E ′
E 英 关 莲
E 沥 医 水 e

点附近解对参数的光滑依赖性定埋胭在″=_暑关于飙苔 Eouoa

E 昙
E 达

从 上 面 的 两 个 引 理 可 得 如 下 推 论 .
引 理 1.4 系 统 (1. 5)a(8 一 0) 存 在 鞍 点 A 的 同 宿 技 , 当 昆 仅

当F[兽默店】 E
I 李

E =I「 吊 2 仪 ( 河 仪
D 1

E
E 沥 月 3 或 3 yy 振 一 ) 志 1 浩河 服
E 厉
E 不 2 沥 2
证 明 “ 由 引 理 1.4 可 得 , 系 统 (1. 5)s 存 在 同 宿 转 的 充 要 条 件

E
人 刃 孙 一 刃 )-ooa 菊 招 a 吴 - 口 分 >eam
E 育 江

E 亭并且 F〔鲁默忑】稀)】 E 园

日 E 政 沥 吴 元 沥 / 发 26 振 力 朐
E 一 河
i
E 林 李n 兄 秉 7 河 437
Page-301
第 二 章 常 见 的 尾 部 与 非 尾 部 分 岔

E E 吊 刑
0 朐 逊 ZCo ,
扬
E E
0
E
不 0, 当 0 人 1 人 工 心 1 [ 沥 )
E 李 2 林 水 E
和 吉 是 乙 ( 的 简 单 零 点 , 我 们 将 此 作 为 习 题 留 给 读 者 .
E

4
二 2 l扁`&震zz煌一1 一 oCl)、
将 它 和 (6. 31) 右 侧 相 比 较 , 由 引 理 6.6 便 知 农 ~~0 时 , 一 0 . 同
E 林

[ 仪
国 t
), 有

0

【

E 7 明 u 芸 八 万 李 一 诊
E 江 江 一 4 伟 吊 人 厂
E 江 洁 水 才 仪 述 余 浩 夕″`它在(0'告〉 园 河

E 一 刀

E

【 沥 木 口 沥 林 沥 仪
圆 t 达 a
Page-302
东 1 二 重 军 案 征 林, Bogdanor-Takene 系 给

E 一

E
们 需 要 证 明 同 宿 分 岔 对 参 数 3 的 一 致 伯 ( 见 第 二 章 $ 4, 即 证 明
存 在 8 丿 0 和 丿 丿 0, 使 得 对 任 意 和 予 只 要 0 一 一 卷 , 匹 一
E

0 东 el
E

《2 7(8) 在 扰 动 下 其 稳 定 流 形 与 不 稳 定 流 形 具 有 固 定 的 相 对
E

E

E 7 刑

许

技 河 沥 招 玟 沥 班2
0
E 。
国 (汊(1 不
由 于 对 任 意 的 3, 系 统 的 鞅 点 均 在 (z,9) 一 (1,0) , 而 且 系 统 在 读 点
的 发 散 量 为
E 氙 0 0

E

67| 一 多 , 就 有 了 (8) 乙 0. 结 合 第 二 章 定 理 4 3 和 宏 理 4. 4 可

E 刑
E

; 力 则 与 第 二 章 (4 19) 式 兵 似 的 线 “ 当

| 旦 ( r OCI8| 十 匹 一 口 (85447

E

M
Page-303
司 题 与 怡 考 颜 二 E

0
E 吊 0 沥 切

a 怦。 E '苛〕Z 】n/〕+ E

E c 李 仪 8 扬
E t 沥

构造沟几它在仙告〉 a d /(Z)在(0'音)
E 圭 标 刑 y 回到原变量丸=告_^则

一 一

0
E

医

2. 1 哲 量 场 契 点 的 双 曲 性 与 非 迹 化 性 有 什 么 联 系 , 有 什 么 区 别 ? 光 潘 向
基 场 的 双 曲 奇 点 或 非 追 化 奇 点 在 吊 量 炼 的 Cr(r 2 1) 扰 动 下 有 何 变 化 规 律 ?

2.2 证 明 当 | 心 1 时 , 例 5.5 中 的 wan der Pol 方 程 (5.14) 的 呐 一 闭 轨
E 1

2.3 眙 (2. 27 式 中 的 啶 量 场 X 是 n 次 多 项 式 系 统 「 试 问 定 理 2.7 的 绪 论
[ 渡 d

2.4 考 虑 方 程 组

E 江 d 江江 李

E 国 技 胡 许
4 一 6 时 的 情 形 来 谋 明 条 休 (Hl) 不 能 娆 少 , 否 则 4,0) 可 既 不 是 方 程 组 的 中
E

E

05 E 沥 玟 吴

E 沥 东 一 一 伟
Page-304
136 E 浩

[ 理 0 一 朐
(0, 加 , 使 只 要 0 一 一 8, 区 一 咤 (8)| 一 加 , 就 有
E 刑 罡
r E

匹
所 以 结 论 (2) 也 成 立 . 引 理 1. 6 得 证 , 〖
E

0
一 万 人 人 品 (80,0 一 8 一 口 时 ,(1,5)8 在 旦 点 的 I 邻 域 内 恰 有
一 个 闭 轨 , 它 是 不 稳 定 的 极 限 环 ; 而 当 吊 (3) =<5 一 多 (8) 十 办 0 一

E 河 i
E 国 沥 437 河

, 目
( 沥 园 穴 河 友 人 0

i 标
E 告殴砧」 园 江 关

E 量〔8 E

E 江 个
(8) 时 ,e(8,f(8)) 一 0,8C8,(8)) 不 0. 即 第 二 章 $ 3 中 的 条 件

(Hi ) 成 立 . 计 算 表 明 ,

E
命
[
E
E 国 林胡
[ 木
Page-305
二 章 帝 见 的 局 郭 与 非 尿 部 分 岑

s 沥
| 一 aiz 一 aay 十 5
D 朋
江 E
[
2.7 证 明 定 理 .15.
E 芸 沥
功 , 服 有 ( 一 0 试 问 显 咏 必 有 8「( 旭 不 09
2.9 试 用 定 理 6.3 木 求 方 程 组
乏_一j=一z一′】+灼z+炮z仪十灼鹦狮
E 李 沥
ctutuot a
d
2.11 试 证 (6, 30) 式 中 的 丿 和 双 都 是 公 式 (6. 29 中 丁 ( 的 简 单
E 伟 江 口

2.13 对 5.15), 试 用 $ 6 中 的 增 推 公 式 , 导 出 a(8) 和 于 ( 所 源 足
E

[ v昙 十 z最' E
E
E 向量场为)圣 E 沥 的 沥 招 沥 北 最汉 E

0 I「…】 E 兰 吴 浩 沥 a

出 万 D G 心 013? 所 满 足 的 Pleard-Fuchs 方 程
2.16 考 E 逊 55 朐 5 才

B
E 了zZ 园 巳 招 水E 招 玟

其 中 为 小 参 数 . 停 设 由 Absl 积 分 (5.12 定 义 的 一 阶 Mettikov 函 数 不 佩 为
雳 , 东其 问 宿 分 岔 的 最 高 阶 数 为 2, 利 用 定 理 6.4 水 其 一 阶 和 二 防 同 寇 分 岑
曲 线 在 ( 平 面 上 的 方 程 ( 即 对 8 的 一 阶 近 伴 方 程 ).
Page-306
n

a
E 一
E
东 , 当 氧 一 缸 一 咤 时 , 一 和 一 肖 一 育 一 育 一 星 , 其 中 肖 一
E t
n

簧`(凡_,0,茗'〉 E n

这 里 要 用 到 y (8“ ) 的 有 界 性 . 事 实 上 , 由 第 二 章 (5. 23) 和 本 节
E

E 朋
T7Ch「) 显 然 是 有 界 的 .

t 取
0 刑 东
E

[
8 国 。
E
一
s
d 一
李 剧
E 关
一

c 空 述
E 怀 标
Page-307
第 三 章 “ 几 类 余 维 2 的 平 面 向 量 场 分 岔

在 本 章 中 , 我 们 将 综 合 运 用 第 二 章 所 介 绍 的 儿 种 典 型 的 向 量
E 河
类 余 维 2 分 岔 .

E 坂 d 仪 [

CKo: 坤 =<fCao,

E 逊
玟
玟
玟

^一′0 1〕 ^_[o O)
】一〔00' 林
,

|
{
|
E 技 育 李 招
若 平 面 呐 重 场 在 奇 点 处 的 线 性 部 分 矩 阵 有 二 重 零 特 征 根 , 并

E
E 沥 园
E 沥
一 一
E 刀 一

化 条 件 下 , 扰 动 系 统 X 具 有 双 参 数 的 普 适 开 折 、 主 耐 参 考 文 献
Page-308
138 E

E
Eotcup t 蓁伍战…禹)伍 E 3 义

E 李 达 标标 水 沥 砂 发 东 取 水 述

闭 轨 是 不 穗 定 的 极 限 环 , 〖

E E 沥 颂 刑 河

E 玟 阮 朐 沥
E & 且 介 于 曲 线 与 曲 线 HL 之 间 时 , 系 统 (1.3) 有 唯 一 闭 转 , 它
是 不 稳 定 的 极 限 环 . 当 ( 趋 于 X 时 , 此 闭 执 缩 向 奇 点 B 当
0
E

5 芸技 江

人 育
E

转
i 胡

为 了 上 系 统 (1. 5)3 返 回 到 系 统 (1. 3), 由 变 换 (1. 4 知
E
Eapogzeeaueos y 生
Page-309
130 第 么 章 儿 类 仁 维 2 的 平 国 向 塔 堤 分 岑

为 :e = 1 的 余 维 2 分 岔 见 [Bol,2] 和 [T], 余 维 3.4 的 讨 论 分 别 见
LDRSL,2] 和 [LR1];4 一 2 和 8 一 3 的 余 维 分 岔 见 [Ho], e 一 2 的
E 如 江 E 述 才
E
有 情 形 的 详 细 介 绍 .

$ 1 二 重 零 特 征 根 : Bogdanov-Takens 系 统
E 描
i
gn 沥
[

Qz 十 bry
当 史 一 0 时 , 由 第 一 章 定 理 5. 13, 系 统 (1, 2) 的 任 一 非 退 化 开 折 可
转 化 为
dz'
05

E 玑

′羞昔 E 河 玟 2 招 沥 2 庞 a

0
其 中 Q,88E C“,Q(0,0) 二 土 1 一 sgn(a5),xE R,m 之 2. 为 确 定
起 见 , 取 Q(0,0) 一 1 Q(0,0) 一 一 1 的 情 况 可 类 似 讨 论 .

分 岑 图 , 轨 线 的 拓 扑 分 类

t
E 技

0 河 t
Page-310
园 国

E 一
E
0 簧赚 E 标 〉0}`

其 中 SN*,H,HL 分 别 为 歌 结 点 分 岔 曲 线 ,Hopf 分 岔 曲 线 和 同 宿
E

E
Eouuaopusetsst 河 b 育
为 了 证 明 定 理 1.1, 注 意 当 h 之 0 时 ,(1. 3) 在 原 点 附 近 无 奇
E 东 园 沥
p

人 人
E 不 许 e 沥 沥 3 沥 人 医 莲 3 扬
Page-311
e

E 沥

E 木
0

E 主
E 命 吴 i

日
uCS 耿 cn

E

0(″【熹). 【
Essa u
6sEuzstuaals 5 标 ri p
untugo d
0 胡

【 t 口

a
E

【
沥

则 当 y 一 0, |z|,|y| 造 当 小 时 ,
Page-312
E 招 d

标 国

E 技 邹
[

水

[
E 一
E
c
一 一
u l t 仪 李 沥

图 3-5
E 吊 ss t 述
E 志江 t 一
E i
的 情 形 ( 图 3-5(b)), 【
E 沥 林园
论 , 得 到 定 理 1. ! 的 结 论

E

E 园 形 扬
E
岔 幽 线 一 与 HL 的 表 达 式 的 高 阶 项 O(xi 中 , 这 种 差 异 为 我 们 证
E
Page-313
e

一
E 沥 刑

园 沥 标 人 阮

E
我 们 可 以 把 (1. 5 看 成 (1. 5)。 的 扰 动 系 统 , 后 者 为 Hamilon 系
统 , 它 有 鞍 点 4(1,0) 的 同 宿 辅 , 以 及 该 同 宿 技 所 园 的 以 点 B( 一 1,
0) 为 中 心 的 周 期 环 域 ( 见 第 二 章 例 5. 6 及 图 2-6) , 周 期 环 域 中 的 闭
E

E {(z,汊)_H(渥'仅)=″,_鲁<″<鲁, 0
E
| …z唧)=誓十z_誓. 医 林 办
当凡令_吾十0时'厂^缩向奇点B【当凡令音~6时,厂″趋于同

E

注 意 对 任 意 的 8,(1. 5)s 都 以 4,B 为 奇 点 , 且 A 为 鞍 点 ,B 为
指 标 十 1 的 奇 点 . 图 此 , 若 (1. 5)3 存 在 闭 辅 , 它 必 定 与 线 段 工 一
0 男 一 方 面 , 由 于 「 与 工 的 交
E 人 逊 国

0

E 阮 圆
正 向 及 负 向 延 续 分 别 与 z 辐 ( 第 一 次 ) 交 于 点 Q, 与 @i. 记 7(&,8,5
E 2 林

E 东 林
]

F仇峨痊)…I E 人
E
E
Page-314
渡

一

E
u e 一 切

E 一
p

林 林 水

(】.5)′
E 吊 3 河 沥 沥 沥 沥 9
E
2
E I 钨`捡 E 秉 沥 9
沥

0
E 圆 |

我 们 可 以 用 取 极 限 的 方 法 把 F(f,3,5) 的 定 义 域 扩 大 到 集 合
i {(^,贪,玄)_、鲁<″<熹瓜<艘<鳃瓢<芗<鲢卜

E 沥 沥 ( d 李 不
E 林匹 中 吊 林 训
E

【 {(′z,贯,玄)_`鲁<凇<暑』<朦<虬瓢<鳖<鲢}
Page-315
n

E
2 宇 辽
标 (z,o? 下 , 它 们 有 表 达 式 y 一 .(z),i = 1,2,3, 并 漪 足 条 件

E 23002 - E 标 技 肖 2
2

d
E 行 芸洽y 1
E 林
怀
E 述 荣 挂 逊 口 一 工 一 一
c
改
E 扬 E
[ 沥 E 刑
E 朐 胺
E
E 沥 1
【26iatiJ2 埕 伟 沥 沥 技 53 圭泓 人 1
Gronm) 一 a(g(ruyGm))
是 对 乙 的 恒 等 式 , 一 1,2,3. 把 上 式 对 z 求 导 , 得 到
E
再 对 丿 求 导 一 次 , 得 到
E 述 仪
E 达
0
E 医 途 0 E
E E 林 3 育
E 沥 李 怡 7 沥 3
Page-316
E E 浩 usssgenlel s

E ′
E 英 关 莲
E 沥 医 水 e

点附近解对参数的光滑依赖性定埋胭在″=_暑关于飙苔 Eouoa

E 昙
E 达

从 上 面 的 两 个 引 理 可 得 如 下 推 论 .
引 理 1.4 系 统 (1. 5)a(8 一 0) 存 在 鞍 点 A 的 同 宿 技 , 当 昆 仅

当F[兽默店】 E
I 李

E =I「 吊 2 仪 ( 河 仪
D 1

E
E 沥 月 3 或 3 yy 振 一 ) 志 1 浩河 服
E 厉
E 不 2 沥 2
证 明 “ 由 引 理 1.4 可 得 , 系 统 (1. 5)s 存 在 同 宿 转 的 充 要 条 件

E
人 刃 孙 一 刃 )-ooa 菊 招 a 吴 - 口 分 >eam
E 育 江

E 亭并且 F〔鲁默忑】稀)】 E 园

日 E 政 沥 吴 元 沥 / 发 26 振 力 朐
E 一 河
i
E 林 李n 兄 秉 7 河 437
Page-317
E p

78 吴 国 酸 伟 ( E 国 胡
00 一 273 万 (g′′(O,0))zy【′〈o)'

E 6 3 十 7737

E 标2 如 认 F
不 y 2 林 河 育 技 2 余 2
E c 江2 有 沥 朐 孕 77
En

22 沥 许 招 (歹_〉,受_(z) E 0'受z(z) 国 '】LTz蔓(z),重〉a(」T) E
E 李 人 技 水 一 根
T

s
E 刹 8E Cr,8C0) 一 1 由 于

F

E 江 仁 不 人

c
v 325
吊 区 墙 DL 一 {(z,o|z 公 0.3 之 乙 ) U (Czr,9)1z 二 01,D: 一 [Cr,
E 沥 仪 林 怀 仪 一
0
E 2
水 一
t

Edtos 0

0

uats n i

设 和 9 一 en 刹 用 了 E C? 和 0) 一 0 可 知 ,(1.18) 在 U f CD, U
E

E 林 圆

系 统 族 , 存 在 ( w) 平 面 上 保 持 原 点 的 C 变 换 , 它 把 其 中 一 个 系
Page-318
东 1 二 重 军 案 征 林, Bogdanor-Takene 系 给

E 一

E
们 需 要 证 明 同 宿 分 岔 对 参 数 3 的 一 致 伯 ( 见 第 二 章 $ 4, 即 证 明
存 在 8 丿 0 和 丿 丿 0, 使 得 对 任 意 和 予 只 要 0 一 一 卷 , 匹 一
E

0 东 el
E

《2 7(8) 在 扰 动 下 其 稳 定 流 形 与 不 稳 定 流 形 具 有 固 定 的 相 对
E

E

E 7 刑

许

技 河 沥 招 玟 沥 班2
0
E 。
国 (汊(1 不
由 于 对 任 意 的 3, 系 统 的 鞅 点 均 在 (z,9) 一 (1,0) , 而 且 系 统 在 读 点
的 发 散 量 为
E 氙 0 0

E

67| 一 多 , 就 有 了 (8) 乙 0. 结 合 第 二 章 定 理 4 3 和 宏 理 4. 4 可

E 刑
E

; 力 则 与 第 二 章 (4 19) 式 兵 似 的 线 “ 当

| 旦 ( r OCI8| 十 匹 一 口 (85447

E

M
Page-319
81 二 重 霖 特 征 松 , BogdenowTakens 系 统

n
E 国 育 c

不 英
E

E 林 二

由 引 理 1. 12 的 统 论 (2) 立 即 推 得 本 引 理 成 立 , 【
引 理 1. 13 具 有 (1. 3) 形 式 相 应 于 不 间 Q, 申 的 两 个 系 统 族
E
E 国 圆 u
[
F 扬

'盖 园 匹 林 或 坂 玟 2 口 朋 班 一

巳 目
E
E
E 沥 园 李 林水
0 河

界
7

一 一

F 二 3

蒜 E 怀 x 坤 技
这 里 2 一 2C1 裕 示 血 一 山(4D , 吊 一 加 ( 一 p 一 .

Esusyscpogetuki t 不 国
E 胡 一
Page-320
136 E 浩

[ 理 0 一 朐
(0, 加 , 使 只 要 0 一 一 8, 区 一 咤 (8)| 一 加 , 就 有
E 刑 罡
r E

匹
所 以 结 论 (2) 也 成 立 . 引 理 1. 6 得 证 , 〖
E

0
一 万 人 人 品 (80,0 一 8 一 口 时 ,(1,5)8 在 旦 点 的 I 邻 域 内 恰 有
一 个 闭 轨 , 它 是 不 稳 定 的 极 限 环 ; 而 当 吊 (3) =<5 一 多 (8) 十 办 0 一

E 河 i
E 国 沥 437 河

, 目
( 沥 园 穴 河 友 人 0

i 标
E 告殴砧」 园 江 关

E 量〔8 E

E 江 个
(8) 时 ,e(8,f(8)) 一 0,8C8,(8)) 不 0. 即 第 二 章 $ 3 中 的 条 件

(Hi ) 成 立 . 计 算 表 明 ,

E
命
[
E
E 国 林胡
[ 木
Page-321
E E 英

E 技
点 相 应 邻 域 乙 (9 中 极 限 集 间 的 同 胚 ( , 然 后 把 此 同 胚 扩 展 到
[ 沥 n t 沥 认 不
E

E 林沥 一 圆 沥 5287

E

FF 扬

E 。
豇=州十″z汊十z 2

是 奇 异 向 量 李 (1. 2)(a5 > 0 的 一 个 普 适 开 折 ,
E 吴
变 换 转 化 为 与 下 列 系 统 等 价 的 系 绕 ( 措 在 原 点 的 邻 域 内 》

[ 二
E
「 Q

鲁萱 E 应 玟 225 标 河 t 林述

E d
2

E
沥
E

E

养 中 (h , 是 猎 立 的 参 数 . 如 果 我 们 把 (1. 19) 看 成 是 含 m 十 2 个
E 不 不 园 E 东
E 沥
unett 8 u 吉
王

巳 国
E
目 莲 2
沥 命
Page-322
n

a
E 一
E
东 , 当 氧 一 缸 一 咤 时 , 一 和 一 肖 一 育 一 育 一 星 , 其 中 肖 一
E t
n

簧`(凡_,0,茗'〉 E n

这 里 要 用 到 y (8“ ) 的 有 界 性 . 事 实 上 , 由 第 二 章 (5. 23) 和 本 节
E

E 朋
T7Ch「) 显 然 是 有 界 的 .

t 取
0 刑 东
E

[
8 国 。
E
一
s
d 一
李 剧
E 关
一

c 空 述
E 怀 标
Page-323
E 河 ss E

拓 扑 等 价 , 这 里 把 (L, 22》 看 成 含 参 数 的 系 统 , 特 别 地 , 在 (1. 21)
医 起 余 i 园
D i
0 怀 i 技
E 伟 l 伟
统 (1. 2) 的 一 个 普 适 开 折 , 【

$ 2 “ 二 重 零 特 征 根 : 1:2 共 振 问 题

E

玟
黯=仍 蕃=僻…十奴钺 [
为 了 陈 述 俞 洁 , 我 们 在 这 里 只 考 虔 三 次 正 规 形 而 去 掉 了 高 阶 项 . 事
玟
质 的 影 响 ( 可 参 考 12.
医 加 仪

鲁言=鹏 盖=士崴一感扔 [

E 沥 小
E 胡

类 似 于 第 一 章 定 理 5. 13, 有 下 面 的 结 果

E 东 园 i 东
D 胡
开 折 化 成 与 下 列 向 量 场 等 价 的 形 式

[
F 河 河 刑

刑
D
Page-324
138 E

E
Eotcup t 蓁伍战…禹)伍 E 3 义

E 李 达 标标 水 沥 砂 发 东 取 水 述

闭 轨 是 不 穗 定 的 极 限 环 , 〖

E E 沥 颂 刑 河

E 玟 阮 朐 沥
E & 且 介 于 曲 线 与 曲 线 HL 之 间 时 , 系 统 (1.3) 有 唯 一 闭 转 , 它
是 不 稳 定 的 极 限 环 . 当 ( 趋 于 X 时 , 此 闭 执 缩 向 奇 点 B 当
0
E

5 芸技 江

人 育
E

转
i 胡

为 了 上 系 统 (1. 5)3 返 回 到 系 统 (1. 3), 由 变 换 (1. 4 知
E
Eapogzeeaueos y 生
Page-325
e

E 沥

E 木
0

E 主
E 命 吴 i

日
uCS 耿 cn

E

0(″【熹). 【
Essa u
6sEuzstuaals 5 标 ri p
untugo d
0 胡

【 t 口

a
E

【
沥

则 当 y 一 0, |z|,|y| 造 当 小 时 ,
Page-326
儿 美 余 纵 2 的 乎 面 问 最 扬 分 岔 -

E dn 圆
训

空 间 的 变 换 , 把 (2. 3)* 化 为

万 二
d′_叟'

窑 E 河 吴 丿 北 学 沥 u 胡 玟 仪 A 技 2

[
E 林 c c 道 河 育 [ 李 02 才
E 沥
Eooo 述 东

(莹=y' 澄=橇z+砀汊士遨_碳拂 [

E
E 东 技 莲

E

0 沥 水
E 2 口 浩 沥 一 不 河
E 汀 行 2 沥 技 沥

E

02 招

E el 训 浩 达 i
Pnpois12 晚 沥 许 2 林 2
林沥
E

u

E
其 中 c < 0. 752,《2. 5)- 的 转 线 拓 扑 分 类 见 图 3-7.
Page-327
E 圆

对 (2.5)+ 的 讨 论 与 1 很 相 似 , 所 以 下 面 仅 对 (2. 5 的 情 形
Page-328
E 招 d

标 国

E 技 邹
[

水

[
E 一
E
c
一 一
u l t 仪 李 沥

图 3-5
E 吊 ss t 述
E 志江 t 一
E i
的 情 形 ( 图 3-5(b)), 【
E 沥 林园
论 , 得 到 定 理 1. ! 的 结 论

E

E 园 形 扬
E
岔 幽 线 一 与 HL 的 表 达 式 的 高 阶 项 O(xi 中 , 这 种 差 异 为 我 们 证
E
Page-329
E 蒙

E 河浩 明 仪

首 先 ,(2.5)- 的 奇 点 满 尼 y 一 0 及 az 一 a 一 0. 与 第 一 章 例
玟

其 次 考 察 e 一 0 的 情 形 这 时 (z,y) 二 (0,0) 是 (2. 5)- 的 唯
E

E
[ 小
医 余 江

程应用第二章公式(3.3〉可得,Re(q)=_告<o. 因 此 , 由 第 二 章

定 理 8. 1 知 , 在 曲 线 H 上 发 生 Hopf 分 岔 .
E

人 一
E 河 加 0
蕃=撕 窑=z〔蹴十厥首_麟澈′
E 月

当 0 一 8 人 1 时 的 扰 动 系 统 . 系 统 (2. 8) 有 首 次 积 分
H位靓)=誓_吾十誓=允 【

E i

E 沥 的两条对称闭轨,当颢峥_盖时,它们分别缩

向 这 两 个 奇 点 f 当 ~ 0* 时 , 它 们 扩 大 而 形 成 鞋 点 (0,0) 的 对 称 双

[ 一

见 图 3-8. 若 令 工 二 D U , 其 中
E
Page-330
E

E
E 沥 不

【 木
园 t 0 园
E

E 一 园 人
0
n 招
M
d
1.2 一 1.4 类 似 可 得

E 标
E

巳 张 一
E L你机n醛 吴 E

而 且 7 成 为 (2.7) 的 同 宿 ( 双 同 宿 ) 扔 , 当 且 仅 当 FCO~,8,6 一 0
0 规
E 江

E L 0
D

其中缥仇)=L ydz, & 一 0,2. 与 第 二 章 中 (5.18) 一 (5. 20) 的
8
Page-331
n

E
2 宇 辽
标 (z,o? 下 , 它 们 有 表 达 式 y 一 .(z),i = 1,2,3, 并 漪 足 条 件

E 23002 - E 标 技 肖 2
2

d
E 行 芸洽y 1
E 林
怀
E 述 荣 挂 逊 口 一 工 一 一
c
改
E 扬 E
[ 沥 E 刑
E 朐 胺
E
E 沥 1
【26iatiJ2 埕 伟 沥 沥 技 53 圭泓 人 1
Gronm) 一 a(g(ruyGm))
是 对 乙 的 恒 等 式 , 一 1,2,3. 把 上 式 对 z 求 导 , 得 到
E
再 对 丿 求 导 一 次 , 得 到
E 述 仪
E 达
0
E 医 途 0 E
E E 林 3 育
E 沥 李 怡 7 沥 3
Page-332
E

结 果 类 似 「 容 易 得 到
E 李 标 责 E

- 目 - 明 a 鲸

R
E
E 团
1 E [ 许

1 当 h 一 二

E

招
E 坂 仪 ( d
引 理 2.5 “ 函 数 P 有 如 下 性 质

0 乏' E
M

03 北 a 噩<丸<汨 E 沥 扬
E2 沥

[ 音'尸′(′【′) 沥 匹 0 [

医 芸沥 育 王
Page-333
E 招 E

E

E 一
从 (2,12 可 知 , 研 究 F(t,0,50 的 零 点 可 用 直 线 5 一 常 数 咤 去 戬
t 2

系统的环绕一个奇点的“大极限环”=当告〈魏<1时,截得两个交

E _瓮〈彻< E

的″小极限环”,后者相应于 M _言时 E

翻双同宿轨,且外侧仍有一个“大极限环墉而当″<巍〈言舶]

fQh“2) 时 , 两 个 交 点 的 槲 垄 标 ,fs 心 0, 系 统 出 现 两 个 “ 大 极 限
E
Page-334
E p

78 吴 国 酸 伟 ( E 国 胡
00 一 273 万 (g′′(O,0))zy【′〈o)'

E 6 3 十 7737

E 标2 如 认 F
不 y 2 林 河 育 技 2 余 2
E c 江2 有 沥 朐 孕 77
En

22 沥 许 招 (歹_〉,受_(z) E 0'受z(z) 国 '】LTz蔓(z),重〉a(」T) E
E 李 人 技 水 一 根
T

s
E 刹 8E Cr,8C0) 一 1 由 于

F

E 江 仁 不 人

c
v 325
吊 区 墙 DL 一 {(z,o|z 公 0.3 之 乙 ) U (Czr,9)1z 二 01,D: 一 [Cr,
E 沥 仪 林 怀 仪 一
0
E 2
水 一
t

Edtos 0

0

uats n i

设 和 9 一 en 刹 用 了 E C? 和 0) 一 0 可 知 ,(1.18) 在 U f CD, U
E

E 林 圆

系 统 族 , 存 在 ( w) 平 面 上 保 持 原 点 的 C 变 换 , 它 把 其 中 一 个 系
Page-335
152 E

江
D
数 学 论 证 要 利 用 ( 对 参 数 一 致 的 )Hopf 分 岔 定 理 , 同 宿 分 岔 定 理 ,
E
对 于 具 有 “8 字 型 “ 双 同 宿 轨 的 平 面 系 统 ( 见 图 3-8), 在 [Lw]
a河

E 技 河 5 浩 北 王 口

由 第 一 章 习 题 1.5, 以 (0,0) 为 奇 点 并 具 有 二 重 零 特 征 根 , 旋
转誓鼬>盼 不 变 的 向 量 场 具 有 如 下 的 复 正 规 形

羞 E 2 河

E

E 扬
E

羞 E 述 江 沥

E 述 u 辽 述 5
E e 沥 胡 一 沥
E ,

盖 E

塞 沥 河 达 人 辽 达

E
星 然 , 当 a 一 0,6 失 0 时 , 发 生 Hopt 分 岔 . 若 a 心 0, 则 由 3. 3 的
第 一 个 方 程 可 知 , 在 r 一 0 的 小 邻 域 内 的 所 有 轨 线 当 5 十 co 均 苔
医
Page-336
81 二 重 霖 特 征 松 , BogdenowTakens 系 统

n
E 国 育 c

不 英
E

E 林 二

由 引 理 1. 12 的 统 论 (2) 立 即 推 得 本 引 理 成 立 , 【
引 理 1. 13 具 有 (1. 3) 形 式 相 应 于 不 间 Q, 申 的 两 个 系 统 族
E
E 国 圆 u
[
F 扬

'盖 园 匹 林 或 坂 玟 2 口 朋 班 一

巳 目
E
E
E 沥 园 李 林水
0 河

界
7

一 一

F 二 3

蒜 E 怀 x 坤 技
这 里 2 一 2C1 裕 示 血 一 山(4D , 吊 一 加 ( 一 p 一 .

Esusyscpogetuki t 不 国
E 胡 一
Page-337
0

罄】=彤,Ez=瞅,r=茆,z=袁'

E 咤 s 李 中 浩 技

城 pd - 网 十 (evp, = R(p,b,8,

0 技

E 木

其 中
E 林
E 达
E 6323

羞=爪1-妨),塞=宣十柘障 《3.6)

此 系 统 有 唯 一 的 不 变 环 不 二 {Cp,b)1p 一 1} , 它 是 股 引 的 . 当 幼 十
0 d t 刑 朗 庞
4
匿>o(相应地`由(3.4),存在勇>o)使得对每】个贪 E 伟
E 加 红 5 s 才
n 技
E 东
E 月 沥
a 标 国 江 政 不 月
E 不 江 u 林( 汀 河
t 沥 口 5 3 木
E
E 林 江 林 一
E 於 八 园 中 p 林技 3 技
E 林国 胡
沥
Page-338
E E 英

E 技
点 相 应 邻 域 乙 (9 中 极 限 集 间 的 同 胚 ( , 然 后 把 此 同 胚 扩 展 到
[ 沥 n t 沥 认 不
E

E 林沥 一 圆 沥 5287

E

FF 扬

E 。
豇=州十″z汊十z 2

是 奇 异 向 量 李 (1. 2)(a5 > 0 的 一 个 普 适 开 折 ,
E 吴
变 换 转 化 为 与 下 列 系 统 等 价 的 系 绕 ( 措 在 原 点 的 邻 域 内 》

[ 二
E
「 Q

鲁萱 E 应 玟 225 标 河 t 林述

E d
2

E
沥
E

E

养 中 (h , 是 猎 立 的 参 数 . 如 果 我 们 把 (1. 19) 看 成 是 含 m 十 2 个
E 不 不 园 E 东
E 沥
unett 8 u 吉
王

巳 国
E
目 莲 2
沥 命
Page-339
s g

E
RCp,b,8) 一 8(p,0,8) 一 det(
E
det〔a(R,6) 仪

E
3(p, 百

2KP: 外
E
DKp,b,8) 一 (3fF 一 Dcos(90) 十 25zsin(ab) 十 O(8)。

医 述
图 此 ,(3. 7) 可 改 写 成
E 《C3.9)
E
巳 仪 2 李 丿
E
E
C
E
由 (3.8) 知 DD|aruco 一 0 在 伟 附 近 对 9 有 2 个 根 . 因 此 , 集 合
E E a 春 许 ;

, E
Page-340
E 河 ss E

拓 扑 等 价 , 这 里 把 (L, 22》 看 成 含 参 数 的 系 统 , 特 别 地 , 在 (1. 21)
医 起 余 i 园
D i
0 怀 i 技
E 伟 l 伟
统 (1. 2) 的 一 个 普 适 开 折 , 【

$ 2 “ 二 重 零 特 征 根 : 1:2 共 振 问 题

E

玟
黯=仍 蕃=僻…十奴钺 [
为 了 陈 述 俞 洁 , 我 们 在 这 里 只 考 虔 三 次 正 规 形 而 去 掉 了 高 阶 项 . 事
玟
质 的 影 响 ( 可 参 考 12.
医 加 仪

鲁言=鹏 盖=士崴一感扔 [

E 沥 小
E 胡

类 似 于 第 一 章 定 理 5. 13, 有 下 面 的 结 果

E 东 园 i 东
D 胡
开 折 化 成 与 下 列 向 量 场 等 价 的 形 式

[
F 河 河 刑

刑
D
Page-341
E 园

E 吊 技 河 满足以卞条件'
林2
5
e
【 述 胡 吴 人 0

E 万 木 晓 发 根

0, 得 刹
E
[ 逊
整 理 得
【 f 尔 J 河 Cu 河
E
E 标 O

E 颂 工 0 沥 河 扬 用 一 一 一 二 ( ).

最后,考虑集合‖仍矾龋谬汁R=0,D=0,6=0),注意

tneC
E d 邦 育 丿
c 吴 0 育
由 对 称 性 知 , 肖 +:( 一 M,(8, 下 匹 佼 计 M:(83 一 M,(8). 在 等
式 (3. 12) 中 取 丿 一 0,1 得
E 述

E 达 仪

利 用 (3. 12) 和 P 一 1 十 0(3) 整 珀 上 式 可 得
国 c c 育 园 0
E
[ 0
Page-342
儿 美 余 纵 2 的 乎 面 问 最 扬 分 岔 -

E dn 圆
训

空 间 的 变 换 , 把 (2. 3)* 化 为

万 二
d′_叟'

窑 E 河 吴 丿 北 学 沥 u 胡 玟 仪 A 技 2

[
E 林 c c 道 河 育 [ 李 02 才
E 沥
Eooo 述 东

(莹=y' 澄=橇z+砀汊士遨_碳拂 [

E
E 东 技 莲

E

0 沥 水
E 2 口 浩 沥 一 不 河
E 汀 行 2 沥 技 沥

E

02 招

E el 训 浩 达 i
Pnpois12 晚 沥 许 2 林 2
林沥
E

u

E
其 中 c < 0. 752,《2. 5)- 的 转 线 拓 扑 分 类 见 图 3-7.
Page-343
E 圆

对 (2.5)+ 的 讨 论 与 1 很 相 似 , 所 以 下 面 仅 对 (2. 5 的 情 形
Page-344
156 E

总 结 上 面 的 讨 论 可 得
引 理 3.1 ([TJ〉 在 (p,0,,8) 空 间 中 印 的 邻 域 内 , 集 合
E

E 沥 2 河 吊
(2K(8,8}(j 一 1,2) 组 成 , 漪 尽 (8) 一 Mu(3) 一 $58《 十
【 江

二 林
下 列 曾 线 组 成 : ,

人 2 才 八 沥 命 江 大 2

s 沥 c 沥
E

顺 唐 n
不 p
[ 沥 2

Ea2 0
Page-345
E 蒙

E 河浩 明 仪

首 先 ,(2.5)- 的 奇 点 满 尼 y 一 0 及 az 一 a 一 0. 与 第 一 章 例
玟

其 次 考 察 e 一 0 的 情 形 这 时 (z,y) 二 (0,0) 是 (2. 5)- 的 唯
E

E
[ 小
医 余 江

程应用第二章公式(3.3〉可得,Re(q)=_告<o. 因 此 , 由 第 二 章

定 理 8. 1 知 , 在 曲 线 H 上 发 生 Hopf 分 岔 .
E

人 一
E 河 加 0
蕃=撕 窑=z〔蹴十厥首_麟澈′
E 月

当 0 一 8 人 1 时 的 扰 动 系 统 . 系 统 (2. 8) 有 首 次 积 分
H位靓)=誓_吾十誓=允 【

E i

E 沥 的两条对称闭轨,当颢峥_盖时,它们分别缩

向 这 两 个 奇 点 f 当 ~ 0* 时 , 它 们 扩 大 而 形 成 鞋 点 (0,0) 的 对 称 双

[ 一

见 图 3-8. 若 令 工 二 D U , 其 中
E
Page-346
沥 人 157

u 一 才
E 标
宁 【

办 (8 一 MMo(3) 一 8 一 十 O(8e3),8 一 0 怡 式 0.
E 园
E
曲 缇

0
[
E 述 芸 切 月
肖 (3) 一 口 ( 二 酥 CM(89 一 Mo(8)) 一 688 十 O(8p-1,
其 中 8 央 0. 定 理 3.2 得 证 , 【

题 与 怡 考 题 三

3 1 在 引 理 1.6 和 引 理 1.7 中 , 为 什 么 要 分 别 利 用 对 参 数 一 致 的 同 宿 分
岷 定 理 和 对 参 数 一 致 的 Hopt 分 岔 定 理 ? 这 对 最 终 得 到 定 理 1.1 有 什 么 作 用 ?
3.2 证 明 引 理 2
$.3 证 明 引 理 2.5. ,
3.4 对 方 程 (2.5)7 讨 论 它 的 分 岔 现 象 , 并 证 明 分 岑 图 3-6 的 正 确 性
E 林 胡
E 技 d

E
E 一

c
E 昔卜《l时,(2^7)在原点的发散量为 弼 奂 0.)
Page-347
E

E
E 沥 不

【 木
园 t 0 园
E

E 一 园 人
0
n 招
M
d
1.2 一 1.4 类 似 可 得

E 标
E

巳 张 一
E L你机n醛 吴 E

而 且 7 成 为 (2.7) 的 同 宿 ( 双 同 宿 ) 扔 , 当 且 仅 当 FCO~,8,6 一 0
0 规
E 江

E L 0
D

其中缥仇)=L ydz, & 一 0,2. 与 第 二 章 中 (5.18) 一 (5. 20) 的
8
Page-348
医 沥 l 李

2
五 章 中 是 我 们 研 究 空 间 R 中 歌 点 同 宿 分 岔 的 基 础 , 同 E
也 有 其 自 身 的 重 要 价 值 . 在 8 1 中 我 们 证 明 一 个 双 曲 不 动 点 定 理 .
在 2 中 引 进 符 号 助 力 学 的 基 本 概 念 . 在 8 3 中 给 出 马 蹄 存 在 定 理
E
引 理 . 在 8 5 中 作 为 $ 2 一 $ 4 中 诺 结 果 的 一 个 应 用 , 我 们 将 给 出 E
中 Birkhoff-Smale 定 理 的 证 明 .

i 园 d
Wiggins 的 书 [Wig] 中 都 能 挂 到 , 但 此 处 所 有 定 理 的 证 明 郭 是 独 立
给 出 的 . 我 们 力 图 把 儿 何 直 观 与 数 学 的 严 密 性 统 一 起 来 , 并 给 予
读 者 一 套 易 于 掌 握 的 方 法 , 用 以 解 决 高 维 空 间 中 其 它 类 似 的 闰 题 .

$ 1 双 曲 不 动 点 定 理

定 理 的 陈 述

育 g
L 沥 c 东 技 才
E 江
E t 沥 东 一 一
图 象 . 当 一 co<a<8<< 十 co 时 , 称 点 (af(a)),(8,/(8) 为 该 曲 线
E 贺 伟 江
为 C 的 y 的 函 数 = 二 g(y),yE Ce, 切 的 图 象 . 特 别 地 , 当 a 一 一 co,
Page-349
E

结 果 类 似 「 容 易 得 到
E 李 标 责 E

- 目 - 明 a 鲸

R
E
E 团
1 E [ 许

1 当 h 一 二

E

招
E 坂 仪 ( d
引 理 2.5 “ 函 数 P 有 如 下 性 质

0 乏' E
M

03 北 a 噩<丸<汨 E 沥 扬
E2 沥

[ 音'尸′(′【′) 沥 匹 0 [

医 芸沥 育 王
Page-350
E 招 E

E

E 一
从 (2,12 可 知 , 研 究 F(t,0,50 的 零 点 可 用 直 线 5 一 常 数 咤 去 戬
t 2

系统的环绕一个奇点的“大极限环”=当告〈魏<1时,截得两个交

E _瓮〈彻< E

的″小极限环”,后者相应于 M _言时 E

翻双同宿轨,且外侧仍有一个“大极限环墉而当″<巍〈言舶]

fQh“2) 时 , 两 个 交 点 的 槲 垄 标 ,fs 心 0, 系 统 出 现 两 个 “ 大 极 限
E
Page-351
E 驱 ri

a 水 纳 t a

东
E
E

E

E 医 一 二 技
技 江
称 为 边 界 的 水 平 部 分 , 记 作 #D: 那 组 A 垂 直 对 边 称 为 边 界 的 坩
E

E 河 0 5
空 间 . 令

2 77 木 人 伟
沥 07 水 7

E

定 义 1.4 令 DCRIXR ` 图
E 仪 a 沥 n 河
n 沥 林
Eupnicsosisi 沥 河

3许 吴
Y E D

[ 刑
E

医

兰 7
E 仁 9

[ 兄 一 万 3

E 沥 技
李
Page-352
E 第 四 章 双 曲 不 动 点 及 马 路 存 圭 定 理

E s
E 亚 n
E i 匕
E
E 怡
202
则 了 在 马 中 有 唯 一 双 曲 不 动 点

上 述 定 理 的 儿 何 直 观 见 囹 4-2(a).

E
E

E 0
E [ 吴 吴 17

E 罚
E 一

e
Page-353
152 E

江
D
数 学 论 证 要 利 用 ( 对 参 数 一 致 的 )Hopf 分 岔 定 理 , 同 宿 分 岔 定 理 ,
E
对 于 具 有 “8 字 型 “ 双 同 宿 轨 的 平 面 系 统 ( 见 图 3-8), 在 [Lw]
a河

E 技 河 5 浩 北 王 口

由 第 一 章 习 题 1.5, 以 (0,0) 为 奇 点 并 具 有 二 重 零 特 征 根 , 旋
转誓鼬>盼 不 变 的 向 量 场 具 有 如 下 的 复 正 规 形

羞 E 2 河

E

E 扬
E

羞 E 述 江 沥

E 述 u 辽 述 5
E e 沥 胡 一 沥
E ,

盖 E

塞 沥 河 达 人 辽 达

E
星 然 , 当 a 一 0,6 失 0 时 , 发 生 Hopt 分 岔 . 若 a 心 0, 则 由 3. 3 的
第 一 个 方 程 可 知 , 在 r 一 0 的 小 邻 域 内 的 所 有 轨 线 当 5 十 co 均 苔
医
Page-354
E 02

第 二 步 证 明 y 一 门 Di 和 一 一 [D-; 分 别 是 端 点 属 于 3eD 和
E 标

第 三 步 证 明 一 V 是 了 的 一 个 双 曲 不 动 点
E

儿 个 引 理

引 理 1. 6 平 面 R!XRI 上 一 条 心 水 平 曳 线 与 一 条 A 垂 直 曲
E

E
厂
[

[ 汀 刑 T
才

[ 芸 c 史 |

E 二
是 端 点 集 属 于 3,D 和 &D 的 A8 水 平 曲 线 及 A 垂 直 曲 线 , 则 上
E 沥

证 明 交 点 的 存 在 性 可 由 连 续 性 得 到 , 而 唯 一 性 由 引 理 1. 6 保
E 河

E 吴 7 李 人
水

5

[ 一
E

0

E 吴 页
E 命a 林 人 水
E
Page-355
0

罄】=彤,Ez=瞅,r=茆,z=袁'

E 咤 s 李 中 浩 技

城 pd - 网 十 (evp, = R(p,b,8,

0 技

E 木

其 中
E 林
E 达
E 6323

羞=爪1-妨),塞=宣十柘障 《3.6)

此 系 统 有 唯 一 的 不 变 环 不 二 {Cp,b)1p 一 1} , 它 是 股 引 的 . 当 幼 十
0 d t 刑 朗 庞
4
匿>o(相应地`由(3.4),存在勇>o)使得对每】个贪 E 伟
E 加 红 5 s 才
n 技
E 东
E 月 沥
a 标 国 江 政 不 月
E 不 江 u 林( 汀 河
t 沥 口 5 3 木
E
E 林 江 林 一
E 於 八 园 中 p 林技 3 技
E 林国 胡
沥
Page-356
E 玟

上 述 引 理 可 参 明 图 4-3.

国 4-3

E 团 医 仪 江 水 技
E t 水 技 振 沥
0
江
沥 不
E 述 江 一 石
7DinD: 是 一 个 (fa,s) 矩 形 , 滔 足 (fDif1D:)CC3Da. 类 似 地 可
E 前

医 沥 2 口 东 9 弃 锋 八 河
0

E y 吊 4 户

2
令

E 沥 0 刑
E
Page-357
s g

E
RCp,b,8) 一 8(p,0,8) 一 det(
E
det〔a(R,6) 仪

E
3(p, 百

2KP: 外
E
DKp,b,8) 一 (3fF 一 Dcos(90) 十 25zsin(ab) 十 O(8)。

医 述
图 此 ,(3. 7) 可 改 写 成
E 《C3.9)
E
巳 仪 2 李 丿
E
E
C
E
由 (3.8) 知 DD|aruco 一 0 在 伟 附 近 对 9 有 2 个 根 . 因 此 , 集 合
E E a 春 许 ;

, E
Page-358
E 技t 163

E 途
E
医 沥 t 仁 水 述
Eopusd y 丞 i
E
E
芸 E

E
2 5
P

E 东 江

E 沥
个 微 分 和 躯 , 满 尸 (Ah,s) 锦 形 条 件 . 设 7CD 是 一 条 C 光 溥 的 A
Eaatey s s 邦
E

[ 李 仁 王

E 园 0

E 河 1十觞'
>1 是 定 义 1.4 中 的 常 数

b 善 林 心 一 t
诉 的 证 明 可 用 类 似 方 法 得 到 -
E
d
E 俊 应 仪
刑
E
入
国 E
E
E 技

0 日
Page-359
E 园

E 吊 技 河 满足以卞条件'
林2
5
e
【 述 胡 吴 人 0

E 万 木 晓 发 根

0, 得 刹
E
[ 逊
整 理 得
【 f 尔 J 河 Cu 河
E
E 标 O

E 颂 工 0 沥 河 扬 用 一 一 一 二 ( ).

最后,考虑集合‖仍矾龋谬汁R=0,D=0,6=0),注意

tneC
E d 邦 育 丿
c 吴 0 育
由 对 称 性 知 , 肖 +:( 一 M,(8, 下 匹 佼 计 M:(83 一 M,(8). 在 等
式 (3. 12) 中 取 丿 一 0,1 得
E 述

E 达 仪

利 用 (3. 12) 和 P 一 1 十 0(3) 整 珀 上 式 可 得
国 c c 育 园 0
E
[ 0
Page-360
E E a g

对 河 1, 令 g 一 乙 , 则 也 滔 足 (x, 户 3 锦 形 条 件 , 只 是 其 中 的 扩 张
s l
证 明 引 理 . 〖

E 不
E

0 园 0
E 吊 5 诊
n
E 技 技 才 2 技

E
上 界 , 则 由 引 理 1. 10, 有

E d 林 i
d
江 )
E
E
故 存 在 常 数 C, 使 得
E
同 理 可 证 存 在 常 数 C, 使 得
E 浩

定 理 1. 5 证 明 的 完 成
E
1
E av…a^D.
E
Page-361
156 E

总 结 上 面 的 讨 论 可 得
引 理 3.1 ([TJ〉 在 (p,0,,8) 空 间 中 印 的 邻 域 内 , 集 合
E

E 沥 2 河 吊
(2K(8,8}(j 一 1,2) 组 成 , 漪 尽 (8) 一 Mu(3) 一 $58《 十
【 江

二 林
下 列 曾 线 组 成 : ,

人 2 才 八 沥 命 江 大 2

s 沥 c 沥
E

顺 唐 n
不 p
[ 沥 2

Ea2 0
Page-362
E 招 E

E 标 an
医 故

E 沥 坂 cnmueyf 妙 口 沥 政 d 河 沥
方 向 拉 伸 , 故 不 动 点 O 是 双 曲 的

E

园
E

符 号 序 列 空 间 及 其 结 构

令
吊 二 人 2oNj, 切 丿 2.
在 5 上 引 进 如 下 度 量 。
【 尘
40 口 二 E 沥 胡 二 团
E 林 沥 一 d 东 东
阁 . 仔

E

的 双 边 无 穷 序 列
[
巳
技

这 里 a 5 是 在 分 量 S 上 的 值 . 在 SY 上 我 们 如 下 引 选 度 重 .
Page-363
i

沥

27 招

5
E anatpe ss
E 招泊 罚
pouE e 真 诊
匹 a
E 命a an 国 2 刑 一

E
E 集 称 为 完 全 不 连
通 的 , 如 果 它 的 每 一 个 连 通 分 支 只 包 含 一 个 点 . 闷 时 具 有 上 面 两 条
Eeaitgpsss 医 口 u

引 理 2.2 空 间 3Y 在 度 量 (2.2) 下 是

0

E

E t 二 日
Page-364
沥 人 157

u 一 才
E 标
宁 【

办 (8 一 MMo(3) 一 8 一 十 O(8e3),8 一 0 怡 式 0.
E 园
E
曲 缇

0
[
E 述 芸 切 月
肖 (3) 一 口 ( 二 酥 CM(89 一 Mo(8)) 一 688 十 O(8p-1,
其 中 8 央 0. 定 理 3.2 得 证 , 【

题 与 怡 考 题 三

3 1 在 引 理 1.6 和 引 理 1.7 中 , 为 什 么 要 分 别 利 用 对 参 数 一 致 的 同 宿 分
岷 定 理 和 对 参 数 一 致 的 Hopt 分 岔 定 理 ? 这 对 最 终 得 到 定 理 1.1 有 什 么 作 用 ?
3.2 证 明 引 理 2
$.3 证 明 引 理 2.5. ,
3.4 对 方 程 (2.5)7 讨 论 它 的 分 岔 现 象 , 并 证 明 分 岑 图 3-6 的 正 确 性
E 林 胡
E 技 d

E
E 一

c
E 昔卜《l时,(2^7)在原点的发散量为 弼 奂 0.)
Page-365
E 招

河 八
s, 使 得 对 其 巾 任 意 序 列 口 一 (aijiez, 古 一 仁 jez, 有 a 二 页 . 下
EoE l 志
医 沥 朱达 朐 许 4
E 0 沥 团 E 沥
u i 志 江 东 5 国 衍
6
浩
E
U 扬 一 才 扬
[ 史
bn d 吴 河
心 巳 经 取 定 , 取 aL1EKu 使 得 心 1 世 i 一 0,1,

2

1 一
E

E 水
人
[

[ 不

一
c 东 |

E
E 沥
2
Page-366
医 沥 l 李

2
五 章 中 是 我 们 研 究 空 间 R 中 歌 点 同 宿 分 岔 的 基 础 , 同 E
也 有 其 自 身 的 重 要 价 值 . 在 8 1 中 我 们 证 明 一 个 双 曲 不 动 点 定 理 .
在 2 中 引 进 符 号 助 力 学 的 基 本 概 念 . 在 8 3 中 给 出 马 蹄 存 在 定 理
E
引 理 . 在 8 5 中 作 为 $ 2 一 $ 4 中 诺 结 果 的 一 个 应 用 , 我 们 将 给 出 E
中 Birkhoff-Smale 定 理 的 证 明 .

i 园 d
Wiggins 的 书 [Wig] 中 都 能 挂 到 , 但 此 处 所 有 定 理 的 证 明 郭 是 独 立
给 出 的 . 我 们 力 图 把 儿 何 直 观 与 数 学 的 严 密 性 统 一 起 来 , 并 给 予
读 者 一 套 易 于 掌 握 的 方 法 , 用 以 解 决 高 维 空 间 中 其 它 类 似 的 闰 题 .

$ 1 双 曲 不 动 点 定 理

定 理 的 陈 述

育 g
L 沥 c 东 技 才
E 江
E t 沥 东 一 一
图 象 . 当 一 co<a<8<< 十 co 时 , 称 点 (af(a)),(8,/(8) 为 该 曲 线
E 贺 伟 江
为 C 的 y 的 函 数 = 二 g(y),yE Ce, 切 的 图 象 . 特 别 地 , 当 a 一 一 co,
Page-367
第 四 敦 双 曲 不 动 点 及 马 路 存 在 定

E
zeanusy
引 理 2.3 移 位 晔 射 v 是 2 到 自 身 的 同 胜 .
E 国 5 兄 招 5
0
令酗=愉h…z, iez)由定义
E

n
E
QKaiyiy 颂 生 1
F
E

E

后 @E 弛 , 集 合

E cod 2 尘
E e
ECuCouot 一 水 Antpts 技 g河 芸 e
期 . 一 个 非 周 期 点 在 的 正 向 及 负 向 迭 代 下 , 如 果 趋 于 同 一 个 周 期
E el E
s

定 理 2.4 对 3 中 的 移 位 映 射 v 下 列 结 论 成 立 .
E 坂
Page-368
E 驱 ri

a 水 纳 t a

东
E
E

E

E 医 一 二 技
技 江
称 为 边 界 的 水 平 部 分 , 记 作 #D: 那 组 A 垂 直 对 边 称 为 边 界 的 坩
E

E 河 0 5
空 间 . 令

2 77 木 人 伟
沥 07 水 7

E

定 义 1.4 令 DCRIXR ` 图
E 仪 a 沥 n 河
n 沥 林
Eupnicsosisi 沥 河

3许 吴
Y E D

[ 刑
E

医

兰 7
E 仁 9

[ 兄 一 万 3

E 沥 技
李
Page-369
E E

s lalt 王 大标 江 沥 沥 二
的 异 宪 点 集 在 V 中 稠 密 ;

[ 一

E 不
的 元 素 构 成 的 一 个 周 期 性 重 复 的 双 边 无 穷 序 列 用 在 其 重 复 段 上 加
一 个 模 线 表 示 . 例 如 (...,1,2,1,2,...} 用 {12} 表 示 . 对 左 ( 右 ) 向 无
穷 的 周 期 重 复 的 序 列 用 一 个 在 其 重 复 段 上 面 向 左 ( 有 ) 的 箭 头 表
示,例如卜",1'2,L2〉用〈迳〉表示,〈l,z,1yzy"-)用(迳〉表示. E
E 一
E 沥 水 水
E 医 程 c 王 一 吴 p 刀 唐 标
E 技
伟

E 莲 中 木 T
E

现 在 证 明 (2). 令 @ 一 {2“ }, 一 他 } 是 阿 个 周 期 点 , 这 里 a“ 与
“ 分 别 是 山 和 二 的 周 期 重 复 郭 分 . 对 任 意 E V, 任 意 给 定 s>-0,
E E

E

E 沥 达
E

E [
E c 林 应 口 林 水
EeEososstol p a 河
E

c 沥

E {鹰′】 ,勿鬟】,." `窿′N′}′
Page-370
E 第 四 章 双 曲 不 动 点 及 马 路 存 圭 定 理

E s
E 亚 n
E i 匕
E
E 怡
202
则 了 在 马 中 有 唯 一 双 曲 不 动 点

上 述 定 理 的 儿 何 直 观 见 囹 4-2(a).

E
E

E 0
E [ 吴 吴 17

E 罚
E 一

e
Page-371
E 玟

E 厂 育 2 前 许

E 〈`“ ,鹰薯,鹰z,鹰】,鹰z,鹰薯"`_}′
这 样 。 包 含 任 意 给 定 长 度 的 所 有 可 能 的 序 列 . 我 们 断 言 u 的 转 道
E 八 伟 认 0 沥 沥 林 加 招
E 广
玟

E 沥

故 乙 的 尬 道 OCa) 在 3 中 稔 密 。 〖

E 匿 浩 u

ss 河
E
E 沥 连 e
园
性 映 射 的 情 况 , 给 出 所 谓 马 踹 存 在 定 理 . 最 后 我 们 将 看 到 马 蹄 可 以
在 三 维 向 量 杨 的 Poincare 映 射 中 出 现 .

E

考 虔 R 上 的 单 位 正 方 形 D 一 [0,1]X [0,1]. 我 们 引 进 DD 到
E 沥 t
压 缩 5 倩 , 然 后 将 所 得 到 的 细 高 矩 形 在 中 部 夺 曲 得 到 马 踹 形 区 域 ,
E 江江 述 c 沥

为 1 家 为 不 的 短 形 V.,V:, 而 广 1CD[1D) 是 两 个 高 为 不 宽 为 1 的
t 刑
王
E 刑 国 一
【 沥
Page-372
E 02

第 二 步 证 明 y 一 门 Di 和 一 一 [D-; 分 别 是 端 点 属 于 3eD 和
E 标

第 三 步 证 明 一 V 是 了 的 一 个 双 曲 不 动 点
E

儿 个 引 理

引 理 1. 6 平 面 R!XRI 上 一 条 心 水 平 曳 线 与 一 条 A 垂 直 曲
E

E
厂
[

[ 汀 刑 T
才

[ 芸 c 史 |

E 二
是 端 点 集 属 于 3,D 和 &D 的 A8 水 平 曲 线 及 A 垂 直 曲 线 , 则 上
E 沥

证 明 交 点 的 存 在 性 可 由 连 续 性 得 到 , 而 唯 一 性 由 引 理 1. 6 保
E 河

E 吴 7 李 人
水

5

[ 一
E

0

E 吴 页
E 命a 林 人 水
E
Page-373
现 在 我 们 考 虚 D 中 的 所 有 在 了 的 任 意 次 迭 代 下 都 不 离 开 D
的 点 集 4, 却

E 林
E

n
u

E
Eeuies 2 E
E

明 沥 水 i 林E 东 一 [
E

L 剧
Page-374
E 玟

上 述 引 理 可 参 明 图 4-3.

国 4-3

E 团 医 仪 江 水 技
E t 水 技 振 沥
0
江
沥 不
E 述 江 一 石
7DinD: 是 一 个 (fa,s) 矩 形 , 滔 足 (fDif1D:)CC3Da. 类 似 地 可
E 前

医 沥 2 口 东 9 弃 锋 八 河
0

E y 吊 4 户

2
令

E 沥 0 刑
E
Page-375
E 技t 163

E 途
E
医 沥 t 仁 水 述
Eopusd y 丞 i
E
E
芸 E

E
2 5
P

E 东 江

E 沥
个 微 分 和 躯 , 满 尸 (Ah,s) 锦 形 条 件 . 设 7CD 是 一 条 C 光 溥 的 A
Eaatey s s 邦
E

[ 李 仁 王

E 园 0

E 河 1十觞'
>1 是 定 义 1.4 中 的 常 数

b 善 林 心 一 t
诉 的 证 明 可 用 类 似 方 法 得 到 -
E
d
E 俊 应 仪
刑
E
入
国 E
E
E 技

0 日
Page-376
172 E

E
技

Dn
E 认
是 一 个 非 空 Cantor 集 .
l 沥 怀 浩 z 四 林 技

故 对 V zE A

E
注 意 到 ACEuU a, 因 此 存 在 一 个 由 1,2 组 成 的 无 限 双 边 序 列 a
[

E 伟

这 根 我 们 定 义 了 一 个 晃 射 申 4> 罗 ,

E
陵 沥 沥 述 河 i 妥上的右移位映射 则 根 据 定 义 有

E 技 技 口
E 倭是^到藩的间胚唧 人
E 丿 吊 沥 el 沥
a 沥 一
E 仪
E 分

E 2 月
【unag E 志
长 度 为 28--1 的 中 间 殷 是 一 样 的

E 门 E
E 不
丿 中 的 距 离 可 以 充 分 的 小 , 这 便 说 明 了 一 是 连 续 的 -
E
Page-377
E E a g

对 河 1, 令 g 一 乙 , 则 也 滔 足 (x, 户 3 锦 形 条 件 , 只 是 其 中 的 扩 张
s l
证 明 引 理 . 〖

E 不
E

0 园 0
E 吊 5 诊
n
E 技 技 才 2 技

E
上 界 , 则 由 引 理 1. 10, 有

E d 林 i
d
江 )
E
E
故 存 在 常 数 C, 使 得
E
同 理 可 证 存 在 常 数 C, 使 得
E 浩

定 理 1. 5 证 明 的 完 成
E
1
E av…a^D.
E
Page-378
E 招 E

E 标 an
医 故

E 沥 坂 cnmueyf 妙 口 沥 政 d 河 沥
方 向 拉 伸 , 故 不 动 点 O 是 双 曲 的

E

园
E

符 号 序 列 空 间 及 其 结 构

令
吊 二 人 2oNj, 切 丿 2.
在 5 上 引 进 如 下 度 量 。
【 尘
40 口 二 E 沥 胡 二 团
E 林 沥 一 d 东 东
阁 . 仔

E

的 双 边 无 穷 序 列
[
巳
技

这 里 a 5 是 在 分 量 S 上 的 值 . 在 SY 上 我 们 如 下 引 选 度 重 .
Page-379
张 理

E
E 坤 2

E 浩
E
E
E 沥 人 木 伟 五
B 竖 连 一 初
E 2 甫n 有
[ 弛 沥 月
E
[
这 与 ax5 矛 盾 .
E 技 5 八 育 英 页 肖 江 胡 5 述 江 育
医 胺
我 们 将 它 作 为 练 习 留 给 读 者 . 、
E 途
玟

定 理 3. 1 马 蹇 春 射 了 在 D 中 有 一 个 不 变 的 Cantor 集 4, 漾
E

【68 s 吊 p a

0 河 5
E 沥

【62 河 坂 a

E 达

0 出 李 口
E 二 技 技
Page-380
i

沥

27 招

5
E anatpe ss
E 招泊 罚
pouE e 真 诊
匹 a
E 命a an 国 2 刑 一

E
E 集 称 为 完 全 不 连
通 的 , 如 果 它 的 每 一 个 连 通 分 支 只 包 含 一 个 点 . 闷 时 具 有 上 面 两 条
Eeaitgpsss 医 口 u

引 理 2.2 空 间 3Y 在 度 量 (2.2) 下 是

0

E

E t 二 日
Page-381
E 第 四 章 双 曲 不 助 点 及 马 跨 存 在 宏 境

i

一 E X 是 一 个 微 分 同 胚 , 并 且
木
[ 木
木
3
则 集 合

是 一 个 不 变 的 Cantor 集 . 丁 限制′羞^上的映射/ Etuest

E 林
医 沥

由 定 理 2. 4 我 们 有
推 论 3.3 定 理 3. 2 中 的 春 射 了 有 一 个 不 变 的 Cantor 集 4, 漾

E
noatzseyap l
〈2) 周 期 软 道 的 同 宿 点 及 异 宿 点 在 4 中 稚 密 :
E 河 沥 5

定 理 3.2 证 明 的 恰 路
E 李
…茅 5 〈窿′〉o<f<n, 命 (膘趸)一″<f<u′
a E
如 果 zE 卫 漾 足 fzE.Du , 0<Sixsn, 则 我 们 记 作
_ E
加 果 zED 满 足 广 !zE D。 ,,0<Si<smv 则 我 们 记 作
6
Page-382
E 招

河 八
s, 使 得 对 其 巾 任 意 序 列 口 一 (aijiez, 古 一 仁 jez, 有 a 二 页 . 下
EoE l 志
医 沥 朱达 朐 许 4
E 0 沥 团 E 沥
u i 志 江 东 5 国 衍
6
浩
E
U 扬 一 才 扬
[ 史
bn d 吴 河
心 巳 经 取 定 , 取 aL1EKu 使 得 心 1 世 i 一 0,1,

2

1 一
E

E 水
人
[

[ 不

一
c 东 |

E
E 沥
2
Page-383
张 胡

沥 人 1 技
3 许

阮
PDlat) CCDCa_a,D(az ) C DCai_0,
E
E
第 二 步 证 明

E 一
E 禺"D(m;)
是 -- 条 端 点 属 于 ,Da 的 丿 垂 直 曲 线
E

a 二 D(or) n DCar ).
E
E Rs 租 s2C)
是一个同胚 E
E ′1正明等式(3.1〉成立'

定 理 3.2 的 证 明
令
人 +】""`″″)〉
E t
E
[
林水
D“″一i′ E
Page-384
第 四 敦 双 曲 不 动 点 及 马 路 存 在 定

E
zeanusy
引 理 2.3 移 位 晔 射 v 是 2 到 自 身 的 同 胜 .
E 国 5 兄 招 5
0
令酗=愉h…z, iez)由定义
E

n
E
QKaiyiy 颂 生 1
F
E

E

后 @E 弛 , 集 合

E cod 2 尘
E e
ECuCouot 一 水 Antpts 技 g河 芸 e
期 . 一 个 非 周 期 点 在 的 正 向 及 负 向 迭 代 下 , 如 果 趋 于 同 一 个 周 期
E el E
s

定 理 2.4 对 3 中 的 移 位 映 射 v 下 列 结 论 成 立 .
E 坂
Page-385
玟

阮
u
E 口
c 木
沥

E 医

E 吴 L 希 技
0
类 似 地 我 们 有
引 理 3.5 E
e
E 宇 口 0 人 八 5
引 理 3.6 存 在 一 个 正 常 数 C, 使 得
0 月
E
E 江

E pn 沥 江 弛 玟 c
b
江
E O 月

Un

E _邑。D(m蒙)
E 沥 一 一 逊
E 剧
E 一
Page-386
E 沥 s

外 史 : 。
E
这 样 我 们 得 到 一 个 眸 射 V 一 A +CoD.
国 [
E
E t
玟
人 设 fzEDa,iEZ, 则
E
E
2. 是 映 上 的 . 这 挡 的 是 对 任 意 给 定 zE 4, 都 存 在 一 个 点 。
EE n
E 李
E
E 团 不
E
团 …_巳″【D(m薯〉 【ULCs p

3. 幼 是 逊 续 的 . 这 指 的 是 任 络 G 3N ,e>0, 都 可 找 到 正 数 8,

使 得 对 任 意
c
着 4(a,m 心 8, 则 Q(ga) 「b(m) 一 s 由 引 理 2.1, 如 果 qCaov 矶 一 8,
D 沥 口 2s 仪
5
0 吴 3

医 沥 林 [

0 引

E 水 二
Page-387
E E

s lalt 王 大标 江 沥 沥 二
的 异 宪 点 集 在 V 中 稠 密 ;

[ 一

E 不
的 元 素 构 成 的 一 个 周 期 性 重 复 的 双 边 无 穷 序 列 用 在 其 重 复 段 上 加
一 个 模 线 表 示 . 例 如 (...,1,2,1,2,...} 用 {12} 表 示 . 对 左 ( 右 ) 向 无
穷 的 周 期 重 复 的 序 列 用 一 个 在 其 重 复 段 上 面 向 左 ( 有 ) 的 箭 头 表
示,例如卜",1'2,L2〉用〈迳〉表示,〈l,z,1yzy"-)用(迳〉表示. E
E 一
E 沥 水 水
E 医 程 c 王 一 吴 p 刀 唐 标
E 技
伟

E 莲 中 木 T
E

现 在 证 明 (2). 令 @ 一 {2“ }, 一 他 } 是 阿 个 周 期 点 , 这 里 a“ 与
“ 分 别 是 山 和 二 的 周 期 重 复 郭 分 . 对 任 意 E V, 任 意 给 定 s>-0,
E E

E

E 沥 达
E

E [
E c 林 应 口 林 水
EeEososstol p a 河
E

c 沥

E {鹰′】 ,勿鬟】,." `窿′N′}′
Page-388
178 E

E
E
[

E
男 一 方 面 , 由 (3. 5) 得
E

这 意 咋 着

E 吴
E

闵
定 理 证 毕 ,

E 颠 c t EE

本 节 将 给 出 关 于 两 个 线 性 眼 射 的 复 合 眸 射 的 双 曲 性 的 一 个 重
要 引 理 , 这 一 引 理 将 在 第 五 章 中 多 次 应 用 .

问 颜 的 提 出

第 五 章 将 研 究 者 点 同 宿 轨 的 分 岗 , 这 一 问 题 的 解 决 是 通 过 研
E
为 两 个 晔 射 的 复 合 : 一 个 陋 射 由 奇 点 的 邻 域 中 的 向 量 场 决 定 , 男 一
E
余
分 小 时 , 它 治 着 稳 定 流 形 方 向 的 压 缩 常 数 和 治 着 不 稳 定 流 形 方 向
的 扩 张 常 数 分 别 先 分 小 和 充 分 大 , 我 们 对 第 二 个 眯 射 除 了 知 道 它
E
余
E
Page-389
E 玟

E 厂 育 2 前 许

E 〈`“ ,鹰薯,鹰z,鹰】,鹰z,鹰薯"`_}′
这 样 。 包 含 任 意 给 定 长 度 的 所 有 可 能 的 序 列 . 我 们 断 言 u 的 转 道
E 八 伟 认 0 沥 沥 林 加 招
E 广
玟

E 沥

故 乙 的 尬 道 OCa) 在 3 中 稔 密 。 〖

E 匿 浩 u

ss 河
E
E 沥 连 e
园
性 映 射 的 情 况 , 给 出 所 谓 马 踹 存 在 定 理 . 最 后 我 们 将 看 到 马 蹄 可 以
在 三 维 向 量 杨 的 Poincare 映 射 中 出 现 .

E

考 虔 R 上 的 单 位 正 方 形 D 一 [0,1]X [0,1]. 我 们 引 进 DD 到
E 沥 t
压 缩 5 倩 , 然 后 将 所 得 到 的 细 高 矩 形 在 中 部 夺 曲 得 到 马 踹 形 区 域 ,
E 江江 述 c 沥

为 1 家 为 不 的 短 形 V.,V:, 而 广 1CD[1D) 是 两 个 高 为 不 宽 为 1 的
t 刑
王
E 刑 国 一
【 沥
Page-390
0 圆 i 179

医 a 吊
两 个 映 射 的 导 算 子 的 积 给 出 的 . 因 而 本 节 我 们 秽 对 两 个 线 性 映 射
的 复 合 晖 射 的 双 曲 性 给 出 一 个 判 别 引 理 .

E

一
E 伟
E
E 木 仁 才

E u 育 芸 木 扬

E
005 e
[ t

0 一 刀

这 里 22>1 是 一 个 常 数 .

E e
E

壮
E
E c 李 p
Page-391
现 在 我 们 考 虚 D 中 的 所 有 在 了 的 任 意 次 迭 代 下 都 不 离 开 D
的 点 集 4, 却

E 林
E

n
u

E
Eeuies 2 E
E

明 沥 水 i 林E 东 一 [
E

L 剧
Page-392
第 四 章 双 曲 不 动 点 及 马 蹭 存 在 定 理

满 足 M 大 0. 令 [>0 是 一 个 常 数 , 使 得

L 万

仁

[ 一 切 E
E i 探 丿 白
如 果 下 列 不 等 式

万 0
匹 2 团 E
[ [
a [
a 0
E yy y 4 扬 水 23 图

E
E 吴 n 育 27724247
E
成 立 的 前 题 下 , 分 三 步 证 明 引 理 .

E 义p

令
E
对 满 尸 (4. 9) 的 常 数 , 我 们 将 证 明
林 应 沥
E
c 人 李
E 沥
d 水 2 肉
C4 1
Page-393
招

E 切
t 0 标 )
E

[ 园 一
沥
E |<#汇】M丨 t
E t
这些不等式与呱.6)和(4'7)】起推出
E y
t 0

E
I 医 圆 1 1<鲍》|M茗+ 1 y
由(4.11)`(4′12),〈4′I3)和(4^16〉我们得
厂 河 胡 〔L〉】 b 瓮)〕- [ 吴 啬乙一外M庸 1

0
E
E 林
Essoc tnpoty i

锥 彭 区 域 的 不 变 性
现 在 证 明
E 0
江
0
[ 《 19
c 刑
Page-394
172 E

E
技

Dn
E 认
是 一 个 非 空 Cantor 集 .
l 沥 怀 浩 z 四 林 技

故 对 V zE A

E
注 意 到 ACEuU a, 因 此 存 在 一 个 由 1,2 组 成 的 无 限 双 边 序 列 a
[

E 伟

这 根 我 们 定 义 了 一 个 晃 射 申 4> 罗 ,

E
陵 沥 沥 述 河 i 妥上的右移位映射 则 根 据 定 义 有

E 技 技 口
E 倭是^到藩的间胚唧 人
E 丿 吊 沥 el 沥
a 沥 一
E 仪
E 分

E 2 月
【unag E 志
长 度 为 28--1 的 中 间 殷 是 一 样 的

E 门 E
E 不
丿 中 的 距 离 可 以 充 分 的 小 , 这 便 说 明 了 一 是 连 续 的 -
E
Page-395
E E

E
E 沥 玟

由 三 角 不 等 式

|87 | 妮 lallh- 1 + 1611D8- 1 十 lallB8 | + 51 |
E

E
lall- [ 十 |6|1D8“ 1 东 2 8 | 4 2D

进 一 步 由 (4. 3)
1B5+ 1 妙 18 | 一 二 | 1.

国 此 由 (4. 1
la| B 1 十 | 阮 | 妙 ( 十 切 | 纳 ( 22)

将 (4 21) 和 (4. 22) 代 人 (4. 20) 得
E 人

这 与 (4 17) 一 起 得
吴 t 人 一

EEasutyy p 技 圭
便 证 明 了 (4 19, 即 (4. 182.

穗 定 锥 形 区 域 内 的 压 绾 性

E
E 林 伟
我 们 将 证 明 不 等 式
[ 沥

令

沥

余
0

(〗雇
E 技

y
Page-396
张 理

E
E 坤 2

E 浩
E
E
E 沥 人 木 伟 五
B 竖 连 一 初
E 2 甫n 有
[ 弛 沥 月
E
[
这 与 ax5 矛 盾 .
E 技 5 八 育 英 页 肖 江 胡 5 述 江 育
医 胺
我 们 将 它 作 为 练 习 留 给 读 者 . 、
E 途
玟

定 理 3. 1 马 蹇 春 射 了 在 D 中 有 一 个 不 变 的 Cantor 集 4, 漾
E

【68 s 吊 p a

0 河 5
E 沥

【62 河 坂 a

E 达

0 出 李 口
E 二 技 技
Page-397
i

E 友'}
E
E 悬【襄〉】【.
E f
[
E
E 吴

E 余
E
s 刑7 命 刀
现 在 我 们 证 明 (4. 23). 注 意 到 4~. 一 厂 &-1, 我 们 有
E 河 芸沥 荣 水 乐
E

| 红 1 芸 | 国 [|ag- | 一 1Beq- | 一 |B+ | 一 | 万 B27+ |
t

由 (4 26)
仪
由 (4. 27,(4. 25) 和 (4. 8), 有
s
E 口 日
E 二

i
由 (4. 27) 和 (4. 1
E
注 意 到 9E K“, 由 (4. 31) 和 (4 32) 得
E
Page-398
E 第 四 章 双 曲 不 助 点 及 马 跨 存 在 宏 境

i

一 E X 是 一 个 微 分 同 胚 , 并 且
木
[ 木
木
3
则 集 合

是 一 个 不 变 的 Cantor 集 . 丁 限制′羞^上的映射/ Etuest

E 林
医 沥

由 定 理 2. 4 我 们 有
推 论 3.3 定 理 3. 2 中 的 春 射 了 有 一 个 不 变 的 Cantor 集 4, 漾

E
noatzseyap l
〈2) 周 期 软 道 的 同 宿 点 及 异 宿 点 在 4 中 稚 密 :
E 河 沥 5

定 理 3.2 证 明 的 恰 路
E 李
…茅 5 〈窿′〉o<f<n, 命 (膘趸)一″<f<u′
a E
如 果 zE 卫 漾 足 fzE.Du , 0<Sixsn, 则 我 们 记 作
_ E
加 果 zED 满 足 广 !zE D。 ,,0<Si<smv 则 我 们 记 作
6
Page-399
E 第 四 章 双 曲 不 助 点 及 马 踊 孛 在 定 理

由 (4. 247,C4. 29),(4. 30) 和 (4. 33) 得 到 (4. 28) 右 端 的 下 界 佼 计 .
E
g | 交 酊 一 匹 十 不 ) 一 8 一 止 十 发 8 |
E
E

医 玟 告(乙 E 沥 夜 5
E 林不

y 沥 江

医 江 2 沥
Smale 定 理 .

定 理 的 陈 述

E 日 孝 c 水 水 一 王 逊 诊
个 特 征 指 数 , 满 足 |2| 一 1 一 |x|. 这 根 的 周 排 执 有 一 维 穗 定 流 形
E 一 国 不
期 乳 5 的 咏 宿 转 , 如 果 7 丿 丘 YCWr(e) iW*Co). 进 一 步 , 同 宿
敏 7 称 为 槲 藏 的 , 如 果 流 形 Pr(o) 与 W*Co) 沿 着 执 道 Y 樟 戳 相 交 ,
E

玟
E
朋 标 林万 玖

事 实 上 , 我 们 将 证 明 比 定 理 5. 1 更 强 的 结 论 : 在 cUY 的 任 意 邻
E 林
行 . ,

周 期 轨 道 的 Poineark 映 射
医 江 c
Page-400
张 胡

沥 人 1 技
3 许

阮
PDlat) CCDCa_a,D(az ) C DCai_0,
E
E
第 二 步 证 明

E 一
E 禺"D(m;)
是 -- 条 端 点 属 于 ,Da 的 丿 垂 直 曲 线
E

a 二 D(or) n DCar ).
E
E Rs 租 s2C)
是一个同胚 E
E ′1正明等式(3.1〉成立'

定 理 3.2 的 证 明
令
人 +】""`″″)〉
E t
E
[
林水
D“″一i′ E
Page-401
[ 渡

E
E 林标 团 达 5 诊
点 处 可 以 〇 线 性 化 ( 参 见 第 一 章 定 理 4 22), 故 在 $ 上 0 点 的 邻 域
ELstotesz
【Cpaispsest
E 沥 技
这 里 | 一 1 人 |A|. 直 线 z 一 0 和 y 一 0 分 别 对 庭 局 部 穗 定 流 形 和 局
E 莲
0
E 沥 L 沥
E 河
Eocut i
e 【
令 Dy 一 P~“ 名 门 B 表 示 8 中 所 有 在 P“ 作 用 下 眶 到 吊 的 那 些 点 所
E 沥 芸 途 训 i
E
E 吴 [

E 沥 吴 tn 沥

0
Page-402
玟

阮
u
E 口
c 木
沥

E 医

E 吴 L 希 技
0
类 似 地 我 们 有
引 理 3.5 E
e
E 宇 口 0 人 八 5
引 理 3.6 存 在 一 个 正 常 数 C, 使 得
0 月
E
E 江

E pn 沥 江 弛 玟 c
b
江
E O 月

Un

E _邑。D(m蒙)
E 沥 一 一 逊
E 剧
E 一
Page-403
玟

E

同 宿 轨 道 的 后 继 映 射

Eapo isky 规 林 s 河 a 水
E 怀 技 技 p 一 沥 林 玟 不 沥
我 们 用 与 表 示 这 一 对 应

E [
由 常 微 分 方 程 的 解 对 初 值 的 光 滑 依 赖 性 知 ,F 是 C! 微 分 同 贴 . 令
E 2 技 应 2 沥 诊 林 t
E
菩 a [
E
卫 一 旦 伙 二 0}, 灵 二 书 n 仁 二 0}.
取 定 8 充 分 小 , 使 得
E 育
沥
0

Ea
E

D
Page-404
E 沥 s

外 史 : 。
E
这 样 我 们 得 到 一 个 眸 射 V 一 A +CoD.
国 [
E
E t
玟
人 设 fzEDa,iEZ, 则
E
E
2. 是 映 上 的 . 这 挡 的 是 对 任 意 给 定 zE 4, 都 存 在 一 个 点 。
EE n
E 李
E
E 团 不
E
团 …_巳″【D(m薯〉 【ULCs p

3. 幼 是 逊 续 的 . 这 指 的 是 任 络 G 3N ,e>0, 都 可 找 到 正 数 8,

使 得 对 任 意
c
着 4(a,m 心 8, 则 Q(ga) 「b(m) 一 s 由 引 理 2.1, 如 果 qCaov 矶 一 8,
D 沥 口 2s 仪
5
0 吴 3

医 沥 林 [

0 引

E 水 二
Page-405
理J

E 深 2 柳

E n 一 沥 一 技 余 扬 月 刑 技
〈5. 9) 可 以 取 定 & 充 分 小 , 使 得

【

[
【

[ 2 二 育 玖
E
现 在 我 们 定 义 后 继 映 射 An: D,-*U 如 下 :
巳 吴

E 吴

E 口
a
林
E

E
和
2 孙 央
E 国 2 肉
u 沥 一 一
0 一
[
标
Page-406
178 E

E
E
[

E
男 一 方 面 , 由 (3. 5) 得
E

这 意 咋 着

E 吴
E

闵
定 理 证 毕 ,

E 颠 c t EE

本 节 将 给 出 关 于 两 个 线 性 眼 射 的 复 合 眸 射 的 双 曲 性 的 一 个 重
要 引 理 , 这 一 引 理 将 在 第 五 章 中 多 次 应 用 .

问 颜 的 提 出

第 五 章 将 研 究 者 点 同 宿 轨 的 分 岗 , 这 一 问 题 的 解 决 是 通 过 研
E
为 两 个 晔 射 的 复 合 : 一 个 陋 射 由 奇 点 的 邻 域 中 的 向 量 场 决 定 , 男 一
E
余
分 小 时 , 它 治 着 稳 定 流 形 方 向 的 压 缩 常 数 和 治 着 不 稳 定 流 形 方 向
的 扩 张 常 数 分 别 先 分 小 和 充 分 大 , 我 们 对 第 二 个 眯 射 除 了 知 道 它
E
余
E
Page-407
玟

E

E
工
朋
E2

E 河
EatwciLuco

由 (5. 10) 及 与 是 一 个 C! 徽 分 同 胚 , 我 们 有
E
医 江
14| 一 | 灵 | 一 0,1M| 二 17 口 一 0, 当 一 十 co,
对 充 分 大 的 a,(4. 3) 一 (4. 8) 显 然 成 立 -
下 面 验 证 边 界 条 件 . 注 意 到

E 园 a 团
以 及 (5. 11) 和 (5. 12), 便 知 迈 界 条 件 成 立 .

最 后 验 证 相 交 条 件 . 在 D; 中 有 两 族 直 线 , 即 所 谓 水 平 直 线 族
E

【 芸 标 吴 汀 0 朋
[ 沥 林
这 里 CDiDo 一 {|Ey 一 11<8}. 注 意 到
鳜@)卫>B们匣鳐位)逞荨凰,当妻玲+的,
再 利 用 (5. 6) 得 到 , 当 ,j 充 分 大 时 , 曲 线 AC6Cz) 与 四 (o) 在 外 点
附 近 相 交 于 唯 一 一 点 - 固 此

3
E
Page-408
E 源 s

E

Eiasnaed |
最 后 , 由 引 理 5. 2 及 定 理 3. 2 可 推 出 定 理 5.1.

医 沥 沥 人 明 沥 不 工 d
果 , 可 参 见 [Wig] 和 〔si11〕.
Page-409
0 圆 i 179

医 a 吊
两 个 映 射 的 导 算 子 的 积 给 出 的 . 因 而 本 节 我 们 秽 对 两 个 线 性 映 射
的 复 合 晖 射 的 双 曲 性 给 出 一 个 判 别 引 理 .

E

一
E 伟
E
E 木 仁 才

E u 育 芸 木 扬

E
005 e
[ t

0 一 刀

这 里 22>1 是 一 个 常 数 .

E e
E

壮
E
E c 李 p
Page-410
第 四 章 双 曲 不 动 点 及 马 蹭 存 在 定 理

满 足 M 大 0. 令 [>0 是 一 个 常 数 , 使 得

L 万

仁

[ 一 切 E
E i 探 丿 白
如 果 下 列 不 等 式

万 0
匹 2 团 E
[ [
a [
a 0
E yy y 4 扬 水 23 图

E
E 吴 n 育 27724247
E
成 立 的 前 题 下 , 分 三 步 证 明 引 理 .

E 义p

令
E
对 满 尸 (4. 9) 的 常 数 , 我 们 将 证 明
林 应 沥
E
c 人 李
E 沥
d 水 2 肉
C4 1
Page-411
巳 江 力 d E

本 章 将 考 虑 空 间 R 中 鞭 点 的 同 宿 分 岔 . 在 8 1 中 将 讨 论 特 征
根 都 为 实 数 的 鞍 点 的 同 宿 分 岔 . 在 8 2 中 将 讨 论 有 复 特 征 根 的 鞍 点
的 间 宿 分 岔 . 在 8 3 中 将 讨 论 由 一 个 奇 点 和 一 条 闭 尧 以 及 连 结 它 们
E 章 的 学 习 , 读 者 将 会 对 如
何 利 用 奇 点 或 不 动 点 附 近 的 线 性 化 理 论 ( 见 第 一 章 $ 4) 来 研 究 非
E

$ 1 具 有 三 个 实 特 征 值 的 鞍 点 的 同 宿 分 岔

河 国 n 沥 沥
E

E p

假 设 中 一 光 溺 向 量 场 有 一 个 特 征 值 都 为 实 数 的 双 曲 鞍 点
及 其 间 宿 技 . 我 们 考 虑 这 样 一 个 向 量 场 在 一 般 的 单 参 数 扰 动 下 所
能 发 生 的 分 岔 . 不 失 一 舫 性 , 我 们 总 可 以 认 为 鞍 点 有 两 个 负 特 征 值
和 一 个 正 牺 征 值 ( 否 则 考 虑 其 时 间 反 向 系 统 ). 这 样 的 鞍 点 具 有 二
维的稳定流形和一维的不稳定流形 我 们 称 最 大 的 负 特 征 值 与 正
特 征 值 之 和 为 鞍 点 量

E 二 园 圆 沥 i
数 e==0 时 , 向 量 场 X 有 一 个 鞍 点 0, 它 具 有 两 个 负 特 征 值 和 一 个
d 5 a
E
E
B
Page-412
招

E 切
t 0 标 )
E

[ 园 一
沥
E |<#汇】M丨 t
E t
这些不等式与呱.6)和(4'7)】起推出
E y
t 0

E
I 医 圆 1 1<鲍》|M茗+ 1 y
由(4.11)`(4′12),〈4′I3)和(4^16〉我们得
厂 河 胡 〔L〉】 b 瓮)〕- [ 吴 啬乙一外M庸 1

0
E
E 林
Essoc tnpoty i

锥 彭 区 域 的 不 变 性
现 在 证 明
E 0
江
0
[ 《 19
c 刑
Page-413
E E

E
E 沥 玟

由 三 角 不 等 式

|87 | 妮 lallh- 1 + 1611D8- 1 十 lallB8 | + 51 |
E

E
lall- [ 十 |6|1D8“ 1 东 2 8 | 4 2D

进 一 步 由 (4. 3)
1B5+ 1 妙 18 | 一 二 | 1.

国 此 由 (4. 1
la| B 1 十 | 阮 | 妙 ( 十 切 | 纳 ( 22)

将 (4 21) 和 (4. 22) 代 人 (4. 20) 得
E 人

这 与 (4 17) 一 起 得
吴 t 人 一

EEasutyy p 技 圭
便 证 明 了 (4 19, 即 (4. 182.

穗 定 锥 形 区 域 内 的 压 绾 性

E
E 林 伟
我 们 将 证 明 不 等 式
[ 沥

令

沥

余
0

(〗雇
E 技

y
Page-414
不 具 有 三 个 实 御 征 值 的 糖 点 的 网 宝 分 吴 E

E
E

E

在 证 明 定 理 之 前 , 我 们 首 先 解 释 定 理 陈 述 中 “ 一 舫 “ 一 词 的 含
E t i 汀 i 沥 D 命诊 5 训 沥 招 河
三 条 是 针 对 E

点 0 的 特 征 值 两 两 不 相 同 且 为 非 共 振 .

d i
在 点 0 处 的 线 性 部 分 ( 参 见 第 一 章 定 理 4 18) , 故 在 绎 性 化 坐 标 系
沥
上 , 除 了 奇 点 0 和 一 条 通 过 O 点 的 直 线 外 , 所 有 辅 道 当 二 十 co 时
t
E

0 吊 p

一
向 量 张 成 一 个 不 变 平 面 仁 , 这 一 不 变 平 面 沿 着 同 宿 轨 延 伸

E
园 5

E 施 t 江 沥 口 p 刃
E

定 理 1. 1 证 明 的 恺 路

玟
E a e 汀 5 扎
伟 t
u 人
E
为 初 值 的 正 半 辉 与 “ 的 第 一 个 交 点 . 假 设 (1 使 我 们 可 以 把 向 量
E 志 不 s 沥n
Page-415
i

E 友'}
E
E 悬【襄〉】【.
E f
[
E
E 吴

E 余
E
s 刑7 命 刀
现 在 我 们 证 明 (4. 23). 注 意 到 4~. 一 厂 &-1, 我 们 有
E 河 芸沥 荣 水 乐
E

| 红 1 芸 | 国 [|ag- | 一 1Beq- | 一 |B+ | 一 | 万 B27+ |
t

由 (4 26)
仪
由 (4. 27,(4. 25) 和 (4. 8), 有
s
E 口 日
E 二

i
由 (4. 27) 和 (4. 1
E
注 意 到 9E K“, 由 (4. 31) 和 (4 32) 得
E
Page-416
第 五 章 宇 间 中 双 曲 鞅 的 问 宿 分 岖

等 函 数 袁 示 出 来 . 当 鞍 点 量 分 别 是 正 和 负 时 ,48“8 分 别 具 有 强 双
E
而 沿 振 道 我 们 可 以 定 义 从 同 宿 轨 与 7“ 的 交 点 的 邻 域 到 吟 宿 技 与
工 的 交 点 的 邺 域 的 晔 射 , 并 将 它 记 作 46, 见 图 5-2. 由 解 对 初 值 及

E gn 沥 y 沥
诊 沥 壮 标

E
最 后 , 我 们 分 别 利 用 双 曲 不 动 点 定 理 和 压 缉 映 象 原 理 , 讨 论 当 鞍 点
i

E 玖 仪
E
Page-417
朱 具 市 三 个 实 後 狄 值 的 鞍 炎 的 问 分 岔
Ezozsoooiumitieosuctuuuioitdi

B

、′

E

E

E 沥 水
{汉 ( 0 口 0 0
E

E 吴

E 应
uaeo e 林 e 7
|os| 一 1 不 失 一 舱 性 , 我 们 还 可 假 设 ,0 点 邻 域 中 正 = 粘 是 同 宿 转
园 c a i

E 坂 t
Page-418
E 第 四 章 双 曲 不 助 点 及 马 踊 孛 在 定 理

由 (4. 247,C4. 29),(4. 30) 和 (4. 33) 得 到 (4. 28) 右 端 的 下 界 佼 计 .
E
g | 交 酊 一 匹 十 不 ) 一 8 一 止 十 发 8 |
E
E

医 玟 告(乙 E 沥 夜 5
E 林不

y 沥 江

医 江 2 沥
Smale 定 理 .

定 理 的 陈 述

E 日 孝 c 水 水 一 王 逊 诊
个 特 征 指 数 , 满 足 |2| 一 1 一 |x|. 这 根 的 周 排 执 有 一 维 穗 定 流 形
E 一 国 不
期 乳 5 的 咏 宿 转 , 如 果 7 丿 丘 YCWr(e) iW*Co). 进 一 步 , 同 宿
敏 7 称 为 槲 藏 的 , 如 果 流 形 Pr(o) 与 W*Co) 沿 着 执 道 Y 樟 戳 相 交 ,
E

玟
E
朋 标 林万 玖

事 实 上 , 我 们 将 证 明 比 定 理 5. 1 更 强 的 结 论 : 在 cUY 的 任 意 邻
E 林
行 . ,

周 期 轨 道 的 Poineark 映 射
医 江 c
Page-419
E

R3 2203
5
注 意 到 在 7 上 z 一 1, 因 此 由 等 式
E
d

哥z(D EzasyJOXG913 对

河 g

P
然 后 代 入 (1. 2 中 前 两 式 , 并 注 意 到 在 7+ 上 z 一 1, 便 可 得 到
E 0
E A A

E 不 不

E 口 林 木
点 加 一 (1,v,0), 由 解 对 初 值 与 参 数 的 依 赖 性 可 知 , 存 在 7“ 上 u
Eeetse e 庞街
吊 江5 加

E
映 射 4P8 是 一 个 光 滢 依 赖 于 的 徽 分 同 胚 - 令
E y c 吊

不 难 看 出 , (Y (e),Z(e)) 是 鞍 点 的 不 稳 定 流 形 与 7 的 交 点 , 面
E

羞Z(鬓〉L-。亨阜 0 2
玟

032 0
Page-420
[ 渡

E
E 林标 团 达 5 诊
点 处 可 以 〇 线 性 化 ( 参 见 第 一 章 定 理 4 22), 故 在 $ 上 0 点 的 邻 域
ELstotesz
【Cpaispsest
E 沥 技
这 里 | 一 1 人 |A|. 直 线 z 一 0 和 y 一 0 分 别 对 庭 局 部 穗 定 流 形 和 局
E 莲
0
E 沥 L 沥
E 河
Eocut i
e 【
令 Dy 一 P~“ 名 门 B 表 示 8 中 所 有 在 P“ 作 用 下 眶 到 吊 的 那 些 点 所
E 沥 芸 途 训 i
E
E 吴 [

E 沥 吴 tn 沥

0
Page-421
y

E 2 怀 李 国 巳 河
E 初
‖4置i”薯(汉`z)‖″″ Etpy an

E
E 达
伟 一 dng:P 一 ,

E

现 在 我 们 对 鞍 点 量 为 负 的 情 况 来 证 明 定 理 1. 1.
E 林0

国 国 ,
月 二 ″(…)>1` E 泓

E 河
相 E
u 涛 | [o M 〕‖″^
E
E 述 林 a
关 于 参 数 一 致 有 界 . 因 此 ,(1. 6) 意 晖 着
L

E

E 沥 E a
e<0 栗 讨 论 映 射 4 的 不 动 点 的 存 在 性 .
【692x44
对 任 意 点 (y,z)E.Lo, 有
U

沥
E 江

CL. 6)

0
Page-422
玟

E

同 宿 轨 道 的 后 继 映 射

Eapo isky 规 林 s 河 a 水
E 怀 技 技 p 一 沥 林 玟 不 沥
我 们 用 与 表 示 这 一 对 应

E [
由 常 微 分 方 程 的 解 对 初 值 的 光 滑 依 赖 性 知 ,F 是 C! 微 分 同 贴 . 令
E 2 技 应 2 沥 诊 林 t
E
菩 a [
E
卫 一 旦 伙 二 0}, 灵 二 书 n 仁 二 0}.
取 定 8 充 分 小 , 使 得
E 育
沥
0

Ea
E

D
Page-423
第 五 章 空 间 中 双 曲 数 点 的 闯 宪 分 盅

E

ELssetesanotulogo a
0nttd p 志 t 人
过 不 动 点 的 转 道 是 一 个 呆 引 周 期 轨 . 而 不 等 式 (1. 7) 意 昝 着 不 动 点
医 余 x 训
宿 软 的 位 置 .

C2) eSo.

d

E t

enst <鲁似鹏z) 5

g

上 边 不 等 式 中 等 叶 成 立 , 当 丁 仅 当

E
E
E

E

鞍 类 周 期 轨 的 产 生

现 在 对 鞍 点 量 为 正 的 倩 况 来 证 明 定 理 . 与 鞍 点 量 为 负 的 情 况
n
E

E
假 设 (3) 意 昭 督
[
- E E
情 况 么 0 和 乙 一 0 分 别 对 应 所 谓 叶 定 向 咖 宿 扔 和 不 可 定 向 闭 宿
E 志2 k 史

E E 《C1. 8)
Page-424
理J

E 深 2 柳

E n 一 沥 一 技 余 扬 月 刑 技
〈5. 9) 可 以 取 定 & 充 分 小 , 使 得

【

[
【

[ 2 二 育 玖
E
现 在 我 们 定 义 后 继 映 射 An: D,-*U 如 下 :
巳 吴

E 吴

E 口
a
林
E

E
和
2 孙 央
E 国 2 肉
u 沥 一 一
0 一
[
标
Page-425
$ 1 具 有 三 个 宗 特 征 例 的 鞑 点 的 同 分 盅

E
沥
[
证 明 我 们 用 第 四 章 引 理 4 1 来 证 明
c
E 招 r
E
E E
l
E E
吉
伟

【nslt 扬 g

刀 二 0, 砺 一 zf-1
由 第 四 章 引 理 4 1 司 知 + 引 理 1. 3 成 立 , 如 果 存 在 常 数 [>0, 使 得

r

【 河 逊 一 仪
2
E 许 技 c

E 政 志 仁 丞 n 口
[

U

加 (

围 E

b 水
E

E

E 责zl一卢镳 E 沥 厂 昔y薪一润伶 E
Page-426
玟

E

E
工
朋
E2

E 河
EatwciLuco

由 (5. 10) 及 与 是 一 个 C! 徽 分 同 胚 , 我 们 有
E
医 江
14| 一 | 灵 | 一 0,1M| 二 17 口 一 0, 当 一 十 co,
对 充 分 大 的 a,(4. 3) 一 (4. 8) 显 然 成 立 -
下 面 验 证 边 界 条 件 . 注 意 到

E 园 a 团
以 及 (5. 11) 和 (5. 12), 便 知 迈 界 条 件 成 立 .

最 后 验 证 相 交 条 件 . 在 D; 中 有 两 族 直 线 , 即 所 谓 水 平 直 线 族
E

【 芸 标 吴 汀 0 朋
[ 沥 林
这 里 CDiDo 一 {|Ey 一 11<8}. 注 意 到
鳜@)卫>B们匣鳐位)逞荨凰,当妻玲+的,
再 利 用 (5. 6) 得 到 , 当 ,j 充 分 大 时 , 曲 线 AC6Cz) 与 四 (o) 在 外 点
附 近 相 交 于 唯 一 一 点 - 固 此

3
E
Page-427
E 源 s

E

Eiasnaed |
最 后 , 由 引 理 5. 2 及 定 理 3. 2 可 推 出 定 理 5.1.

医 沥 沥 人 明 沥 不 工 d
果 , 可 参 见 [Wig] 和 〔si11〕.
Page-428
198 雉 五 章 空 间 中 双 幼 骇 炯 阡 问 宿 分 盅

y 巳 a 月
第 二 个 不 等 式 必 然 成 立 , 引 理 证 毕 . ‖
i
E 李 d
E
E
g
e eg
=簪汊z″十a仝予 8 沥 吴

巴 E
E
E 尚 i 吊 吴 A 口
E 命 育 c 芸 3 理25 命

E

再 设 e<0, 下 面 用 双 曲 不 助 点 定 理 来 证 明 ,4 在 T 内 有 唯 一

歌 类 不 动 点 因 为 4eCy,0) 一 e 一 0, 我 们 可 以 找 到 这 桦 一 个 依 赖 于
n

医 不 0

沥 _
E 怀 刑 人

则 短 形 D 的 边 界 的 水 平 部 分 为

2 L 辽
E

2 一 者 许
E

R 口 3

沥
Page-429
巳 江 力 d E

本 章 将 考 虑 空 间 R 中 鞭 点 的 同 宿 分 岔 . 在 8 1 中 将 讨 论 特 征
根 都 为 实 数 的 鞍 点 的 同 宿 分 岔 . 在 8 2 中 将 讨 论 有 复 特 征 根 的 鞍 点
的 间 宿 分 岔 . 在 8 3 中 将 讨 论 由 一 个 奇 点 和 一 条 闭 尧 以 及 连 结 它 们
E 章 的 学 习 , 读 者 将 会 对 如
何 利 用 奇 点 或 不 动 点 附 近 的 线 性 化 理 论 ( 见 第 一 章 $ 4) 来 研 究 非
E

$ 1 具 有 三 个 实 特 征 值 的 鞍 点 的 同 宿 分 岔

河 国 n 沥 沥
E

E p

假 设 中 一 光 溺 向 量 场 有 一 个 特 征 值 都 为 实 数 的 双 曲 鞍 点
及 其 间 宿 技 . 我 们 考 虑 这 样 一 个 向 量 场 在 一 般 的 单 参 数 扰 动 下 所
能 发 生 的 分 岔 . 不 失 一 舫 性 , 我 们 总 可 以 认 为 鞍 点 有 两 个 负 特 征 值
和 一 个 正 牺 征 值 ( 否 则 考 虑 其 时 间 反 向 系 统 ). 这 样 的 鞍 点 具 有 二
维的稳定流形和一维的不稳定流形 我 们 称 最 大 的 负 特 征 值 与 正
特 征 值 之 和 为 鞍 点 量

E 二 园 圆 沥 i
数 e==0 时 , 向 量 场 X 有 一 个 鞍 点 0, 它 具 有 两 个 负 特 征 值 和 一 个
d 5 a
E
E
B
Page-430
1 具 有 三 个 实 待 招 信 的 靼 点 的 闭 宿 分 盅

c 麾鸭(0,0)i E E

掸' E 丨蓥亡卜 d 途n

注 意 到 |ov|<<1, 我 们 有
E
由 (1.10) 有
E n 2 n 北 a
(l, 14) 与 (L 10) 一 起 推 出
E 招
以 及
l 0
E cp t enptyy e sR 圆 7 E 沥 河
兰 上 满 足 第 四 章 定 理 1. 5 的 所 有 条 件 . 因 而 4 在 De 内 有 晴 -- 双
D 不 胺
沥
E 朝
ss
【

E
p
t 吊 2 河 述

E E
[

E
〈蔷邈十| 园

b 玲
Page-431
不 具 有 三 个 实 御 征 值 的 糖 点 的 网 宝 分 吴 E

E
E

E

在 证 明 定 理 之 前 , 我 们 首 先 解 释 定 理 陈 述 中 “ 一 舫 “ 一 词 的 含
E t i 汀 i 沥 D 命诊 5 训 沥 招 河
三 条 是 针 对 E

点 0 的 特 征 值 两 两 不 相 同 且 为 非 共 振 .

d i
在 点 0 处 的 线 性 部 分 ( 参 见 第 一 章 定 理 4 18) , 故 在 绎 性 化 坐 标 系
沥
上 , 除 了 奇 点 0 和 一 条 通 过 O 点 的 直 线 外 , 所 有 辅 道 当 二 十 co 时
t
E

0 吊 p

一
向 量 张 成 一 个 不 变 平 面 仁 , 这 一 不 变 平 面 沿 着 同 宿 轨 延 伸

E
园 5

E 施 t 江 沥 口 p 刃
E

定 理 1. 1 证 明 的 恺 路

玟
E a e 汀 5 扎
伟 t
u 人
E
为 初 值 的 正 半 辉 与 “ 的 第 一 个 交 点 . 假 设 (1 使 我 们 可 以 把 向 量
E 志 不 s 沥n
Page-432
E E

0
E 吊 7
2
再 设 e>0. 因 为 4z(o,0) 一 e, 故 存 在 一 个 依 赖 于 参 数 e 的 常

E

4z(y,z〉>青,对0<z<Cg′ 0
E
E <
〈z(二v,言) 〈一歹<0′ 1

E

p
E 沥
E
E 园
i
E E E e 根 p
似 , 我 们 可 以 证 明 映 射 在 De 内 有 唯 一 双 曲 不 动 点 . 定 理 证 毕 ,

E 沥 洁 E toLork l

我 们 将 R 中 具 有 一 对 复 特 征 值 和 一 实 特 征 值 的 双 曲 鞍 点 称
E
假 设 鞋 焦 点 的 一 对 复 特 征 值 具 有 负 实 部 , 而 它 的 实 特 征 值 为 正 数 ,
E 一
实 部 的 和 称 作 鞍 点 量 . 鞋 点 量 为 正 或 为 负 的 鞍 焦 点 所 对 应 的 同 容
分 岗 有 着 本 质 的 不 同 . 当 鞍 点 重 为 负 时 , 分 盆 与 本 章 8 1 中 相 应 的
情 况 类 似 1 而 当 鞍 点 重 为 正 时 , 在 鞍 焦 点 的 同 宿 轨 的 任 意 邺 域 中 定
Page-433
第 五 章 宇 间 中 双 曲 鞅 的 问 宿 分 岖

等 函 数 袁 示 出 来 . 当 鞍 点 量 分 别 是 正 和 负 时 ,48“8 分 别 具 有 强 双
E
而 沿 振 道 我 们 可 以 定 义 从 同 宿 轨 与 7“ 的 交 点 的 邻 域 到 吟 宿 技 与
工 的 交 点 的 邺 域 的 晔 射 , 并 将 它 记 作 46, 见 图 5-2. 由 解 对 初 值 及

E gn 沥 y 沥
诊 沥 壮 标

E
最 后 , 我 们 分 别 利 用 双 曲 不 动 点 定 理 和 压 缉 映 象 原 理 , 讨 论 当 鞍 点
i

E 玖 仪
E
Page-434
朱 具 市 三 个 实 後 狄 值 的 鞍 炎 的 问 分 岔
Ezozsoooiumitieosuctuuuioitdi

B

、′

E

E

E 沥 水
{汉 ( 0 口 0 0
E

E 吴

E 应
uaeo e 林 e 7
|os| 一 1 不 失 一 舱 性 , 我 们 还 可 假 设 ,0 点 邻 域 中 正 = 粘 是 同 宿 转
园 c a i

E 坂 t
Page-435
不 2 圭 闹 中 驶 焦 点 的 同街 分 彻 E

界
5 二

具 有 负 鞍 点 量 的 鞍 焦 点 的 同 宿 分 岐

本 节 第 一 个 主 要 结 果 如 下 :

定 理 2. 1 令 X 是 R 中 一 个 一 舫 的 单 参 数 向 量 场 族 . 假 设 当
uap a 2ss p
的 同 密 轨 7, 则 存 在 YUO 的 邻 域 D 和 参 数 空 间 中 零 值 的 邹 域 7
E 途 余 t t
玟
E c

在 定 理 2 1 的 陈 述 中 “ 一 舫 “ 词 的 含 义是

( 向 量 场 X 的 鞍 焦 点 的 特 征 值 非 共 振 :

052 技t

Eagtult e

定 理 2. 1 证 明 的 直 现 含 义

在 给 出 严 格 证 明 以 前 , 为 便 于 读 者 理 解 , 我 们 先 给 出 证 明 的 直
观 揩 述 , 为 此 我 们 将 尽 量 把 问 题 简 化 - 首 先 我 们 假 设 , 在 鞍 焦 点 的
Rs t

E
c
0
E E
E t
E
8 二 外 门 I:Croy: 切 一 (000D0.

E

《C2.3)

会
E
E 2 东 林 t
Page-436
E

R3 2203
5
注 意 到 在 7 上 z 一 1, 因 此 由 等 式
E
d

哥z(D EzasyJOXG913 对

河 g

P
然 后 代 入 (1. 2 中 前 两 式 , 并 注 意 到 在 7+ 上 z 一 1, 便 可 得 到
E 0
E A A

E 不 不

E 口 林 木
点 加 一 (1,v,0), 由 解 对 初 值 与 参 数 的 依 赖 性 可 知 , 存 在 7“ 上 u
Eeetse e 庞街
吊 江5 加

E
映 射 4P8 是 一 个 光 滢 依 赖 于 的 徽 分 同 胚 - 令
E y c 吊

不 难 看 出 , (Y (e),Z(e)) 是 鞍 点 的 不 稳 定 流 形 与 7 的 交 点 , 面
E

羞Z(鬓〉L-。亨阜 0 2
玟

032 0
Page-437
208 第 五 章 空 间 中 双 幼 鞑 怒 的 问 宿 分 岖

E l gzg i 技怀 汀
起
E
E 5 江 述 汀
E dattuydumde i t
fz 一 0} 上 出 发 的 正 半 扬 线 的 并 组 成 . 我 们 用 4fog 表 示 泳 系 统
《2.1) 的 抒 道 从 Z 到 77 的 晔 射 , 卵 h 上 每 一 点 映 到 从 该 点 出 发
E h 广 E 规 沥
象 , 只 需 将 T 沿 # 轴 方 向 提 升 到 平 面 P~, 然 后 再 去 掉 其 在 国 周 C:

[ i 羞之外的部分' 有 两 种 情 况 需 要 考 虚
0 沥 不 c 应 梁
0 怀 2 东n 育 梁
见 图 5-3.

E
在 定 理 2. 1 中 只 考 虑 第 一 种 情 况 . 根 据 假 设 同 宿 执 逊 结 T+ 的 点 4
沥 林秉 北 扬 水 92
不
e
是 一 个 平 移 :
E l 5 江许 3
Page-438
y

E 2 怀 李 国 巳 河
E 初
‖4置i”薯(汉`z)‖″″ Etpy an

E
E 达
伟 一 dng:P 一 ,

E

现 在 我 们 对 鞍 点 量 为 负 的 情 况 来 证 明 定 理 1. 1.
E 林0

国 国 ,
月 二 ″(…)>1` E 泓

E 河
相 E
u 涛 | [o M 〕‖″^
E
E 述 林 a
关 于 参 数 一 致 有 界 . 因 此 ,(1. 6) 意 晖 着
L

E

E 沥 E a
e<0 栗 讨 论 映 射 4 的 不 动 点 的 存 在 性 .
【692x44
对 任 意 点 (y,z)E.Lo, 有
U

沥
E 江

CL. 6)

0
Page-439
沥 20

E 林口 沥
E 招
E 2
E 述 e 口 43
E e an
E 技
E d 一

定 理 2. 1 的 证 明

我 们 将 证 明 分 成 儿 步
0 标 p 江
由 假 设 1, 应 用 第 一 章 定 理 4. 27, 在 鞍 焦 点 的 一 个 邹 域 [ 里 存
在 一 个 坐 标 系 (z,y,z), 使 得 X: 在 0 内 是 线 性 系 统
E 3 朋
哥 E 怡 103 E
O
这 里 A(0) 一 0 人 x(0),e 一 (0) 十 pC0 一 0 如 果 必 要 的 话 , 可 对
吴
人 史 不 失 一 舫 性 , 我 们 还 可 进 一 步 假 设 , 当 e 一 0 时 ,[ 中 正 轻 是
同 宿 技 的 一 部 分 、 令
i
【 2 林李 圆 [ 朋
厂 二 {Cruy: 切 | 与 十 万 人 1 一 1
酝 c 木
z(震) E exp(入(s)z)〔c。s(m(s)′)堑 g sin(m(龌)!)y〕'
庆 XCIO3 ENCYG3 河 许 2
00
从 上 面 第 三 个 等 式 解 得 执 道 到 达 7 所 需 时 间 为 4 气 一 xCe)-:Inz.
Page-440
第 五 章 空 间 中 双 曲 数 点 的 闯 宪 分 盅

E

ELssetesanotulogo a
0nttd p 志 t 人
过 不 动 点 的 转 道 是 一 个 呆 引 周 期 轨 . 而 不 等 式 (1. 7) 意 昝 着 不 动 点
医 余 x 训
宿 软 的 位 置 .

C2) eSo.

d

E t

enst <鲁似鹏z) 5

g

上 边 不 等 式 中 等 叶 成 立 , 当 丁 仅 当

E
E
E

E

鞍 类 周 期 轨 的 产 生

现 在 对 鞍 点 量 为 正 的 倩 况 来 证 明 定 理 . 与 鞍 点 量 为 负 的 情 况
n
E

E
假 设 (3) 意 昭 督
[
- E E
情 况 么 0 和 乙 一 0 分 别 对 应 所 谓 叶 定 向 咖 宿 扔 和 不 可 定 向 闭 宿
E 志2 k 史

E E 《C1. 8)
Page-441
E 第 五 章 空 间 中 双 胡 积 点 的 问 裂 分 岔

将 其 代 人 前 两 个 等 式 , 得 到 抒 道 与 P 的 交 点 坐 标 Cz「 ,y「 ,1) 满 趸
[ 沥 技 c

E 一 A 簧羞岫. E

E 东 一
0

E 伟 E

《2. 6)
微 分 上 式 得 到
E z”_】【 俊 Sin辉〕 E 0 办
E
此 处

力 工 E 矗忐(入(s)楂。Se BaEX3SX)

E 1

0

E 河

E 朋

E
E R E 沥 口 吴

F
[ 诊
E 述 王

E 怀 c 玖
用 5,8 分 别 表 示 咏 寇 转 Y 与 7r+ 和 - 的 交 点 . 将 坐 标 系 做 一
E
E 述 [
E 万 圭 c 沥
M 伟
园 兰
Page-442
$ 1 具 有 三 个 宗 特 征 例 的 鞑 点 的 同 分 盅

E
沥
[
证 明 我 们 用 第 四 章 引 理 4 1 来 证 明
c
E 招 r
E
E E
l
E E
吉
伟

【nslt 扬 g

刀 二 0, 砺 一 zf-1
由 第 四 章 引 理 4 1 司 知 + 引 理 1. 3 成 立 , 如 果 存 在 常 数 [>0, 使 得

r

【 河 逊 一 仪
2
E 许 技 c

E 政 志 仁 丞 n 口
[

U

加 (

围 E

b 水
E

E

E 责zl一卢镳 E 沥 厂 昔y薪一润伶 E
Page-443
198 雉 五 章 空 间 中 双 幼 骇 炯 阡 问 宿 分 盅

y 巳 a 月
第 二 个 不 等 式 必 然 成 立 , 引 理 证 毕 . ‖
i
E 李 d
E
E
g
e eg
=簪汊z″十a仝予 8 沥 吴

巴 E
E
E 尚 i 吊 吴 A 口
E 命 育 c 芸 3 理25 命

E

再 设 e<0, 下 面 用 双 曲 不 助 点 定 理 来 证 明 ,4 在 T 内 有 唯 一

歌 类 不 动 点 因 为 4eCy,0) 一 e 一 0, 我 们 可 以 找 到 这 桦 一 个 依 赖 于
n

医 不 0

沥 _
E 怀 刑 人

则 短 形 D 的 边 界 的 水 平 部 分 为

2 L 辽
E

2 一 者 许
E

R 口 3

沥
Page-444
国 摄 E

E 途 c
d

0020 0 兰 7
则 假 设 (2) 的 确 切 含 义 为

基z佑〉 a

E .
E
E 用
0
E 林 a
E 林
u

由(2.8)'当/【充分小时,鸟是」个压缩常数小于噩的压缩映射一

河 沥
E 胡 [ 沥
E

鞍 点 量 为 正 的 鞍 焦 点 的 同 宿 分 岔 : 马 跟 的 存 在 性

i
形 的 歌 焦 点 的 同 宿 转 . 英 鞋 焦 点 的 鞍 点 量 为 正 , 则 在 同 宿 尬 的 任 意
E

定 理 2. 2 证 明 的 恺 路 我 们 将 通 过 验 证 同 宪 承 的 Poincare 映
E 小 技
0
河
t 江 t 莲
为 O , 集 合 4UAMtz 一 0} 有 可 数 个 途 通 分 岔 , 我 们 考 虑 那 些 位
Page-445
E 霓五量空间中双曲幔点的同宿分岔

于 f 一 0} 上 方 的 连 通 分 岔 的 原 象 , 这 些 原 象 都 是 曲 边 短 形 , 晓 射 4
E

E
n 河
E 达

止 一 z 一 ay
{夕 二 az 十 加 ,
c

E 沥

E 扬
E 仪 招 加 罡

鸡鬣蚯疃(枳'z) E 标 玟 小 沥 招 雇))Y 《C2.12)

E

[ 林二 浩 [ 林亚 二 阮
由(2.12),象肥叩碗是厂_上的以(0,0)为极点的对数嫖线. E
到 4 是 一 个 微 分 同 胡 , 故 象 t =4…〖'鲳叩碗是厂十上以点诙
Page-446
1 具 有 三 个 实 待 招 信 的 靼 点 的 闭 宿 分 盅

c 麾鸭(0,0)i E E

掸' E 丨蓥亡卜 d 途n

注 意 到 |ov|<<1, 我 们 有
E
由 (1.10) 有
E n 2 n 北 a
(l, 14) 与 (L 10) 一 起 推 出
E 招
以 及
l 0
E cp t enptyy e sR 圆 7 E 沥 河
兰 上 满 足 第 四 章 定 理 1. 5 的 所 有 条 件 . 因 而 4 在 De 内 有 晴 -- 双
D 不 胺
沥
E 朝
ss
【

E
p
t 吊 2 河 述

E E
[

E
〈蔷邈十| 园

b 玲
Page-447
0 招 207

E 沥 0 一 如
E eln 浩g a 心 A
园 述
述 i 心 t 河 河 e 浩
二 是 一 个 曲 边 矩 形 . 它 的 上 下 边 属 于 直 线 {z 一 0} 的 原 象 , 我 们 晤
医 育 c 江 连 E s

orml- 羯 <oosernsaml - 献 |
0

【 <‖腑胖川4肛 丿 Ciexp E 814
_m_ |

这 里 b,z)1 表 示 炭 (0,m) 到 点 8 一 (0,0) 之 间 的 贸 高 - 为 了 证 明 上
述 两 不 等 式 , 在 (b, 史 平 面 上 对 点 (9,2) E Z 我 们 如 此 定 义 其 象 点
E
E 沥

[ 河 0

EE
由 2 12) , 有

[ \羞lnz 十 0GD.

E
许 汀

E 牡羞』Z牡十O〈1)' [
E

二 2nx 十 O(D。

由 上 式 便 推 出 (2. 13). 注 意 到
Page-448
E E

0
E 吊 7
2
再 设 e>0. 因 为 4z(o,0) 一 e, 故 存 在 一 个 依 赖 于 参 数 e 的 常

E

4z(y,z〉>青,对0<z<Cg′ 0
E
E <
〈z(二v,言) 〈一歹<0′ 1

E

p
E 沥
E
E 园
i
E E E e 根 p
似 , 我 们 可 以 证 明 映 射 在 De 内 有 唯 一 双 曲 不 动 点 . 定 理 证 毕 ,

E 沥 洁 E toLork l

我 们 将 R 中 具 有 一 对 复 特 征 值 和 一 实 特 征 值 的 双 曲 鞍 点 称
E
假 设 鞋 焦 点 的 一 对 复 特 征 值 具 有 负 实 部 , 而 它 的 实 特 征 值 为 正 数 ,
E 一
实 部 的 和 称 作 鞍 点 量 . 鞋 点 量 为 正 或 为 负 的 鞍 焦 点 所 对 应 的 同 容
分 岗 有 着 本 质 的 不 同 . 当 鞍 点 重 为 负 时 , 分 盆 与 本 章 8 1 中 相 应 的
情 况 类 似 1 而 当 鞍 点 重 为 正 时 , 在 鞍 焦 点 的 同 宿 轨 的 任 意 邺 域 中 定
Page-449
i

a
医 招
D

E 怀 国 a 沥 木
L 口

E 国 ttut fn
E n 国 沥 河
E 标 p 汪

E 沥 伟 t 河
E 一
林工 初
E

E

52 林 沥 E 2 东 衍 7
E

0
一 sin(9 十 . 等 式 (2. 17) 两 边 对 9

里鱼暨

″zd朦
E 江
团 zd″十z〔] 友 一 d″〕

d三 5 1 里 1 里
胁〔 E ″汊/ 【 命 '′z〕

E 沥 玟 2 余 22
将 (2. 18) 除 以 z, 得

0 国 吴 不 巳
丽〔 ″/′(z)+″z E 有 z/′(z)十 '′]

[
Page-450
不 2 圭 闹 中 驶 焦 点 的 同街 分 彻 E

界
5 二

具 有 负 鞍 点 量 的 鞍 焦 点 的 同 宿 分 岐

本 节 第 一 个 主 要 结 果 如 下 :

定 理 2. 1 令 X 是 R 中 一 个 一 舫 的 单 参 数 向 量 场 族 . 假 设 当
uap a 2ss p
的 同 密 轨 7, 则 存 在 YUO 的 邻 域 D 和 参 数 空 间 中 零 值 的 邹 域 7
E 途 余 t t
玟
E c

在 定 理 2 1 的 陈 述 中 “ 一 舫 “ 词 的 含 义是

( 向 量 场 X 的 鞍 焦 点 的 特 征 值 非 共 振 :

052 技t

Eagtult e

定 理 2. 1 证 明 的 直 现 含 义

在 给 出 严 格 证 明 以 前 , 为 便 于 读 者 理 解 , 我 们 先 给 出 证 明 的 直
观 揩 述 , 为 此 我 们 将 尽 量 把 问 题 简 化 - 首 先 我 们 假 设 , 在 鞍 焦 点 的
Rs t

E
c
0
E E
E t
E
8 二 外 门 I:Croy: 切 一 (000D0.

E

《C2.3)

会
E
E 2 东 林 t
Page-451
E 209

E l+量/′彻苡 。 E

注 意 到 当 z令0时,量令/′ E
E 述 [
蔫 E

再 由 (2.13), 当 nrvoo 时 有 z*0. 引 理 证 毕 , 【
E

E {(廖yz) E 缙【〕 }

E 技 东 技 余 a 技
Anitz 一 0}- 四 为 鞍 点 量 c 一 X 十 /2>0, 故 当 xS>1 时

E 懈【〉 E eXp〈脊了r).

由 不 等 式 (2. 13) 和 (2. 14) ,Poincar 映 射 4 限 制 在 区 域 D 上 类 似
一 个 马 路 映 射 , 参 见 图 5-5. 下 面 证 明 Dof1ADn 确 实 包 含 了 一 个 马
eo iiretaina s
理 3.2 的 所 有 条 件 . 首 先 , 我 们 证 明 映 射 4 在 区 域 Dfi14““Dn 上
Page-452
b E

E

E 国 a 贾 d
E u 沥 u 7 命 寸 河
E 江 江 诊

E

医 e [ 彗 E

E 林口 许 莲 人 /
E E 圆 国

因为向童〔盂〕是矩阵靥 的 特 征 向 量 , 故

a 12
0

[ 责lnz一>婶。(mod E n 浩 脱 白 4

由 区 域 D 的 定 义 ,(2. 21) 中 第 一 个 极 限 是 显 然 的 . 现 在 我 们 证 明
E , 故 4(6,z) E Z 门 A 由
《2. 13) 和 (2. 14) 有

E 朋
L
E 刑

现 在 我 们 来 计 算 雅 可 比 阵 D4. 对 (2. 12) 求 微 分 得

E sin'聋〕^′

sin8 cosp

DQreg(a) [

|

E (

E .
门 「 E 帼加仍]_
E 圆
因 此 , 由 链 式 法 则 有
E iC 3 D4【噜(4血疃(仇z)) * D4酶叩(仇z〉
Page-453
208 第 五 章 空 间 中 双 幼 鞑 怒 的 问 宿 分 岖

E l gzg i 技怀 汀
起
E
E 5 江 述 汀
E dattuydumde i t
fz 一 0} 上 出 发 的 正 半 扬 线 的 并 组 成 . 我 们 用 4fog 表 示 泳 系 统
《2.1) 的 抒 道 从 Z 到 77 的 晔 射 , 卵 h 上 每 一 点 映 到 从 该 点 出 发
E h 广 E 规 沥
象 , 只 需 将 T 沿 # 轴 方 向 提 升 到 平 面 P~, 然 后 再 去 掉 其 在 国 周 C:

[ i 羞之外的部分' 有 两 种 情 况 需 要 考 虚
0 沥 不 c 应 梁
0 怀 2 东n 育 梁
见 图 5-3.

E
在 定 理 2. 1 中 只 考 虑 第 一 种 情 况 . 根 据 假 设 同 宿 执 逊 结 T+ 的 点 4
沥 林秉 北 扬 水 92
不
e
是 一 个 平 移 :
E l 5 江许 3
Page-454
E

=疃/=〔

a 凸) (雅 巴
4

E

cosg “ 一 sing

E 河 5
国 E
0 麽j E (夕'z))〔蚯n辫 匹 春

5

E
一 D M}'
e
E
玲
E 途

二 之 max(BH- 弓 a 1BM7 1 E
以 及

maxt| 取 一 M D 4 |

E 目
成 立 . 由 (2.217, 对 (9,z)E n, 当 z“*oo 时 有 ~ 应 或 一 应 . 故 由
0

1

国 医

注 意 到 当 w-voo 时 ,8~~0, 我 们 有
团 E
1BMT| E

E d p
u a

〈韫L,″》1' E

E

n
b 沥 述 沥

最 后 来 验 证 ,4 在 Df14““D 上 满 足 边 界 条 件 和 相 交 条 件 -
Page-455
沥 20

E 林口 沥
E 招
E 2
E 述 e 口 43
E e an
E 技
E d 一

定 理 2. 1 的 证 明

我 们 将 证 明 分 成 儿 步
0 标 p 江
由 假 设 1, 应 用 第 一 章 定 理 4. 27, 在 鞍 焦 点 的 一 个 邹 域 [ 里 存
在 一 个 坐 标 系 (z,y,z), 使 得 X: 在 0 内 是 线 性 系 统
E 3 朋
哥 E 怡 103 E
O
这 里 A(0) 一 0 人 x(0),e 一 (0) 十 pC0 一 0 如 果 必 要 的 话 , 可 对
吴
人 史 不 失 一 舫 性 , 我 们 还 可 进 一 步 假 设 , 当 e 一 0 时 ,[ 中 正 轻 是
同 宿 技 的 一 部 分 、 令
i
【 2 林李 圆 [ 朋
厂 二 {Cruy: 切 | 与 十 万 人 1 一 1
酝 c 木
z(震) E exp(入(s)z)〔c。s(m(s)′)堑 g sin(m(龌)!)y〕'
庆 XCIO3 ENCYG3 河 许 2
00
从 上 面 第 三 个 等 式 解 得 执 道 到 达 7 所 需 时 间 为 4 气 一 xCe)-:Inz.
Page-456
g i

E tiessiu r
医
上 下 两 边 属 于 D 的 水 平 边 界 D 的 原 象 4“8%Du, 因 而 HY, F
E

E 沥 莲 仁 【
注 意 到 43yHY 人 %Do , 由 (2.26) 有 42HY n (HY U H 一 幼 . 由
E 沥
E
娆 立 . '
上 面 的 论 证 表 明 , 对 所 有 充 分 大 的 自 熊 数 , 映 射 4 在 区 域
E 磁 匹 ss t a
拓 扑 共 轲 . 图 而 4 在 Ln 上 有 无 穷 多 个 周 期 点 . 因 而 向 量 场 在 同 宿
辅 附 近 有 无 穷 多 条 双 曲 周 期 转 .

定 理 证 毕 【

转 道 等 价 不 变 量 的 存 在 性

医
E 达

沥
E 林 (

表 示 与 友 等 价 的 向 量 场 的 集 合 . 称 XE 心 “M) ( 关 于 等 价 关 系
E 沥 春 [

定 义 2.5 一 个 复 数 C 称 为 向 量 场 XE “ “CM ) 的 模 , 如 果 对
玟
上 的 一 个 非 常 值 连 续 函 数 广 漾 足

E

E 技 X 园 4
Page-457
E 第 五 章 空 间 中 双 胡 积 点 的 问 裂 分 岔

将 其 代 人 前 两 个 等 式 , 得 到 抒 道 与 P 的 交 点 坐 标 Cz「 ,y「 ,1) 满 趸
[ 沥 技 c

E 一 A 簧羞岫. E

E 东 一
0

E 伟 E

《2. 6)
微 分 上 式 得 到
E z”_】【 俊 Sin辉〕 E 0 办
E
此 处

力 工 E 矗忐(入(s)楂。Se BaEX3SX)

E 1

0

E 河

E 朋

E
E R E 沥 口 吴

F
[ 诊
E 述 王

E 怀 c 玖
用 5,8 分 别 表 示 咏 寇 转 Y 与 7r+ 和 - 的 交 点 . 将 坐 标 系 做 一
E
E 述 [
E 万 圭 c 沥
M 伟
园 兰
Page-458
E E

E
E 达
征 根 的 比 值 即 为 模 . 因 而 , 任 意 有 孤 立 契 点 的 向 量 场 都 不 是 结 构 稳
定 的 . 再 如 , 如 果 我 们 在 拓 扑 轨 道 等 价 中 要 求 相 应 同 背 保 持 时 间 ,
则 向 量 场 的 孤 立 周 期 转 的 周 期 便 是 模 . 因 而 , 任 意 有 学 立 周 期 轨 的
向 量 场 不 是 结 构 稳 定 的 . 显 然 , 上 述 两 种 等 价 关 系 过 于 “ 严 格 “, 使
国 2 志 相
道 等 价 ( 见 第 一 章 定 义 1.7) 下 , 研 究 结 构 稳 定 问 题 和 分 岔 问 题 . 寻
找 有 模 的 系 统 是 分 会 理 论 中 一 个 很 有 意 义 的 问 题

E
Etstusuroh gn i
t

n
明 , 是 辐 道 等 价 不 变 量 . 在 下 而 的 讨 论 中 , 我 们 仍 利 用 前 面 的
E

E 沥 a 江 0

A(m 二 min 代 以 N,Zn 门 A 世 灭 )

E 仪 江 ( 余

n( 吴 一 pmaxfm 仪 月 , 五 仁 A 央 如 ]
E

吴 弘
E E E 7

b
E
E
E 仪
园

根
Page-459
国 摄 E

E 途 c
d

0020 0 兰 7
则 假 设 (2) 的 确 切 含 义 为

基z佑〉 a

E .
E
E 用
0
E 林 a
E 林
u

由(2.8)'当/【充分小时,鸟是」个压缩常数小于噩的压缩映射一

河 沥
E 胡 [ 沥
E

鞍 点 量 为 正 的 鞍 焦 点 的 同 宿 分 岔 : 马 跟 的 存 在 性

i
形 的 歌 焦 点 的 同 宿 转 . 英 鞋 焦 点 的 鞍 点 量 为 正 , 则 在 同 宿 尬 的 任 意
E

定 理 2. 2 证 明 的 恺 路 我 们 将 通 过 验 证 同 宪 承 的 Poincare 映
E 小 技
0
河
t 江 t 莲
为 O , 集 合 4UAMtz 一 0} 有 可 数 个 途 通 分 岔 , 我 们 考 虑 那 些 位
Page-460
E 霓五量空间中双曲幔点的同宿分岔

于 f 一 0} 上 方 的 连 通 分 岔 的 原 象 , 这 些 原 象 都 是 曲 边 短 形 , 晓 射 4
E

E
n 河
E 达

止 一 z 一 ay
{夕 二 az 十 加 ,
c

E 沥

E 扬
E 仪 招 加 罡

鸡鬣蚯疃(枳'z) E 标 玟 小 沥 招 雇))Y 《C2.12)

E

[ 林二 浩 [ 林亚 二 阮
由(2.12),象肥叩碗是厂_上的以(0,0)为极点的对数嫖线. E
到 4 是 一 个 微 分 同 胡 , 故 象 t =4…〖'鲳叩碗是厂十上以点诙
Page-461
214 E

Eugsss
n
由 (2. 13) 和 (2. 147 易 知 , 如 果

[

[ 仪 一 一
E
[

t

E 一 一 述
李 7
E

巳 c 巴
′e(″)<噜_ A E

由 (2. 30) 得
InC 2 E
E

E
E
幽丨nc】_, E 园 苴_ 国 矶nC】

E ′

E 沥 或 t
E 英
E
E 木 t 沥 p
伟

工 一 Aiz 一 ag,
纠沙 人 a

5
Page-462
0 招 207

E 沥 0 一 如
E eln 浩g a 心 A
园 述
述 i 心 t 河 河 e 浩
二 是 一 个 曲 边 矩 形 . 它 的 上 下 边 属 于 直 线 {z 一 0} 的 原 象 , 我 们 晤
医 育 c 江 连 E s

orml- 羯 <oosernsaml - 献 |
0

【 <‖腑胖川4肛 丿 Ciexp E 814
_m_ |

这 里 b,z)1 表 示 炭 (0,m) 到 点 8 一 (0,0) 之 间 的 贸 高 - 为 了 证 明 上
述 两 不 等 式 , 在 (b, 史 平 面 上 对 点 (9,2) E Z 我 们 如 此 定 义 其 象 点
E
E 沥

[ 河 0

EE
由 2 12) , 有

[ \羞lnz 十 0GD.

E
许 汀

E 牡羞』Z牡十O〈1)' [
E

二 2nx 十 O(D。

由 上 式 便 推 出 (2. 13). 注 意 到
Page-463
E b
Dnudttiii

通 过 一 个 线 性 变 换 , 我 们 可 使 中 的 点 丸 一 (L,0,0) 命 于 oi 的 同
E
c 2 罚 月 诊

e 贺 朝
E

s2 玖 命 目 医 吊

E 沥 逊 怀
园 a 伟 理 江
沥 i 一
E
tto 022
月 述
[
E 达 荣 p 江 才
E 河 口
医 u i 吴 行
E 达 沥 振 训
E 河 p 2
的 整 数 , 使 得
c

Ed

0 俊

0 沥 玟 达 5 万
E

吴
2

E 国
Page-464
i

a
医 招
D

E 怀 国 a 沥 木
L 口

E 国 ttut fn
E n 国 沥 河
E 标 p 汪

E 沥 伟 t 河
E 一
林工 初
E

E

52 林 沥 E 2 东 衍 7
E

0
一 sin(9 十 . 等 式 (2. 17) 两 边 对 9

里鱼暨

″zd朦
E 江
团 zd″十z〔] 友 一 d″〕

d三 5 1 里 1 里
胁〔 E ″汊/ 【 命 '′z〕

E 沥 玟 2 余 22
将 (2. 18) 除 以 z, 得

0 国 吴 不 巳
丽〔 ″/′(z)+″z E 有 z/′(z)十 '′]

[
Page-465
E

才 c

552 怀 e
道 兮 , 满 足 a(2 一 史 和 o(73 一 ai+lsi 心 0,1,...,& 一 1, 斧 构 成 的 捉
e
E
双 曲 奇 点 和 一 条 双 曲 闭 转 以 及 两 条 连 结 它 们 的 抒 道 所 构 成 的 环 .
值 得 指 出 的 是 , 在 著 名 的 Lorenz 方 程 中 , 对 某 些 参 数 值 上 述 环 是
Eseaaizinlaktsppigsa 达 [ d
E

环 的 分 类
巴 el c

E 沥
0 i d
稳 定 流 形 “(oo.
0 沥 s 沥
E
E 沥 c
E 技t
(4) 如 果 奇 点 r 的 特 征 值 都 为 实 数 <<A 一 0 一 v 则 c 是 2 阶
E
【 志 a 怀 c
E p
E
E 怀 d 立
满 足 上 面 四 条 假 设 的 环 4 有 三 种 不 合 的 类 型 :
p t
Page-466
E 209

E l+量/′彻苡 。 E

注 意 到 当 z令0时,量令/′ E
E 述 [
蔫 E

再 由 (2.13), 当 nrvoo 时 有 z*0. 引 理 证 毕 , 【
E

E {(廖yz) E 缙【〕 }

E 技 东 技 余 a 技
Anitz 一 0}- 四 为 鞍 点 量 c 一 X 十 /2>0, 故 当 xS>1 时

E 懈【〉 E eXp〈脊了r).

由 不 等 式 (2. 13) 和 (2. 14) ,Poincar 映 射 4 限 制 在 区 域 D 上 类 似
一 个 马 路 映 射 , 参 见 图 5-5. 下 面 证 明 Dof1ADn 确 实 包 含 了 一 个 马
eo iiretaina s
理 3.2 的 所 有 条 件 . 首 先 , 我 们 证 明 映 射 4 在 区 域 Dfi14““Dn 上
Page-467
E 腾 s E

E ss ss 水
示 向 量 场 乃 限 制 在 4 的 棠 个 邻 域 U 内 的 非 游 荡 集 , 这 时 A 称 为
E 不

E
E

上 面 三 种 类 珩 的 环 可 参 见 图 5-6. 下 面 我 们 首 先 证 明 鞍 焦 环 和
E 不 述
量场在一般的单参敷扰动下的分岔y还没有被研究渭楚'本节的最
E d

鞍 焦 环

定 理 3. 1 设 光 滑 向 量 场 丿 有 一 个 鞍 焦 环 4, 漾 足 假 设 (1) 一
2

证 明 定 理 结 论 的 儿 何 意 义 可 参 见 囹 5-7-

形 沥 沥 tn 述 林
E

0

E

0 2 a 光

E O 扬 2

E
E 在该邻域里可由下面线
E

宋 心 矩 一 a
卡 江5 [ 沥
c
吴 5 江应 才
E 沥
E 命 巳 匹
Page-468
b E

E

E 国 a 贾 d
E u 沥 u 7 命 寸 河
E 江 江 诊

E

医 e [ 彗 E

E 林口 许 莲 人 /
E E 圆 国

因为向童〔盂〕是矩阵靥 的 特 征 向 量 , 故

a 12
0

[ 责lnz一>婶。(mod E n 浩 脱 白 4

由 区 域 D 的 定 义 ,(2. 21) 中 第 一 个 极 限 是 显 然 的 . 现 在 我 们 证 明
E , 故 4(6,z) E Z 门 A 由
《2. 13) 和 (2. 14) 有

E 朋
L
E 刑

现 在 我 们 来 计 算 雅 可 比 阵 D4. 对 (2. 12) 求 微 分 得

E sin'聋〕^′

sin8 cosp

DQreg(a) [

|

E (

E .
门 「 E 帼加仍]_
E 圆
因 此 , 由 链 式 法 则 有
E iC 3 D4【噜(4血疃(仇z)) * D4酶叩(仇z〉
Page-469
经
E
[
区
意
国
蟹
梁
国
[
区
E
E
团
E
Page-470
E 胡2

E

E 命 口

E 沥 林 圆 一 不
(3'1)所导出的从碲 E
E 沥 力

d 不 辽 江 园 3
E 2 15 顺 05 沥

这 里 a=A/A<0, 婶=_责bz′ E

【 芸
转 道 丿 连 结 点 加 与 点 , 由 方 程 的 解 对 初 值 的 依 赖 性 定 理 , 每 一 条
E 林东
宇 廷 明 小 l 育 0 朋
c 一
截 相 交 , 故 曲 线 WPr(oi) V 可 以 表 示 成
E 52 沥 一 = 人 罗 0 才 、

E 达
Page-471
E

=疃/=〔

a 凸) (雅 巴
4

E

cosg “ 一 sing

E 河 5
国 E
0 麽j E (夕'z))〔蚯n辫 匹 春

5

E
一 D M}'
e
E
玲
E 途

二 之 max(BH- 弓 a 1BM7 1 E
以 及

maxt| 取 一 M D 4 |

E 目
成 立 . 由 (2.217, 对 (9,z)E n, 当 z“*oo 时 有 ~ 应 或 一 应 . 故 由
0

1

国 医

注 意 到 当 w-voo 时 ,8~~0, 我 们 有
团 E
1BMT| E

E d p
u a

〈韫L,″》1' E

E

n
b 沥 述 沥

最 后 来 验 证 ,4 在 Df14““D 上 满 足 边 界 条 件 和 相 交 条 件 -
Page-472
220 E 摄 t

EEok t 沥 s 河 木 洁 技 5
螺 线

【 0 沥
上 述 曲 线 属 于 何 扬 r 的 不 穗 定 流 形 WP“(o) 与 平 面 5 的 交 . 下 面 将
通 过 验 证 曲 线 (3. 2) 与 曲 线 P“CotfS: z 一 0) 有 横 戳 相 交 点 来 证 ˇ
明 定 理 3.1. 令 F 一 (Fo,Fy), 则 我 们 只 需 诛 明 方 程

y( 友 (8(a), 驯 , 丁 (b(a),z)) 一 0 医

a

国 史 国 一

园 E ,

那 么 (3. 3) 可 写 成
E d 2 述 3

园 ,
E E 林 E 刑

E
E 李 玟 人 应 玟 圭 玟 人 河
E
E 江 2 仪 才 步 2 河
的 函 数 加 上 一 个 小 重 . 因 此 (3. 4 有 无 穷 多 个 根 s 一 0. 下 面 证 明
E 江 应

u 水

0

这 里 当 & 一 oo 时 ,o(1) 一 0. 另 一 方 面
E A 2 育 圆 林 6 育 切
_"′ (z′)I 0 亩 7 5 + _苏_ F 万 E

团 _ E 江 述 途
【 羞)〕 玟 沥 伟 吴 23 丞通

罡
Page-473
g i

E tiessiu r
医
上 下 两 边 属 于 D 的 水 平 边 界 D 的 原 象 4“8%Du, 因 而 HY, F
E

E 沥 莲 仁 【
注 意 到 43yHY 人 %Do , 由 (2.26) 有 42HY n (HY U H 一 幼 . 由
E 沥
E
娆 立 . '
上 面 的 论 证 表 明 , 对 所 有 充 分 大 的 自 熊 数 , 映 射 4 在 区 域
E 磁 匹 ss t a
拓 扑 共 轲 . 图 而 4 在 Ln 上 有 无 穷 多 个 周 期 点 . 因 而 向 量 场 在 同 宿
辅 附 近 有 无 穷 多 条 双 曲 周 期 转 .

定 理 证 毕 【

转 道 等 价 不 变 量 的 存 在 性

医
E 达

沥
E 林 (

表 示 与 友 等 价 的 向 量 场 的 集 合 . 称 XE 心 “M) ( 关 于 等 价 关 系
E 沥 春 [

定 义 2.5 一 个 复 数 C 称 为 向 量 场 XE “ “CM ) 的 模 , 如 果 对
玟
上 的 一 个 非 常 值 连 续 函 数 广 漾 足

E

E 技 X 园 4
Page-474
医 胡 e

5 兽土z…『】(肃 E 6

上 面 最 后 一 个 等 式 是 由 (3. 4 和 (3. 5) 推 出 的 , 定 理 证 毕 , 〖

E

E 东 志 2 u s 怡 a 租 沥 河
E t 东 5 i 丞 志 朐

E 国 英
E
E 怀 0

源 万 水 P。擅黜崖 E

医 沥 a 兰 一 公 i
邹 基 可 以 定 义 闭 轨 口 的 Poincare 映 射 九 因 为 @ 点 是 一 维 晔 射 P 的
E 00 招
E

E

0

0 林

E 4 芒

[
E 沥
E
国
E
玟
令
E
Page-475
E E

E
E 达
征 根 的 比 值 即 为 模 . 因 而 , 任 意 有 孤 立 契 点 的 向 量 场 都 不 是 结 构 稳
定 的 . 再 如 , 如 果 我 们 在 拓 扑 轨 道 等 价 中 要 求 相 应 同 背 保 持 时 间 ,
则 向 量 场 的 孤 立 周 期 转 的 周 期 便 是 模 . 因 而 , 任 意 有 学 立 周 期 轨 的
向 量 场 不 是 结 构 稳 定 的 . 显 然 , 上 述 两 种 等 价 关 系 过 于 “ 严 格 “, 使
国 2 志 相
道 等 价 ( 见 第 一 章 定 义 1.7) 下 , 研 究 结 构 稳 定 问 题 和 分 岔 问 题 . 寻
找 有 模 的 系 统 是 分 会 理 论 中 一 个 很 有 意 义 的 问 题

E
Etstusuroh gn i
t

n
明 , 是 辐 道 等 价 不 变 量 . 在 下 而 的 讨 论 中 , 我 们 仍 利 用 前 面 的
E

E 沥 a 江 0

A(m 二 min 代 以 N,Zn 门 A 世 灭 )

E 仪 江 ( 余

n( 吴 一 pmaxfm 仪 月 , 五 仁 A 央 如 ]
E

吴 弘
E E E 7

b
E
E
E 仪
园

根
Page-476
第 五 蜕 空 间 中 双 曲 粤 炎 的 间 宿 分 盆

E 技
E
医 沥 L
E 沥 2 技

E

国 t

由 假 设 (43, 我 们 可 以 在 鞍 炒 % 的 一 个 邻 域 外 中 选 取 一 个 C!
E 河 t 、

0 沥

2 向 量 场 X 由 下 列 线 性 系 统 给 出

E
{仪 E 林

5

E

E

沥

E 沥 22 技 林 江 一 “
E

E
Euztse 河 E 唐 芸
有 形 式
[

E

C 卫 到 广 的 映 射

E
E 沥 命 林 沥 东
Page-477
214 E

Eugsss
n
由 (2. 13) 和 (2. 147 易 知 , 如 果

[

[ 仪 一 一
E
[

t

E 一 一 述
李 7
E

巳 c 巴
′e(″)<噜_ A E

由 (2. 30) 得
InC 2 E
E

E
E
幽丨nc】_, E 园 苴_ 国 矶nC】

E ′

E 沥 或 t
E 英
E
E 木 t 沥 p
伟

工 一 Aiz 一 ag,
纠沙 人 a

5
Page-478
E 脱 y

Egoam 河 y
【 a 0 2 673
由 假 设 (3) 流 形 P*(ou) 与 W“(oo) 沿 . 朱截 相 交 , 故 有

E

面 a

囹 此 , 可 选 联 充 分 小 , 使 得 .
E 林 朋

a

颜 e

[ 一 林 许
E 木
技 _ _

E
E _

闵 0 E

[

浩 ′)姜0,阊 [

曦>0和碗<o分别对应非游荡环和游荡环′

E 膏 e 江 许 沥
令
砺 一 G 人 H Kn 一 P eK8 口 丞 n
E d 吊 洁 玟
〈愫”:F”配岫 匹
令
@ 二 { E D|Rral 十 切 一 5 丢 不 叶 3G 一 动 j
Page-479
224 l

则 有
[ 吴 2

令
E
则 映 射 4 在 区 域 DY U DY 上 是 马 蹄 映 射 , 见 图 5-8-

E 林 国 胺
E
E
E s2
b 沥
[6 t 32

因 为 (&,EQn+a, 故
大 [ 沥

0 =壬

E
Page-480
E b
Dnudttiii

通 过 一 个 线 性 变 换 , 我 们 可 使 中 的 点 丸 一 (L,0,0) 命 于 oi 的 同
E
c 2 罚 月 诊

e 贺 朝
E

s2 玖 命 目 医 吊

E 沥 逊 怀
园 a 伟 理 江
沥 i 一
E
tto 022
月 述
[
E 达 荣 p 江 才
E 河 口
医 u i 吴 行
E 达 沥 振 训
E 河 p 2
的 整 数 , 使 得
c

Ed

0 俊

0 沥 玟 达 5 万
E

吴
2

E 国
Page-481
E 浩 2

E E
E

E

E
E E

命 砌 ) 门
明
玄剔<破<2j腥′y 对 一 余 1

这 与 (3. 11) 一 起 推 出
4

(

现 在 我 们 计 算 雅 可 比 阵 D4. 由 链 式 法 则 有
E 沥 3

孕
E [0

E

E

y

E
。 潺〕 e 达 一

园 东 林
E 【(z抛 +叮抛%
Page-482
226 d

E 江
maxtlDRF[IDPF-i a 1BM-| 一 , (8.13)
E 一 沥

E 东
max(|4| “ | 取 M 1D|,14 一 BMTD|,
c [ 浩 5
E 水
E 途 沥 沥 d 丿 厂 2 余 技 沥 0 办 育 3
E 告蔚一婶(二菖…厂】 z箐 E 咿彗 a
1 沥
国 5
E
E 江 当鹫+o时 5

M [ 一 肉 RmFTmar-ece 驿 土 心 骆 0 一 o,

t 夏_离一-|(一)一】(一)| 一 0.

E
林李
D ,
[ 一 木 标
E 达

1
P 雇_″〔 E 谚〕 崛翼 。 】肯互血+”

(1 +盘_市湟…」+…_〉玮0y 门
园 技 沥 沥 圆
Page-483
E

才 c

552 怀 e
道 兮 , 满 足 a(2 一 史 和 o(73 一 ai+lsi 心 0,1,...,& 一 1, 斧 构 成 的 捉
e
E
双 曲 奇 点 和 一 条 双 曲 闭 转 以 及 两 条 连 结 它 们 的 抒 道 所 构 成 的 环 .
值 得 指 出 的 是 , 在 著 名 的 Lorenz 方 程 中 , 对 某 些 参 数 值 上 述 环 是
Eseaaizinlaktsppigsa 达 [ d
E

环 的 分 类
巴 el c

E 沥
0 i d
稳 定 流 形 “(oo.
0 沥 s 沥
E
E 沥 c
E 技t
(4) 如 果 奇 点 r 的 特 征 值 都 为 实 数 <<A 一 0 一 v 则 c 是 2 阶
E
【 志 a 怀 c
E p
E
E 怀 d 立
满 足 上 面 四 条 假 设 的 环 4 有 三 种 不 合 的 类 型 :
p t
Page-484
不 3 环 的 分 盆

颠 L

沥
图
E 林 7

【 关 许

[ 怀 沥 c 技 胡

《3) 边 界 条 件 ,4aeDf 门 D? 一 史 ,ADFM13,D9 一 奶 心 j 一 1,2.

a
盘的一部分,而它的上卞两边分别是Q″+z上下边在映射4下的原
象 ; 因 而 是 18 水 平 曲 线 , 故 条 件 (1) 成 立 , 其 次 , 相 交 条 件 不 难 由 述
E 东 E 伟 行

【 述

交 , 故 49.DFTIDY 一 亿 . 另 一 方 面 , 注 意 到

E 22 沥
E 训 沥 扬 达 江 王 月

E

E |

游 茆 环

4
所 发 生 的 分 岗 . 设 光 滑 向 最 场 X 有 一 个 游 荧 环 4, 满 尼 假 设 (0) 一
E 江 林 n 仪
和 闭 轨 u 是 双 曲 的 , 故 向 量 场 Xe 在 ca,oi 及 软 道 4 附 近 有 双 曲 奇
沥
E 7 朐

定 义 3.4 称 参 数 值 e 为 向 量 场 族 Xe 的 同 宣 分 岔 值 ( 环 分 岔
Page-485
E 腾 s E

E ss ss 水
示 向 量 场 乃 限 制 在 4 的 棠 个 邻 域 U 内 的 非 游 荡 集 , 这 时 A 称 为
E 不

E
E

上 面 三 种 类 珩 的 环 可 参 见 图 5-6. 下 面 我 们 首 先 证 明 鞍 焦 环 和
E 不 述
量场在一般的单参敷扰动下的分岔y还没有被研究渭楚'本节的最
E d

鞍 焦 环

定 理 3. 1 设 光 滑 向 量 场 丿 有 一 个 鞍 焦 环 4, 漾 足 假 设 (1) 一
2

证 明 定 理 结 论 的 儿 何 意 义 可 参 见 囹 5-7-

形 沥 沥 tn 述 林
E

0

E

0 2 a 光

E O 扬 2

E
E 在该邻域里可由下面线
E

宋 心 矩 一 a
卡 江5 [ 沥
c
吴 5 江应 才
E 沥
E 命 巳 匹
Page-486
经
E
[
区
意
国
蟹
梁
国
[
区
E
E
团
E
Page-487
228 i

朐 东 仪 u y 5
Ecouy s i 应 招 3 李
E 力

园
E
沥 扬
E 人
i
医
E 才
E 汪

( 双 曲 周 期 转 e) 有 一 条 横 截 同 宿 转 :

(2) 集 合 Bc 的 闭 包 B 是 一 个 包 含 s 一 0 的 Cantor 集 , 且 满 足
yogpz 吊 浩 s 途 2

E 述 一 余 21
道 的 双 曲 不 变 集 } ,
E

E 半m(〈0<s<叶s 途 2
D 巳

这 里 _m(“) 表 示 RI 上 的 集 合 (.) 的 Lebesgue 测 度 .

E b b
E 述 g p y 泓
uty c 林 e
E

医 e p 标 育 述 水 E 技
E 吴

E 林 s

10

定 理 3.5 的 证 明
Page-488
E 胡2

E

E 命 口

E 沥 林 圆 一 不
(3'1)所导出的从碲 E
E 沥 力

d 不 辽 江 园 3
E 2 15 顺 05 沥

这 里 a=A/A<0, 婶=_责bz′ E

【 芸
转 道 丿 连 结 点 加 与 点 , 由 方 程 的 解 对 初 值 的 依 赖 性 定 理 , 每 一 条
E 林东
宇 廷 明 小 l 育 0 朋
c 一
截 相 交 , 故 曲 线 WPr(oi) V 可 以 表 示 成
E 52 沥 一 = 人 罗 0 才 、

E 达
Page-489
E 踊 c
证 明 将 分 成 儿 步 进 行

团 腾 l

医 林 林 仪 达 园 技
E 林 东
参 数 5 的 有 限 光 滑 坂 标 卡 (x,o), 使 得

【69 0 途 胡 2

(2) {Goo|jx|,|a| 妇 3 人 (@), 闭 辅 e) 的 Poinears 晖
射 在 UCQ) 上 有 形 式

E n t 2 有
这 里 0 一 XCe) 心 1<<RCej
E 林 572 东 沥
E
E 2 一 人 c
和

E
E 李
E

p

因 为 鞍 点 c 非 共 振 , 故 在 c 的 一 个 邻 域 仁 中 , 可 以 取 一 个 依
E 述
E
并 且 向 量 场 族 Xe 有 形 步
E
河
E
E 从喊<唰翰<0<″胤 E
E l),
Page-490
220 E 摄 t

EEok t 沥 s 河 木 洁 技 5
螺 线

【 0 沥
上 述 曲 线 属 于 何 扬 r 的 不 穗 定 流 形 WP“(o) 与 平 面 5 的 交 . 下 面 将
通 过 验 证 曲 线 (3. 2) 与 曲 线 P“CotfS: z 一 0) 有 横 戳 相 交 点 来 证 ˇ
明 定 理 3.1. 令 F 一 (Fo,Fy), 则 我 们 只 需 诛 明 方 程

y( 友 (8(a), 驯 , 丁 (b(a),z)) 一 0 医

a

国 史 国 一

园 E ,

那 么 (3. 3) 可 写 成
E d 2 述 3

园 ,
E E 林 E 刑

E
E 李 玟 人 应 玟 圭 玟 人 河
E
E 江 2 仪 才 步 2 河
的 函 数 加 上 一 个 小 重 . 因 此 (3. 4 有 无 穷 多 个 根 s 一 0. 下 面 证 明
E 江 应

u 水

0

这 里 当 & 一 oo 时 ,o(1) 一 0. 另 一 方 面
E A 2 育 圆 林 6 育 切
_"′ (z′)I 0 亩 7 5 + _苏_ F 万 E

团 _ E 江 述 途
【 羞)〕 玟 沥 伟 吴 23 丞通

罡
Page-491
第 五 章 空 间 中 双 曲 数 点 的 间 宿 分 盆

E

E t动
E

而 转 道 与 7+ 相 交 于 丿 点

吴 加 月 动 二 (Cyom 二 Ooo0,|y| 一 1

E y 技

T 有 形 式

【n e t 3 月
E

团 吴 1C 助 吴 刹
D ″(s)〈 ″〈s)_m

25 河 u

Epotauas tests 芸 e 河 沥 3 沥 北 砂 河
Eneti i p
0 t 2 00 吊 3 罡 03
由 假 设 (3), 不 穗 定 流 形 W7“(ot) 与 稳 定 流 形 WP“(oo) 泰 7 模 截 相 交 ,
E

[

G

t

Epoeutusduesl e 育 0
E 描
E t 23
E 林技
E
Page-492
医 胡 e

5 兽土z…『】(肃 E 6

上 面 最 后 一 个 等 式 是 由 (3. 4 和 (3. 5) 推 出 的 , 定 理 证 毕 , 〖

E

E 东 志 2 u s 怡 a 租 沥 河
E t 东 5 i 丞 志 朐

E 国 英
E
E 怀 0

源 万 水 P。擅黜崖 E

医 沥 a 兰 一 公 i
邹 基 可 以 定 义 闭 轨 口 的 Poincare 映 射 九 因 为 @ 点 是 一 维 晔 射 P 的
E 00 招
E

E

0

0 林

E 4 芒

[
E 沥
E
国
E
玟
令
E
Page-493
E 脱 p
游 草 环 对 应 情 泓 do<<0.

E yy

假 设 (6) 的 确 切 含 意 是

E
张 [n

E

E y 剧

F <0 时 结 论 的 证 明
对 e<0, 我 们 有
E 巴

Ere 2 颐仪′ E 苔歹z′

a

国 s+票燮z″十誓剽= E 吊 265 李 人

E
E airpsueetate yy E 河 春
s
所 有 的 执 道 都 将 离 开 A 的 邻 域

G 结 论 (1) 的 证 明

现 在 证 明 , 当 e>0 时 , 闭 执 ci(e) 有 横 截 同 宝 转 . 这 一 结 论 从 图
沥
E 志 改 s 江 仪
E 河
的 在 o 处 与 丶 , 仁 的 特 征 方 向 相 切 的 不 资 流 形 ) 在 7。 上 的 任 意
处 相 切 , 故 由 假 设 (43 的 (a), 它 与 流 形 WV“Co 泪 7 模 截 相 交 . 另 一
E pn 李 一
n
E
Page-494
第 五 蜕 空 间 中 双 曲 粤 炎 的 间 宿 分 盆

E 技
E
医 沥 L
E 沥 2 技

E

国 t

由 假 设 (43, 我 们 可 以 在 鞍 炒 % 的 一 个 邻 域 外 中 选 取 一 个 C!
E 河 t 、

0 沥

2 向 量 场 X 由 下 列 线 性 系 统 给 出

E
{仪 E 林

5

E

E

沥

E 沥 22 技 林 江 一 “
E

E
Euztse 河 E 唐 芸
有 形 式
[

E

C 卫 到 广 的 映 射

E
E 沥 命 林 沥 东
Page-495
空 间 中 双 尔 敏 炉 的 闭 宿 分 吴

E 国 L t

为 了 证 明 结 论 (2) , 我 们 需 要 下 面 的
引 理 3.7 存 在 正 规 同 宿 分 岑 值 序 列 哎 和 正 规 环 分 岔 值 序 列

E

a

E k 一 p 2
E l an
d
Page-496
E 脱 y

Egoam 河 y
【 a 0 2 673
由 假 设 (3) 流 形 P*(ou) 与 W“(oo) 沿 . 朱截 相 交 , 故 有

E

面 a

囹 此 , 可 选 联 充 分 小 , 使 得 .
E 林 朋

a

颜 e

[ 一 林 许
E 木
技 _ _

E
E _

闵 0 E

[

浩 ′)姜0,阊 [

曦>0和碗<o分别对应非游荡环和游荡环′

E 膏 e 江 许 沥
令
砺 一 G 人 H Kn 一 P eK8 口 丞 n
E d 吊 洁 玟
〈愫”:F”配岫 匹
令
@ 二 { E D|Rral 十 切 一 5 丢 不 叶 3G 一 动 j
Page-497
224 l

则 有
[ 吴 2

令
E
则 映 射 4 在 区 域 DY U DY 上 是 马 蹄 映 射 , 见 图 5-8-

E 林 国 胺
E
E
E s2
b 沥
[6 t 32

因 为 (&,EQn+a, 故
大 [ 沥

0 =壬

E
Page-498
[ 贺 c

此 , 曲 线 Ci 可 表 示 为
〉 [ 国 A 吊 502 达 ( C3.19)
标 r 罪
技
A 《C3.20)
因 此 , 如 果
吊 2
野
巳 A 友 A t 河
[ 述 )
E 沥
E A z 不 园
颂 水 沥
噩町徇 1_皇v懈〉

p

E (E)‖〈e〉十翌」夏(罄)'″′ (E)十誓〉 〕
E , 【
E 国 述 25 顺 朐 吊 a 标
y
2 团
E Y 5

- [
E 胡

23seet 5
的 重 要 附 注
E 怀 政 仪
Page-499
E 浩 2

E E
E

E

E
E E

命 砌 ) 门
明
玄剔<破<2j腥′y 对 一 余 1

这 与 (3. 11) 一 起 推 出
4

(

现 在 我 们 计 算 雅 可 比 阵 D4. 由 链 式 法 则 有
E 沥 3

孕
E [0

E

E

y

E
。 潺〕 e 达 一

园 东 林
E 【(z抛 +叮抛%
Page-500
E 第 五 定 空 间 中 双 兽 敲 点 的 名 宪 分 岑

Euatnsy 命 t 2 吊 河
e 园 s 切
E L 0 芸 河 吊 林命 |
速 度 接 近 及 离 开 曲 线 Pef15.

p gr 浩
国
D

b 余 林22 明 Iea3y

这 里 (0,0) 二 0, 四 目 哪>0. E (E))n
沥 伟 林 一

E e E
医 技

5 一 ′l〈仅辜聂yE) 0
令 (oo,0) 是 点 兮 门 7 的 土 标 , 我 们 对 方 程 (3. Z们庄黄(y试黯)_
E y

z=互(汊,s)歹零嬗屿赋 E
我 们 断 言 , 函 数 东 有 下 面 的 性 质
Page-501
226 d

E 江
maxtlDRF[IDPF-i a 1BM-| 一 , (8.13)
E 一 沥

E 东
max(|4| “ | 取 M 1D|,14 一 BMTD|,
c [ 浩 5
E 水
E 途 沥 沥 d 丿 厂 2 余 技 沥 0 办 育 3
E 告蔚一婶(二菖…厂】 z箐 E 咿彗 a
1 沥
国 5
E
E 江 当鹫+o时 5

M [ 一 肉 RmFTmar-ece 驿 土 心 骆 0 一 o,

t 夏_离一-|(一)一】(一)| 一 0.

E
林李
D ,
[ 一 木 标
E 达

1
P 雇_″〔 E 谚〕 崛翼 。 】肯互血+”

(1 +盘_市湟…」+…_〉玮0y 门
园 技 沥 沥 圆
Page-502
E

为 了 得 到 (3. 28) 中 后 两 个 估 计 , 我 们 对 (3. 25) 分 别 关 于 3 和 求
医

E E
J
i

伟 韶z婶丨nz 一 萝誓(瞰一l

医 沥 扬 E E
E
0 吴 3 [ 沥
E 才
E 江 3 育
医

E

59 F E

E E
哥O(s『′〉 D
令 Cn 一 P-oCS, 则 曲 线 C,n 上 的 点 (ovo7 满 尸

azcynAlucea E
因 为 曲 线 C.n 腾 于 闭 轨 ri() 的 稳 定 流 形 W“(o(6)), 故 若 点
E 沥 3 国
Page-503
236 第 二 章 宇 间 中 双 幼 敌 炳 的 闯 分 盆

E 技 b 丶 林 江
林
哥 EAc uAC1IC 7
令 <<1 是 一 个 小 正 数 , 滔 足

万 E
E
E

](1 溃)浑″[

日

工 〕(l十谬) E 技

E
分 大 时 有

E 【
E 沥
证 明 史 是 单 根 , 即

0 沥

〔″雇(氨)一“「1 业王璧v八“庐) O 切 〔″互(E)″_】 E

E P

E 河 _ 李

十 X ) E E “=又(罄)_″(星)yE=s二
E

E P E
n 兰 园

由 (3. 33) 可 知 , 如 果 8S1, 则 有
D
0 0 E 吴 s 辜二

医 一 口

i
Page-504
不 3 环 的 分 盆

颠 L

沥
图
E 林 7

【 关 许

[ 怀 沥 c 技 胡

《3) 边 界 条 件 ,4aeDf 门 D? 一 史 ,ADFM13,D9 一 奶 心 j 一 1,2.

a
盘的一部分,而它的上卞两边分别是Q″+z上下边在映射4下的原
象 ; 因 而 是 18 水 平 曲 线 , 故 条 件 (1) 成 立 , 其 次 , 相 交 条 件 不 难 由 述
E 东 E 伟 行

【 述

交 , 故 49.DFTIDY 一 亿 . 另 一 方 面 , 注 意 到

E 22 沥
E 训 沥 扬 达 江 王 月

E

E |

游 茆 环

4
所 发 生 的 分 岗 . 设 光 滑 向 最 场 X 有 一 个 游 荧 环 4, 满 尼 假 设 (0) 一
E 江 林 n 仪
和 闭 轨 u 是 双 曲 的 , 故 向 量 场 Xe 在 ca,oi 及 软 道 4 附 近 有 双 曲 奇
沥
E 7 朐

定 义 3.4 称 参 数 值 e 为 向 量 场 族 Xe 的 同 宣 分 岔 值 ( 环 分 岔
Page-505
E 源 s 237

E |庭(s)一″s_霹撞 』=g

_雇(辱) 园

“_X(鲁〉″″(E) E

0
严(

″ 一 I 卯 1
E 一 P E
〈1)[雇(0) 一 1 凌)_膘互(″(o) E
二 0 [ RKo) 一 薯…锦」〕″ 一 0
E
[ el t
芸 工 河 口 一

巳 A 0(l)燮羞' [ 沥
j 28 技 公 扬 知 汀 江
E
E 标
E 仪 0 河 达 0
Eudttucsucayun y

E 0(1)灭(莓)_″(铭)羞.

[
一 OCDR(s-at 加 n
E

i
E e y
E
Page-506
228 i

朐 东 仪 u y 5
Ecouy s i 应 招 3 李
E 力

园
E
沥 扬
E 人
i
医
E 才
E 汪

( 双 曲 周 期 转 e) 有 一 条 横 截 同 宿 转 :

(2) 集 合 Bc 的 闭 包 B 是 一 个 包 含 s 一 0 的 Cantor 集 , 且 满 足
yogpz 吊 浩 s 途 2

E 述 一 余 21
道 的 双 曲 不 变 集 } ,
E

E 半m(〈0<s<叶s 途 2
D 巳

这 里 _m(“) 表 示 RI 上 的 集 合 (.) 的 Lebesgue 测 度 .

E b b
E 述 g p y 泓
uty c 林 e
E

医 e p 标 育 述 水 E 技
E 吴

E 林 s

10

定 理 3.5 的 证 明
Page-507
第 五 章 空 间 中 双 曲 鞍 灰 阡 同 街分 动

E 赖 b 62p

由 引 理 3.7, 集 合 Bw 和 Be 非 空 . 对 任 意 e,E Bevt0} 由 假 设
〈3) 和 结 论 (1), 流 形 IP*Coo(so) 和 IPr(oCe)) 与 流 形 W*(ou(ey)》
E c e 浩 水
E 东
明 B 是 Cantor 集 . 为 此 只 霁 证 明 它 不 含 任 何 开 区 间 . 令 s,E B,
则 系 统 丿 的 奇 点 %(e) 有 一 条 咖 宿 转 . 因 此 , 对 e 的 紫 一 侧 e, 附 近
E
应
E

了 结 论 (3) 的 证 明

ENC2

2C:
一 0.
河 子

E 庆
医 吴 cn b 招

到 区 域 Do. 令 Ku 表 示 G 与 C$ 之 间 的 区 域 ,
医 一 (Guoo) P 乐 一 付 GusJ, 0 丿 4 一 酊 .
令

EstcR2 一 吊 一 李 李 胡
巳
E 一 3
s 吴 育
医 不 t 园

r 吴
则 由 口 , 咤 的 渐 近 表 达 式 可 得 a 一 . 令 Dn 表 示 区 间 (8n, 气 . 对
Page-508
E 踊 c
证 明 将 分 成 儿 步 进 行

团 腾 l

医 林 林 仪 达 园 技
E 林 东
参 数 5 的 有 限 光 滑 坂 标 卡 (x,o), 使 得

【69 0 途 胡 2

(2) {Goo|jx|,|a| 妇 3 人 (@), 闭 辅 e) 的 Poinears 晖
射 在 UCQ) 上 有 形 式

E n t 2 有
这 里 0 一 XCe) 心 1<<RCej
E 林 572 东 沥
E
E 2 一 人 c
和

E
E 李
E

p

因 为 鞍 点 c 非 共 振 , 故 在 c 的 一 个 邻 域 仁 中 , 可 以 取 一 个 依
E 述
E
并 且 向 量 场 族 Xe 有 形 步
E
河
E
E 从喊<唰翰<0<″胤 E
E l),
Page-509
E 颖

于 eEUn_-1, 我 们 定 义 由 网 二 KoU Ln 刨 万 的 映 射 如 下 。
E 东

E

E

令
E
则 4 在 ZrUZn 上 看 上 去 像 一 个 马 踹 眺 射 . 参 见 图 5-10.

1 Eoale
E 一
沥 命 史 诊 2
不 二 三

7
s

E n u
E 才 d
E
<1, 它 就 满 足 (/avo) 锥 形 条 件 . 对 于 (uoo)E , 有
E zn
在 利 用 第 四 章 引 理 4. 1 来 验 证 4 源 足 锦 形 条 件 时 , 所 有 运 算
与 引 理 3. 3 的 证 明 中 完 全 一 样 , 从 那 里 可 以 看 出 我 们 只 需 验 证
Page-510
第 五 章 空 间 中 双 曲 数 点 的 间 宿 分 盆

E

E t动
E

而 转 道 与 7+ 相 交 于 丿 点

吴 加 月 动 二 (Cyom 二 Ooo0,|y| 一 1

E y 技

T 有 形 式

【n e t 3 月
E

团 吴 1C 助 吴 刹
D ″(s)〈 ″〈s)_m

25 河 u

Epotauas tests 芸 e 河 沥 3 沥 北 砂 河
Eneti i p
0 t 2 00 吊 3 罡 03
由 假 设 (3), 不 穗 定 流 形 W7“(ot) 与 稳 定 流 形 WP“(oo) 泰 7 模 截 相 交 ,
E

[

G

t

Epoeutusduesl e 育 0
E 描
E t 23
E 林技
E
Page-511
玟

[
E 东 不 沥 木
E 园

E 一
E
因 此
E 训

s
国 C羞一1雇(s)_″〔亘(0)(1一羞)(1+羞】〕一】
r
这 里 r 一 F(o30- 劳 Q+ 声 / FCe. 注 意 到
E
故 3. 40) 成 立 . 引 理 证 毕 , 【

s 浩 t
E =′gz4塞(乙; Uucopeomeei e

E Ln
E 述
E 河 i
E 达 n 吴 s 余 2n
E

E
E

E
c

E
Page-512
E 脱 p
游 草 环 对 应 情 泓 do<<0.

E yy

假 设 (6) 的 确 切 含 意 是

E
张 [n

E

E y 剧

F <0 时 结 论 的 证 明
对 e<0, 我 们 有
E 巴

Ere 2 颐仪′ E 苔歹z′

a

国 s+票燮z″十誓剽= E 吊 265 李 人

E
E airpsueetate yy E 河 春
s
所 有 的 执 道 都 将 离 开 A 的 邻 域

G 结 论 (1) 的 证 明

现 在 证 明 , 当 e>0 时 , 闭 执 ci(e) 有 横 截 同 宝 转 . 这 一 结 论 从 图
沥
E 志 改 s 江 仪
E 河
的 在 o 处 与 丶 , 仁 的 特 征 方 向 相 切 的 不 资 流 形 ) 在 7。 上 的 任 意
处 相 切 , 故 由 假 设 (43 的 (a), 它 与 流 形 WV“Co 泪 7 模 截 相 交 . 另 一
E pn 李 一
n
E
Page-513
E 河 e

[
湟够0′l+玄焊;

医 沥 4 林c 4 技丞
命 1
E 林
1 沥 沥

E 0(])亘(0)_u+圭】″,
故

E ()(1〉雇(o)一“+竟‖.
D

男 一 方 面
B

E

因 而

R3

-【量

Hm

E [〗 (′″)〕】+硐

EE

凸 (3. 43) 不 难 导 出 (3. 42). 定 理 证 毕 , 〖

E 述 a 不 一 c
的 敌 点 量 为 负 , 则 使 系 统 X 具 有 稳 定 周 期 转 的 参 数 值 集 是 一 个
边 界 包 含 原 点 的 开 集

E
邰 域 UCeo) , 使 得 对 <EU(e2vteo},Xe 有 一 个 稳 定 周 期 轨 . 另 一 方
E

E 刑 e 兰 s
论 也 成 立 , 读 者 可 参 见 [Sil2,3] 和 [ILJ

E
Page-514
空 间 中 双 尔 敏 炉 的 闭 宿 分 吴

E 国 L t

为 了 证 明 结 论 (2) , 我 们 需 要 下 面 的
引 理 3.7 存 在 正 规 同 宿 分 岑 值 序 列 哎 和 正 规 环 分 岔 值 序 列

E

a

E k 一 p 2
E l an
d
Page-515
[ 贺 c

此 , 曲 线 Ci 可 表 示 为
〉 [ 国 A 吊 502 达 ( C3.19)
标 r 罪
技
A 《C3.20)
因 此 , 如 果
吊 2
野
巳 A 友 A t 河
[ 述 )
E 沥
E A z 不 园
颂 水 沥
噩町徇 1_皇v懈〉

p

E (E)‖〈e〉十翌」夏(罄)'″′ (E)十誓〉 〕
E , 【
E 国 述 25 顺 朐 吊 a 标
y
2 团
E Y 5

- [
E 胡

23seet 5
的 重 要 附 注
E 怀 政 仪
Page-516
第 六 章 “ 实 二 次 单 峥 映 射 族 的 吸 引 子

从 60 年 代 兴 起 的 动 力 系 统 的 现 代 研 究 , 其 中 心 课 题 之 一 是 双
曲 理 论 . 一 个 系 统 ( 流 , 微 分 间 胚 , 胥 射 ) 是 双 曲 的 , 如 果 它 的 极 限
集 是 双 曲 的 , 即 极 限 集 中 所 有 转 道 的 Liapunov 指 数 一 致 非 0( 在 讨
论 洪 的 双 曲 性 时 , 不 考 虑 沿 流 方 向 的 Liapunow 指 数 , 它 为 0. 从 那
E 东
E 技 标
古 外 , 当 时 人 们 曾 相 信 动 力 系 统 基 本 上 是 由 双 曲 系 统 构 成 的 . 这 一
看 法 的 根 本 转 变 是 由 于 在 70 年 代 受 到 物 理 、 天 文 学 等 领 域 的 一 些
E 林技 人 育 y 育 沥 林55 尿 技y 命 i 招 s
Lorenz( 他 们 的 工 作 在 70 年 代 开 始 才 受 到 数 学 家 的 重 视 等 人 对
这 些 模 型 的 算 工 作 表 明 , 这 些 模 型 具 有 极 其 复 杂 的 动 力 学
行 为 , 它 们 似 乎 不 具 有 双 曲 结 构 , 相 反 它 们 应 当 履 于 非 双 曲 苑 畴

河
Jakobsonp 在 80 年 代 初 取 得 的 , 他 证 明 了 对 实 二 次 映 射 族 的 一 个
E 途 c 北 二 江 标
E
2
Jakobson 的 结 果 和 方 法 , 并 在 此 基 础 上 , 讨 论 了 Hnon 映 射 在 具 有
E t t
E 江 河 d t
D
E 沥
E 沥 E
对 菪 个 系 统 有 很 好 的 了 解 ( 例 如 ,Logistie 映 射 族 在 a 一 2 时 HHenon
Page-517
E 第 五 定 空 间 中 双 兽 敲 点 的 名 宪 分 岑

Euatnsy 命 t 2 吊 河
e 园 s 切
E L 0 芸 河 吊 林命 |
速 度 接 近 及 离 开 曲 线 Pef15.

p gr 浩
国
D

b 余 林22 明 Iea3y

这 里 (0,0) 二 0, 四 目 哪>0. E (E))n
沥 伟 林 一

E e E
医 技

5 一 ′l〈仅辜聂yE) 0
令 (oo,0) 是 点 兮 门 7 的 土 标 , 我 们 对 方 程 (3. Z们庄黄(y试黯)_
E y

z=互(汊,s)歹零嬗屿赋 E
我 们 断 言 , 函 数 东 有 下 面 的 性 质
Page-518
E

为 了 得 到 (3. 28) 中 后 两 个 估 计 , 我 们 对 (3. 25) 分 别 关 于 3 和 求
医

E E
J
i

伟 韶z婶丨nz 一 萝誓(瞰一l

医 沥 扬 E E
E
0 吴 3 [ 沥
E 才
E 江 3 育
医

E

59 F E

E E
哥O(s『′〉 D
令 Cn 一 P-oCS, 则 曲 线 C,n 上 的 点 (ovo7 满 尸

azcynAlucea E
因 为 曲 线 C.n 腾 于 闭 轨 ri() 的 稳 定 流 形 W“(o(6)), 故 若 点
E 沥 3 国
Page-519
0 243

E
E 技
E yy 吴 E
和 证 明 方 法 . 应 当 指 出 , 这 里 介 绍 的 证 明 思 想 在 当 前 这 一 方 向 的 研
究 中 是 十 分 重 要 的 .

E 沥 刑 心 余 c

江
t

小 不 园 i 江 园 c 浩 江
e

20 尿

0

005 仪 坂 政 E c d M

E 沥 技 t

【UUD 明 志 t 林 一 北 不

医 沥 怀 e 沥 许 玟 东 步 2 江
E 亚 招
b 学 述 玲 技 振
E ,有静噩荤阗/″P(仪)今z E

E

现 在 我 们 要 讨 论 的 问 题 ; 一 个 单 峰 昭 射 可 以 有 多 少 条 穗 定 的
E 兰 一 i 水 椿 中
玟
E

E 春 应 述

国 u 河 秉 扬 A
A 浩河 ′(′)〕 「
Page-520
236 第 二 章 宇 间 中 双 幼 敌 炳 的 闯 分 盆

E 技 b 丶 林 江
林
哥 EAc uAC1IC 7
令 <<1 是 一 个 小 正 数 , 滔 足

万 E
E
E

](1 溃)浑″[

日

工 〕(l十谬) E 技

E
分 大 时 有

E 【
E 沥
证 明 史 是 单 根 , 即

0 沥

〔″雇(氨)一“「1 业王璧v八“庐) O 切 〔″互(E)″_】 E

E P

E 河 _ 李

十 X ) E E “=又(罄)_″(星)yE=s二
E

E P E
n 兰 园

由 (3. 33) 可 知 , 如 果 8S1, 则 有
D
0 0 E 吴 s 辜二

医 一 口

i
Page-521
E 源 s 237

E |庭(s)一″s_霹撞 』=g

_雇(辱) 园

“_X(鲁〉″″(E) E

0
严(

″ 一 I 卯 1
E 一 P E
〈1)[雇(0) 一 1 凌)_膘互(″(o) E
二 0 [ RKo) 一 薯…锦」〕″ 一 0
E
[ el t
芸 工 河 口 一

巳 A 0(l)燮羞' [ 沥
j 28 技 公 扬 知 汀 江
E
E 标
E 仪 0 河 达 0
Eudttucsucayun y

E 0(1)灭(莓)_″(铭)羞.

[
一 OCDR(s-at 加 n
E

i
E e y
E
Page-522
244 E

E 不 医 伟

[ 沥 明 北

【 明 志 s

【 仪

0 李 伟 22

《Se 一 (0) 一 0.

E 林 M 国 育 u 沥 招 技t
E 李 a 余

Eos 国 圆 D 河 2 河
的 , 此 处 不 再 赘 述 , ,

2
Sg(z.

E 沥 技 e 伟 2
【 仪

3. | 一 | 在 ( 一 1,1 中 没 有 正 的 局 部 最 小 .

E
E 仪 d
医 芸 2 沥 伟
d
2( 天 士 1) 是 了 的 不 动 点 并 东 | 乜 (97| 一 1, 那 么 这 个 不 助 点 至 少
在 一 边 是 稳 定 的 .

沥 政 2 技
2 沥 王 一
E 途

如 若 不 然 , 令 g 一 户 并 设 有 无 穷 多 = E , 满 足 g(z) 一 z. 由
t 东
医 e 伟
E

5. 如 果 a 一 & 一 c 是 8 一 八 的 相 邻 不 动 点 , 并 且 在 区 间 [avc]
Page-523
第 五 章 空 间 中 双 曲 鞍 灰 阡 同 街分 动

E 赖 b 62p

由 引 理 3.7, 集 合 Bw 和 Be 非 空 . 对 任 意 e,E Bevt0} 由 假 设
〈3) 和 结 论 (1), 流 形 IP*Coo(so) 和 IPr(oCe)) 与 流 形 W*(ou(ey)》
E c e 浩 水
E 东
明 B 是 Cantor 集 . 为 此 只 霁 证 明 它 不 含 任 何 开 区 间 . 令 s,E B,
则 系 统 丿 的 奇 点 %(e) 有 一 条 咖 宿 转 . 因 此 , 对 e 的 紫 一 侧 e, 附 近
E
应
E

了 结 论 (3) 的 证 明

ENC2

2C:
一 0.
河 子

E 庆
医 吴 cn b 招

到 区 域 Do. 令 Ku 表 示 G 与 C$ 之 间 的 区 域 ,
医 一 (Guoo) P 乐 一 付 GusJ, 0 丿 4 一 酊 .
令

EstcR2 一 吊 一 李 李 胡
巳
E 一 3
s 吴 育
医 不 t 园

r 吴
则 由 口 , 咤 的 渐 近 表 达 式 可 得 a 一 . 令 Dn 表 示 区 间 (8n, 气 . 对
Page-524
E 颖

于 eEUn_-1, 我 们 定 义 由 网 二 KoU Ln 刨 万 的 映 射 如 下 。
E 东

E

E

令
E
则 4 在 ZrUZn 上 看 上 去 像 一 个 马 踹 眺 射 . 参 见 图 5-10.

1 Eoale
E 一
沥 命 史 诊 2
不 二 三

7
s

E n u
E 才 d
E
<1, 它 就 满 足 (/avo) 锥 形 条 件 . 对 于 (uoo)E , 有
E zn
在 利 用 第 四 章 引 理 4. 1 来 验 证 4 源 足 锦 形 条 件 时 , 所 有 运 算
与 引 理 3. 3 的 证 明 中 完 全 一 样 , 从 那 里 可 以 看 出 我 们 只 需 验 证
Page-525
t

E 李 林

E
训 林
E hean 怀 租

一
E 刑

E 标 东 荣
E 人 一
t 沥
E 芸
0

[

[

E 李

【
林
E

E
沥

木 一
2
招

【 d 月 仪 育 尘 E

E 伟
E
2
水
玟 2
E 5 王
Page-526
玟

[
E 东 不 沥 木
E 园

E 一
E
因 此
E 训

s
国 C羞一1雇(s)_″〔亘(0)(1一羞)(1+羞】〕一】
r
这 里 r 一 F(o30- 劳 Q+ 声 / FCe. 注 意 到
E
故 3. 40) 成 立 . 引 理 证 毕 , 【

s 浩 t
E =′gz4塞(乙; Uucopeomeei e

E Ln
E 述
E 河 i
E 达 n 吴 s 余 2n
E

E
E

E
c

E
Page-527
E 河 e

[
湟够0′l+玄焊;

医 沥 4 林c 4 技丞
命 1
E 林
1 沥 沥

E 0(])亘(0)_u+圭】″,
故

E ()(1〉雇(o)一“+竟‖.
D

男 一 方 面
B

E

因 而

R3

-【量

Hm

E [〗 (′″)〕】+硐

EE

凸 (3. 43) 不 难 导 出 (3. 42). 定 理 证 毕 , 〖

E 述 a 不 一 c
的 敌 点 量 为 负 , 则 使 系 统 X 具 有 稳 定 周 期 转 的 参 数 值 集 是 一 个
边 界 包 含 原 点 的 开 集

E
邰 域 UCeo) , 使 得 对 <EU(e2vteo},Xe 有 一 个 稳 定 周 期 轨 . 另 一 方
E

E 刑 e 兰 s
论 也 成 立 , 读 者 可 参 见 [Sil2,3] 和 [ILJ

E
Page-528
246 第 六 章 宗 二 次 单 峰 胸 射 旌 的 吟 引 孔

E 关 人 一 兰 e 招
乙 , 使 得 8 ( 一 1. 如 榔 有 y E (d,z), 使 得 8 (9》 一 0, 则 利 用 性
质 6, 吞 则 由 情 况 (iit) , 我 们 完 成 了 性 质 7 的 证 明 , 并 因 此 证 明 了 定
E

定 理 1.3 有 以 下 儿 个 推 论 .

E 江 园
期 点 , 加 上 在 区 间 [ 一 1,f(1)] 中 的 一 个 可 能 的 稳 定 不 动 点

E 园 2
玟
园 达 L 8 芸 s
a 口 i
如 果 是 第 一 种 情 况 ,z 炜 引 [0,z] 并 因 此 吸 引 0 点 ! 如 果 是 第 二 种
E
论 证 .

如 果 有 一 个 周 期 p 之 2 的 穗 定 周 期 轨 , 那 么 用 完 全 类 似 的 讨
E
引 0 点 或 者 它 吸 引 1. 于 是 , 我 们 证 明 了 予 在 J( 丿 中 至 多 有 一 条 稿
定 周 期 转 . 另 外 , 从 上 面 的 证 明 我 们 也 可 以 看 到 , 在 ( 户 中 没 有 稿
E

现 在 我 们 考 虑 了 的 稳 定 周 期 软 , 它 不 盼 引 0 或 1. 由 定 理 1. 3,
E 不 仪

下 面 我 们 证 明 , 这 样 的 穗 定 周 期 软 是 JC+ 一 ( 一 1,A(1) 中
的 一 个 穗 定 不 动 点 . 如 同 我 们 已 经 看 到 的 , 如 果 这 条 转 道 有 一 个 点
在 J 丿 中 , 那 么 整 条 执 道 在 ( 丿 中 并 豚 引 0 和 1 因 此 , 这 桦 的 轨
道 必 在 JCA5+ 中 . 由 性 质 3, 丫 在 JC/)+ 中 至 多 有 两 个 不 动 点 , 因 为
了 在 [ 一 1,0] 中 至 多 有 两 个 不 动 点 . 如 果 丨 在 ( 乙 + 中 没 有 不 动
P 技 i e 唐 八 i
没 有 周 期 炭 ! 如 果 丁 在 T 一 中 只 有 一 个 不 动 焯 , 当 人 (z) 丿 1 时 ,
E
Page-529
第 六 章 “ 实 二 次 单 峥 映 射 族 的 吸 引 子

从 60 年 代 兴 起 的 动 力 系 统 的 现 代 研 究 , 其 中 心 课 题 之 一 是 双
曲 理 论 . 一 个 系 统 ( 流 , 微 分 间 胚 , 胥 射 ) 是 双 曲 的 , 如 果 它 的 极 限
集 是 双 曲 的 , 即 极 限 集 中 所 有 转 道 的 Liapunov 指 数 一 致 非 0( 在 讨
论 洪 的 双 曲 性 时 , 不 考 虑 沿 流 方 向 的 Liapunow 指 数 , 它 为 0. 从 那
E 东
E 技 标
古 外 , 当 时 人 们 曾 相 信 动 力 系 统 基 本 上 是 由 双 曲 系 统 构 成 的 . 这 一
看 法 的 根 本 转 变 是 由 于 在 70 年 代 受 到 物 理 、 天 文 学 等 领 域 的 一 些
E 林技 人 育 y 育 沥 林55 尿 技y 命 i 招 s
Lorenz( 他 们 的 工 作 在 70 年 代 开 始 才 受 到 数 学 家 的 重 视 等 人 对
这 些 模 型 的 算 工 作 表 明 , 这 些 模 型 具 有 极 其 复 杂 的 动 力 学
行 为 , 它 们 似 乎 不 具 有 双 曲 结 构 , 相 反 它 们 应 当 履 于 非 双 曲 苑 畴

河
Jakobsonp 在 80 年 代 初 取 得 的 , 他 证 明 了 对 实 二 次 映 射 族 的 一 个
E 途 c 北 二 江 标
E
2
Jakobson 的 结 果 和 方 法 , 并 在 此 基 础 上 , 讨 论 了 Hnon 映 射 在 具 有
E t t
E 江 河 d t
D
E 沥
E 沥 E
对 菪 个 系 统 有 很 好 的 了 解 ( 例 如 ,Logistie 映 射 族 在 a 一 2 时 HHenon
Page-530
不 1 关 孔 单 峰 咏 射 租 定 闸 期 点 的 存 在 性 247

c
一 1, 团 此 JCD+ 中 没 有 其 它 的 稳 定 周 期 轨 . 最 后 , 假 设 广 在 J 一
E 口 沥 3
E

E 育
E

E 河 u tn i 技 2 才
i

推 论 1. 5 可 以 看 成 是 推 论 1.4 的 部 分 证 明 过 程 , 用 它 可 以 断
E

推 论 ! 6 “ 存 在 没 有 穗 定 周 期 转 的 5 单 峰 映 射 .

D 国 圆 2
Neumann 在 1947 年 给 出 的 . 容 易 验 证 , 丫 是 5 单 峰 晏 射 . 经 0 点 的
扬
弓 林 不 p en 林 y 1
个 点 . 因 此 , 它 们 不 能 被 其 它 周 期 扬 道 妓 引 . 但 是 , 由 于 一 ( 一 1) 一
E 沥 沥 22 标

附 注 1. 7 “ 在 本 节 我 们 讨 论 了 5 单 峻 眸 射 的 穗 定 周 期 轨 的 存
Eouaogs i 浩 园 东 i 江许 永 途 沥 3
沥 i 技
el s 河 沥 招 圆
E t 园
E
e
e 芬
E
E 河 技 85 深
Page-531
E

e 技 芸 一 一

玟

( 一 8,8) 的 方 式 , 其 中 8 一 exp( 一 V e ) 特 别 要 讨 论 它 们 是 怎 样
带 近 原 点 的 , 以 及 靠 近 原 点 的 速 度 . 而 FCz,a) 的 送 代 的 一 些 基 本
性 质 对 于 理 解 这 些 问 题 起 着 重 要 作 用 . 应 当 指 出 , 尽 管 我 们 只 对 这
一 特 殊 眨 射 展 开 讨 论 , 但 是 这 种 以 Jakobson 开 始 ,Benedicks 和
E 育 沥 林 E
g =l′器J>″/″,

巳 沥 22n 林庆 不
E 厂 不 余

E 东 莲 育 2
如 果 加 E [ 一 L,1J 和 满 趸

3 才 52 明
E
e 丞 3 河
E
E C2. 1

n 河 =sin昔廖. E
为

E 熹m恤0 b 奎廖).
团
0s 认 史

E 育 2
4 C。sz音没苛_〔s_n 昔核)[1 歹)

i
Page-532
0 243

E
E 技
E yy 吴 E
和 证 明 方 法 . 应 当 指 出 , 这 里 介 绍 的 证 明 思 想 在 当 前 这 一 方 向 的 研
究 中 是 十 分 重 要 的 .

E 沥 刑 心 余 c

江
t

小 不 园 i 江 园 c 浩 江
e

20 尿

0

005 仪 坂 政 E c d M

E 沥 技 t

【UUD 明 志 t 林 一 北 不

医 沥 怀 e 沥 许 玟 东 步 2 江
E 亚 招
b 学 述 玲 技 振
E ,有静噩荤阗/″P(仪)今z E

E

现 在 我 们 要 讨 论 的 问 题 ; 一 个 单 峰 昭 射 可 以 有 多 少 条 穗 定 的
E 兰 一 i 水 椿 中
玟
E

E 春 应 述

国 u 河 秉 扬 A
A 浩河 ′(′)〕 「
Page-533
E 浩 u E

E 育
园 2 育 2 江 口

d P′(枳′)T[a′雇〈″”瓜) 薹俨′【(%)II(_ 2a87,C2. 2
E

E <1 E 江 ,
团 因此cos告庞尝0 E 育

日
Y1 一 办

沥 …|<缪以及_旺沪蝴)卜 春

I
e 二
引 理 2.2 “ 存 在 常 数 7 丿 0 和 A(8》 又 0, 当 m 充 分 靠 近 2 时 ,
E
t
10 沥 c
(2) log13cF(z,a)| 交

E 一 e 一 青 , 而 东
E 沥
0
[

0

玟

一 (2a327( 一 52 标 切 ( 一 4522 纳 ] ( 一 270

人
t
[ 水
n
Page-534
244 E

E 不 医 伟

[ 沥 明 北

【 明 志 s

【 仪

0 李 伟 22

《Se 一 (0) 一 0.

E 林 M 国 育 u 沥 招 技t
E 李 a 余

Eos 国 圆 D 河 2 河
的 , 此 处 不 再 赘 述 , ,

2
Sg(z.

E 沥 技 e 伟 2
【 仪

3. | 一 | 在 ( 一 1,1 中 没 有 正 的 局 部 最 小 .

E
E 仪 d
医 芸 2 沥 伟
d
2( 天 士 1) 是 了 的 不 动 点 并 东 | 乜 (97| 一 1, 那 么 这 个 不 助 点 至 少
在 一 边 是 稳 定 的 .

沥 政 2 技
2 沥 王 一
E 途

如 若 不 然 , 令 g 一 户 并 设 有 无 穷 多 = E , 满 足 g(z) 一 z. 由
t 东
医 e 伟
E

5. 如 果 a 一 & 一 c 是 8 一 八 的 相 邻 不 动 点 , 并 且 在 区 间 [avc]
Page-535
第 六 竟 宗 二 次 单 峰 哈 尊 旋 的 媒 引 孔

最 后 , 取 &(8) 一 (log2)71 E
东
区 间 (ao,23 中 存 在 一 个 小 区 间 4, 使 得 映 射 k : 心 一 工 是 一 一 对
应 的 , 并 东 保 持 指 数 扩 张 .
E 不 技 宏
E
s 木 一 用
[ t d
《383 |8 砂 (La)| 交 (1. 9 一 史 二 2 一 工
E 园
[ 一
E 改 才 月

E 告 [ 河 河 述 鲁U E 命

E 吴 述 沥 一 命
E

E E
「蒜_^鬓_脓龟盂<一毓

可 归 纳 地 证 明 a 一 史 (a 是 单 调 下 降 的 . 由 这 些 事 实 , 结 论 (1) 和
〈2) 得 证 . 这 是 因 为 对 上 述 的 8 和 丶 , 我 们 逃 出 包 含 2 的 小 参 数 区
[oteyaE 连 驱 述 沥 t
t 林
一 , 其 中 B 是 。 的 左 端 点 - 注 意 到 Sm (a),a E 。 是 单 调 下 陆
的 , 所 以 存 在 心 一 4o, 得 当 eE 心 时 ,Sn :4 一 厂 是 一 一 映 射 . 最
后 我 们 证 明 指 数 扩 张 伯 . 图 为 ˇ

a
Page-536
一 2 PCrva] 一 1 一 a 的 葛 本 性 质 E

E 园 理
E

园
E 江江 河 0 盼

E 技
和
E

E

Enal
E 胺

并 归 纳 地 得 到
E
7 E
[ 壶墓u 十

0

s
[ 命
引 理 2.4 “ 存 在 充 分 小 的 > 0 和 m 一 2,m 帝 近 2, 如 果
0 技
[ 扬
0

s
2

E 园
E

M
[ 不 1
I 伟 y

E

z2
Page-536
t

E 李 林

E
训 林
E hean 怀 租

一
E 刑

E 标 东 荣
E 人 一
t 沥
E 芸
0

[

[

E 李

【
林
E

E
沥

木 一
2
招

【 d 月 仪 育 尘 E

E 伟
E
2
水
玟 2
E 5 王
Page-538
第 六 章 “ 实 二 次 单 峰 胥 寺 族 的 哉 引 二

[ 吊 ]
0

n _

E

E [

E
l一酝韧廷卜十

E 一

[
其 它 情 况 可 类 似 证 明 .

5 九 东 5
E >仍羞扁>仪 E
5

E 朐

E 命 谅
D 之 0. 所 以 1 十
医
Z
E 沥 圭a′庐 5 责exp基″气 园
E

E B
E
最 后 , 我 们 来 归 纳 地 证 明 引 理 结 论 . 一 7 时 无 需 证 明 , 设 7 女 ? 一
E

E

[
忐<‖ l 命

a 「
回 为
Page-539
246 第 六 章 宗 二 次 单 峰 胸 射 旌 的 吟 引 孔

E 关 人 一 兰 e 招
乙 , 使 得 8 ( 一 1. 如 榔 有 y E (d,z), 使 得 8 (9》 一 0, 则 利 用 性
质 6, 吞 则 由 情 况 (iit) , 我 们 完 成 了 性 质 7 的 证 明 , 并 因 此 证 明 了 定
E

定 理 1.3 有 以 下 儿 个 推 论 .

E 江 园
期 点 , 加 上 在 区 间 [ 一 1,f(1)] 中 的 一 个 可 能 的 稳 定 不 动 点

E 园 2
玟
园 达 L 8 芸 s
a 口 i
如 果 是 第 一 种 情 况 ,z 炜 引 [0,z] 并 因 此 吸 引 0 点 ! 如 果 是 第 二 种
E
论 证 .

如 果 有 一 个 周 期 p 之 2 的 穗 定 周 期 轨 , 那 么 用 完 全 类 似 的 讨
E
引 0 点 或 者 它 吸 引 1. 于 是 , 我 们 证 明 了 予 在 J( 丿 中 至 多 有 一 条 稿
定 周 期 转 . 另 外 , 从 上 面 的 证 明 我 们 也 可 以 看 到 , 在 ( 户 中 没 有 稿
E

现 在 我 们 考 虑 了 的 稳 定 周 期 软 , 它 不 盼 引 0 或 1. 由 定 理 1. 3,
E 不 仪

下 面 我 们 证 明 , 这 样 的 穗 定 周 期 软 是 JC+ 一 ( 一 1,A(1) 中
的 一 个 穗 定 不 动 点 . 如 同 我 们 已 经 看 到 的 , 如 果 这 条 转 道 有 一 个 点
在 J 丿 中 , 那 么 整 条 执 道 在 ( 丿 中 并 豚 引 0 和 1 因 此 , 这 桦 的 轨
道 必 在 JCA5+ 中 . 由 性 质 3, 丫 在 JC/)+ 中 至 多 有 两 个 不 动 点 , 因 为
了 在 [ 一 1,0] 中 至 多 有 两 个 不 动 点 . 如 果 丨 在 ( 乙 + 中 没 有 不 动
P 技 i e 唐 八 i
没 有 周 期 炭 ! 如 果 丁 在 T 一 中 只 有 一 个 不 动 焯 , 当 人 (z) 丿 1 时 ,
E
Page-540
医 浩 a

E 忡】丨
[
E
E
[
a巍痿`′+I 12
萜廷善洲<m 1

E
阮
i
根 据 / 的 送 代 的 这 种 特 性 , 分 别 讨 论 了 当 参 数 攀 近 2 时 ,/a 进 入
乙 前 和 走 出 习 后 , 在 厂 外 的 扩 张 性 质 . 引 理 2.3 将 作 为 我 们 归 络
地 得 到 正 Lebesgue 参 数 集 4.. 的 基 础 . 我 们 将 FCz,a) 看 成 二 元 函
数 , 引 理 2.4 说 明 FCz,a) 的 送 代 对 <= 和 a 是 等 度 增 长 的 , 这 一 结 果
的 好 处 在 于 对 参 数 或 对 变 量 的 估 算 可 以 相 互 转 化 . 在 $ 3 定 理 3.1
E e 胡
引 理 的 意 义 还 不 仅 仅 如 化 - 对 于 一 舫 的 眺 射 族 FCz,a), 我 们 可 以
玲
E 沥
结 中 闸 述 .

(l十

$ 3 (rva) 不 存 在 穗 定 屉 期 轨 问 题

a
e
n

伟 [
E

2 沥 林 一
Page-541
E E 河 s2

而 在 [F(L,a),1] 中 至 多 有 一 条 穗 定 周 期 豹 ; 并 且 偎 如 x 一 0 没 有
被 吴 引 到 一 条 周 朝 轨 , 那 么 F(z,a) 没 有 穗 定 周 期 轨 . 于 恬 , 定 理
3.1 可 以 从 下 面 的 定 理 3.2 得 到 .

定 理 3.2 存 在 4 CC (0,2),4. 具 有 正 Lebesgue 测 度 , 以 及
二

[ 刑 国 仪 命

E 图胡 a 扬 巳
E

A 我 们 根 据 之 (0,a) 不 断 返 回 到 “ 的 特 点 , 讨 论 返 回 的 形 式 .
与 此 同 时 用 自 由 返 回 概 念 , 归 纳 地 将 参 数 区 间 4 分 类 .

由 引 理 2. 3, 存 在 一 个 最 小 整 数 za(>> N) , 使 得

E
是 丿 到 T“ 上 的 1 一 1 映 射 . 称 m 是 映 射 的 首 次 自 由 返 回 的 指 标 ,
相 应 的 参 数 首 欣 分 类 为
[ t

h 途
E 二 d 八 厂 5
使 得 0 点 是 F(z,eo 的 周 期 为 z 的 超 稳 定 周 期 点 . 为 了 保 证 定 理
匹
数 区 间 4 中 , 我 们 全 部 排 除 了 使 / 可 能 有 小 于 等 于 m 的 稳 定 周
期 转 的 参 数

设 口 是 第 & 次 分 类 的 任 意 一 个 构 成 区 闽 , 丿 1. 下 面 我 们 定 义
第 十 1 欣 自 由 返 回 以 及 相 应 的 分 类 ˇ

E 2 技
t 伟

[ 一 _占′(″)_
成立的最大值.更确切地讲,对所有窿…m和所有″仨 a
Page-542
不 1 关 孔 单 峰 咏 射 租 定 闸 期 点 的 存 在 性 247

c
一 1, 团 此 JCD+ 中 没 有 其 它 的 稳 定 周 期 轨 . 最 后 , 假 设 广 在 J 一
E 口 沥 3
E

E 育
E

E 河 u tn i 技 2 才
i

推 论 1. 5 可 以 看 成 是 推 论 1.4 的 部 分 证 明 过 程 , 用 它 可 以 断
E

推 论 ! 6 “ 存 在 没 有 穗 定 周 期 转 的 5 单 峰 映 射 .

D 国 圆 2
Neumann 在 1947 年 给 出 的 . 容 易 验 证 , 丫 是 5 单 峰 晏 射 . 经 0 点 的
扬
弓 林 不 p en 林 y 1
个 点 . 因 此 , 它 们 不 能 被 其 它 周 期 扬 道 妓 引 . 但 是 , 由 于 一 ( 一 1) 一
E 沥 沥 22 标

附 注 1. 7 “ 在 本 节 我 们 讨 论 了 5 单 峻 眸 射 的 穗 定 周 期 轨 的 存
Eouaogs i 浩 园 东 i 江许 永 途 沥 3
沥 i 技
el s 河 沥 招 圆
E t 园
E
e
e 芬
E
E 河 技 85 深
Page-543
E

e 技 芸 一 一

玟

( 一 8,8) 的 方 式 , 其 中 8 一 exp( 一 V e ) 特 别 要 讨 论 它 们 是 怎 样
带 近 原 点 的 , 以 及 靠 近 原 点 的 速 度 . 而 FCz,a) 的 送 代 的 一 些 基 本
性 质 对 于 理 解 这 些 问 题 起 着 重 要 作 用 . 应 当 指 出 , 尽 管 我 们 只 对 这
一 特 殊 眨 射 展 开 讨 论 , 但 是 这 种 以 Jakobson 开 始 ,Benedicks 和
E 育 沥 林 E
g =l′器J>″/″,

巳 沥 22n 林庆 不
E 厂 不 余

E 东 莲 育 2
如 果 加 E [ 一 L,1J 和 满 趸

3 才 52 明
E
e 丞 3 河
E
E C2. 1

n 河 =sin昔廖. E
为

E 熹m恤0 b 奎廖).
团
0s 认 史

E 育 2
4 C。sz音没苛_〔s_n 昔核)[1 歹)

i
Page-544
述 胡

16(a 一 RCya1 一 官髌赫 二 2 力 G8.D)

Eabosuta 一

[
c r [

E 林 沥 孝 0
0

林 胺
E 林
E 沥 达
E
0

沥 怀 d 政
E 一 吴
医

述 怀 芸 林一 命
0 根 t 东 沥 仪 才
E 河 芸 c 才 b 浩 莲 达 江 水 一 河
C
E J 商
E
c 东
E 朐 沥

5 林江标 木 木 沥 才 医 伟
那 么 71 心 a 此 时 ,PoeheC0,m) 未 必 可 以 严 裁 地 表 为 区 间 加 的 并 .
团

E F瓢蟹+…(0y唧〉\〈(7』J鬣′7) U ( 一 eakgHFly csHH1D7.
Page-545
E 浩 u E

E 育
园 2 育 2 江 口

d P′(枳′)T[a′雇〈″”瓜) 薹俨′【(%)II(_ 2a87,C2. 2
E

E <1 E 江 ,
团 因此cos告庞尝0 E 育

日
Y1 一 办

沥 …|<缪以及_旺沪蝴)卜 春

I
e 二
引 理 2.2 “ 存 在 常 数 7 丿 0 和 A(8》 又 0, 当 m 充 分 靠 近 2 时 ,
E
t
10 沥 c
(2) log13cF(z,a)| 交

E 一 e 一 青 , 而 东
E 沥
0
[

0

玟

一 (2a327( 一 52 标 切 ( 一 4522 纳 ] ( 一 270

人
t
[ 水
n
Page-546
256 第 六 章 “ 实 二 次 单 峰 吹 射 族 的 呆 引 孔

E 李 2 ,

E 沥
的 gz, 此 时 我 们 只 要 稍 稍 扩 展 相 应 的 参 数 区 间 4,r, 使 得 Frvhi(0,
7) 一 丁 U 匕 . 这 桦 的 构 造 意 显 着 对 恰 当 的 7 和 符 号 土 有

E g 兄 15 [
由 此 , 我 们 可 以 定 义 第 史 十 1 欣 自 由 返 回 .

[63 述 u 河 异 t
仪
林 芸
[ny c 水

4′+l E U m′+l(酗).
E

E 不 技 莲 江 林
Po, 都 与 工 中 的 一 个 区 间 关 联 . 这 样 的 区 间 分 成 两 类 , 一 类 区 间 是
In, 而 另 一 类 区 间 除 包 含 形 奶 Z 的 区 间 外 , 它 的 端 点 葛 入 I 毗 邻
的 区 间 内 ( 见 公 式 (3. 3 为 了 暮 免 引 入 复 杂 的 记 号 , 在 下 面 的 讨
玟
可 能 性 . 读 者 可 以 仿 照 证 明 思 想 , 对 第 二 类 区 间 得 到 同 桦 的 结 论 .

E 怡 30 15EsEupotsyna 的 沥 技 s
面 的 事 实 成 立 .

(D 如 果 m 一 mv(a) 是 第 % 次 返 回 的 指 标 , 志 之 1「 那 么

[ 林

G 途

e

0 达 东 木

E 育 江
【 李

下 面 我 们 对 & 十 1 证 明 (,(ii) 和 (ii) 成 立 . 令 是 心 中 一 个
Page-547
第 六 竟 宗 二 次 单 峰 哈 尊 旋 的 媒 引 孔

最 后 , 取 &(8) 一 (log2)71 E
东
区 间 (ao,23 中 存 在 一 个 小 区 间 4, 使 得 映 射 k : 心 一 工 是 一 一 对
应 的 , 并 东 保 持 指 数 扩 张 .
E 不 技 宏
E
s 木 一 用
[ t d
《383 |8 砂 (La)| 交 (1. 9 一 史 二 2 一 工
E 园
[ 一
E 改 才 月

E 告 [ 河 河 述 鲁U E 命

E 吴 述 沥 一 命
E

E E
「蒜_^鬓_脓龟盂<一毓

可 归 纳 地 证 明 a 一 史 (a 是 单 调 下 降 的 . 由 这 些 事 实 , 结 论 (1) 和
〈2) 得 证 . 这 是 因 为 对 上 述 的 8 和 丶 , 我 们 逃 出 包 含 2 的 小 参 数 区
[oteyaE 连 驱 述 沥 t
t 林
一 , 其 中 B 是 。 的 左 端 点 - 注 意 到 Sm (a),a E 。 是 单 调 下 陆
的 , 所 以 存 在 心 一 4o, 得 当 eE 心 时 ,Sn :4 一 厂 是 一 一 映 射 . 最
后 我 们 证 明 指 数 扩 张 伯 . 图 为 ˇ

a
Page-548
E 浩 i 257

E
s d
E 沥 伟
(a 一 砺 (9,a) 二 F1Ct,a) 一 R1(1 一 af,a)
E 医 述
E
E

u 伟
园 吴 5 5 E
a 1

E
E

12cFV(1 一 a8a)|
c

E 音. [
技
E 江2 刃

[ <羞陶伽 一 卯 | 万

& Z
d

只 要 j 一 1 丿 m, 邦 么 由 (iD,exp(2 V 乐 exp(j4). 但 是 由 引 理 .
E 林 林

E 沥 述 5

[ 15′】(音矶】 5 告胍薹) E 林 E 0

武 式 也 说 明 , 当 了 增 加 时 不 等 式 了 << 23x4 在 一 劝 之 前 就 破 坏 掉
E

E [
Page-549
一 2 PCrva] 一 1 一 a 的 葛 本 性 质 E

E 园 理
E

园
E 江江 河 0 盼

E 技
和
E

E

Enal
E 胺

并 归 纳 地 得 到
E
7 E
[ 壶墓u 十

0

s
[ 命
引 理 2.4 “ 存 在 充 分 小 的 > 0 和 m 一 2,m 帝 近 2, 如 果
0 技
[ 扬
0

s
2

E 园
E

M
[ 不 1
I 伟 y

E

z2
Page-550
258 E

令 山 一 (a,5). 我 们 讨 论 史 二 Pov+f+1(0, 的 长 度 的 下
乙 可 以 表 为
[ 张
E [

[ 咤

沥 '忐expm乏"' E

医 c
E 比较是微不足道的(也可参考〔TTY〕). 因 此

E 暑2″|左m′(亿)| [

E 辽 t [

其 中 1 一 a 丿 8 女 1
由(3.2)和〈3.4),存在仍俨,0〈劝<v<知_n使得
国 吴
d 沥
E 门 江 2 沥 技
18Fp(ga1 乐 不 18FpG 一 aqoom1.

将 (3. 11) 和 上 式 代 入 (3. 10

[

我 们 要 进 一 步 证 明 |2| 乐 exp( 一 2144). 事 实 上 , 由 假 设 ( 口 ) ,
[ 人

log2〔啬) Eitsssut

日

P 二 exp( 一 YR 一 1 一 exp( 一 YR ) 乐 ce- ,
仪 0 【 达 ″12/歹
Page-551
第 六 章 “ 实 二 次 单 峰 胥 寺 族 的 哉 引 二

[ 吊 ]
0

n _

E

E [

E
l一酝韧廷卜十

E 一

[
其 它 情 况 可 类 似 证 明 .

5 九 东 5
E >仍羞扁>仪 E
5

E 朐

E 命 谅
D 之 0. 所 以 1 十
医
Z
E 沥 圭a′庐 5 责exp基″气 园
E

E B
E
最 后 , 我 们 来 归 纳 地 证 明 引 理 结 论 . 一 7 时 无 需 证 明 , 设 7 女 ? 一
E

E

[
忐<‖ l 命

a 「
回 为
Page-552
道 志 河 d

王
6pHlCa1 芸 exp[ 一 (1 十 硕 砺 ) 07,

EJEdcpe
E

王
E 1 壬/_(] 十 z号#昙)_zexPG〈″) E 水
命 5

荣 砺 自
E [ 吴n
E E z/〖〕_
u 林述 t n
1asPm+pCLa1 节 exp| 2Cmx 十 刑 才 十 圭涣暑〕. G3.16
E 人
E E 吊 2 2鹰左″_(鹰)) *

E 命 c 沥 石

吴

G
之 2aexp( 一 YR) 。 玉exp(2 E 愤 0

王 一 A ss
>zoexp( Y 十 2 医 (P+l)z

[ 一 扬 明 沥

E
e 7

yeomor A >exp〈啬潍菩〉.

医 玟 5
E

[ 噩P菩〕

>叠xp〔 E 戍)鲁 E 忐棘吾] ,
Page-553
医 浩 a

E 忡】丨
[
E
E
[
a巍痿`′+I 12
萜廷善洲<m 1

E
阮
i
根 据 / 的 送 代 的 这 种 特 性 , 分 别 讨 论 了 当 参 数 攀 近 2 时 ,/a 进 入
乙 前 和 走 出 习 后 , 在 厂 外 的 扩 张 性 质 . 引 理 2.3 将 作 为 我 们 归 络
地 得 到 正 Lebesgue 参 数 集 4.. 的 基 础 . 我 们 将 FCz,a) 看 成 二 元 函
数 , 引 理 2.4 说 明 FCz,a) 的 送 代 对 <= 和 a 是 等 度 增 长 的 , 这 一 结 果
的 好 处 在 于 对 参 数 或 对 变 量 的 估 算 可 以 相 互 转 化 . 在 $ 3 定 理 3.1
E e 胡
引 理 的 意 义 还 不 仅 仅 如 化 - 对 于 一 舫 的 眺 射 族 FCz,a), 我 们 可 以
玲
E 沥
结 中 闸 述 .

(l十

$ 3 (rva) 不 存 在 穗 定 屉 期 轨 问 题

a
e
n

伟 [
E

2 沥 林 一
Page-554
260 第 大 章 实 二 次 单 珠 胥 射 旌 的 呆 引 孔

E
由 分 类 定 义 , 设 对 f 一 丶 ,JomtAHt(0,a 与 工 相 交 . 那 么
[ 6 命 弘 林 扬
E 又 技
0
2RhT1C8) | 交 1978 一 1

E
[ 沥 5 9)f】一】exp[ E 2 击涮

之 exp(2Cmy 十 力 十 趸】)晏)′
E 许标 许 刑
E 标 一
E 一 一
迷 回 时 , 我 们 有
12Rma1Ca| >exp(z″置).

如 此 继 缓 下 去 直 至 到 性 时 , 返 团 是 自 由 的 时 侯 为 止 . 自 由 返 回 在
E 东
我 们 在 A 中 看 到 的 一 桦 , 在 迷 回 时 是 非 常 快 地 增 长 的 . 我 们 完 成 了
i `

zn 育
E 不

E 沥铁 沥 林 技 5 沥 朐
那 么 我 们 有

肖

gn 吴 c 顶
E 一
Page-555
E E 河 s2

而 在 [F(L,a),1] 中 至 多 有 一 条 穗 定 周 期 豹 ; 并 且 偎 如 x 一 0 没 有
被 吴 引 到 一 条 周 朝 轨 , 那 么 F(z,a) 没 有 穗 定 周 期 轨 . 于 恬 , 定 理
3.1 可 以 从 下 面 的 定 理 3.2 得 到 .

定 理 3.2 存 在 4 CC (0,2),4. 具 有 正 Lebesgue 测 度 , 以 及
二

[ 刑 国 仪 命

E 图胡 a 扬 巳
E

A 我 们 根 据 之 (0,a) 不 断 返 回 到 “ 的 特 点 , 讨 论 返 回 的 形 式 .
与 此 同 时 用 自 由 返 回 概 念 , 归 纳 地 将 参 数 区 间 4 分 类 .

由 引 理 2. 3, 存 在 一 个 最 小 整 数 za(>> N) , 使 得

E
是 丿 到 T“ 上 的 1 一 1 映 射 . 称 m 是 映 射 的 首 次 自 由 返 回 的 指 标 ,
相 应 的 参 数 首 欣 分 类 为
[ t

h 途
E 二 d 八 厂 5
使 得 0 点 是 F(z,eo 的 周 期 为 z 的 超 稳 定 周 期 点 . 为 了 保 证 定 理
匹
数 区 间 4 中 , 我 们 全 部 排 除 了 使 / 可 能 有 小 于 等 于 m 的 稳 定 周
期 转 的 参 数

设 口 是 第 & 次 分 类 的 任 意 一 个 构 成 区 闽 , 丿 1. 下 面 我 们 定 义
第 十 1 欣 自 由 返 回 以 及 相 应 的 分 类 ˇ

E 2 技
t 伟

[ 一 _占′(″)_
成立的最大值.更确切地讲,对所有窿…m和所有″仨 a
Page-556
医 源 s

E 林 沥 加 肉
[ 许
>EXp〈2泅詹 5 、/页 E 吴 ′″翼)昙) >exP著鲁'
如 果 又 & 十 力 , 由 (8. 147
E
E 命 3 E 吊
E 蛊擒吾〉 [ 吴 吊
E
3
[
医
』 命 5 1 怀
E 林 述 r 沥 木
从 而 (iD 得 证 .
E
b 园 仪 沥 人 东 15
伟
[ 一
[

[ 不

E
测 度 . 更 确 切 地 讲 , 我 们 要 证 环
n [

其中0<鲇<1且置u_鲇)〉0.如果(3.16)得以证明,那么由
l
E
Page-557
第 六 章 实 二 次 单 峰 咏 射 族 的 吴 引 扎

1al 芸 lldl 节 小 丿 真 d 口 a
注 意 到 4+i C 心 , 我 们 有 国
E 」亘叫>飙
并 完 或 了 定 理 3. 2 的 证 明 . 在 证 明 (3. 16) 之 前 , 我 们 暂 时 便 设 对 所

E 沥

E
Pn 5
E 团 汞

人
Ln 0 的 | 一 |PowuC0,a 一 Fou 0
E 2
E
与 推 导 (3.10》 相 似 , 我 们 得 到
E

z 一
i

E d
5 沥 水 沥 沥 网 孙
[ 沥 E 弘
河
E

沥

E

二 鲁(L E 7
我 们 也 有

[ 刑 E 发 小 刀
E 林跃 江 林
Page-558
述 胡

16(a 一 RCya1 一 官髌赫 二 2 力 G8.D)

Eabosuta 一

[
c r [

E 林 沥 孝 0
0

林 胺
E 林
E 沥 达
E
0

沥 怀 d 政
E 一 吴
医

述 怀 芸 林一 命
0 根 t 东 沥 仪 才
E 河 芸 c 才 b 浩 莲 达 江 水 一 河
C
E J 商
E
c 东
E 朐 沥

5 林江标 木 木 沥 才 医 伟
那 么 71 心 a 此 时 ,PoeheC0,m) 未 必 可 以 严 裁 地 表 为 区 间 加 的 并 .
团

E F瓢蟹+…(0y唧〉\〈(7』J鬣′7) U ( 一 eakgHFly csHH1D7.
Page-559
医 胡 s

王
园 in
1 尹 [ 日

男 一 方 面 , 令 仁 表 示 中 可 以 包 含 于 @+ 的 小 区 间 全 体 . 根
据 +i 定 义 ; 记 仁 + 一 v , 则

e 沥

00
[ E
其 中 w E s 因 此

, | 二 sgC- vETKFI
[

利 用 (3. 17), 我 们 可 以 佼 计 从 Ax 中 排 除 的 参 数 集 的 测 度
I 吴 E 史 扬 视 园 X 王 35 E
团 1 E
团 此 , 有
_4叠| g |4蘑+l| E exp(_ 霜斗_媲-卜 )
1 “ E

1g+l 交 C1 一 aD|Q|.
E 述

H 许5

容易验证H a 因此II [

耐注3 4 读 者 可 以 看 到 , 在证明(3 16) 时 , 我 们 仅 考 虑 了 从
E 招
英

玟
t
Page-560
256 第 六 章 “ 实 二 次 单 峰 吹 射 族 的 呆 引 孔

E 李 2 ,

E 沥
的 gz, 此 时 我 们 只 要 稍 稍 扩 展 相 应 的 参 数 区 间 4,r, 使 得 Frvhi(0,
7) 一 丁 U 匕 . 这 桦 的 构 造 意 显 着 对 恰 当 的 7 和 符 号 土 有

E g 兄 15 [
由 此 , 我 们 可 以 定 义 第 史 十 1 欣 自 由 返 回 .

[63 述 u 河 异 t
仪
林 芸
[ny c 水

4′+l E U m′+l(酗).
E

E 不 技 莲 江 林
Po, 都 与 工 中 的 一 个 区 间 关 联 . 这 样 的 区 间 分 成 两 类 , 一 类 区 间 是
In, 而 另 一 类 区 间 除 包 含 形 奶 Z 的 区 间 外 , 它 的 端 点 葛 入 I 毗 邻
的 区 间 内 ( 见 公 式 (3. 3 为 了 暮 免 引 入 复 杂 的 记 号 , 在 下 面 的 讨
玟
可 能 性 . 读 者 可 以 仿 照 证 明 思 想 , 对 第 二 类 区 间 得 到 同 桦 的 结 论 .

E 怡 30 15EsEupotsyna 的 沥 技 s
面 的 事 实 成 立 .

(D 如 果 m 一 mv(a) 是 第 % 次 返 回 的 指 标 , 志 之 1「 那 么

[ 林

G 途

e

0 达 东 木

E 育 江
【 李

下 面 我 们 对 & 十 1 证 明 (,(ii) 和 (ii) 成 立 . 令 是 心 中 一 个
Page-561
E

[
不 伟 心 口
E 沥 林 招 沥

历
E

E 【

E 刑

E 刃

E 圆
E 园 沥 ) 林 林技 2

[ 2
团 n

E E 途

由 归 纳 的 结 论 ( , 得
E
[

[
p 沥
| el 园 0
P <〔1十茼] 廷h+曲岖_m川 E

E 述

春
u
E

0 逵叶] 园 愉砧)_锹叭‖

[ l

所 以 我 们 只 要 佶 计
Page-562
68 颜 uesposssaey

t
国 O
s “Rar
国 22 E 吴
E
i w
吴
2

E

乙 林 沥 东 顶 仁 取 洁 0
成 两 部 分

园 s 沥
国 0

与
n

首 免 佶 计 第 一 个 和 式 , 当 y 一 一 时 , r
E 伟
20 |F…(左′′(′z) E F窜(虔′翼(′′),丑〉|
e
同 时
E
之 | 严 (0,a)| 一 】F醴(蔓z′(″〉`鹰) g
E 一 T 认
r
E

江

l

【

月
国 2

| 十 26 ‖′茎〕 日 2
Page-563
E 浩 i 257

E
s d
E 沥 伟
(a 一 砺 (9,a) 二 F1Ct,a) 一 R1(1 一 af,a)
E 医 述
E
E

u 伟
园 吴 5 5 E
a 1

E
E

12cFV(1 一 a8a)|
c

E 音. [
技
E 江2 刃

[ <羞陶伽 一 卯 | 万

& Z
d

只 要 j 一 1 丿 m, 邦 么 由 (iD,exp(2 V 乐 exp(j4). 但 是 由 引 理 .
E 林 林

E 沥 述 5

[ 15′】(音矶】 5 告胍薹) E 林 E 0

武 式 也 说 明 , 当 了 增 加 时 不 等 式 了 << 23x4 在 一 劝 之 前 就 破 坏 掉
E

E [
Page-564
E

lay H 门 aRFeTTCLa|

我 们 把 不 等 式 有 边 的 和 式 分 成 两 部 分
0 0
国

E E 蠹_′j′+1
E
壬 L

d
第 二 部 分 我 们 利 用 (3. 1),(3.4) 和 (3. 5) 推 导 出

l ]oz|6s(镳)|`

E 吴 沥 训
0 技 [ 颂 闯 木 c
x

E 技

0
许
D E

E
[opo s 达

[
其 中 9 在 &.(a) 与 吕 (8) 之 间 , 我 们 有
Page-565
258 E

令 山 一 (a,5). 我 们 讨 论 史 二 Pov+f+1(0, 的 长 度 的 下
乙 可 以 表 为
[ 张
E [

[ 咤

沥 '忐expm乏"' E

医 c
E 比较是微不足道的(也可参考〔TTY〕). 因 此

E 暑2″|左m′(亿)| [

E 辽 t [

其 中 1 一 a 丿 8 女 1
由(3.2)和〈3.4),存在仍俨,0〈劝<v<知_n使得
国 吴
d 沥
E 门 江 2 沥 技
18Fp(ga1 乐 不 18FpG 一 aqoom1.

将 (3. 11) 和 上 式 代 入 (3. 10

[

我 们 要 进 一 步 证 明 |2| 乐 exp( 一 2144). 事 实 上 , 由 假 设 ( 口 ) ,
[ 人

log2〔啬) Eitsssut

日

P 二 exp( 一 YR 一 1 一 exp( 一 YR ) 乐 ce- ,
仪 0 【 达 ″12/歹
Page-566
医溥 s

16.(4b 一 051 〈鱼惮 r uCa 一 咤 、
[ l

E 一 一 个 一
t 林2 浩
于 是
16,(a 一 6(01 万 暑〔 罡] ′]+】一”|怠'… 1 洁 工
[
人 玑 国
,=′鲈熹′+] 国 X

国
E Ezssorsoyooon
19] 1

,( 一 品 ,(8)|

a

限 2
[

c '量】 E
s <m瞄〗 国 驿 上让'
/ [ 命
为 完 成 (3. 18) 的 估 计 , 我 们 只 要 佶 计 上 面 不 等 式 右 边 和 式 的
界 . 完 全 类 似 于 (3. 13) 的 佼 计 , 我 们 有

u (3.21)
E 沥

5协仁为此,先证明在有限的时间段1<i廷m中,胁
Page-567
道 志 河 d

王
6pHlCa1 芸 exp[ 一 (1 十 硕 砺 ) 07,

EJEdcpe
E

王
E 1 壬/_(] 十 z号#昙)_zexPG〈″) E 水
命 5

荣 砺 自
E [ 吴n
E E z/〖〕_
u 林述 t n
1asPm+pCLa1 节 exp| 2Cmx 十 刑 才 十 圭涣暑〕. G3.16
E 人
E E 吊 2 2鹰左″_(鹰)) *

E 命 c 沥 石

吴

G
之 2aexp( 一 YR) 。 玉exp(2 E 愤 0

王 一 A ss
>zoexp( Y 十 2 医 (P+l)z

[ 一 扬 明 沥

E
e 7

yeomor A >exp〈啬潍菩〉.

医 玟 5
E

[ 噩P菩〕

>叠xp〔 E 戍)鲁 E 忐棘吾] ,
Page-568
E
′T[ E 汀

u (

C25 月 吉

E

E
E

E
d

我 们 得 到
[

lD”z〈|左y(亿】)| c 命 En c

1(1 ze L,eE a

和
FPCruvao)| 之 告_尸(0鸪z)卜

E

cn
dcao

6ao 一 a , 16.Ca 一 6Cao|
m讲[2十 L )

由归纳的结论佃%引璎2 E

[ 刑
E e

<mn鱿(

因 止
[ 3
E
Page-569
E 浩 a

妙 const exp(8 丫 内 )|8 一 马 |,

) 悼 [ 仙)_
0 慨码 e
江

E
团 园
_az矗寸(z】 E
E
E
田 引 理 2. 1 和 上 述 佶 计 , 我 们 得 到
5 莲n
P _尸…_(′′十p′+1)〈左′′十′′+】(“),′z) 1
E
E
E

E

Esdaou

E

吊 园
因 为 8 移 小 , 所 以 (3. 22) 意 昧 着 lov+i| 丿 5|oy|, 最 后 , 我 们 估 计
和 式

【
内

E 某些叨可能同时位于某个区间驷中 Eaulne
E 李 永 有 阮 河 t
Page-570
260 第 大 章 实 二 次 单 珠 胥 射 旌 的 呆 引 孔

E
由 分 类 定 义 , 设 对 f 一 丶 ,JomtAHt(0,a 与 工 相 交 . 那 么
[ 6 命 弘 林 扬
E 又 技
0
2RhT1C8) | 交 1978 一 1

E
[ 沥 5 9)f】一】exp[ E 2 击涮

之 exp(2Cmy 十 力 十 趸】)晏)′
E 许标 许 刑
E 标 一
E 一 一
迷 回 时 , 我 们 有
12Rma1Ca| >exp(z″置).

如 此 继 缓 下 去 直 至 到 性 时 , 返 团 是 自 由 的 时 侯 为 止 . 自 由 返 回 在
E 东
我 们 在 A 中 看 到 的 一 桦 , 在 迷 回 时 是 非 常 快 地 增 长 的 . 我 们 完 成 了
i `

zn 育
E 不

E 沥铁 沥 林 技 5 沥 朐
那 么 我 们 有

肖

gn 吴 c 顶
E 一
Page-571
270 第 失 章 _ 官 二 次 单 峰 眸 射 消 的 吟 引 孔

最 大 下 标 , 由 此 有

U 是剩余指标集 E
吴 巳 刑
′…/】厂‖|

E
E

a 乏】 歹廷叩n歇,

E 林 沥 述
lex| 芸 const |ajlexpCV 为 ) exp| 一 z#量〕

艺 const V 内 羞‖吟恤帐伽募 |
3

乃 exp( 一 3 序 ).
沥罪 一 孙 李 人 厉
Page-572
医 源 s

E 林 沥 加 肉
[ 许
>EXp〈2泅詹 5 、/页 E 吴 ′″翼)昙) >exP著鲁'
如 果 又 & 十 力 , 由 (8. 147
E
E 命 3 E 吊
E 蛊擒吾〉 [ 吴 吊
E
3
[
医
』 命 5 1 怀
E 林 述 r 沥 木
从 而 (iD 得 证 .
E
b 园 仪 沥 人 东 15
伟
[ 一
[

[ 不

E
测 度 . 更 确 切 地 讲 , 我 们 要 证 环
n [

其中0<鲇<1且置u_鲇)〉0.如果(3.16)得以证明,那么由
l
E
Page-573
J…J】 ^/趸 0
E 5`从而定理3 i 国 L

E
玟

目
E 青〗鳃 [
c

E 吴 i
的 基 本 定 理

E 林木 水
r 王 述
E 林

附 注 4.2 这 个 定 理 表 明 , 测 度 x 不 是 很 特 殊 的 , 并 且 临 界 点
[ 水 沥 力 邹 河 租 途 运
国
才
E 林 浩 国 口 月
E

E 林 一 一
巳 阮 邦 院 才 s 江 沥 史 江
沥 个 u
本 质 返 回 (sw } 的 分 布 具 有 有 界 密 度 . 我 们 最 后 证 明 有 界 返 回 { }
E 一 玟

E

1. 自 由 返 回 的 分 布 问 题 的 讨 论 :
Page-574
第 六 章 实 二 次 单 峰 咏 射 族 的 吴 引 扎

1al 芸 lldl 节 小 丿 真 d 口 a
注 意 到 4+i C 心 , 我 们 有 国
E 」亘叫>飙
并 完 或 了 定 理 3. 2 的 证 明 . 在 证 明 (3. 16) 之 前 , 我 们 暂 时 便 设 对 所

E 沥

E
Pn 5
E 团 汞

人
Ln 0 的 | 一 |PowuC0,a 一 Fou 0
E 2
E
与 推 导 (3.10》 相 似 , 我 们 得 到
E

z 一
i

E d
5 沥 水 沥 沥 网 孙
[ 沥 E 弘
河
E

沥

E

二 鲁(L E 7
我 们 也 有

[ 刑 E 发 小 刀
E 林跃 江 林
Page-575
E n

E
我 们 可 以 把 它 看 成 概 率 测 度 . 在 此 基 础 上 , 我 们 可 以 引 入 期 望 三 和
2

E

E 如
或 者 .“ 一 4 由 前 面 的 讨 论 我 们 知 道 , 这 种 形 式 的 参 数 集 合 可 以 分
成 一 列 子 集 . 一 G , 其 中 每 个 心 ( 相 应 于 一 条 不 同 的 “ 路 径 吊 , 该
E 沥 才 a 东 河 儿 孙
E 振 化 工 0 志 = 仪

0 人 0 一 5 朐
颂 匹
对 每 个 路 往 2 由 (3. 13) , 可 扩 增 为 一 个 区 间 .(5) , 其 中
E [ 途
由 分 类 的 意 义 , 我 们 知 道 00.(5) 可 以 分 成 一 些 区 间 , 的 五 , 由 前 面

的 引 理 3. 5, 我 们 得 到

, 命 ,
E 茎〕m4耿0). [

为 完 成 情 形 1 的 证 明 , 我 们 首 先 给 出 两 个 事 实 , 并 把 它 们 归 纳
为 下 面 两 个 引 理 .
E 东 吴 阮 才
E
D
对 所 有 的 y 成 立 . 郁 么 , 对 所 有 x 我 们 有
E 二 月
E 根 水
Page-576
医 胡 s

王
园 in
1 尹 [ 日

男 一 方 面 , 令 仁 表 示 中 可 以 包 含 于 @+ 的 小 区 间 全 体 . 根
据 +i 定 义 ; 记 仁 + 一 v , 则

e 沥

00
[ E
其 中 w E s 因 此

, | 二 sgC- vETKFI
[

利 用 (3. 17), 我 们 可 以 佼 计 从 Ax 中 排 除 的 参 数 集 的 测 度
I 吴 E 史 扬 视 园 X 王 35 E
团 1 E
团 此 , 有
_4叠| g |4蘑+l| E exp(_ 霜斗_媲-卜 )
1 “ E

1g+l 交 C1 一 aD|Q|.
E 述

H 许5

容易验证H a 因此II [

耐注3 4 读 者 可 以 看 到 , 在证明(3 16) 时 , 我 们 仅 考 虑 了 从
E 招
英

玟
t
Page-577
g

E , d
FF05 许 2 仪 技 ,
E 标

E 3 仪 E 水 李 河
B 笋忐 菖m彻们 一 呈m膘 、

E

不
P 刑 应 3

E 07

其 中 , 之 (2e/Zo). 于 是
】′D~^′′′< [譬 E 暑〕 ′Z′]′】【】】′<Q|′″|m】_ 5 _
D 江 沥 国
引 理 4 3 的 一 种 特 殊 情 况 为
引 理 4.5 “ 假 如 -
E 怀 - 有 沥 不 明 训
鄱 么 对 所 有 的 ,
E
E 圆 e
2.3 可 得
E
如 果 上 述 不 等 式 对 n 时 成 立 , 下 面 证 明
E
事 实 上 , 如 果 我 们 记 心 一 U,dm, 那 么
E P
a
人
其 中 Q 一 (mw ) 一 , 二 是 , 由 引 理 4 3, 我 们 立 得
Page-578
E

[
不 伟 心 口
E 沥 林 招 沥

历
E

E 【

E 刑

E 刃

E 圆
E 园 沥 ) 林 林技 2

[ 2
团 n

E E 途

由 归 纳 的 结 论 ( , 得
E
[

[
p 沥
| el 园 0
P <〔1十茼] 廷h+曲岖_m川 E

E 述

春
u
E

0 逵叶] 园 愉砧)_锹叭‖

[ l

所 以 我 们 只 要 佶 计
Page-579
E

有

a
t 园 育

匹 el 牛I^ E

E 沥
E u t 东 沥 技 伟 i
密 度 . 因 此

E 亩 E 0
其 中 第 二 个 不 等 式 由 引 理 4. 5 得 到 . 因 此 , 如 果 定 义
林

那 么
E

医 i 门

E(R 二 一 N一寸 E 发 0 吊 伟
s ^

我 们 要 证 明
f E
E 命a 扬 形 2 江 2
t 木 u 许
并 注 意 骁
L棚勾R…唰驯) 一 一 1

E e 4 河
u 李 林 技
Page-580
68 颜 uesposssaey

t
国 O
s “Rar
国 22 E 吴
E
i w
吴
2

E

乙 林 沥 东 顶 仁 取 洁 0
成 两 部 分

园 s 沥
国 0

与
n

首 免 佶 计 第 一 个 和 式 , 当 y 一 一 时 , r
E 伟
20 |F…(左′′(′z) E F窜(虔′翼(′′),丑〉|
e
同 时
E
之 | 严 (0,a)| 一 】F醴(蔓z′(″〉`鹰) g
E 一 T 认
r
E

江

l

【

月
国 2

| 十 26 ‖′茎〕 日 2
Page-581
医 圆 2 E

ECu l yy i 芸
un 林林 目 剧
s 东 院 沥 辽
时 , 我 们 在 读 引 理 中 取 Q 一 | 令
E 不 朋

由 引 理 4 3, 我 们 有

E 朝
[pulol 阮 伟
E 朋

E 朝 月

E

E 水
E

E
E 沥

E
t

E 沥
E 沥 2 庞
p C4 7

(4. 7) 式 意 显 着 自 由 返 回 的 极 限 分 布 在 . 一 0 外 有 有 界 密 度 ( 界 为
c). 由 于 在 x 不 0 处 密 度 一 致 有 界 , 以 及 对 所 有 的 x,zs 久 0, 所 以
E 沥 t

利 用 非 本 质 返 回 的 定 义 和 性 质 , 类 似 地 可 以 证 明 , 非 本 质 返 回
的 极 限 分 布 具 有 有 界 密 度 .

E 沥 p
Page-582
E

lay H 门 aRFeTTCLa|

我 们 把 不 等 式 有 边 的 和 式 分 成 两 部 分
0 0
国

E E 蠹_′j′+1
E
壬 L

d
第 二 部 分 我 们 利 用 (3. 1),(3.4) 和 (3. 5) 推 导 出

l ]oz|6s(镳)|`

E 吴 沥 训
0 技 [ 颂 闯 木 c
x

E 技

0
许
D E

E
[opo s 达

[
其 中 9 在 &.(a) 与 吕 (8) 之 间 , 我 们 有
Page-583
u E

我 们 要 讨 论 有 界 返 回 {ow}, n 一 .18,,N, 一 1,2,.. 的 分
E 晓noltereaol e
JJ 固 定 ) 的 分 布 , 然 后 对 所 有 了 >> J 进 行 分 析 .

E 东 育
t 不
E
为
[
[

E ={

E
口
E 李 3
E

沥
E
土 _i) 二 I 由 中 值 定 理 和 (3. 5 式 , 我 们 有

[ 刀 团 于
这 里 我 们 要 特 别 指 出 , 如 果 {6,}8-, 是 一 个 奇 ( 偶 ) 接 序 列 , 那 么 a;
E 振

E
因 此 , 一 个 点 zn E T 的 第 了 次 有 界 返 回 属 于 (a) 的 “ 条 件 概
E 标 纳 a
Page-584
医溥 s

16.(4b 一 051 〈鱼惮 r uCa 一 咤 、
[ l

E 一 一 个 一
t 林2 浩
于 是
16,(a 一 6(01 万 暑〔 罡] ′]+】一”|怠'… 1 洁 工
[
人 玑 国
,=′鲈熹′+] 国 X

国
E Ezssorsoyooon
19] 1

,( 一 品 ,(8)|

a

限 2
[

c '量】 E
s <m瞄〗 国 驿 上让'
/ [ 命
为 完 成 (3. 18) 的 估 计 , 我 们 只 要 佶 计 上 面 不 等 式 右 边 和 式 的
界 . 完 全 类 似 于 (3. 13) 的 佼 计 , 我 们 有

u (3.21)
E 沥

5协仁为此,先证明在有限的时间段1<i廷m中,胁
Page-585
E
′T[ E 汀

u (

C25 月 吉

E

E
E

E
d

我 们 得 到
[

lD”z〈|左y(亿】)| c 命 En c

1(1 ze L,eE a

和
FPCruvao)| 之 告_尸(0鸪z)卜

E

cn
dcao

6ao 一 a , 16.Ca 一 6Cao|
m讲[2十 L )

由归纳的结论佃%引璎2 E

[ 刑
E e

<mn鱿(

因 止
[ 3
E
Page-586
4 分 布 问 题

爽<垩const丽, L
0, a 一
E 沥 i 河

3
C
3

AEM 0

其 中

、 E 灿愤>删跳仰「圩
Eots 水

, 招
E 。

从 而 得

2 秉 沥 E
玟
形 式

E 皇 大 圆 A
0
玟

E atteohiet
到 , 对 充 分 大 的 丿 , 我 们 有

E 92 东 e^尝昔″~ [ 大
由(4'9〉式我们得到,相应于y″j,极限分布对几乎所有的窿仨^有
2

一

原
I″(″)g/(况) E D

E tgsnatlp 规 d
E 胡
Page-587
E 浩 a

妙 const exp(8 丫 内 )|8 一 马 |,

) 悼 [ 仙)_
0 慨码 e
江

E
团 园
_az矗寸(z】 E
E
E
田 引 理 2. 1 和 上 述 佶 计 , 我 们 得 到
5 莲n
P _尸…_(′′十p′+1)〈左′′十′′+】(“),′z) 1
E
E
E

E

Esdaou

E

吊 园
因 为 8 移 小 , 所 以 (3. 22) 意 昧 着 lov+i| 丿 5|oy|, 最 后 , 我 们 估 计
和 式

【
内

E 某些叨可能同时位于某个区间驷中 Eaulne
E 李 永 有 阮 河 t
Page-588
E

E 页

E
, 一
″′〈“)_{0' E
E

〗缆 E 沥 用

E 由上式还可以得到″、有懈|>J 医 途
20

E 〗 E 湘 e E e一告 `/7'

E

E 刑 医 p
E 时分两种情况】吨n岫H 庞 木

渡 A 李 t

e 个 ,
E 人 吊 述 渡 吴 P 水 扬
定 理 4. 6 “ 对 儿 乎 所 有 的 a E A, 自 由 返 回 和 非 本 质 自 由 返 固
的 极 限 分 布 具 有 一 致 有 界 的 密 度 , 而 有 界 返 回 极 限 分 布 具 有 密 度
&(z)、g(z) 满 足 不 等 式
[ E

E a
5 7

0
E
Page-589
270 第 失 章 _ 官 二 次 单 峰 眸 射 消 的 吟 引 孔

最 大 下 标 , 由 此 有

U 是剩余指标集 E
吴 巳 刑
′…/】厂‖|

E
E

a 乏】 歹廷叩n歇,

E 林 沥 述
lex| 芸 const |ajlexpCV 为 ) exp| 一 z#量〕

艺 const V 内 羞‖吟恤帐伽募 |
3

乃 exp( 一 3 序 ).
沥罪 一 孙 李 人 厉
Page-590
团

* E
E

69 E 0,
, E 史
E 渡 2
玟
i

2 E n ‖
1
3 林3 林3 林 林 33
E E E 2 h

0
E 引 i 0 许 技

0

医 达
_m'嚷萝 E 林

E

E 沥 伟 李
E
E 一 育

E
ERCD

(z) 二

E

E I″U 〉′L(算) E

E

医 梁

其 中 ACz) 二 const 〗吾亓工. E
E

上 式 的 右 端 具 形 式

8
E 5 0
I″U(一″)′【(箕) E L 幄碧颢(F【 0 怡 沥
Page-591
J…J】 ^/趸 0
E 5`从而定理3 i 国 L

E
玟

目
E 青〗鳃 [
c

E 吴 i
的 基 本 定 理

E 林木 水
r 王 述
E 林

附 注 4.2 这 个 定 理 表 明 , 测 度 x 不 是 很 特 殊 的 , 并 且 临 界 点
[ 水 沥 力 邹 河 租 途 运
国
才
E 林 浩 国 口 月
E

E 林 一 一
巳 阮 邦 院 才 s 江 沥 史 江
沥 个 u
本 质 返 回 (sw } 的 分 布 具 有 有 界 密 度 . 我 们 最 后 证 明 有 界 返 回 { }
E 一 玟

E

1. 自 由 返 回 的 分 布 问 题 的 讨 论 :
Page-592
280 l
E 庆

z
Ltocoytny t 标 c
[

E

1
g Efoal 12 江 E
I |RCF-Gm)) 1F-i(ay | dz 二 I…z)| …彻)怦

<c呱【I …叫′_缸<伽酞 〗 _[ |′′(z〉|P_d丈

pca

3 医 l)/_I L

P

国

E
1 I [
林 8

E
E
- 盲e一″盖扒< [ 朋 英

E
E 艾` ^舌″宜] _z′′_歹zze兰′" E e【″”_一″T〕

用 积 分 佼 计 式 , 我们容易证明
0 限 刃 dz
E 吴
c0【ISt'=〔′熹…〕+】e z [I_丨 恤一灼瞎]

3

E
EEeseity 怡 招 5556 芸 河 5 国 子 氏 标 淅
E 【
Page-593
E n

E
我 们 可 以 把 它 看 成 概 率 测 度 . 在 此 基 础 上 , 我 们 可 以 引 入 期 望 三 和
2

E

E 如
或 者 .“ 一 4 由 前 面 的 讨 论 我 们 知 道 , 这 种 形 式 的 参 数 集 合 可 以 分
成 一 列 子 集 . 一 G , 其 中 每 个 心 ( 相 应 于 一 条 不 同 的 “ 路 径 吊 , 该
E 沥 才 a 东 河 儿 孙
E 振 化 工 0 志 = 仪

0 人 0 一 5 朐
颂 匹
对 每 个 路 往 2 由 (3. 13) , 可 扩 增 为 一 个 区 间 .(5) , 其 中
E [ 途
由 分 类 的 意 义 , 我 们 知 道 00.(5) 可 以 分 成 一 些 区 间 , 的 五 , 由 前 面

的 引 理 3. 5, 我 们 得 到

, 命 ,
E 茎〕m4耿0). [

为 完 成 情 形 1 的 证 明 , 我 们 首 先 给 出 两 个 事 实 , 并 把 它 们 归 纳
为 下 面 两 个 引 理 .
E 东 吴 阮 才
E
D
对 所 有 的 y 成 立 . 郁 么 , 对 所 有 x 我 们 有
E 二 月
E 根 水
Page-594
E

E
P 〗麟枫岫 [ 达 )

我们已经知道对于z [ 〖az,l) 2 伟 3 林沥 沥 久
沥
[ 肖 沥 [ 途 )
用 此 事 实 以 及 完 全 类 伸 前 面 的 方 法 , 我 们 有 下 面 的 估 计

E <I 7355 浩 兰 技 技 技
其 中 8xjjs 二 const Xr, 力 一 2, 由 此 推 得

0 <I 〗gz(z) dz,

其中〗例…猷 E

小结4 0 技 生 s 途 达
a 一 2 附 近 的 动 力 学 行 为 , 证 明 过 程 也 说 明 ,[Yo] 中 闸 述 的 关 于 当
前 如 何 研 究 非 双 曲 系 统 的 看 法 的 实 用 性 . 下 面 我 们 简 单 地 介 绍 , 关
于 一 舫 单 峰 映 射 族 /:(z 的 新 结 果 , 以 此 作 为 对 上 面 特 例 的 总 结 。

E 怀 训 述 t 国 伟
[ 沥 育 印 2 技 沥 吴 ndU i 技

林 顶

[ 芸 林a 芸
[ 沥 7 5
林 仪 不

O 2R 河 c 技 u
有 r E I

E

以 及
Page-595
g

E , d
FF05 许 2 仪 技 ,
E 标

E 3 仪 E 水 李 河
B 笋忐 菖m彻们 一 呈m膘 、

E

不
P 刑 应 3

E 07

其 中 , 之 (2e/Zo). 于 是
】′D~^′′′< [譬 E 暑〕 ′Z′]′】【】】′<Q|′″|m】_ 5 _
D 江 沥 国
引 理 4 3 的 一 种 特 殊 情 况 为
引 理 4.5 “ 假 如 -
E 怀 - 有 沥 不 明 训
鄱 么 对 所 有 的 ,
E
E 圆 e
2.3 可 得
E
如 果 上 述 不 等 式 对 n 时 成 立 , 下 面 证 明
E
事 实 上 , 如 果 我 们 记 心 一 U,dm, 那 么
E P
a
人
其 中 Q 一 (mw ) 一 , 二 是 , 由 引 理 4 3, 我 们 立 得
Page-596
第 六 辅 “ 实 二 次 单 峰 吾 射 焊 的 吴 引 孔

E :. | 井
E

人 招

E s 东 不
动 参 数 的 概 念 -

木
动 参 数 , 如 果 它 满 足 如 下 条 件 :

E 林 沥 东 胡2
周 期 转 ;

E 2 坂 u 沥 沥 ( 沥
E 技 0 林 沥 江 林 芸 0 李 林 沥 庆
E 刑 co 不
E 沥 G 沥 才
翦丽 E Q' 姜 0

E
临 界 点 c 一 0 是 非 回 复 的 ; 面 么 没 有 稳 定 周 期 转 可 以 保 证 在 临 界
点 的 任 意 邻 域 外 的 任 何 足 够 长 的 执 道 段 的 指 数 增 长 ,CCEu) 条 件
是 证 明 进 程 中 技 术 上 的 要 求 , 它 保 证 软 道 的 增 长 指 数 ( 比 方 说 忍
伟
E aze 规 沥 口 扬 汀

沥
E 尘 林 园 河 5

0 沥 人
b

其 中 |Q| 表 示 集 合 的 Lebesgue 测 度 . 特 别 , 如 果 该 极 限 值 为 1,
则 称 a, 为 Lebesgue 全 翟 点 .
E s 沥
Page-597
E

有

a
t 园 育

匹 el 牛I^ E

E 沥
E u t 东 沥 技 伟 i
密 度 . 因 此

E 亩 E 0
其 中 第 二 个 不 等 式 由 引 理 4. 5 得 到 . 因 此 , 如 果 定 义
林

那 么
E

医 i 门

E(R 二 一 N一寸 E 发 0 吊 伟
s ^

我 们 要 证 明
f E
E 命a 扬 形 2 江 2
t 木 u 许
并 注 意 骁
L棚勾R…唰驯) 一 一 1

E e 4 河
u 李 林 技
Page-598
E 283

E
E

E 河

E 仪
园
E

EouEy A ss t 才

E 技 d dn

0 8 东 孙2

E 才 2
0 技 野 2 5许
0

由 于 对 每 个 a E , 务 滢 足 条 件 (NS),(CE1) 和 CCE2), 利 用
DNS] 的 结 果 , 立 得 对 乙 存 在 一 个 关 于 Lebesgue 测 度 绝 对 连 绩 的
技 途

近 年 来 , 在 非 双 曲 系 统 研 究 中 不 断 有 较 深 刻 的 结 果 出 现 , 我 们
在 本 章 中 不 准 备 介 绍 了 . 但 有 一 点 要 指 出 , 我 们 在 这 里 介 绍 的
Benedicks 和 Carleson 的 证 明 思 想 ( 简 称 BC 方 法 ), 近 儿 年 来 在 这 个
d
射 族 还 是 对 平 面 间 宿 相 切 ( 例 如 Henon 映 射 , 敲 结 分 岔 甚 至 高 维
的 同 宿 分 岔 现 象 所 得 到 的 主 要 绪 果 的 证 明 方 法 , 基 本 上 是 BC 方
法 , 或 者 是 从 这 一 方 法 中 派 生 出 来 的 -
Page-599
医 圆 2 E

ECu l yy i 芸
un 林林 目 剧
s 东 院 沥 辽
时 , 我 们 在 读 引 理 中 取 Q 一 | 令
E 不 朋

由 引 理 4 3, 我 们 有

E 朝
[pulol 阮 伟
E 朋

E 朝 月

E

E 水
E

E
E 沥

E
t

E 沥
E 沥 2 庞
p C4 7

(4. 7) 式 意 显 着 自 由 返 回 的 极 限 分 布 在 . 一 0 外 有 有 界 密 度 ( 界 为
c). 由 于 在 x 不 0 处 密 度 一 致 有 界 , 以 及 对 所 有 的 x,zs 久 0, 所 以
E 沥 t

利 用 非 本 质 返 回 的 定 义 和 性 质 , 类 似 地 可 以 证 明 , 非 本 质 返 回
的 极 限 分 布 具 有 有 界 密 度 .

E 沥 p
Page-600
附 “ 录

即 袖 我 们 只 考 虑 R 上 的 向 量 场 , 在 讨 论 分 岔 的 佘 维 数 时 , 也
E L e
述
在 附 录 4~C 中 介 绍 一 些 有 关 微 分 流 形 与 徽 分 拓 扑 的 概 念 、 名 词
E
医
D

[

微 分 流 形 是 欧 氏 空 间 中 光 滑 曲 面 椿 念 的 抽 象 和 推 广 . 它 的 基
s
立 微 分 同 背 而 引 迹 相 应 的 代 数 和 拓 扑 结 构 , 然 后 再 把 这 些 局 部 结
构 光 滑 地 粘 接 起 来 . 因 此 , 我 们 可 以 把 Banach 空 间 中 的 运 算 推 广
到 微 分 流 形 上 .

征 分 流 形 的 定 义

朋 江标 国 育 李 i 门
Banach 空 间 . 假 设 口 是 M 的 开 子 集 ,p 是 从 口 到 8 中 开 子 集 CDU)
E

切 的 两 个 坐 标 卡 (U,g),(7,43 称 为 是 C“ 相 容 的 , 如 果 当 口
【 绍

E
Page-601
u E

我 们 要 讨 论 有 界 返 回 {ow}, n 一 .18,,N, 一 1,2,.. 的 分
E 晓noltereaol e
JJ 固 定 ) 的 分 布 , 然 后 对 所 有 了 >> J 进 行 分 析 .

E 东 育
t 不
E
为
[
[

E ={

E
口
E 李 3
E

沥
E
土 _i) 二 I 由 中 值 定 理 和 (3. 5 式 , 我 们 有

[ 刀 团 于
这 里 我 们 要 特 别 指 出 , 如 果 {6,}8-, 是 一 个 奇 ( 偶 ) 接 序 列 , 那 么 a;
E 振

E
因 此 , 一 个 点 zn E T 的 第 了 次 有 界 返 回 属 于 (a) 的 “ 条 件 概
E 标 纳 a
Page-602
4 分 布 问 题

爽<垩const丽, L
0, a 一
E 沥 i 河

3
C
3

AEM 0

其 中

、 E 灿愤>删跳仰「圩
Eots 水

, 招
E 。

从 而 得

2 秉 沥 E
玟
形 式

E 皇 大 圆 A
0
玟

E atteohiet
到 , 对 充 分 大 的 丿 , 我 们 有

E 92 东 e^尝昔″~ [ 大
由(4'9〉式我们得到,相应于y″j,极限分布对几乎所有的窿仨^有
2

一

原
I″(″)g/(况) E D

E tgsnatlp 规 d
E 胡
Page-603
E

是 旦 上 的 C 微 分 同 胚 .

E
为 一 个 C 坐 标 綦 , 如 果

【 江 c

0

p
版 的 Cr 坐 标 系 . 队 上 C“ 坐 标 系 的 一 个 等 价 类 三 称 作 元 的 一 个 C“
E 吴 沥 a
的 - 个 极 火 Cr 坐 标 系 , 而 (U,8》 E - 必 称 作 一 个 容 许 坐 标 卡 .

如 果 在 吊 上 给 定 了 一 个 C“ 微 分 结 构 则 称 6 一 ( 史 ,3 是 一
E t
E 育
( 注 意 , 由 吊 的 连 通 性 和 坐 标 卡 的 相 容 性 易 知 ,5 与 坂 标 卡 的 选 取
无 关 ). 特 别 , 当 8 为 Hilbert 空 间 时 , 称 M 为 Hilbert 流 渺 ; 而 当 5
Eoratstsaediuhe t

E 玟 吴 a
以 把 与 a 等 价 的 全 部 坐 标 系 合 起 来 而 得 到 一 个 极 大 坐 标 系 , 从 而
生 成 队 上 的 一 个 微 分 结 构 . 图 此 , 只 须 给 定 M 上 一 个 特 定 的 坐 标
系 , 就 可 以 决 定 这 个 微 分 流 形 .

E
应 地 得 到 C“ 微 分 流 形 . C“ 微 分 流 形 也 称 为 光 滑 流 形
国 0n
( 是 旦 的 佰 同 映 射 ), 这 个 坐 标 卡 显 然 就 构 成 日 上 的 一 个 C“ 坐 标 .
系 . 因 此 , 任 何 Banach 空 间 都 是 装 备 在 它 自 身 上 的 一 个 光 谊
Banach 流 形

(2) 8 一 tz E Rr++llia 一 1} 是 一 个 n 雕 流 形 . 事 实 上 , 记
足 一 人 ,0,.0},5 二 { 一 一 0,.0} 分 别 是 “ 的 北 极 与 家 极 , 取
坐 标 卡 (SeNtN}, ),《SeMtS}, , 其 中
Page-604
E

E 页

E
, 一
″′〈“)_{0' E
E

〗缆 E 沥 用

E 由上式还可以得到″、有懈|>J 医 途
20

E 〗 E 湘 e E e一告 `/7'

E

E 刑 医 p
E 时分两种情况】吨n岫H 庞 木

渡 A 李 t

e 个 ,
E 人 吊 述 渡 吴 P 水 扬
定 理 4. 6 “ 对 儿 乎 所 有 的 a E A, 自 由 返 回 和 非 本 质 自 由 返 固
的 极 限 分 布 具 有 一 致 有 界 的 密 度 , 而 有 界 返 回 极 限 分 布 具 有 密 度
&(z)、g(z) 满 足 不 等 式
[ E

E a
5 7

0
E
Page-605
E E 吴

E
0

E 叽〈渥】,""艾”鼻】〉=

。

沥 招
容 易 得 出

8 RM0) 一 RM(0j, 8 0 一 命
是 C“ 微 分 同 胚 .

E E 绅_〕
E i 左

流 形 间 的 映 射

定 义 A.5 “ 设 友 , 分 别 是 装 备 在 Banach 空 间 4, 上 的 C
E
氓 53 诊 2 52 技 沥 i
E u 坤 林 f

E
是 C“ 的 .

E

E 胡2 东 技
医 5 标 g
E

E 丶 沥 标 育 2 林 述 林 标 江
E 沥 |

定 义 A.8 “ 设 肌 , 义 是 Cr 徽 分 流 形 , 称 M ~ 丶 为 C 微 分
同 胚 , 如 果 它 是 一 一 的 C 映 射 , 并 且 其 逆 映 射 广 !:N 一 吊 也 是 Cr
的 . 如 果 两 个 流 形 间 存 在 一 个 C 徽 分 同 胚 , 则 称 这 两 个 流 形 C“ 微
分 同 胚 .

子 流 形 与 积 流 形
类 似 于 向 量 空 间 的 子 空 间 与 乘 积 空 间 , 微 分 流 形 也 存 在 予 流
Page-606
团

* E
E

69 E 0,
, E 史
E 渡 2
玟
i

2 E n ‖
1
3 林3 林3 林 林 33
E E E 2 h

0
E 引 i 0 许 技

0

医 达
_m'嚷萝 E 林

E

E 沥 伟 李
E
E 一 育

E
ERCD

(z) 二

E

E I″U 〉′L(算) E

E

医 梁

其 中 ACz) 二 const 〗吾亓工. E
E

上 式 的 右 端 具 形 式

8
E 5 0
I″U(一″)′【(箕) E L 幄碧颢(F【 0 怡 沥
Page-607
280 l
E 庆

z
Ltocoytny t 标 c
[

E

1
g Efoal 12 江 E
I |RCF-Gm)) 1F-i(ay | dz 二 I…z)| …彻)怦

<c呱【I …叫′_缸<伽酞 〗 _[ |′′(z〉|P_d丈

pca

3 医 l)/_I L

P

国

E
1 I [
林 8

E
E
- 盲e一″盖扒< [ 朋 英

E
E 艾` ^舌″宜] _z′′_歹zze兰′" E e【″”_一″T〕

用 积 分 佼 计 式 , 我们容易证明
0 限 刃 dz
E 吴
c0【ISt'=〔′熹…〕+】e z [I_丨 恤一灼瞎]

3

E
EEeseity 怡 招 5556 芸 河 5 国 子 氏 标 淅
E 【
Page-608
E 标

E
E 林
E 林c 园
国 江2 p
河 江
E
E 技 5
【 a 浩
E 应 辽 述 逊 s
CU, 外 , 和 8 的 直 和 分 解 8 = B 申 Ba, 使 得 E D, 干 一
E 林 0203 0 相
p 林 林 国 吊
E 一
E 沥 d
个 微 分 流 形 , 并 旦 它 的 微 分 结 构 可 由 下 面 的 坂 标 系 生 成 :
d
s 吴
E 水
流 形 的 方 法
E z i 朐
t 小 不
E R 西 一 明 春 述
分 流 形 , 称 它 为 M 与 N 的 积 流 形 , 仍 记 为 友 X N-

附 星 B “ 切 丛 与 切 映 射 , 向 量 场 及 其 流 , 漫 人 与 漫 盐

向 量 丛 是 积 流 形 的 推 广 , 而 流 形 上 的 向 量 场 , 则 是 作 为 一 个 特
E u
Page-609
E

E
P 〗麟枫岫 [ 达 )

我们已经知道对于z [ 〖az,l) 2 伟 3 林沥 沥 久
沥
[ 肖 沥 [ 途 )
用 此 事 实 以 及 完 全 类 伸 前 面 的 方 法 , 我 们 有 下 面 的 估 计

E <I 7355 浩 兰 技 技 技
其 中 8xjjs 二 const Xr, 力 一 2, 由 此 推 得

0 <I 〗gz(z) dz,

其中〗例…猷 E

小结4 0 技 生 s 途 达
a 一 2 附 近 的 动 力 学 行 为 , 证 明 过 程 也 说 明 ,[Yo] 中 闸 述 的 关 于 当
前 如 何 研 究 非 双 曲 系 统 的 看 法 的 实 用 性 . 下 面 我 们 简 单 地 介 绍 , 关
于 一 舫 单 峰 映 射 族 /:(z 的 新 结 果 , 以 此 作 为 对 上 面 特 例 的 总 结 。

E 怀 训 述 t 国 伟
[ 沥 育 印 2 技 沥 吴 ndU i 技

林 顶

[ 芸 林a 芸
[ 沥 7 5
林 仪 不

O 2R 河 c 技 u
有 r E I

E

以 及
Page-610
E 附 录 B “ 切 屿 与 切 际 驯 , 名 量 圭 友 其 浇 浦 人 与 深 范

Banach 空 间 中 的 隐 函 数 ( 反 函 数 ) 定 理 以 及 局 部 潍 入 和 浸 盎 定 理
E 沥 吊 c 胡
E

向 量 丛

E 孙 林团
开 集 , 秒 乙 X 日 为 局 部 向 量 丛 , 并 称 U 为 廉 空 间 , 它 可 以 等 同 于
E i 国 林
团 伟
人

E 园 一 一
一 个 开 子 流 形 .

定 义 B.2 设 D X 旦 和 D X X 都 是 局 部 向 量 仰 . 如 果 眸 射
E 「

E
E
这 个 映 射 还 是 一 一 的 , 则 称 它 为 C 局 邵 向 量 丛 同 构

E
t
E 东

类 似 于 微 分 流 形 的 定 义 , 我 们 可 以 把 局 部 向 量 丛 粘 接 起 来 , 得
E

定 义 B.3 设 5 是 集 合 . 称 (P7, 旭 是 5 的 一 个 局 部 向 重 仰 卡 ,
E 沥
(U,B 可 能 与 g 有 关 ) 称 这 桦 的 卡 集 畅 一 (CWo,8)|a E 4 是 5

的 C7 向 量 丛 垫 标 系 , 如 果
E05 林 t A 沥
(2) 著 (Wav8),(Wo8) E 男 , Ja n W 夺 切 , 则 (
Page-611
第 六 辅 “ 实 二 次 单 峰 吾 射 焊 的 吴 引 孔

E :. | 井
E

人 招

E s 东 不
动 参 数 的 概 念 -

木
动 参 数 , 如 果 它 满 足 如 下 条 件 :

E 林 沥 东 胡2
周 期 转 ;

E 2 坂 u 沥 沥 ( 沥
E 技 0 林 沥 江 林 芸 0 李 林 沥 庆
E 刑 co 不
E 沥 G 沥 才
翦丽 E Q' 姜 0

E
临 界 点 c 一 0 是 非 回 复 的 ; 面 么 没 有 稳 定 周 期 转 可 以 保 证 在 临 界
点 的 任 意 邻 域 外 的 任 何 足 够 长 的 执 道 段 的 指 数 增 长 ,CCEu) 条 件
是 证 明 进 程 中 技 术 上 的 要 求 , 它 保 证 软 道 的 增 长 指 数 ( 比 方 说 忍
伟
E aze 规 沥 口 扬 汀

沥
E 尘 林 园 河 5

0 沥 人
b

其 中 |Q| 表 示 集 合 的 Lebesgue 测 度 . 特 别 , 如 果 该 极 限 值 为 1,
则 称 a, 为 Lebesgue 全 翟 点 .
E s 沥
Page-612
E 283

E
E

E 河

E 仪
园
E

EouEy A ss t 才

E 技 d dn

0 8 东 孙2

E 才 2
0 技 野 2 5许
0

由 于 对 每 个 a E , 务 滢 足 条 件 (NS),(CE1) 和 CCE2), 利 用
DNS] 的 结 果 , 立 得 对 乙 存 在 一 个 关 于 Lebesgue 测 度 绝 对 连 绩 的
技 途

近 年 来 , 在 非 双 曲 系 统 研 究 中 不 断 有 较 深 刻 的 结 果 出 现 , 我 们
在 本 章 中 不 准 备 介 绍 了 . 但 有 一 点 要 指 出 , 我 们 在 这 里 介 绍 的
Benedicks 和 Carleson 的 证 明 思 想 ( 简 称 BC 方 法 ), 近 儿 年 来 在 这 个
d
射 族 还 是 对 平 面 间 宿 相 切 ( 例 如 Henon 映 射 , 敲 结 分 岔 甚 至 高 维
的 同 宿 分 岔 现 象 所 得 到 的 主 要 绪 果 的 证 明 方 法 , 基 本 上 是 BC 方
法 , 或 者 是 从 这 一 方 法 中 派 生 出 来 的 -
Page-613
附 源 B 切 世 与 切 险 射 , 名 量 行 及 其 流 , 清 人 与 渥 敏 289

L 林t t

称 5 的 两 个 向 量 从 坂 标 系 88 与 田 蔽 价 , 如 果 幺 lU 口 ; 还 是
l 吴
E 应 不 不 不 i
伟
医 i 2

[

0 微分流形

ECutopasaopo 志 s t

E 胡 c 李
达

0
E

E 江
E 述
E n U
Ep,a 线 性 拓 扑 同 构 ( 作 为 线 性 空 间 是 同 构 的 , 作 为 拓 扑 空 间 是 同
胃 的 . 在 这 个 意 义 下 , 可 以 认 为 E 与 8 无 关 , 可 记 为 E

【 坂 孙 育 的婵…M(底空间) E
可 定 义 投 影

E i
i
有 的 Banach 结 构 称 为 纤 维 型 . 在 有 些 书 上 , 把 上 面 的 性 质 作 为 向
Eatpys t t
E

从几何上粗略地说,以流形M为底空间的向丛y就是在M

上 的 每 一 点 “ 附 着 “ 一 个 以 该 点 为 零 元 素 的 Banach 空 间 , 而 不 囡 点
E
Page-614
附 “ 录

即 袖 我 们 只 考 虑 R 上 的 向 量 场 , 在 讨 论 分 岔 的 佘 维 数 时 , 也
E L e
述
在 附 录 4~C 中 介 绍 一 些 有 关 微 分 流 形 与 徽 分 拓 扑 的 概 念 、 名 词
E
医
D

[

微 分 流 形 是 欧 氏 空 间 中 光 滑 曲 面 椿 念 的 抽 象 和 推 广 . 它 的 基
s
立 微 分 同 背 而 引 迹 相 应 的 代 数 和 拓 扑 结 构 , 然 后 再 把 这 些 局 部 结
构 光 滑 地 粘 接 起 来 . 因 此 , 我 们 可 以 把 Banach 空 间 中 的 运 算 推 广
到 微 分 流 形 上 .

征 分 流 形 的 定 义

朋 江标 国 育 李 i 门
Banach 空 间 . 假 设 口 是 M 的 开 子 集 ,p 是 从 口 到 8 中 开 子 集 CDU)
E

切 的 两 个 坐 标 卡 (U,g),(7,43 称 为 是 C“ 相 容 的 , 如 果 当 口
【 绍

E
Page-615
E E

列 两 种 方 式 构 造 出 不 同 的 向 量 丛 :Y # E 5, 取 过 的 法 线 为 纤 维
七 ; 或 取 过 的 切 平 面 为 Ep, 前 者 的 纤 维 型 是 R「, 而 后 者 是 R「、

切 空 间 与 切 丛

我 们 可 以 通 过 坐 标 卡 把 Banach 空间中曲线相切的概含诱导
到 流 形 上 , 从 而 建 立 切 空 间 与 切 东 .

空
江 江 32 技 45 林2标 史 江 E
力 则 称 曲 线 以 为 基 点 . 设 c,es 是 以 8 为 基 点 的 两 条 曲 线 , 并 且
d

E
( 即 Banach 空 间 中 的 曲 线 8“e 与 c: 在 0 点 相 切 , 则 称 切 上 的
曲 线 c 与 cs 在 丿 点 相 切 .

注 意 , 利 用 流 形 史 上 坐 标 卡 的 相 宰 性 ,ei 与 c 在 点 的 相 切 性
医 沥
标 卡 , E D [ Da- 设 D(p。 c(0) 二 DCp。 c(0). 由 于 .

画 6 二 ( 8 p D 王 一 工 2
E
E t

一 DC8 。 877Cp ( 力 ) D 。co(0) 一 D 。 67(0).

这 桦 , 我 们 在 同 基 点 的 曲 线 之 间 规 定 了 一 个 等 价 关 系 :a! 一 c
E
力 点 的 一 个 切 向 量

E
E

E 林口
E 沥 朐
E
E
Page-616
E

是 旦 上 的 C 微 分 同 胚 .

E
为 一 个 C 坐 标 綦 , 如 果

【 江 c

0

p
版 的 Cr 坐 标 系 . 队 上 C“ 坐 标 系 的 一 个 等 价 类 三 称 作 元 的 一 个 C“
E 吴 沥 a
的 - 个 极 火 Cr 坐 标 系 , 而 (U,8》 E - 必 称 作 一 个 容 许 坐 标 卡 .

如 果 在 吊 上 给 定 了 一 个 C“ 微 分 结 构 则 称 6 一 ( 史 ,3 是 一
E t
E 育
( 注 意 , 由 吊 的 连 通 性 和 坐 标 卡 的 相 容 性 易 知 ,5 与 坂 标 卡 的 选 取
无 关 ). 特 别 , 当 8 为 Hilbert 空 间 时 , 称 M 为 Hilbert 流 渺 ; 而 当 5
Eoratstsaediuhe t

E 玟 吴 a
以 把 与 a 等 价 的 全 部 坐 标 系 合 起 来 而 得 到 一 个 极 大 坐 标 系 , 从 而
生 成 队 上 的 一 个 微 分 结 构 . 图 此 , 只 须 给 定 M 上 一 个 特 定 的 坐 标
系 , 就 可 以 决 定 这 个 微 分 流 形 .

E
应 地 得 到 C“ 微 分 流 形 . C“ 微 分 流 形 也 称 为 光 滑 流 形
国 0n
( 是 旦 的 佰 同 映 射 ), 这 个 坐 标 卡 显 然 就 构 成 日 上 的 一 个 C“ 坐 标 .
系 . 因 此 , 任 何 Banach 空 间 都 是 装 备 在 它 自 身 上 的 一 个 光 谊
Banach 流 形

(2) 8 一 tz E Rr++llia 一 1} 是 一 个 n 雕 流 形 . 事 实 上 , 记
足 一 人 ,0,.0},5 二 { 一 一 0,.0} 分 别 是 “ 的 北 极 与 家 极 , 取
坐 标 卡 (SeNtN}, ),《SeMtS}, , 其 中
Page-617
E 291

E E t 园

E
E 林2

2

E
E 河
人
空 间 . 这 桦 ,Y [c]Jp E Tp(M3, 有 二 DCp。c)(0) E 为 与 之 对 应 .
E 2 怡 52
下 而 DGp。c)(0) 二 o, 从 c 可 得 [c]. 于 是 , 得 到 T「( 肌 ) 到 B, 的 一
沥
E 不 育 玟

再 来 证 明 (TM ,z,M) 是 向 量 丛 , 其 中 投 影 rxTM - M 通 过
0 p
外 ) |a E 4}, 则 出 下 文 的 定 理 B 10 可 知 , 吊 一 {(TUe,7W)laE 4}
E 沥 李 沥
t 的 切 晃 封 -

切 映 射

E 木 用
E 绍 匹 t t 许 儿 应 沥 希 余 江 逊
l t 永 P 林史 日
2 2s e G 30
注 意

伟 东 3 才
则 由 Banach 空 间 中 导 算 子 的 链 式 法 则 可 得
DC8 。 一 67(0) 二 D 。 丁 g (6))。DCp。c2(0),
0
Page-618
E E 吴

E
0

E 叽〈渥】,""艾”鼻】〉=

。

沥 招
容 易 得 出

8 RM0) 一 RM(0j, 8 0 一 命
是 C“ 微 分 同 胚 .

E E 绅_〕
E i 左

流 形 间 的 映 射

定 义 A.5 “ 设 友 , 分 别 是 装 备 在 Banach 空 间 4, 上 的 C
E
氓 53 诊 2 52 技 沥 i
E u 坤 林 f

E
是 C“ 的 .

E

E 胡2 东 技
医 5 标 g
E

E 丶 沥 标 育 2 林 述 林 标 江
E 沥 |

定 义 A.8 “ 设 肌 , 义 是 Cr 徽 分 流 形 , 称 M ~ 丶 为 C 微 分
同 胚 , 如 果 它 是 一 一 的 C 映 射 , 并 且 其 逆 映 射 广 !:N 一 吊 也 是 Cr
的 . 如 果 两 个 流 形 间 存 在 一 个 C 徽 分 同 胚 , 则 称 这 两 个 流 形 C“ 微
分 同 胚 .

子 流 形 与 积 流 形
类 似 于 向 量 空 间 的 子 空 间 与 乘 积 空 间 , 微 分 流 形 也 存 在 予 流
Page-619
用 录 B “ 切 坚 与 奶 限 射 , 向 重 姬 及 其 沛 , 描 入 与 抒 堵

J 【
D 技 8 ( 坂 沥 30

E
E

玟
为 了 的 切 晖 射 . 有 时 也 把 T 记 为 4 或 了 .

E 诊 0 e d 江
E

E 9 2592 中 东 技 2

国 【
E 0 日
许

[ 河 江5 林 玟

月

林一 人 一 目

沥

是 T 一 TK 的 C 一 晓 射 . 1

E 河
射 .

万
沥

E t

[ 李
木 刀 一 一 刑
E 林

E 万
0

E
Page-620
贺 录 卫 “ 切 此 与 切 院 射 , 向 量 李 及 其 流 , 揉 人 与 抒 故 2

n
n
E
E
医 沥 伟 林 才
[ 国 , t 圆 19 顺 阮

玟
东 结 构 , 例 如 , 叟 AC{#} X B) 为 过 E 伟 的 纤 维 , 则 恶 是 TW 与
L 志 Eopettiot 2s 圆

引 理 B.9 _ 设 仁 和 伟 「 分 别 是 Banach 空 间 日 和 B 的 开 子 集 ,
乙 印 一 仁 「 是 Cr 徽 分 同 胚 , 则 Tf:V7 X 8 一 IP「 X B 是 局 部 向 量
东 同 构 映 射 .

证 明 “ 因 为

【 095 才 62

E 林林
玟
也 是 一 个 C“ 局 部 向 量 丛 映 射 , 从 而 T/ 是 一 向 量 丛 同 松 映 射 , ‖

现 在 可 以 证 明

E 二 209 才
} 是 胆 的 一 个 Cr 坐 标 系 , 则 T.ar 二 ((T(Ua) ,TCR2)[aE 心 是
切 丛 TM 的 C! 向 量 丛 坐 标 系 , 从 而 TWU 是 C““* 向 量 仰 . 此 外 , 如
E 沥 李 沥 江
团 0 园 人 )
E 肖 技 川 E 沥

E 732 述

是 局 部 向 量 丛 , 并 且

E |T'″(7〈U-】门7(U′〉) 一 工 (8 “ 秀 |T′"(7(″″)门T【U′〉〉
( 见 定 义 B 7 下 的 性 质 (3) 和 (5)). 注 意 8 。 8 是 9(U。) 所 在 的
Page-621
E 标

E
E 林
E 林c 园
国 江2 p
河 江
E
E 技 5
【 a 浩
E 应 辽 述 逊 s
CU, 外 , 和 8 的 直 和 分 解 8 = B 申 Ba, 使 得 E D, 干 一
E 林 0203 0 相
p 林 林 国 吊
E 一
E 沥 d
个 微 分 流 形 , 并 旦 它 的 微 分 结 构 可 由 下 面 的 坂 标 系 生 成 :
d
s 吴
E 水
流 形 的 方 法
E z i 朐
t 小 不
E R 西 一 明 春 述
分 流 形 , 称 它 为 M 与 N 的 积 流 形 , 仍 记 为 友 X N-

附 星 B “ 切 丛 与 切 映 射 , 向 量 场 及 其 流 , 漫 人 与 漫 盐

向 量 丛 是 积 流 形 的 推 广 , 而 流 形 上 的 向 量 场 , 则 是 作 为 一 个 特
E u
Page-622
E E t

Banach 空 间 五 中 的 开 二 集 9(7CUa) 门 T(Do)) 到 自 身 的 〇 映 射 .
EEi 2 匹 a 北 2 圆 政 林s 吊

E 一
E 沥 沥 李 b

附 注 B. 11 “ 前 面 已 经 提 到 , 流 形 间 的 切 映 射 是 Banach 空 间
中 导 算 子 的 推 广 . 闷 此 , 有 时 把 它 狞 为 导 昭 射 - 设 M 和 丶 是 充 分 光
玟
江 仁 a e
续 求 二 防 导 晃 射 7(TA 一 72f7(TM3 一 (CTN) , 并 可 通 推 地 定
Eolss

向 量 场 及 其 流

王
6sneaeolols l

玟

E 万
E
量 场 . M 上 一 切 Cr 向 量 场 的 集 合 记 为 -gT“C2.

[ 达
E 应泓
园
椿 p

E 林Ba 李 梁 d
王
映 射

E 林林 i

E 肖 吴 0 李 水 5 沥 技 0 一

E
E
Page-623
E 附 录 B “ 切 屿 与 切 际 驯 , 名 量 圭 友 其 浇 浦 人 与 深 范

Banach 空 间 中 的 隐 函 数 ( 反 函 数 ) 定 理 以 及 局 部 潍 入 和 浸 盎 定 理
E 沥 吊 c 胡
E

向 量 丛

E 孙 林团
开 集 , 秒 乙 X 日 为 局 部 向 量 丛 , 并 称 U 为 廉 空 间 , 它 可 以 等 同 于
E i 国 林
团 伟
人

E 园 一 一
一 个 开 子 流 形 .

定 义 B.2 设 D X 旦 和 D X X 都 是 局 部 向 量 仰 . 如 果 眸 射
E 「

E
E
这 个 映 射 还 是 一 一 的 , 则 称 它 为 C 局 邵 向 量 丛 同 构

E
t
E 东

类 似 于 微 分 流 形 的 定 义 , 我 们 可 以 把 局 部 向 量 丛 粘 接 起 来 , 得
E

定 义 B.3 设 5 是 集 合 . 称 (P7, 旭 是 5 的 一 个 局 部 向 重 仰 卡 ,
E 沥
(U,B 可 能 与 g 有 关 ) 称 这 桦 的 卡 集 畅 一 (CWo,8)|a E 4 是 5

的 C7 向 量 丛 垫 标 系 , 如 果
E05 林 t A 沥
(2) 著 (Wav8),(Wo8) E 男 , Ja n W 夺 切 , 则 (
Page-624
E 78 刑 E

E 才 c 木
d 王 tpu 莲 E 林 命
E
E
E 0020 [ 沥
技
E 伟 团 沥 林 5 罡伟 技
朱5 林 a 技 伯
E
蛊 0
p
含 0 的 开 区 间 , 都 有 J CC 厂 且 al, = .
取 切 上 的 坐 标 卡 (C,9) ,po E U 设 向 量 场 &: 伦 一 TWU 有 局 部
玟

表 示 相 对 照 , 就 得 到 Banach 空 间 中 关 于 (p。 ) 的 微 分 方 程 初 值 闭
5

青 E
命 ( 201 水
浩
玟
E

利 用 Banach 空 间 中 微 分 方 程 初 值 问 题 解 的 存 在 和 唯 一 性 定
理 , 可 以 得 到

E 一 吴 沥 孝 小
E .

E 55 林招 人
沥 达
出 的 切 向 量 相 君 合 , 这 与 欧 氏 空 间 中 向 量 场 的 流 的 概 念 相 一 致

[ 怀 u e 扬
Page-625
附 源 B 切 世 与 切 险 射 , 名 量 行 及 其 流 , 清 人 与 渥 敏 289

L 林t t

称 5 的 两 个 向 量 从 坂 标 系 88 与 田 蔽 价 , 如 果 幺 lU 口 ; 还 是
l 吴
E 应 不 不 不 i
伟
医 i 2

[

0 微分流形

ECutopasaopo 志 s t

E 胡 c 李
达

0
E

E 江
E 述
E n U
Ep,a 线 性 拓 扑 同 构 ( 作 为 线 性 空 间 是 同 构 的 , 作 为 拓 扑 空 间 是 同
胃 的 . 在 这 个 意 义 下 , 可 以 认 为 E 与 8 无 关 , 可 记 为 E

【 坂 孙 育 的婵…M(底空间) E
可 定 义 投 影

E i
i
有 的 Banach 结 构 称 为 纤 维 型 . 在 有 些 书 上 , 把 上 面 的 性 质 作 为 向
Eatpys t t
E

从几何上粗略地说,以流形M为底空间的向丛y就是在M

上 的 每 一 点 “ 附 着 “ 一 个 以 该 点 为 零 元 素 的 Banach 空 间 , 而 不 囡 点
E
Page-626
E E

E 兰 达
E
E 许 达 一 颂 技 一 t
[ 一 才
满 足 微 分 方 程
D 沥 伟 3
E 林 i 李
E 水
E 梁 y 关 达
、 友 一 Tg*X(g(za)), 乙 E 伊 口 巴
E 沥 g
E 沥 林吴 八 技
Da(o) 一 E 0
E 汀 沥
u 诊 李 才 一
【 吴 2s 沥 技 沥 3 振
它 成 为 积 分 曲 线 的 条 件 是

a
因 此 可 以 说 , 对 于 维 流 形 上 的 向 量 场 而 言 ; 其 积 分 曲 线 的 局 部 表
E
E
隐 函 数 定 理 是 徽 分 学 中 最 重 要 的 定 理 之 一 . 设 R“ -~ R“ 是
E 伟 c 仪n t
Page-627
E E

列 两 种 方 式 构 造 出 不 同 的 向 量 丛 :Y # E 5, 取 过 的 法 线 为 纤 维
七 ; 或 取 过 的 切 平 面 为 Ep, 前 者 的 纤 维 型 是 R「, 而 后 者 是 R「、

切 空 间 与 切 丛

我 们 可 以 通 过 坐 标 卡 把 Banach 空间中曲线相切的概含诱导
到 流 形 上 , 从 而 建 立 切 空 间 与 切 东 .

空
江 江 32 技 45 林2标 史 江 E
力 则 称 曲 线 以 为 基 点 . 设 c,es 是 以 8 为 基 点 的 两 条 曲 线 , 并 且
d

E
( 即 Banach 空 间 中 的 曲 线 8“e 与 c: 在 0 点 相 切 , 则 称 切 上 的
曲 线 c 与 cs 在 丿 点 相 切 .

注 意 , 利 用 流 形 史 上 坐 标 卡 的 相 宰 性 ,ei 与 c 在 点 的 相 切 性
医 沥
标 卡 , E D [ Da- 设 D(p。 c(0) 二 DCp。 c(0). 由 于 .

画 6 二 ( 8 p D 王 一 工 2
E
E t

一 DC8 。 877Cp ( 力 ) D 。co(0) 一 D 。 67(0).

这 桦 , 我 们 在 同 基 点 的 曲 线 之 间 规 定 了 一 个 等 价 关 系 :a! 一 c
E
力 点 的 一 个 切 向 量

E
E

E 林口
E 沥 朐
E
E
Page-628
E 291

E E t 园

E
E 林2

2

E
E 河
人
空 间 . 这 桦 ,Y [c]Jp E Tp(M3, 有 二 DCp。c)(0) E 为 与 之 对 应 .
E 2 怡 52
下 而 DGp。c)(0) 二 o, 从 c 可 得 [c]. 于 是 , 得 到 T「( 肌 ) 到 B, 的 一
沥
E 不 育 玟

再 来 证 明 (TM ,z,M) 是 向 量 丛 , 其 中 投 影 rxTM - M 通 过
0 p
外 ) |a E 4}, 则 出 下 文 的 定 理 B 10 可 知 , 吊 一 {(TUe,7W)laE 4}
E 沥 李 沥
t 的 切 晃 封 -

切 映 射

E 木 用
E 绍 匹 t t 许 儿 应 沥 希 余 江 逊
l t 永 P 林史 日
2 2s e G 30
注 意

伟 东 3 才
则 由 Banach 空 间 中 导 算 子 的 链 式 法 则 可 得
DC8 。 一 67(0) 二 D 。 丁 g (6))。DCp。c2(0),
0
Page-629
E 2标 E

E
E 门 逊
E '

王 2 颂 木 朐
E 胡2 阮
s 沥 c t 万 玟 唐
E

E 沥 林

这 称 为 ( 局 部 ) 正 则 漫 羞 ( 投 影 ).

一
E 不
过 坐 标 卡 推 广 到 微 分 草 形 上 , 并 田 此 得 到 构 造 ( 或 判 断 ) 于 流 形 的
方 法 . 对 于 无 穷 维 的 Banach 空 间 , 仅 有 DfCz) 的 单 射 ( 或 满 射 ) 条 “
件 是 不 够 的 , 要 附 加 适 当 的 可 裂 性 条 件 . 首 先 给 出 下 面 的

E L 余 林 二 t 国
a p

E 林 L 玟 不 沥
E
c
E
[ 人 才
E

n
[ 述
E 0 水
[ 许 芸
E 伟 沥 扬
E e 沥
Page-630
用 录 B “ 切 坚 与 奶 限 射 , 向 重 姬 及 其 沛 , 描 入 与 抒 堵

J 【
D 技 8 ( 坂 沥 30

E
E

玟
为 了 的 切 晖 射 . 有 时 也 把 T 记 为 4 或 了 .

E 诊 0 e d 江
E

E 9 2592 中 东 技 2

国 【
E 0 日
许

[ 河 江5 林 玟

月

林一 人 一 目

沥

是 T 一 TK 的 C 一 晓 射 . 1

E 河
射 .

万
沥

E t

[ 李
木 刀 一 一 刑
E 林

E 万
0

E
Page-631
E 附 录 B “ 切 坤 与 切 陋 财 , 吊 探 埃 及 其 旁 , 淑 人 不 清 盎

[ 仪 2

E 亚 园 芸 人 0 技 沥 育
炼 (CDJCuo)) 人 七 的 代 数 与 拓 扑 同 构 . 令
E 李 李 e 林 林 沥 动 03 河
E 沥

E
E

0 国
是 口 X 卷 一 吊 田 P 一 丁 的 Cr 徽 分 同 背 , 由 隐 函 数 定 理 , 存 在 开
水 2 E 021
[ 7 一 技 一 7
[

寸
E 肖 口 林
E 8 应n 沥
E
n
E

E 国 0

E 技 月
EE

D [

0 Dz/(吻)〕
0 E
王
[ 2
EiSE
[ 吊 小 沥 2
E 林2
Page-632
贺 录 卫 “ 切 此 与 切 院 射 , 向 量 李 及 其 流 , 揉 人 与 抒 故 2

n
n
E
E
医 沥 伟 林 才
[ 国 , t 圆 19 顺 阮

玟
东 结 构 , 例 如 , 叟 AC{#} X B) 为 过 E 伟 的 纤 维 , 则 恶 是 TW 与
L 志 Eopettiot 2s 圆

引 理 B.9 _ 设 仁 和 伟 「 分 别 是 Banach 空 间 日 和 B 的 开 子 集 ,
乙 印 一 仁 「 是 Cr 徽 分 同 胚 , 则 Tf:V7 X 8 一 IP「 X B 是 局 部 向 量
东 同 构 映 射 .

证 明 “ 因 为

【 095 才 62

E 林林
玟
也 是 一 个 C“ 局 部 向 量 丛 映 射 , 从 而 T/ 是 一 向 量 丛 同 松 映 射 , ‖

现 在 可 以 证 明

E 二 209 才
} 是 胆 的 一 个 Cr 坐 标 系 , 则 T.ar 二 ((T(Ua) ,TCR2)[aE 心 是
切 丛 TM 的 C! 向 量 丛 坐 标 系 , 从 而 TWU 是 C““* 向 量 仰 . 此 外 , 如
E 沥 李 沥 江
团 0 园 人 )
E 肖 技 川 E 沥

E 732 述

是 局 部 向 量 丛 , 并 且

E |T'″(7〈U-】门7(U′〉) 一 工 (8 “ 秀 |T′"(7(″″)门T【U′〉〉
( 见 定 义 B 7 下 的 性 质 (3) 和 (5)). 注 意 8 。 8 是 9(U。) 所 在 的
Page-633
E 林 E

现 在 把 上 面 的 结 果 推 广 到 徽 分 流 形 之 间 的 晔 射 广 M - ,
医 罡 医 命 述
件是合理的.

E 尘 史 希7 述 355 河 招 秉 5
沥 林

E
是 满 射 , 而 且 ker(TA7) 作 为 TAM 的 闭 子 空 间 是 可 裂 的 . 在 流 形
n
氓

定 理 B. 23 “ 设 传 ,W 是 Banach 流 形 , /:M 一 N 是 C“ 映 射 (r
E

【 化 ; 江 i

0 沥 en 2 t
E l 标 林技 t
包 吾 际 射 A * (,03;

[ 是N E
【

证 明 “CU) 与 (2) 的 等 价 性 可 由 定 理 B. 20 得 到 .(2 与 (3) 的
等 价 性 可 由 子 流 形 定 义 A. 9 得 到 , 注 意 Y 是 N 中 的 开 集 【

E
E 伟
E 林 i
的 子 流 形 . 事 实 上 , 涉 入 的 局 部 单 射 性 不 能 保 证 整 体 的 单 射 性 .
例 如 , 由 平 面 极 坐 标 方 程 r 一 cos2g 定 义 的 映 射 八 S「 - R 是 一 个
E 技
E 政 吊
i 江 吊 p 沥
( 见 图 2). 上 面 两 例 中 的 问 题 都 出 在 R 原 点 附 近 的 邻 域 内 、
Page-634
E

图 1 图 2
丁 D 利 用 了 从 友 诱 导 的 拓 扑 与 /CM) 作 为 * 的 孙 集 而 获 得 的 承
扑 是 不 同 的 , 在 前 一 种 拓 扑 下 ,f(M) 作 成 一 微 分 流 形 , 因 而 有 时
称 单 射 浸 入 的 象 集 fCM) 是 一 个 漫 人 子 流 形 ; 但 是 在 后 一 种 拓 扑
2 园
E 江 唐

E
相 关 拓 扑 下 ) 的 同 胚 , 则 称 它 是 一 个 崴 人 . 此 时 , 称 一 M 是 N 的
E 志 八 汪

容 易 证 明 下 面 的

定 理 B. 26 “ 设 广 Mf 一 N 是 单 值 的 漫 入 , 若 它 是 M 到 f )
E 育

E 胡2
0 沥 1 河
一 个 C“ 子 流 形

53 圆圆 丿 P 孙 32 育 5 江江 林 仪 |

E 沥
E 述 林 一
R 为 了 的 一 切 正 则 值 的 集 合 * 则 定 理 B: 27 有 如 下 等 价 的 陈 述 :

E 沥 2
Page-635
E E t

Banach 空 间 五 中 的 开 二 集 9(7CUa) 门 T(Do)) 到 自 身 的 〇 映 射 .
EEi 2 匹 a 北 2 圆 政 林s 吊

E 一
E 沥 沥 李 b

附 注 B. 11 “ 前 面 已 经 提 到 , 流 形 间 的 切 映 射 是 Banach 空 间
中 导 算 子 的 推 广 . 闷 此 , 有 时 把 它 狞 为 导 昭 射 - 设 M 和 丶 是 充 分 光
玟
江 仁 a e
续 求 二 防 导 晃 射 7(TA 一 72f7(TM3 一 (CTN) , 并 可 通 推 地 定
Eolss

向 量 场 及 其 流

王
6sneaeolols l

玟

E 万
E
量 场 . M 上 一 切 Cr 向 量 场 的 集 合 记 为 -gT“C2.

[ 达
E 应泓
园
椿 p

E 林Ba 李 梁 d
王
映 射

E 林林 i

E 肖 吴 0 李 水 5 沥 技 0 一

E
E
Page-636
E 江 浩 育 水 江 页 大

E

附 录 C Thom 横 戬 定 理

在 寻 拖 结 构 不 稳 定 向 量 场 的 普 适 开 折 时 , 有 时 要 确 定 开 折 中
的 通 有 族 (generic family), 或 称 为 一 般 族 , 非 退 化 族 . 为 此 , 零 要 利
用 Thom 模 截 定 理 ( 它 是 从 著 名 的 Sard 定 理 导 出 的 . 事 实 上 , 为 丁
玟
e
E

E ′ 日

E
E

定 友 C. 1 ( 骏 拓 扑 , 即 compact-open 拓 扑 》 设 孔 E CrCM,
),(U,9 和 (C7, 办 分 别 是 肖 和 义的 容 许 坐 标 卡 ; 令 史 口 口 是 紧
E 林伟 关 途 技 5 江2 扬 广

a c

E 0

E

E 红 L 木 c
包 含 有 限 个 这 种 集 司 交 集 的 集 合 都 是 了 的 一 个 邻 域 . 所 得 的 拓 扑
室 间 记 为 Cr(M,N).

E 出 伟 c
{GUn,8o |a E 4} 是 余 的 一 个 局 部 有 限 坐 标 系 , 即 M 的 每 一 点 都
有 一 个 邹 域 , 它 只 与 有 限 个 Dn 相 京 ; 记 2 一 (Kale E 4},Ku 人
Ue 是 胡 上 的 紧 集 ; 画 二 {CVaugp) | B} 是 丶 的 坐 标 系 , 对 任 一
D
Page-637
E 78 刑 E

E 才 c 木
d 王 tpu 莲 E 林 命
E
E
E 0020 [ 沥
技
E 伟 团 沥 林 5 罡伟 技
朱5 林 a 技 伯
E
蛊 0
p
含 0 的 开 区 间 , 都 有 J CC 厂 且 al, = .
取 切 上 的 坐 标 卡 (C,9) ,po E U 设 向 量 场 &: 伦 一 TWU 有 局 部
玟

表 示 相 对 照 , 就 得 到 Banach 空 间 中 关 于 (p。 ) 的 微 分 方 程 初 值 闭
5

青 E
命 ( 201 水
浩
玟
E

利 用 Banach 空 间 中 微 分 方 程 初 值 问 题 解 的 存 在 和 唯 一 性 定
理 , 可 以 得 到

E 一 吴 沥 孝 小
E .

E 55 林招 人
沥 达
出 的 切 向 量 相 君 合 , 这 与 欧 氏 空 间 中 向 量 场 的 流 的 概 念 相 一 致

[ 怀 u e 扬
Page-638
E E

E 兰 达
E
E 许 达 一 颂 技 一 t
[ 一 才
满 足 微 分 方 程
D 沥 伟 3
E 林 i 李
E 水
E 梁 y 关 达
、 友 一 Tg*X(g(za)), 乙 E 伊 口 巴
E 沥 g
E 沥 林吴 八 技
Da(o) 一 E 0
E 汀 沥
u 诊 李 才 一
【 吴 2s 沥 技 沥 3 振
它 成 为 积 分 曲 线 的 条 件 是

a
因 此 可 以 说 , 对 于 维 流 形 上 的 向 量 场 而 言 ; 其 积 分 曲 线 的 局 部 表
E
E
隐 函 数 定 理 是 徽 分 学 中 最 重 要 的 定 理 之 一 . 设 R“ -~ R“ 是
E 伟 c 仪n t
Page-639
E E

E 伟 技
u

0

E 不
d

L 孝 莲 园 口 河
E t
沥

E
E 余 c 江
2 水 吴
C“ 胥 射 在 “ 无 穷 远 “ 的 伯 质 , 强 拓 扑 就 成 为 需 要 的 了 , 注 意 , 在 办 ˇ
玟
E
集 在 其 中 稠 密 . 这 对 研 究 通 有 性 质 是 很 重 要 的 , 如 无 牺 别 声 明 ,
下 文 中 都 取 强 拓 扑 , 并 把 CS(MM , 简 记 为 CrCM ,MD.

U 途 i

一 怀 国 0
1 位 7 妙 svdimM 一 十 eo. 则 Gr(M,N) 在 Gr(MM,N) 中 稠 密 , 其 中
E

g 扬 玖 河 厂 标
紧 集 5 【

2
胚 于 一 个 C“ 微 分 流 形 ,

0
它 们 C 微 分 同 胚 ,

定 理 C.7 (Whitney 定 理 〉 设 1 口 r 口 十 co, 则 任 何 a 维 的 C“
Page-640
E

微 分 流 形 都 C7 微 分 同 胚 于 R*“「 的 一 个 闭 子 流 形 , 【

E 0 河 沥 i 沥
到 C“ , 二 不 是 一 个 很 严 重 的 事 情 . 今 后 我 们 将 经 常 作 这 样 的 假 定 .
虽 然 ^ 维 微 分 流 形 是 维 欧 氏 空 间 非 常 一 舱 的 推 广 , 但 定 理 C. 7 说
basaeetu pote y 林水 d

射 式 e6) 流 形

E E
E 伟
E 0

E 吊
E 应

林医 一
t

显 然 , 上 面 的 定 义 与 坐 标 卡 的 选 取 无 关 , 从 而 也 与 口 的 选 取 无
关 . 因 此 记

E 仪
E 林c 政b 河 c
记 力 (M ,N) 为 全 体 从 此 到 为 的 r-jet 之 集 合 , 我 们 把 一 (M,
E 沥沙 技 招 朐
Epeuuepnutisg u
现 在 考 虑 一 个 特 殊 情 形 :M 一 R“,N 一 R“. 此 时 简 记
E d
E 5 林 L 2 育 t 5
式 就 给 出 了 在 z 点 的 r-jet 一 种 自 然 的 表 示 . 这 个 从 R“ 到 R“ 的 多

项 式 眺 射 可 由 一 在 z 点 直 到 r 阶 ( 包 括 0 阶 ) 导 算 予 所 唯 一 决 定 . 这
医 江志
Page-641
E 2标 E

E
E 门 逊
E '

王 2 颂 木 朐
E 胡2 阮
s 沥 c t 万 玟 唐
E

E 沥 林

这 称 为 ( 局 部 ) 正 则 漫 羞 ( 投 影 ).

一
E 不
过 坐 标 卡 推 广 到 微 分 草 形 上 , 并 田 此 得 到 构 造 ( 或 判 断 ) 于 流 形 的
方 法 . 对 于 无 穷 维 的 Banach 空 间 , 仅 有 DfCz) 的 单 射 ( 或 满 射 ) 条 “
件 是 不 够 的 , 要 附 加 适 当 的 可 裂 性 条 件 . 首 先 给 出 下 面 的

E L 余 林 二 t 国
a p

E 林 L 玟 不 沥
E
c
E
[ 人 才
E

n
[ 述
E 0 水
[ 许 芸
E 伟 沥 扬
E e 沥
Page-642
E

E 振 目 沥
E

医 林
量 空 间 . 这 说 明 , 对 Pr(m, 中 的 每 一 个 元 素 , 都 有 且 仅 有 一 个
E 木 c l 园
E t 之
E 林
[ 阮 河
E 伟
E
E
技t 技 李
E d d tot d
E d D 沥 t
氏 空 间 同 构 , 所 以 ,yvCLD,V)) 就 是 一 (M ,N) 的 一 个 坐 标 卡 , 这
E
河 t 仪 2
Eapyteu2 沥 河

Thom 槲 戢 定 理

E a ritases e 沥 5
p

E 吴 i
E
空 间 中 树 子 流 形 的 机 截 性 .

E 北 伟 0 5
E 技
E 李 述
Page-643
E 附 录 B “ 切 坤 与 切 陋 财 , 吊 探 埃 及 其 旁 , 淑 人 不 清 盎

[ 仪 2

E 亚 园 芸 人 0 技 沥 育
炼 (CDJCuo)) 人 七 的 代 数 与 拓 扑 同 构 . 令
E 李 李 e 林 林 沥 动 03 河
E 沥

E
E

0 国
是 口 X 卷 一 吊 田 P 一 丁 的 Cr 徽 分 同 背 , 由 隐 函 数 定 理 , 存 在 开
水 2 E 021
[ 7 一 技 一 7
[

寸
E 肖 口 林
E 8 应n 沥
E
n
E

E 国 0

E 技 月
EE

D [

0 Dz/(吻)〕
0 E
王
[ 2
EiSE
[ 吊 小 沥 2
E 林2
Page-644
E E

E t p
E d 化 明 ,
E 心 c 林5 浩 a

(D ( 一 (TpM) 十 Tyek 二 TyeN, 面 东

0 芸 李
E 怀 吴 沥

林
E 江 s

E
E 应

102 怡 u 园
E E
Et 述 仪 沥 e 纳 d

由 于 机 蒙 相 交 性 与 土 标 卡 的 选 取 无 关 , 所 以 当 史 与 丶 都 为 有
E

英
E t 江 5 脱
心 点 附 近 有 局 部 坐 标 z「,.,z“, X 在 zo) 点 附 近 有 局 部 坐 标

E 沥 u i

沥 扬 2
区

E

江
E

E 林0 汀

E 达
定 理 C.14 CThom 横 戳 定 理 ) .w“(M ,Ny4) 是 CrCM,N) 中

E s i 沥
Page-645
E 林 E

现 在 把 上 面 的 结 果 推 广 到 徽 分 流 形 之 间 的 晔 射 广 M - ,
医 罡 医 命 述
件是合理的.

E 尘 史 希7 述 355 河 招 秉 5
沥 林

E
是 满 射 , 而 且 ker(TA7) 作 为 TAM 的 闭 子 空 间 是 可 裂 的 . 在 流 形
n
氓

定 理 B. 23 “ 设 传 ,W 是 Banach 流 形 , /:M 一 N 是 C“ 映 射 (r
E

【 化 ; 江 i

0 沥 en 2 t
E l 标 林技 t
包 吾 际 射 A * (,03;

[ 是N E
【

证 明 “CU) 与 (2) 的 等 价 性 可 由 定 理 B. 20 得 到 .(2 与 (3) 的
等 价 性 可 由 子 流 形 定 义 A. 9 得 到 , 注 意 Y 是 N 中 的 开 集 【

E
E 伟
E 林 i
的 子 流 形 . 事 实 上 , 涉 入 的 局 部 单 射 性 不 能 保 证 整 体 的 单 射 性 .
例 如 , 由 平 面 极 坐 标 方 程 r 一 cos2g 定 义 的 映 射 八 S「 - R 是 一 个
E 技
E 政 吊
i 江 吊 p 沥
( 见 图 2). 上 面 两 例 中 的 问 题 都 出 在 R 原 点 附 近 的 邻 域 内 、
Page-646
E E

果 么 是 闭 子 流 形 , 则 它 还 是 开 的 ,

E
E 志 技 e 医
成 为 横 截 , 而 原 来 樟 截 的 晔 射 休 可 保 持 横 戳 . 因 此 “ 利 用 槲 戳 性 可
e 个 中 一

Thom 横 截 定 理 可 以 推 广 到 jet 形 式 . 这 样 就 可 把 无 限 维 空 间
CrQd , 中 的 通 有 性 闭 题 , 转 化 到 有 限 维 空 间 J7CM ,N) 中 的 模
截 闭 题 `

国 u t
水
E }

E 技t 朋

E 河 园
|

证 明 可 见 [Hirpp80 一 81] 下 面 的 定 理 对 于 确 定 子 流 形 的 余
维 数 是 重 要 的 .

E 李
了 与 么 横 戳 , 则 广 “(4) 是 仁 的 子 流 形 . 如 果 4 在 丶 中 有 有 限 余
维 , 则

codim( 广 1(4D) 二 eodim(4)。 【
特 别 地 , 当 人 育 一 丶 是 浸 草 时 , 条 件 “ 历 与 横 截 “ 总 是 满 趸
医
Page-647
E

图 1 图 2
丁 D 利 用 了 从 友 诱 导 的 拓 扑 与 /CM) 作 为 * 的 孙 集 而 获 得 的 承
扑 是 不 同 的 , 在 前 一 种 拓 扑 下 ,f(M) 作 成 一 微 分 流 形 , 因 而 有 时
称 单 射 浸 入 的 象 集 fCM) 是 一 个 漫 人 子 流 形 ; 但 是 在 后 一 种 拓 扑
2 园
E 江 唐

E
相 关 拓 扑 下 ) 的 同 胚 , 则 称 它 是 一 个 崴 人 . 此 时 , 称 一 M 是 N 的
E 志 八 汪

容 易 证 明 下 面 的

定 理 B. 26 “ 设 广 Mf 一 N 是 单 值 的 漫 入 , 若 它 是 M 到 f )
E 育

E 胡2
0 沥 1 河
一 个 C“ 子 流 形

53 圆圆 丿 P 孙 32 育 5 江江 林 仪 |

E 沥
E 述 林 一
R 为 了 的 一 切 正 则 值 的 集 合 * 则 定 理 B: 27 有 如 下 等 价 的 陈 述 :

E 沥 2
Page-648
医 招 怀 生 人

[A1] Armold V L Geometrie Methods in the Theory of Ordinary Diflerental
E
E 东 不 e
E
〖nouoe 志 怀 坂 园 0 技砂
E

E
E 林 日
E
Program for Sci Tranal , Wiley, 1973
[AMRJ Abraham R,Marsden ] E, Ratiu T Manilolds,Tensor Analysis,
E 口
Publishing Company, Inc, 1983
E 江 技
of the coefticients from an equitbriam Position al focus or ceoter type、
E 沥 沥 E 李 应 一
E
E
E 沥
[BC2] 一 一 . The dynamics of the Htnon mag,Aan Math。 1991, 133: 73 一
E
E
3 印 永 X 前 日
[BL] Bonin G Legalt J Comparision de l methode des constantes de Lia-
Punov et la Mfreation de Hopf,Canad,Math、Bull 1988, 31(2): 200
g
E 林 伟
Page-649
E 江 浩 育 水 江 页 大

E

附 录 C Thom 横 戬 定 理

在 寻 拖 结 构 不 稳 定 向 量 场 的 普 适 开 折 时 , 有 时 要 确 定 开 折 中
的 通 有 族 (generic family), 或 称 为 一 般 族 , 非 退 化 族 . 为 此 , 零 要 利
用 Thom 模 截 定 理 ( 它 是 从 著 名 的 Sard 定 理 导 出 的 . 事 实 上 , 为 丁
玟
e
E

E ′ 日

E
E

定 友 C. 1 ( 骏 拓 扑 , 即 compact-open 拓 扑 》 设 孔 E CrCM,
),(U,9 和 (C7, 办 分 别 是 肖 和 义的 容 许 坐 标 卡 ; 令 史 口 口 是 紧
E 林伟 关 途 技 5 江2 扬 广

a c

E 0

E

E 红 L 木 c
包 含 有 限 个 这 种 集 司 交 集 的 集 合 都 是 了 的 一 个 邻 域 . 所 得 的 拓 扑
室 间 记 为 Cr(M,N).

E 出 伟 c
{GUn,8o |a E 4} 是 余 的 一 个 局 部 有 限 坐 标 系 , 即 M 的 每 一 点 都
有 一 个 邹 域 , 它 只 与 有 限 个 Dn 相 京 ; 记 2 一 (Kale E 4},Ku 人
Ue 是 胡 上 的 紧 集 ; 画 二 {CVaugp) | B} 是 丶 的 坐 标 系 , 对 任 一
D
Page-650
E 卷 考 文 欣

吴
012 技 移 2 圆
E

[ 水
玟
Sov. 1981, 1+ 373 一 387 Gin English》

[C] 陈 郎 炎 , 含 参 数 循 分 方 程 的 周 排 解 与 极 限 环 , 数 学 学 报 , 1963, 13(42:

807 一 609

[ 李 述 扬

30 刑 c
Sinica (Series A7y 1995, 38: 29 一 35

E
Springer-Verlag, 1982 ,

[CLW] Chow ShuiNee, Li Chengzhi, Wang Duo Mormal Forms and Bifar-

0 沥
Press 1994

[CMJ 蔡 暗 林 , 马 晔 . 广 义 Litnard 方 程 的 奇 点 的 中 心 焦 点 判 定 闭 题 . 浙 江 大
E 技

[Ca] 蔡 腾 林 , 二 次 系 统 研 究 近 况 . 数 学 进 展 , 1989, 18(+ 5 一 21

[ 口 水 沥 沥
E 水 园 广

[CW] 陈 兰 荪 , 王 明 淑 , 二 次 微 分 系 统 极 限 环 的 相 对 位 置 和 数 目 数 学 学
E 振 E 莲

[
玟

E
E 玲 汀 沥 5
Delae H,Sar les cycles linites,BullSoc,Math,Fr,1923,51+ 柯 一
E
E

ECICL32SSL2EXIS u
Page-651
E E

E 伟 技
u

0

E 不
d

L 孝 莲 园 口 河
E t
沥

E
E 余 c 江
2 水 吴
C“ 胥 射 在 “ 无 穷 远 “ 的 伯 质 , 强 拓 扑 就 成 为 需 要 的 了 , 注 意 , 在 办 ˇ
玟
E
集 在 其 中 稠 密 . 这 对 研 究 通 有 性 质 是 很 重 要 的 , 如 无 牺 别 声 明 ,
下 文 中 都 取 强 拓 扑 , 并 把 CS(MM , 简 记 为 CrCM ,MD.

U 途 i

一 怀 国 0
1 位 7 妙 svdimM 一 十 eo. 则 Gr(M,N) 在 Gr(MM,N) 中 稠 密 , 其 中
E

g 扬 玖 河 厂 标
紧 集 5 【

2
胚 于 一 个 C“ 微 分 流 形 ,

0
它 们 C 微 分 同 胚 ,

定 理 C.7 (Whitney 定 理 〉 设 1 口 r 口 十 co, 则 任 何 a 维 的 C“
Page-652
E
e
quadratic vector felds, J.Reine Angew、Math, 1987, 3825 165 一 180
[DL] 丁 同 仁 , 李 承 治 . 常 微 分 方 程 敬 程 . 北 京 : 高 等 教 育 出 版 社 , 1991
[DLZ] Daumaortier F, Li Chengzhi, Zhang Zhi-Fen、 Unfolding of a quadreti
integrable sywtem Wmith tmo centers and two umbounded hetroctinie
loops, Preprit, 1996
CDRR1J Dumortier F, Rousearie R, Roussese C。Hilberth 16th problem for
E 圆圆 一 木
E

1994, 7 86 一 133 、
[DRSL] Dumortier F,Roussare 叉 , Sotomayor J Generic 3-parameter family

吴
Hnear part, The cep case of codimension 3,Ergodie Theory end Dy-
E
[DRS5] 一 一 , Genetie 3-parameter family ot planar vector fields, anfolding of
E
ture Notes in Meth,1991, 1480: 1 一 164
CDz] 杜 之 林 , 曾 宪 武 . 计 算 焦 点 量 的 一 粤 通 推 公 式 , 科 学 通 报 , 1994,39
E b
Ecalle 一 丁 Finiude des eyoes bmites 吊 acotlero-sommation de
Dapplication de retour,、 Lecture Notes in Math。 1990, 14551 74 一 159
n
E
[F] 冯 贝 名 , 临 界 情 泓 下 奇 环 的 猪 定 伯 - 数 学 学 报 , 1990, 33(1): 113--134
[ 皇 E 育 河
玟
Anal. 1989, 20: 13 一 30
E
and Bifurcations af VWector Fields,New Yotk: SpringerVeriag, 1983
E
E 沥 途 浩 浩
Page-653
E

微 分 流 形 都 C7 微 分 同 胚 于 R*“「 的 一 个 闭 子 流 形 , 【

E 0 河 沥 i 沥
到 C“ , 二 不 是 一 个 很 严 重 的 事 情 . 今 后 我 们 将 经 常 作 这 样 的 假 定 .
虽 然 ^ 维 微 分 流 形 是 维 欧 氏 空 间 非 常 一 舱 的 推 广 , 但 定 理 C. 7 说
basaeetu pote y 林水 d

射 式 e6) 流 形

E E
E 伟
E 0

E 吊
E 应

林医 一
t

显 然 , 上 面 的 定 义 与 坐 标 卡 的 选 取 无 关 , 从 而 也 与 口 的 选 取 无
关 . 因 此 记

E 仪
E 林c 政b 河 c
记 力 (M ,N) 为 全 体 从 此 到 为 的 r-jet 之 集 合 , 我 们 把 一 (M,
E 沥沙 技 招 朐
Epeuuepnutisg u
现 在 考 虑 一 个 特 殊 情 形 :M 一 R“,N 一 R“. 此 时 简 记
E d
E 5 林 L 2 育 t 5
式 就 给 出 了 在 z 点 的 r-jet 一 种 自 然 的 表 示 . 这 个 从 R“ 到 R“ 的 多

项 式 眺 射 可 由 一 在 z 点 直 到 r 阶 ( 包 括 0 阶 ) 导 算 予 所 唯 一 决 定 . 这
医 江志
Page-654
E 动 考 文 献

E 河

Math, 1990, 1455: 160 一 196 ′

[Go] Touesoa B IL Summaneuricors covweter E
E 沥 E 吴 E
E

E

计

E 沥 一 园 一 工 刑
E

[Ha] Haysshi S. On the solution ofCtstabtlity conjecture for floms、Preprint、

D
turbations of quadretie Hamitonian systems, J Dif,Eq、1994,113
y

Euulor 吴 林浩 b 沥 沥 工
Springer-Verlag, 1976

[HL.zJ 韩 苗 安 , 罗 定 军 , 朱 循 明 , 奇 闯 捉 分 支 出 极 限 环 的 唯 一 性 ( I .C 王 ,

玟

LHmJ 韩 茂 安 . 周 期 抢 动 系 统 的 不 变 环 面 与 亚 调 和 解 的 分 支 . 中 国 科 学 A
n

[Ho] Horozoy E Vereal deformetions afeduivanant vector ficlds in the case of
E 一
一 192 Gin Russian)

[Biw] Huang Wenzao,The bifureation theory for nonlinear equations, Lecture
E
E

E

招
E i

E 沥 林 沥 林口 工 一 一 力 孙 水 门 吊 才 才 用 河
E

E

E 门 林 河

Sarveys 1990, 40 148 一 200
Page-655
E

E 振 目 沥
E

医 林
量 空 间 . 这 说 明 , 对 Pr(m, 中 的 每 一 个 元 素 , 都 有 且 仅 有 一 个
E 木 c l 园
E t 之
E 林
[ 阮 河
E 伟
E
E
技t 技 李
E d d tot d
E d D 沥 t
氏 空 间 同 构 , 所 以 ,yvCLD,V)) 就 是 一 (M ,N) 的 一 个 坐 标 卡 , 这
E
河 t 仪 2
Eapyteu2 沥 河

Thom 槲 戢 定 理

E a ritases e 沥 5
p

E 吴 i
E
空 间 中 树 子 流 形 的 机 截 性 .

E 北 伟 0 5
E 技
E 李 述
Page-656
E E

[ 助 育 2 技
腾 史 当
E 江 英 0 胡
E
河
[JaJIakobson M V Absoiately continuous invariant messtre for one-parame-
n
E

DJoj Joysl P Generatized HopE bifurcation and its dual generalized homoelinic
Hifurceation、SIAM J Math,1988, 48, 481 一 486
] Khovanslkey A G,Real analytic manifolds with finiteness properties and
complex Abetian integrals, Funct,Anal, AppL 1984, 18: 119 一 128
E e

5 兄 口 un 胡 y

tons, hi Chinese Mathemates into 21et Century-, Peking University
Press, 1992.
[Le] 李 承 治 . 关 于 平 面 二 次 系 统 的 两 个 阿 题 . 中 国 科 学 (A 辑 )+1982(12。
E
2 胡 玲 王 3 江2 埕 u
the wealkened 16th Hilbert prablem,]MAA 1995, 190: 489 一 516
[ 胡 八 沥 河 技
3
阮
玟
[URI] Li Chengzhi, Rouseeau C。 A system with three linit eyes eppearing
训 s Hopf bifureation and dying 记 a homoclinic biforcation, the cusp of
order 4 丁 Diff, Eq, 1989, 78: 132 一 167
[UR2] 一 一 ,Codimension 2 symmetric homodinie biforeation,Can, 了
Math、 1980, 42: 191 一 212
[Lnw] Li Weiga,The bHforeation of “eight ure“ of separatiz of saddle with
zero seddle alue in the Plane,Preprint of Peking University,Research
Report No 46, 1995
Page-657
E E

E t p
E d 化 明 ,
E 心 c 林5 浩 a

(D ( 一 (TpM) 十 Tyek 二 TyeN, 面 东

0 芸 李
E 怀 吴 沥

林
E 江 s

E
E 应

102 怡 u 园
E E
Et 述 仪 沥 e 纳 d

由 于 机 蒙 相 交 性 与 土 标 卡 的 选 取 无 关 , 所 以 当 史 与 丶 都 为 有
E

英
E t 江 5 脱
心 点 附 近 有 局 部 坐 标 z「,.,z“, X 在 zo) 点 附 近 有 局 部 坐 标

E 沥 u i

沥 扬 2
区

E

江
E

E 林0 汀

E 达
定 理 C.14 CThom 横 戳 定 理 ) .w“(M ,Ny4) 是 CrCM,N) 中

E s i 沥
Page-658
312 参 考 文 献

E
E

i

玲
E
MaRe R A proof af theCtstabiity conjecttze Inst, Hautes,Sci,PubiL
E 吴 E
E
Hiamiltonign veotor fald,Ergod Th, 吊 Dynsm, Sys , 1990, 10: 523
国 J 「
[Me] Mourtada A, Degenerate and non-trivial hypetbolic polyeycles with wo
E 吴 园 振n
E 国 园
河
[Ma] 马 知 恩 . 种 羯 生 怪 学 的 数 学 建 模 与 研 究 合 肥 : 安 微 救 育 出 版 社 , 1996
[NS] Nowiki 个 , Strien Y S,Absclately continuous invariant measures for C+
E 育 诊汀
E
E 振
E
[
1988, 22: 72 一 78
[ 沥
EE 沥 弋 2 沥 江 10
yyslioyxXEHEAXLL3IIX335SLuy
n
[PTJ Pais J, Tikens F,Hyperbolicity and sensitive chaotic dyaemics t ho-
meotinic bifureations Cambridge University Press, 1992
[Q] 秽 元 助 、 微 分 方 程 所 定 义 的 积 分 曲 缇 , 上 下 册 , 北 京 : 科 学 出 版 社 ,
1956.1959 【
3
E
Page-659
E E

果 么 是 闭 子 流 形 , 则 它 还 是 开 的 ,

E
E 志 技 e 医
成 为 横 截 , 而 原 来 樟 截 的 晔 射 休 可 保 持 横 戳 . 因 此 “ 利 用 槲 戳 性 可
e 个 中 一

Thom 横 截 定 理 可 以 推 广 到 jet 形 式 . 这 样 就 可 把 无 限 维 空 间
CrQd , 中 的 通 有 性 闭 题 , 转 化 到 有 限 维 空 间 J7CM ,N) 中 的 模
截 闭 题 `

国 u t
水
E }

E 技t 朋

E 河 园
|

证 明 可 见 [Hirpp80 一 81] 下 面 的 定 理 对 于 确 定 子 流 形 的 余
维 数 是 重 要 的 .

E 李
了 与 么 横 戳 , 则 广 “(4) 是 仁 的 子 流 形 . 如 果 4 在 丶 中 有 有 限 余
维 , 则

codim( 广 1(4D) 二 eodim(4)。 【
特 别 地 , 当 人 育 一 丶 是 浸 草 时 , 条 件 “ 历 与 横 截 “ 总 是 满 趸
医
Page-660
医 招 怀 生 人

[A1] Armold V L Geometrie Methods in the Theory of Ordinary Diflerental
E
E 东 不 e
E
〖nouoe 志 怀 坂 园 0 技砂
E

E
E 林 日
E
Program for Sci Tranal , Wiley, 1973
[AMRJ Abraham R,Marsden ] E, Ratiu T Manilolds,Tensor Analysis,
E 口
Publishing Company, Inc, 1983
E 江 技
of the coefticients from an equitbriam Position al focus or ceoter type、
E 沥 沥 E 李 应 一
E
E
E 沥
[BC2] 一 一 . The dynamics of the Htnon mag,Aan Math。 1991, 133: 73 一
E
E
3 印 永 X 前 日
[BL] Bonin G Legalt J Comparision de l methode des constantes de Lia-
Punov et la Mfreation de Hopf,Canad,Math、Bull 1988, 31(2): 200
g
E 林 伟
Page-661
E 313

E
E
cho Paciticoed , Longman, Scientific and Technial, Pitman Research
Notes in Math,Series 180, 1987: 377 一 385
b 技
E 莲 E
E

Roassesu C,Universal untolding f a Singularity f a symumetric vector

E 林 圆
E E

Notes in Math、1989,1455, 334 一 354
e
tions to lanar qundratic gystems,Ann,Pclon, Math. 1988, 49: 1 一 16
n
Phys 1971, 20: 167 一 192
Singer D Stable ofbits and bifurcation of maps of the interval,SIAM Ap-
E 朋
[Sh] Shoashitaishvif A N Bfurcetion of topalogical type a sngular points f
psrameterized veetor falda,Fumet,AnalAppl 1972(2): 169 一 170
E
E 吴 技 唐 E 沥
[SHj] Sijbrand J, Properties of center mantfolda,Trans,AMS 1985, 289: 431
E
E 园 e 招
353 一 371
[SH2] 一 一 . On the generation of a periodic motion fzom a trajectory doubly
aspmptotic t an edilibriam state af snddle type,Math USSR Sb 1968,
沥 江
[
neighborhood cf a rough equitibrium state f saddls-foous type,Math
E 吴
[SJ] Shen Jaqi, Jing Zhjun, A new detecting method a conditions for the ex-
Page-662
E 卷 考 文 欣

吴
012 技 移 2 圆
E

[ 水
玟
Sov. 1981, 1+ 373 一 387 Gin English》

[C] 陈 郎 炎 , 含 参 数 循 分 方 程 的 周 排 解 与 极 限 环 , 数 学 学 报 , 1963, 13(42:

807 一 609

[ 李 述 扬

30 刑 c
Sinica (Series A7y 1995, 38: 29 一 35

E
Springer-Verlag, 1982 ,

[CLW] Chow ShuiNee, Li Chengzhi, Wang Duo Mormal Forms and Bifar-

0 沥
Press 1994

[CMJ 蔡 暗 林 , 马 晔 . 广 义 Litnard 方 程 的 奇 点 的 中 心 焦 点 判 定 闭 题 . 浙 江 大
E 技

[Ca] 蔡 腾 林 , 二 次 系 统 研 究 近 况 . 数 学 进 展 , 1989, 18(+ 5 一 21

[ 口 水 沥 沥
E 水 园 广

[CW] 陈 兰 荪 , 王 明 淑 , 二 次 微 分 系 统 极 限 环 的 相 对 位 置 和 数 目 数 学 学
E 振 E 莲

[
玟

E
E 玲 汀 沥 5
Delae H,Sar les cycles linites,BullSoc,Math,Fr,1923,51+ 柯 一
E
E

ECICL32SSL2EXIS u
Page-663
E E

istence af Hopf bifurcation in “Dynamical Systetms“, Nankai Seies 训
E 朋 一 罡 途
E 吴 5 团 一 不 院 日

[ 芸

Nonlinear Science The Next Decade, 见 * 数 学 注 林 “, 动 力 系 统 学 的 团
顾 ; 重 大 闭 题 , 失 败 前 尝 谈 ,1993(47: 262 一 269,.

[Ss] 史 松 龄 . 平 面 二 次 系 统 存 在 四 个 极 限 环 的 具 体 值 子 中 国 科 学 , 1979
《LDy 1081 一 1056

[TJ Tkens F,Forced oseilhtions and biftrcstions: Applications of global
E

[TTYJ Thisullen P, Tresser C+ Young L-S, Positive Lispunov exbonent for

gereie one-parameter familes af unimodal tmape,Peprint
E 林 E
E 吴 一 芸
E
[Va] Varchenko A N Estimate of the number of zeros ot an Abetian integeal
E 莲
E 技 n
月
E
[Wd] Weng Dao。An introduetion ta the normal jorm theory of cdinary dif-
ferental equations,Adyances 训 Math,1990, 19(17 38 一 仁

[WiH] Wiggins S8,Introduction to Applied Nonlinear Dynamical Systems and
i

[Wia] 一 一 . Glbal Bifureations and Chaoe, Analytical Methode, New York
E

[W0 王 兰 字 , 多 峰 晔 射 的 动 力 学 . 北 京 大 学 博 士 论 文 ,1996

[X] 肖 冬 梅 . 一 类 余 维 3 歌 点 坦 平 面 向 量 杨 的 分 支 , 中 国 科 学 CA 辑 ) 1993
0

[XY1 叶 咤 谥 等 , 极 限 环 论 , 第 二 版 , 上 海 ; 上 海 科 学 技 术 出 版 社 , 1964

[Y2] 叶 彦 谦 , 多 项 式 徽 分 系 统 定 性 理 论 . 上 海 ; 上 海 科 学 技 术 出 版 社 , 1995

E
Page-664
E
e
quadratic vector felds, J.Reine Angew、Math, 1987, 3825 165 一 180
[DL] 丁 同 仁 , 李 承 治 . 常 微 分 方 程 敬 程 . 北 京 : 高 等 教 育 出 版 社 , 1991
[DLZ] Daumaortier F, Li Chengzhi, Zhang Zhi-Fen、 Unfolding of a quadreti
integrable sywtem Wmith tmo centers and two umbounded hetroctinie
loops, Preprit, 1996
CDRR1J Dumortier F, Rousearie R, Roussese C。Hilberth 16th problem for
E 圆圆 一 木
E

1994, 7 86 一 133 、
[DRSL] Dumortier F,Roussare 叉 , Sotomayor J Generic 3-parameter family

吴
Hnear part, The cep case of codimension 3,Ergodie Theory end Dy-
E
[DRS5] 一 一 , Genetie 3-parameter family ot planar vector fields, anfolding of
E
ture Notes in Meth,1991, 1480: 1 一 164
CDz] 杜 之 林 , 曾 宪 武 . 计 算 焦 点 量 的 一 粤 通 推 公 式 , 科 学 通 报 , 1994,39
E b
Ecalle 一 丁 Finiude des eyoes bmites 吊 acotlero-sommation de
Dapplication de retour,、 Lecture Notes in Math。 1990, 14551 74 一 159
n
E
[F] 冯 贝 名 , 临 界 情 泓 下 奇 环 的 猪 定 伯 - 数 学 学 报 , 1990, 33(1): 113--134
[ 皇 E 育 河
玟
Anal. 1989, 20: 13 一 30
E
and Bifurcations af VWector Fields,New Yotk: SpringerVeriag, 1983
E
E 沥 途 浩 浩
Page-665
城 考 文 献

E 职 。
[

n
【
E
[Za1] Zhu Deming,Meiaikov vector and heteroctinie manifolds。Science in
D
E
E
[Z4s] 一 一 . Tranaersal hetrocinic orbits in gSneral degenerate ceses, Science
E
[zDHD] 张 芷 芬 , 丁 同 仁 , 黄 文 灶 , 董 镐 善 , 徽 分 方 程 定 性 理 论 , 北 京 : 科 学
出 版 社 , 1985
[Zg] 张 恭 庆 , 临 界 点 理 论 及 其 应 用 上 海 , 上 海 科 学 技 术 出 版 社 , 1886
[Z 永 张 镯 粉 , 常 微 分 方 程 儿 何 理 论 与 分 支 闭 题 ( 修 订 本 》, 北 京 : 北 京 大 学 出
版 社 , 1987
0
Progress in Natural Science, 1996, 6( 旦 , 401 一 407
[zQ] 张 锦 炎 , 钱 敏 , 徽 分 动 力 系 统 导 引 、 北 京 , 北 京 大 学 出 版 社 ,1991
[Zol] 2oladek H,Ou the versality of ceriain family of vector felds on the
E
E 唐 0 沥 刑 沥 沥 江 目
E 发 压 巩 玟 2 吴 y
[N] 涨 梅 斯 基 B B, 四 十 年 李 的 苏 联 数 学 (1917 一 1957), 常 微 分 方 程 部 分 ,
饶 生 惠 译 , 北 京 : 科 学 出 版 社 , 1960
[Zafl] Zhang Zhifen, On the uniqueness ol the limit cycles of some noninear
E
e
b
Dignard equations,Applicable Analysis, 1986, 23 83 一 67.
gZze] 张 筑 生 , 微 分 助 力 系 统 原 理 , 北 京 , 科 学 出 版 社 ,1987
Page-666
索

l

中 文 词 条 拳 首 字 的 笔 下 排 列 , 西 文 开 头 的 词 条 接 字 母 颂 序 排 列 .

E

t

医
四

双 公 韶 点

双 曾 闭 辅

双 曲 不 动 点 定 理

分 员

E

分 盆 值

分 岔 图

分 岔 曲 线

分 岔 方 程

分 岔 函 数

分 岔 的 余 维

无 穷 阶 非 共 振

无 限 C 水 平 曲 线

无 限 C 垂 直 曲 线

E

E

切 向 量

切 空 间

E

E

7,46 一 50,286
E
173

63
53

159

4

E

E

E

E
291292

E
3s E
E E
a E
对 参 数 一 致 的 Hop 分 岔 定 理 82
p 282
E E
边 翔 的 水 平 部 分 159,198,200
边 界 的 垂 宇 部 分 E
边 弈 条 件 E
主 穗 定 方 向 E
E
E
10
165
15,85
10,23,114
E
146
Page-667
E 动 考 文 献

E 河

Math, 1990, 1455: 160 一 196 ′

[Go] Touesoa B IL Summaneuricors covweter E
E 沥 E 吴 E
E

E

计

E 沥 一 园 一 工 刑
E

[Ha] Haysshi S. On the solution ofCtstabtlity conjecture for floms、Preprint、

D
turbations of quadretie Hamitonian systems, J Dif,Eq、1994,113
y

Euulor 吴 林浩 b 沥 沥 工
Springer-Verlag, 1976

[HL.zJ 韩 苗 安 , 罗 定 军 , 朱 循 明 , 奇 闯 捉 分 支 出 极 限 环 的 唯 一 性 ( I .C 王 ,

玟

LHmJ 韩 茂 安 . 周 期 抢 动 系 统 的 不 变 环 面 与 亚 调 和 解 的 分 支 . 中 国 科 学 A
n

[Ho] Horozoy E Vereal deformetions afeduivanant vector ficlds in the case of
E 一
一 192 Gin Russian)

[Biw] Huang Wenzao,The bifureation theory for nonlinear equations, Lecture
E
E

E

招
E i

E 沥 林 沥 林口 工 一 一 力 孙 水 门 吊 才 才 用 河
E

E

E 门 林 河

Sarveys 1990, 40 148 一 200
Page-668
E 园 E

自 由 返 团 E 园 E
a 国 巳 唐
E 万 既
全 局 中 心 流 形 定 理 途 8
E 口 圆 英 E
沥 E E
E 十 画

颖 玟
b E 45
E 通 有 族 网
E 1 16
E 0 E
b 税 资 形
更 荐 法 渥 入

E E
E
E

坐 标 卡

u

移 位 映 射

符 号 动 加 系 统
E

E 、

尿 朝 辅 腑 适 开 折

E 焦 炉 敦
环 E
河 超 稳 定 周 婵 软 道
非 游 济 环 逗 历
E 园 刹
E s
E

单 峰 眯 射

仪

E

组 歉 点
Page-669
E E

[ 助 育 2 技
腾 史 当
E 江 英 0 胡
E
河
[JaJIakobson M V Absoiately continuous invariant messtre for one-parame-
n
E

DJoj Joysl P Generatized HopE bifurcation and its dual generalized homoelinic
Hifurceation、SIAM J Math,1988, 48, 481 一 486
] Khovanslkey A G,Real analytic manifolds with finiteness properties and
complex Abetian integrals, Funct,Anal, AppL 1984, 18: 119 一 128
E e

5 兄 口 un 胡 y

tons, hi Chinese Mathemates into 21et Century-, Peking University
Press, 1992.
[Le] 李 承 治 . 关 于 平 面 二 次 系 统 的 两 个 阿 题 . 中 国 科 学 (A 辑 )+1982(12。
E
2 胡 玲 王 3 江2 埕 u
the wealkened 16th Hilbert prablem,]MAA 1995, 190: 489 一 516
[ 胡 八 沥 河 技
3
阮
玟
[URI] Li Chengzhi, Rouseeau C。 A system with three linit eyes eppearing
训 s Hopf bifureation and dying 记 a homoclinic biforcation, the cusp of
order 4 丁 Diff, Eq, 1989, 78: 132 一 167
[UR2] 一 一 ,Codimension 2 symmetric homodinie biforeation,Can, 了
Math、 1980, 42: 191 一 212
[Lnw] Li Weiga,The bHforeation of “eight ure“ of separatiz of saddle with
zero seddle alue in the Plane,Preprint of Peking University,Research
Report No 46, 1995
Page-670
B E

E E
锋 分 绪 构 E
徽 分 同 酗 E
i E

十 四 画

E
稳 定 周 期 扔 道
E

十 五 面
E 四
E E
E ′ 国
E 216
E 46,304

E 99v0L,106 一 109
E 沥 E
Bogdanov-Takens 系 统 47,109,130
Biekhof-Smale 定 理 E
E E
E E
[ 159
C 垂 直 带 域 E
E

E 0
E D
E 6
Hilbert 第 16 问 题 E
E 园
E E

引

吴 2 34
【 吴 4
0 园
0 0
江 0
防 细 焦 点 [
Lebesgue 测 度 242
E E
Lebestgoe 全 稿 点 E
Liapunov 系 数 法 78,79
E
Maigrange 定 理 18
E 河 刑 90,97,108
E 282
E

Pichfork 分 员

Pieari-Fuchs 方 程

E

Pioncare 分 纭

Pliss 约 化 原 理

E

Smale 马 路

E E
E t E
【 扬 33 斧 江 江 E
0 形 达 江 E
E 育 江 162,168
E 159
E 228
Page-671

Page-672
京 )112 号

伟 习 分 为 六 耿 , 各 章 内 容 分 泉 是 ; 基 本 概 伊 积 准 备 知 识 , 帝 见 的 局 部 与 非 局 部 分
岔 , 儿 兴 余 绵 2 的 平 面 暨 圭 分 岔 , 双 丞 不 劲 炕 及 驯 晃 存 在 定 璐 , 空 间 中 友 曲 驿 炳 的 固
E 的 铁 章 之 后 , 都 配 备 了 一 定 数 最 的 习

E
木 书 可 作 为 高 等 学 校 数 学 考 业 高 年 镁 本 移 生 的 法 修 谅 教 材 , 或 相 关 专 业 研 究 生 的

药 磁 课 教 材 以 司 供 莲 振 了 焊 分 盆 理 论 这 门 学 科 的 学 生 . 敏 帝 或 科 技 人 员 作 为 坂 考 书 .

固 书 在 版 编 目 (CIP) 数 据

i
育 出 版 社 ,1997
E

a 途 的

旦

高 等 救 育 出 版 社 出 版
北 京 沙 滩 后 街 55 号
E 2 玟
新 华 书 店 总 店 北 京 发 行 所 发 行
固 货 工 业 出 版 社 印 刷 厂 印 刷

E

玟
E 的 1997 年 10 月 第 1 次 卯 刹
印 数 0 001-1 715
定 价 10.20 元

凡 购 买 高 等 敏 育 出 版 社 的 图 书 , 如 有 狱 页 . 倒 页 , 育 页 等
E

、 圃 版 扑 所 有 , 不 得 番 印
Page-673

Page-674

Page-675

Page-676
312 参 考 文 献

E
E

i

玲
E
MaRe R A proof af theCtstabiity conjecttze Inst, Hautes,Sci,PubiL
E 吴 E
E
Hiamiltonign veotor fald,Ergod Th, 吊 Dynsm, Sys , 1990, 10: 523
国 J 「
[Me] Mourtada A, Degenerate and non-trivial hypetbolic polyeycles with wo
E 吴 园 振n
E 国 园
河
[Ma] 马 知 恩 . 种 羯 生 怪 学 的 数 学 建 模 与 研 究 合 肥 : 安 微 救 育 出 版 社 , 1996
[NS] Nowiki 个 , Strien Y S,Absclately continuous invariant measures for C+
E 育 诊汀
E
E 振
E
[
1988, 22: 72 一 78
[ 沥
EE 沥 弋 2 沥 江 10
yyslioyxXEHEAXLL3IIX335SLuy
n
[PTJ Pais J, Tikens F,Hyperbolicity and sensitive chaotic dyaemics t ho-
meotinic bifureations Cambridge University Press, 1992
[Q] 秽 元 助 、 微 分 方 程 所 定 义 的 积 分 曲 缇 , 上 下 册 , 北 京 : 科 学 出 版 社 ,
1956.1959 【
3
E
Page-677
E 313

E
E
cho Paciticoed , Longman, Scientific and Technial, Pitman Research
Notes in Math,Series 180, 1987: 377 一 385
b 技
E 莲 E
E

Roassesu C,Universal untolding f a Singularity f a symumetric vector

E 林 圆
E E

Notes in Math、1989,1455, 334 一 354
e
tions to lanar qundratic gystems,Ann,Pclon, Math. 1988, 49: 1 一 16
n
Phys 1971, 20: 167 一 192
Singer D Stable ofbits and bifurcation of maps of the interval,SIAM Ap-
E 朋
[Sh] Shoashitaishvif A N Bfurcetion of topalogical type a sngular points f
psrameterized veetor falda,Fumet,AnalAppl 1972(2): 169 一 170
E
E 吴 技 唐 E 沥
[SHj] Sijbrand J, Properties of center mantfolda,Trans,AMS 1985, 289: 431
E
E 园 e 招
353 一 371
[SH2] 一 一 . On the generation of a periodic motion fzom a trajectory doubly
aspmptotic t an edilibriam state af snddle type,Math USSR Sb 1968,
沥 江
[
neighborhood cf a rough equitibrium state f saddls-foous type,Math
E 吴
[SJ] Shen Jaqi, Jing Zhjun, A new detecting method a conditions for the ex-
Page-678
E E

istence af Hopf bifurcation in “Dynamical Systetms“, Nankai Seies 训
E 朋 一 罡 途
E 吴 5 团 一 不 院 日

[ 芸

Nonlinear Science The Next Decade, 见 * 数 学 注 林 “, 动 力 系 统 学 的 团
顾 ; 重 大 闭 题 , 失 败 前 尝 谈 ,1993(47: 262 一 269,.

[Ss] 史 松 龄 . 平 面 二 次 系 统 存 在 四 个 极 限 环 的 具 体 值 子 中 国 科 学 , 1979
《LDy 1081 一 1056

[TJ Tkens F,Forced oseilhtions and biftrcstions: Applications of global
E

[TTYJ Thisullen P, Tresser C+ Young L-S, Positive Lispunov exbonent for

gereie one-parameter familes af unimodal tmape,Peprint
E 林 E
E 吴 一 芸
E
[Va] Varchenko A N Estimate of the number of zeros ot an Abetian integeal
E 莲
E 技 n
月
E
[Wd] Weng Dao。An introduetion ta the normal jorm theory of cdinary dif-
ferental equations,Adyances 训 Math,1990, 19(17 38 一 仁

[WiH] Wiggins S8,Introduction to Applied Nonlinear Dynamical Systems and
i

[Wia] 一 一 . Glbal Bifureations and Chaoe, Analytical Methode, New York
E

[W0 王 兰 字 , 多 峰 晔 射 的 动 力 学 . 北 京 大 学 博 士 论 文 ,1996

[X] 肖 冬 梅 . 一 类 余 维 3 歌 点 坦 平 面 向 量 杨 的 分 支 , 中 国 科 学 CA 辑 ) 1993
0

[XY1 叶 咤 谥 等 , 极 限 环 论 , 第 二 版 , 上 海 ; 上 海 科 学 技 术 出 版 社 , 1964

[Y2] 叶 彦 谦 , 多 项 式 徽 分 系 统 定 性 理 论 . 上 海 ; 上 海 科 学 技 术 出 版 社 , 1995

E
Page-679
城 考 文 献

E 职 。
[

n
【
E
[Za1] Zhu Deming,Meiaikov vector and heteroctinie manifolds。Science in
D
E
E
[Z4s] 一 一 . Tranaersal hetrocinic orbits in gSneral degenerate ceses, Science
E
[zDHD] 张 芷 芬 , 丁 同 仁 , 黄 文 灶 , 董 镐 善 , 徽 分 方 程 定 性 理 论 , 北 京 : 科 学
出 版 社 , 1985
[Zg] 张 恭 庆 , 临 界 点 理 论 及 其 应 用 上 海 , 上 海 科 学 技 术 出 版 社 , 1886
[Z 永 张 镯 粉 , 常 微 分 方 程 儿 何 理 论 与 分 支 闭 题 ( 修 订 本 》, 北 京 : 北 京 大 学 出
版 社 , 1987
0
Progress in Natural Science, 1996, 6( 旦 , 401 一 407
[zQ] 张 锦 炎 , 钱 敏 , 徽 分 动 力 系 统 导 引 、 北 京 , 北 京 大 学 出 版 社 ,1991
[Zol] 2oladek H,Ou the versality of ceriain family of vector felds on the
E
E 唐 0 沥 刑 沥 沥 江 目
E 发 压 巩 玟 2 吴 y
[N] 涨 梅 斯 基 B B, 四 十 年 李 的 苏 联 数 学 (1917 一 1957), 常 微 分 方 程 部 分 ,
饶 生 惠 译 , 北 京 : 科 学 出 版 社 , 1960
[Zafl] Zhang Zhifen, On the uniqueness ol the limit cycles of some noninear
E
e
b
Dignard equations,Applicable Analysis, 1986, 23 83 一 67.
gZze] 张 筑 生 , 微 分 助 力 系 统 原 理 , 北 京 , 科 学 出 版 社 ,1987
Page-680
索

l

中 文 词 条 拳 首 字 的 笔 下 排 列 , 西 文 开 头 的 词 条 接 字 母 颂 序 排 列 .

E

t

医
四

双 公 韶 点

双 曾 闭 辅

双 曲 不 动 点 定 理

分 员

E

分 盆 值

分 岔 图

分 岔 曲 线

分 岔 方 程

分 岔 函 数

分 岔 的 余 维

无 穷 阶 非 共 振

无 限 C 水 平 曲 线

无 限 C 垂 直 曲 线

E

E

切 向 量

切 空 间

E

E

7,46 一 50,286
E
173

63
53

159

4

pE

E

E

E
291292

E
3s E
E E
a E
对 参 数 一 致 的 Hop 分 岔 定 理 82
p 282
E E
边 翔 的 水 平 部 分 159,198,200
边 界 的 垂 宇 部 分 E
边 弈 条 件 E
主 穗 定 方 向 E
E
E
10
165
15,85
10,23,114
E
146
Page-681
E 园 E

自 由 返 团 E 园 E
a 国 巳 唐
E 万 既
全 局 中 心 流 形 定 理 途 8
E 口 圆 英 E
沥 E E
E 十 画

颖 玟
b E 45
E 通 有 族 网
E 1 16
E 0 E
b 税 资 形
更 荐 法 渥 入

E E
E
E

坐 标 卡

u

移 位 映 射

符 号 动 加 系 统
E

E 、

尿 朝 辅 腑 适 开 折

E 焦 炉 敦
环 E
河 超 稳 定 周 婵 软 道
非 游 济 环 逗 历
E 园 刹
E s
E

单 峰 眯 射

仪

E

组 歉 点
Page-682
B E

E E
锋 分 绪 构 E
徽 分 同 酗 E
i E

十 四 画

E
稳 定 周 期 扔 道
E

十 五 面
E 四
E E
E ′ 国
E 216
E 46,304

E 99v0L,106 一 109
E 沥 E
Bogdanov-Takens 系 统 47,109,130
Biekhof-Smale 定 理 E
E E
E E
[ 159
C 垂 直 带 域 E
E

E 0
E D
E 6
Hilbert 第 16 问 题 E
E 园
E E

引

吴 2 34
【 吴 4
0 园
0 0
江 0
防 细 焦 点 [
Lebesgue 测 度 242
E E
Lebestgoe 全 稿 点 E
Liapunov 系 数 法 78,79
E
Maigrange 定 理 18
E 河 刑 90,97,108
E 282
E

Pichfork 分 员

Pieari-Fuchs 方 程

E

Pioncare 分 纭

Pliss 约 化 原 理

E

Smale 马 路

E E
E t E
【 扬 33 斧 江 江 E
0 形 达 江 E
E 育 江 162,168
E 159
E 228
Page-683

Page-684
京 )112 号

伟 习 分 为 六 耿 , 各 章 内 容 分 泉 是 ; 基 本 概 伊 积 准 备 知 识 , 帝 见 的 局 部 与 非 局 部 分
岔 , 儿 兴 余 绵 2 的 平 面 暨 圭 分 岔 , 双 丞 不 劲 炕 及 驯 晃 存 在 定 璐 , 空 间 中 友 曲 驿 炳 的 固
E 的 铁 章 之 后 , 都 配 备 了 一 定 数 最 的 习

E
木 书 可 作 为 高 等 学 校 数 学 考 业 高 年 镁 本 移 生 的 法 修 谅 教 材 , 或 相 关 专 业 研 究 生 的

药 磁 课 教 材 以 司 供 莲 振 了 焊 分 盆 理 论 这 门 学 科 的 学 生 . 敏 帝 或 科 技 人 员 作 为 坂 考 书 .

固 书 在 版 编 目 (CIP) 数 据

i
育 出 版 社 ,1997
E

a 途 的

旦

高 等 救 育 出 版 社 出 版
北 京 沙 滩 后 街 55 号
E 2 玟
新 华 书 店 总 店 北 京 发 行 所 发 行
固 货 工 业 出 版 社 印 刷 厂 印 刷

E

玟
E 的 1997 年 10 月 第 1 次 卯 刹
印 数 0 001-1 715
定 价 10.20 元

凡 购 买 高 等 敏 育 出 版 社 的 图 书 , 如 有 狱 页 . 倒 页 , 育 页 等
E

、 圃 版 扑 所 有 , 不 得 番 印
Page-685

Page-686

Page-687
