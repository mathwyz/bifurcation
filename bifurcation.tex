
Page-29
第一章
E
B园
E河
E沥
E

【
E
E

.不3
E
E
E

第三章
E
B
E

第四章
E
E

[

基本概念和准备知识

E伟莲

分岔与分岔何题的提法
E

E

E
E

常见的局部与非局部分岔

奇点分岗

c

Hopt分岔E
E

Poincart分岛与弘Hilbert第16何题
关于Petrov定理的诛明
E
E
E沥林标许一
河
E
E

友曲不动点及马蹄存在定理
E

E
Page-30

Page-31

Page-32

Page-33
[

述

江万人
沥

高等教育出版社
Page-34
史3“Smale马蹄。i
$4线恒胥射的复吊昼尉的双峡何
5“Birkhoff-Smale定理
第五章空间中双曲鞍点的同宿分岔
Et
2n
医
邹公章“实二次单峰脓射旋的颂引子
E河玲
E沥沥腾3浩5
$3PCzva)不存在稳定周期狸闵题
荃4分布问题......2
耐′暴E
附录ADenzct浑形和流形问的际躬

E
附录CThom模截定理-

参考文献
E王
Page-35
[一

怀沥沥
E
突砾性进展,对结衿不穗定素统的研究(即分岛理论)便取到超来
超多的关注,分岔理论具有深厚的实际胍景,又雷借助于现代数
E吴东
i、

E
E

E
Ee
医不
E

既焰分岛现象智速地存在于自熊界中,因而在描译自然现象
E圭政东

E

s訾+侣一D莹+毓=&s岫'

E一沙仪不脱政沥
引起了人们的关注.随后发现当8取菜些特定值时,系统有非通
怀怡不2河林东
沥
s
Page-36

\part{基本概念和准备知识}
第一章基本概念和准备知识

作为全书的准备,我们在$1中简述有关动力系统和结构程
定的基本概念,不加证明地陈述一些重要结果;在82中引入分岑
E
心流形定理和正规形理论,在研究分岔问题时它们是进行简化处
理的有效手段!最后,在85中介绍奇异向量场的普适开折和分岔
的余维这两个重要的概念.

E宇

动力系统的概念和理论是从人们对常微分方程的研究中产生
和发展起来的,而丁对常微分方程的研究,至今仍是动力系统理论
a

E

董=/彻)】0

其中广R*一R“是C向量场,7之1.由常微分方程中熟知的缤
E
含t一0的棠区间上存在.如果/Cz)满足适当条件(或在某种等
E江标2李p水i
E

0李2

422E玖
E河3技2
Page-37
E国胡

又如,60年代从气象学研究中提出的Lorenz方程
巳
5E

盖=叩_彻,

1雹=一zz+胤叭

E伟扬达
E

岔)值7】~13.926,7】~24'06和7a~z4.74附近时,相应系统的
目玟玖江
玟
e
物理和数学界关注的热点问题之一.

胡如,从生态学中提出的虫巳差分模塑

王2沥E水
E租训2
E

英用2
0途缠增加时,尕(z》不断出现借周期分岛点,一明对应于出现鹤
定周期点的那些分员值具有很强的规律性,从而发现了一个新的
河
关注、

敏学上作为研究分岔现象的理论一-分岔理论主要研究三类
匹氓
甾眺艇所定义的离敬动力系统的分岔;函数方程的零解随参数炕
化而产生的分岔.前两类分岔称为动态分贪,一第三粤分员称为
木河
技
Page-38
E河

E

e
E

Ed2oto

不隼证明,对Vz,z:ER“,O。(z)和O。(za)或者重合,或
者(对有限的时间:)不相交.因此,(1.1)的转道集合依一的不同
E

常微分方程定性理论(或称为儿何理论)的首要目标,就是对
E林td2s
构(扬道集合的拓扑结构图称为相图,通常在相图上用箭头标明对
D
E肉
E[
倩形,人们至今尚不清楚相图的确切结构.参见[YL,2],[Q]

玟
E木
E林口e一c
玟
C1,定义C.2,附注C.4).Y叉E&“(M),存在弋过8E趸的概
大流ar(见附录B中定义B.13,定义B15,和定理B.16),为了讨
1
无限延伸;使XXE&r“(M)的极大流在(一co,十co)上存在(香
E

伟

0颖5

E林i
E

王
Page-39
国B

动态分岔理论生要研究动力系统的转道族的拓扑结构随参数
变化所发生的变化及关规律.例如,奇点(成不动点)的汇聚与分
E
(或环的形成与破梁!以及一些更复机的动力学行为(例如浑沛
态的出现与消失等.

日然分会理论的森些方面可以追潼到Poineare阡代,但坂这
E江江吊沥林
一,大郭分工作集中于平面上逼化程度不禽(即余维二2的分岛,
包揣闷宿分岔和异宿分岛问题等.分岔理论的发展很大程度上依
r
的结构稳定性有较完境的结昂.因而,当祖空间维数增大或系统的
遥化程度增大时,问题的复机性大大增加,完整的工作尚属少见。
此外,最初人们席望在分岔值附近都能进行开折,即在分岔值附近
E

应结构龚定的系统.但人们递淅认识到,在不少情况下分岛值附
医志沥c沥
迷,本节的第五章$3和第六章将涉及这一问题

E
E沥招
笔,第六章由郑志明执笔,附录凸李承治执笔,最后经集体讨论定
E沥

下面简要介绍本书的内容安排。

第一章介绍基木概念和准备知识。我们假定读者具有E
玟
只作了简略的介绍.然后通过实例引进分岔的概念及分岛问题的
i
e
第二竟介绍儿类平面向量场的最典型的分岔现象,如资点分岛,闭
执分岔、Hopf分呕\同窥分岛等,以及研究这些闰题的典珩方法;
Page-40
E骏ssE

园2[

医吊5志5
参数变换群;对固定+ER和所有的E聪,ex(t,办是故一版的
微分同胚.

[

2

氏2a

E林

0

0

[

有时也把句称为壮上的Cr徽分动力票统特别,当+EZ,
称为穴散动力系统,本书主要讨论连续流、由于它与离散流有密
一王沥国许汀江士江梁圆国

E怀园
E沥述林2

Ed2opssssples命t
改为R+(或Z+)或者R-(或Z-,则相应地得到过z的正半软道
或者负半尬道,并分别记为Oz+(z)或者O(z).
定义12过x的正(或负)半扬道的极限点称为的e(戡口
林totesuto沥2
E芸c,
[肉
E人c
E
E主阮一孙

的邹域口人吊,和菜个正整数丶,恋V|丿,有8CD)口一
必y不是游荫点的点称为非渡茹点,#的所有非游茹点的集合称
Page-41
B园`

迦介绍了臂Hilbert第16问题,在第三章中,我们综合运用第二
E述
E口一东沥u
集(Smale马踵)存在性的简浩而严格的判别方法.逄些结果在研
2
E
的由一个双曲鞋点和一个双曲闽轨形成的环的分岛、第六章介绍
医
沥途一
E怡1河E东A达
圣吊
江n
E
园2
E
E不命沥命
E
上,但不可邀免地涉及到一些离散动力系统的情形.我们力求在迹
医逊不
长的结果戡证明;在着力于可读性的同时,尽量兼顾一定的理论
深度;并疫注重解林推理的同时,薛顾几何直觅.本习的大部分
园t吴
了使读者晟于接受,我们对这些材料做了整理和加工,例如,第三
沥
作者重新络出的}在第六章大部分定理的证明中,作者对原始材料
沥扬不尘c胡I怡ci二
如,第二章中对参数一致的Hopf分岛定理;对Abel积分霆点个数
E
和能力,书中难兖有不奚或错误之处,我们热诚放迎读者们的批评
Page-42
a河n

Et「
E怀2
E

星然,wCz)巳Q(8),a(z)CCQ(8).因此,当M素致时,Q(9》
Et

E不伟园不吊

E医
称为一个临界元.

给出临界元的定义是为了陈述上筒洁.有时要对临界元进行
如下细致的区分.

在徽分间狄的情形,如果存在E2+,恋得(8)一,则2为
ul
E吴a怀
t2ctstunates怀沥
园阡

在向坤场的惊形,临界元工有两种类型.一种类型是,工由一
Eust刑述422i92
Ec
夺0是口的一个双曲不动点(或等价地,DXCB)的所有特征根
Ei沥余p规林d
E红东Lt不王2纳
E八水吴余政技
E
E
1为特征值的特征向量

在研究向量场的轨道结构时,局部的困集在奇点附近(就奇
E述
E命河
Page-43
述颠

e
E化坤医沥李p25林丞
c
不隼证明,Oo(z),w(z),a(z),0(9)都是p的不变集.
n
达
定义1.6设7是
E虑a
E
E沥5
e途贤水
E
向量场在口上每一点与
E怀5小
U,pEDCU,和在
【坤标
使得TC》等于?的周
E吴孙亚
E林0江

E0林9
E

国玟2技刑东口技
射,可以把对C向量场在闭转7附近执道绪构的研究,转化为对
E述25

我们现在转向结构稳定问题,篓言之,在“小扰动“下不改变
E

E不王
E余2
Page-44
E江沥

园渡沥怀吴一沥
党生的基硬课教材!也可供相关学科学生或科技人员当作参考书。
明沥述
1990年“南开动力系统年*期间被用作敏材,根据我们的经验,对
轻河达沥
Ei汀孝沥一一一芸沥河江
的\二个学期的谜稿,则可以讲授第一章到第五章的全部内容,第
E汀
立即转入第二\三章,而第一章的其它各节可在适当时候再学,这
园邹育沥人

E
武、五竹君,王铉、高素志,唐云\张伟年、李宝殿、李举萱,肖冬梅、
齐东文.曹欢罗、王兰孙,赵尔琴,彭临平等同志和我们的研究生们
表示愚谢,他们的报告和讨论使我们曼益匪浅其中有的间志诈帮
n
E
E扬兰不述述
坦的吊位专家,他们从审定本书的撞写计划到审说书稿都提出了
E河述东东
援,仪们不仅对本习提出了很多建设性的意见,西东还提供了部分
习题;.感谢张装庆教捷,他在百忍中审阅了本书的附录,并提出了
宝贵的惧见;感谢高等敬育出版社的杨芸馨等同志,变有他们的
E江江t东

圭描写本书期间,作者们得到国家自然科学基金和离等学校
e
E

L
1995年8月于北京大学
Page-45
第一章
E
B园
E河
E沥
E

【
E
E

.不3
E
E
E

第三章
E
B
E

第四章
E
E

[

基本概念和准备知识

E伟莲

分岔与分岔何题的提法
E

E

E
E
\part{}
常见的局部与非局部分岔

奇点分岗

c

Hopt分岔E
E

Poincart分岛与弘Hilbert第16何题
关于Petrov定理的诛明
E
E
E沥林标许一
河
E
E

友曲不动点及马蹄存在定理
E

E
Page-46
9E浩

n
水
E
520技标t标沥伟
【t江y不

当同背还保持儿与X:相应转道的时间对应时,秒为拓扑
转
此时常省略“C1“,简称为结构稳定.

诊

面的局部结果.-
玟

的开集;向量场元以O为双曲奇炉(或g:口一RF以O为双曲不
EJRS8沥rii沥俊s沥p
E述绍e

扑共软,〖

E沥珑d
为双曲奇点,则一在O点附近局部绪构稳定,即存在丿在.“CRT)
中的一个C+邻域,YZE,在0附近有唯一的双曲奇炉,万
EppasEueoet沥怀e

E

附注110“如果把定义1.7中的拓扑扔道等价从C?加强到
E心述nl
性化系统的特征根有相同的比值(见[GH,p42].这就把等价关系
限制过严,使得两个软道结构相同的向量场也未必等价例如,由
定理1.9可知,二维系统

五二z予二3
e
E林逊
Page-47
史3“Smale马蹄。i
$4线恒胥射的复吊昼尉的双峡何
5“Birkhoff-Smale定理
第五章空间中双曲鞍点的同宿分岔
Et
2n
医
邹公章“实二次单峰脓射旋的颂引子
E河玲
E沥沥腾3浩5
$3PCzva)不存在稳定周期狸闵题
荃4分布问题......2
耐′暴E
附录ADenzct浑形和流形问的际躬

E
附录CThom模截定理-

参考文献
E王
Page-48
Ec0

(不0)是C等价的.但它们不是C等价的,另一方西,如果把定
uy
E
一是局部结构稳定的,但它的C扰动系统丿一z十xVTzT与
玟
人1因此,如无特别声明,下文中的等价都挡C“等价,而扰动都
栗C抓动.
在给出进一步的结果之前,我们需要下面的定义.
定义1.11设口,p咏上.称集合
E林
和
E一伟
分别为p在双曲不动点0的穗定流渺与不穗定流形
对二削量场的情形,有类似的定义.
E
林
E
A孝仪)
分别称为又在?的穗定流形和不稳定流形.
利用Poincart晓射,可以把吊量场在双曲闭轨附近的研究转
中育沥
Agd沥芸E

以
E
(O))在0点附近都是M的子流形进而可以证明,它们在O
E沥沥林5振
E沥
Page-49
第一章基本概念和准备知识

作为全书的准备,我们在$1中简述有关动力系统和结构程
定的基本概念,不加证明地陈述一些重要结果;在82中引入分岑
E
心流形定理和正规形理论,在研究分岔问题时它们是进行简化处
理的有效手段!最后,在85中介绍奇异向量场的普适开折和分岔
的余维这两个重要的概念.

E宇

动力系统的概念和理论是从人们对常微分方程的研究中产生
和发展起来的,而丁对常微分方程的研究,至今仍是动力系统理论
a

E

董=/彻)】0

其中广R*一R“是C向量场,7之1.由常微分方程中熟知的缤
E
含t一0的棠区间上存在.如果/Cz)满足适当条件(或在某种等
E江标2李p水i
E

0李2

422E玖
E河3技2
Page-50
BE

形,见附录日中附注B,24》,证明可参考[ZQ,定理4.9和4.10].

E芸沥闵团李国口
全局绪构稳定性的结果(参见[ZDHD]).

E一
s

265东

《2成的临界元(奇点或闭轨)个数有限,并且它们都是双曲
的j

[沥5
流形槲截相交,【

这里朱截性的定义见附录C中定义C10.注意,不相交也算
E

定理1.14“二维可定向紧敌流形M上C“结构穗定向量场的
E

附注1.15Peixoto等人的上述结果,从60年代开始呆引了
E
E河
把满足定理113中三个条件的向量场称为Morse-Smale向量场
E沥
E沥沥招达
著名侧子(见第四章),以及Newhouse随后对“马跨“的改造,说明
野B命河
(或微分同胚)未必是M-S的,百全体结构的稳定向最来(或徽分
同胚)的集合在““(M)(或Ditf“CM))中不一定是稠集.进一步
i
Eurseceouetenseoaeatss述
M-S条件更广的公理A条件,并东给出两个结构穗定性猜滁;结
Ed
结构稳定33公理A十无环条件.这两个猜洪的充分性部分已为
Page-51
E河

E

e
E

Ed2oto

不隼证明,对Vz,z:ER“,O。(z)和O。(za)或者重合,或
者(对有限的时间:)不相交.因此,(1.1)的转道集合依一的不同
E

常微分方程定性理论(或称为儿何理论)的首要目标,就是对
E林td2s
构(扬道集合的拓扑结构图称为相图,通常在相图上用箭头标明对
D
E肉
E[
倩形,人们至今尚不清楚相图的确切结构.参见[YL,2],[Q]

玟
E木
E林口e一c
玟
C1,定义C.2,附注C.4).Y叉E&“(M),存在弋过8E趸的概
大流ar(见附录B中定义B.13,定义B15,和定理B.16),为了讨
1
无限延伸;使XXE&r“(M)的极大流在(一co,十co)上存在(香
E

伟

0颖5

E林i
E

王
Page-52
E振g9

Smale本人和其它人所证明;必要性部分则于1987年分别由
ucutoueotuteuooalotaos招应5兰阮l汀
的第一个猪测,直到最近才由廖山溥0“,胡森“7,和Hayashi
s

$2-分岔与分岔问题的提法

怡故江林林余
a沥
研究结构不穗定向量场在“扰动“下捉道结构的变化规律,是分岔
E

分岔的概念

E
e

E沥沥20
则称;为族丿(6)的分岱值,当参数s通过分岔值时,在相空间中
sss

对于离敬动力系统,也可给出类似的定义.

[园u
奇点必是非双曲的.

医肖
E1
或者它的奇点和闭轨是双曲的,伯它们的穗定流形和不稳定流形
i
游荣集含有无限多个临界元.

E林
E不b
Page-53
E骏ssE

园2[

医吊5志5
参数变换群;对固定+ER和所有的E聪,ex(t,办是故一版的
微分同胚.

[

2

氏2a

E林

0

0

[

有时也把句称为壮上的Cr徽分动力票统特别,当+EZ,
称为穴散动力系统,本书主要讨论连续流、由于它与离散流有密
一王沥国许汀江士江梁圆国

E怀园
E沥述林2

Ed2opssssples命t
改为R+(或Z+)或者R-(或Z-,则相应地得到过z的正半软道
或者负半尬道,并分别记为Oz+(z)或者O(z).
定义12过x的正(或负)半扬道的极限点称为的e(戡口
林totesuto沥2
E芸c,
[肉
E人c
E
E主阮一孙

的邹域口人吊,和菜个正整数丶,恋V|丿,有8CD)口一
必y不是游荫点的点称为非渡茹点,#的所有非游茹点的集合称
Page-54
EE

E仪是这个开稠集的〖边界[但在高维流形的情形,

E胡3

E怀坂国圆y
E林
极限集与。极限集都与这奇点(或闭辅)一致-相轨线称为异宿
Cheterolinie)扬,如果它的e极限集和。极限集是不同的契点或
沥

定义2.4发生在奇炉(或闭转)的小邻域内,并且与它的双
E不
i
E

一面的定义2.1一2.4叟自[AAIS].我们将在后面看到,在研
野不
也可能伴随出现全局分岔.

附注2.58“如上所述,分岔集心(M)一4“CMJNSrCM)是

t述i述
Ed不育1s浩6儿4河浩
仁GM中具有更复杂分岔现象的闭子集.我们可以继绪对

47.QK)进行这种剔分,得到各种层次的分岔集.
E技

盖=焯一鲈=瞄″】磺沈0

E
0
e
Eii
是双幼的,图1-2中给出(2.1皇奇点分布对参数x的依赖关系.
E
Page-55
a河n

Et「
E怀2
E

星然,wCz)巳Q(8),a(z)CCQ(8).因此,当M素致时,Q(9》
Et

E不伟园不吊

E医
称为一个临界元.

给出临界元的定义是为了陈述上筒洁.有时要对临界元进行
如下细致的区分.

在徽分间狄的情形,如果存在E2+,恋得(8)一,则2为
ul
E吴a怀
t2ctstunates怀沥
园阡

在向坤场的惊形,临界元工有两种类型.一种类型是,工由一
Eust刑述422i92
Ec
夺0是口的一个双曲不动点(或等价地,DXCB)的所有特征根
Ei沥余p规林d
E红东Lt不王2纳
E八水吴余政技
E
E
1为特征值的特征向量

在研究向量场的轨道结构时,局部的困集在奇点附近(就奇
E述
E命河
Page-56
E源e

一

EE

|
薯
|

丨D
薯
EE

E怡玟应38in诊沥才仪
i
E水5
刑(2.1的转道拓扑结构所发生的变化.显然,一0是唯一的
E

例2.7“考虑R上的向量场

蔷=#_z=,E

与上例同样的分析可知,k一0是(22)的唯一分岔值;一0时向
量场(2.2)无奇点;当p>0时,(2.2)有二个双曲奇炎当一0时
[应沥
(sadale-uode),并把这种分岗现象称为鞍结点分岔,与上例相似,
可分别作出图1-4与图1-5.从下面的例中,我们可以对这一名称
E'

例2.8“二维的鞍结点分岔.考虑R上的向量地

{0
F'

国dO是唯一的分岔值,并可作出轨道拓扑分类图
Page-57
述颠

e
E化坤医沥李p25林丞
c
不隼证明,Oo(z),w(z),a(z),0(9)都是p的不变集.
n
达
定义1.6设7是
E虑a
E
E沥5
e途贤水
E
向量场在口上每一点与
E怀5小
U,pEDCU,和在
【坤标
使得TC》等于?的周
E吴孙亚
E林0江

E0林9
E

国玟2技刑东口技
射,可以把对C向量场在闭转7附近执道绪构的研究,转化为对
E述25

我们现在转向结构稳定问题,篓言之,在“小扰动“下不改变
E

E不王
E余2
Page-58
E途

云素轼
五M

E
E
例2.9“考虑R上(0,0)点附近的单参数系统族
i
【y
E沥

它的线伯部分短阵以x土i为特征值.在极坐标变换下,方程(2.4
变形为「
Page-59
EbE

当口0时,原点是租定的焦点(x一0时非双曲)当x丿0时,原
点为不稳定焦点,并有唯一闭轨r一VA,它是租定的极限环T.,
E不
捉道以它为极限集,因此称它为稳定的极限环.注意,当x一0
(

图1-7

Et
E
E

例2.19([DL]》首先考虑一个平面系统

0

E
E

F(z,汊)亭鲈十如z_藁氢=c`E

其中C为任意常数.为了下文中的方便,记

巳招「
C_27E0
Page-60
E

E
则方程(2.5的扬道依x的不同选取,有如下分布,

EpN
E

EC2te皇东
所国有界区域之外和Z的左仁当~>0-时,2x收缉到PU
E所

E
T它位于e所围有界区域的内部;另一条(无界)转道x在H
un

令函数

E(龛5″〕,

Ei技tn沥
D
于
5

一3

1
Page-61
9E浩

n
水
E
520技标t标沥伟
【t江y不

当同背还保持儿与X:相应转道的时间对应时,秒为拓扑
转
此时常省略“C1“,简称为结构稳定.

诊

面的局部结果.-
玟

的开集;向量场元以O为双曲奇炉(或g:口一RF以O为双曲不
EJRS8沥rii沥俊s沥p
E述绍e

扑共软,〖

E沥珑d
为双曲奇点,则一在O点附近局部绪构稳定,即存在丿在.“CRT)
中的一个C+邻域,YZE,在0附近有唯一的双曲奇炉,万
EppasEueoet沥怀e

E

附注110“如果把定义1.7中的拓扑扔道等价从C?加强到
E心述nl
性化系统的特征根有相同的比值(见[GH,p42].这就把等价关系
限制过严,使得两个软道结构相同的向量场也未必等价例如,由
定理1.9可知,二维系统

五二z予二3
e
E林逊
Page-62
弧

E堇虹_国
紫|,-媒訾+要刑|-arceouoy,

由此可知,当|x|心1时,系统(2.8)的轨道分布如图1-9所示.(所

Q

EE《op>0

国
用的论探,类似于用Liapunov函数判断奇点的穗定性).并东容易
2
E
[
E
玟
种半局部分会.
y沥5
0沥
【na汀E
其中a七0.在乙一0附近的一个邰域内,晏射尹以z一0为唯一
不动点.注意,由隐函数定理可知,(2.9)的任一扰动系统在z一0
附近仍有唯一的不动点,所以我们不妙取它的扰动系统保持z一0
为不助点,东具有下面的形式
E应2n
E中(
性Fx的两次境代晓射,得到
Page-63
Ec0

(不0)是C等价的.但它们不是C等价的,另一方西,如果把定
uy
E
一是局部结构稳定的,但它的C扰动系统丿一z十xVTzT与
玟
人1因此,如无特别声明,下文中的等价都挡C“等价,而扰动都
栗C抓动.
在给出进一步的结果之前,我们需要下面的定义.
定义1.11设口,p咏上.称集合
E林
和
E一伟
分别为p在双曲不动点0的穗定流渺与不穗定流形
对二削量场的情形,有类似的定义.
E
林
E
A孝仪)
分别称为又在?的穗定流形和不稳定流形.
利用Poincart晓射,可以把吊量场在双曲闭轨附近的研究转
中育沥
Agd沥芸E

以
E
(O))在0点附近都是M的子流形进而可以证明,它们在O
E沥沥林5振
E沥
Page-64
E国a振s

p胡2仪4
从面硼(z一z可表示为
z[x(2+闪十O(UAJz~-(2+0CO)Jazz+O(|z[-
E不
E河河芬河沥扬
点)之外,又有两个新的不动点,它们是F,的2周期点(图1-100a)

-训
E
E
e@A
\
E
史EE

0

E
E
的Poincare映射,则当x的值从负到正的瞬间(设a>>0),原有的稚
E
儿乎两倍于原周期的闭轨兰,这种分岔现象称为借周期分岑,它发
E浩东达应2
这带的边界

分岔问题的提法
E
寄x[

E吴技l一江4
Page-65
BE

形,见附录日中附注B,24》,证明可参考[ZQ,定理4.9和4.10].

E芸沥闵团李国口
全局绪构稳定性的结果(参见[ZDHD]).

E一
s

265东

《2成的临界元(奇点或闭轨)个数有限,并且它们都是双曲
的j

[沥5
流形槲截相交,【

这里朱截性的定义见附录C中定义C10.注意,不相交也算
E

定理1.14“二维可定向紧敌流形M上C“结构穗定向量场的
E

附注1.15Peixoto等人的上述结果,从60年代开始呆引了
E
E河
把满足定理113中三个条件的向量场称为Morse-Smale向量场
E沥
E沥沥招达
著名侧子(见第四章),以及Newhouse随后对“马跨“的改造,说明
野B命河
(或微分同胚)未必是M-S的,百全体结构的稳定向最来(或徽分
同胚)的集合在““(M)(或Ditf“CM))中不一定是稠集.进一步
i
Eurseceouetenseoaeatss述
M-S条件更广的公理A条件,并东给出两个结构穗定性猜滁;结
Ed
结构稳定33公理A十无环条件.这两个猜洪的充分性部分已为
Page-66
Ess

蛀E0

是结构不稳定的,当川人1时,常把(2.10)称寺(2.11的一个
Eu1

[i
7中变动时,弄清系统(2.10的转道结构如何变化?

E余
们对应(2.10的轨道拓扑结构的不同等价类?在例2.6中,一0
Ed东
0

进一歪的问题是。

间题B“能否找到(2.11)的开折,它“包含*了(2.113的任一
开折所能出现的转道结构?

[
d

对上述问题中一些名词的确切含义进行澄清,将会引导到“普
适开扳“和“分岔的余维“这样一些深刻的概念,我们将在85中
介绍,本节先对这些概念给出直观的描述,这里需要指出的是,由
E
Badc
E浩d人2
E

E

茁E0

i
折.我们要证明,(2.1不是满足间题B要求的那种开折E
感(2.127的任意一个C阗E

茁AE
Page-67
E振g9

Smale本人和其它人所证明;必要性部分则于1987年分别由
ucutoueotuteuooalotaos招应5兰阮l汀
的第一个猪测,直到最近才由廖山溥0“,胡森“7,和Hayashi
s

$2-分岔与分岔问题的提法

怡故江林林余
a沥
研究结构不穗定向量场在“扰动“下捉道结构的变化规律,是分岔
E

分岔的概念

E
e

E沥沥20
则称;为族丿(6)的分岱值,当参数s通过分岔值时,在相空间中
sss

对于离敬动力系统,也可给出类似的定义.

[园u
奇点必是非双曲的.

医肖
E1
或者它的奇点和闭轨是双曲的,伯它们的穗定流形和不稳定流形
i
游荣集含有无限多个临界元.

E林
E不b
Page-68
EE

Ed许
d
E

林
开集,丁EC“(U,R)并且漪足ft,0)一tg(,其中&E2+,g在
t才
仪8i
E河
E人2

E(红,(omE人

E
到有限光滑性.由上面的定理可知,对于开折(13)存在R““,中
ta
E(zE
E

E
E05仪
玟

黯E林达A225

E应
E

c
+2C08Z,xto=c林

2改写为z则系统(2.15)变为
盖:E怡n
星然,(2.16)所能出现的转道拓扑类型不趣出系统
Page-69
E

莹沥[
Eppb
们证明了(2.12)的任一开折(可以含有任意多个参数)所能出现
E标沥
c
cd
园沥吊d训

集合为原点O(0,0)和由
伟a人

玲
3

余吴

Eotuteegcutot芸ap

图-11E
岔图1-11.相应于不同的,开折(2,17》的5种转道拓扑分类中有
E
E匹才
E浩育一
Page-70
EE

E仪是这个开稠集的〖边界[但在高维流形的情形,

E胡3

E怀坂国圆y
E林
极限集与。极限集都与这奇点(或闭辅)一致-相轨线称为异宿
Cheterolinie)扬,如果它的e极限集和。极限集是不同的契点或
沥

定义2.4发生在奇炉(或闭转)的小邻域内,并且与它的双
E不
i
E

一面的定义2.1一2.4叟自[AAIS].我们将在后面看到,在研
野不
也可能伴随出现全局分岔.

附注2.58“如上所述,分岔集心(M)一4“CMJNSrCM)是

t述i述
Ed不育1s浩6儿4河浩
仁GM中具有更复杂分岔现象的闭子集.我们可以继绪对

47.QK)进行这种剔分,得到各种层次的分岔集.
E技

盖=焯一鲈=瞄″】磺沈0

E
0
e
Eii
是双幼的,图1-2中给出(2.1皇奇点分布对参数x的依赖关系.
E
Page-71
E源e

一

EE

|
薯
|

丨D
薯
EE

E怡玟应38in诊沥才仪
i
E水5
刑(2.1的转道拓扑结构所发生的变化.显然,一0是唯一的
E

例2.7“考虑R上的向量场

蔷=#_z=,E

与上例同样的分析可知,k一0是(22)的唯一分岔值;一0时向
量场(2.2)无奇点;当p>0时,(2.2)有二个双曲奇炎当一0时
[应沥
(sadale-uode),并把这种分岗现象称为鞍结点分岔,与上例相似,
可分别作出图1-4与图1-5.从下面的例中,我们可以对这一名称
E'

例2.8“二维的鞍结点分岔.考虑R上的向量地

{0
F'

国dO是唯一的分岔值,并可作出轨道拓扑分类图
Page-72
20E渡s

E沥
E伟江医一
E一
妮

E沥医不
E述E
引出的分岔为标维2的事实上,由于(2.12的任一开折都可等
E木
l刑
E达才
E
tC八月3相肖匹77技2江许河aE办水
E一

2颖兄二莲沥

沥江
E
河
E仪
绍的曲面(相应于
[胡
开折)才能与4模
朐
1-13,而(2.12)的
E
的工中的曲线蛎然闯命
E
相交,但在任意小的批动下,它都可以与口分离.换男话说,至少
E一
【
Page-73
E途

云素轼
五M

E
E
例2.9“考虑R上(0,0)点附近的单参数系统族
i
【y
E沥

它的线伯部分短阵以x土i为特征值.在极坐标变换下,方程(2.4
变形为「
Page-74
EbE

当口0时,原点是租定的焦点(x一0时非双曲)当x丿0时,原
点为不稳定焦点,并有唯一闭轨r一VA,它是租定的极限环T.,
E不
捉道以它为极限集,因此称它为稳定的极限环.注意,当x一0
(

图1-7

Et
E
E

例2.19([DL]》首先考虑一个平面系统

0

E
E

F(z,汊)亭鲈十如z_藁氢=c`E

其中C为任意常数.为了下文中的方便,记

巳招「
C_27E0
Page-75
0国

纳
E

水
E木东木

0
E朋

E沥圆河丿

s
uss述i江
Ey沥朐述i一
E不
E河i月
把结果推广到微分同狐的情形(例如,参见[Wil]).

观察上节图1-6可以发现,二维空间上的分岔现象其实主要
E水
图1-5),而在这不变流形之外的执线,无非是向这不变流形的压
绘.出现这种规律并不是健然的,系统(2.8在(0,0点的线性部
分短阵为

[
1

E
El
[
,近限制在树一个m维的不变流形上,从而使和题的隼度得以降
[
Page-76
E

E
则方程(2.5的扬道依x的不同选取,有如下分布,

EpN
E

EC2te皇东
所国有界区域之外和Z的左仁当~>0-时,2x收缉到PU
E所

E
T它位于e所围有界区域的内部;另一条(无界)转道x在H
un

令函数

E(龛5″〕,

Ei技tn沥
D
于
5

一3

1
Page-77
第一章药本李伊和淅备知订

线性情形
先考虑线性方程
蕃E医

玟

5[
的性态完全被矩阵4的特征值的性质所决定.

Ep
E育7河
E
E吴F刑
0

E

记口为R中相应于2Ec的那些特征值的广义特征向量所张成
E
E沥八达述

邵相应的投影
动t酥一,i职一酥t联一阮
E
ker(m)二E′“鼻E翻E
ker(xu)二丘v…鼻Ec申吴,
EE^矗鼻E翼日.
E仪

医
的,而在B中则是指数型“增长“的(t一一eo时的情况相反.所
Page-78
弧

E堇虹_国
紫|,-媒訾+要刑|-arceouoy,

由此可知,当|x|心1时,系统(2.8)的轨道分布如图1-9所示.(所

Q

EE《op>0

国
用的论探,类似于用Liapunov函数判断奇点的穗定性).并东容易
2
E
[
E
玟
种半局部分会.
y沥5
0沥
【na汀E
其中a七0.在乙一0附近的一个邰域内,晏射尹以z一0为唯一
不动点.注意,由隐函数定理可知,(2.9)的任一扰动系统在z一0
附近仍有唯一的不动点,所以我们不妙取它的扰动系统保持z一0
为不助点,东具有下面的形式
E应2n
E中(
性Fx的两次境代晓射,得到
Page-79
EE

有对f~士c有界的执道(特别地,所有契点,闭轨)都停国在砂
医s技不园2
Esu
E渡c
E

非线性情形
E
需=Az十八z兆【

E技0
程(3.5的转道结构是否仍然具有方程(3.1)的上述规律?下面
Ed沥i
0浩g途途木

常实用的还是在契点z一0的局部,那些“复杂现象“(特别地,所
有奇点.闭转.同宿转、异宪执等)都发生在W上;在一定条件下,
国O述沥
过对E上诱导的方程的研究而得到.本节的内容主要参考了[V]
E

我们先陈述整佛的结果.

不伟沥园育20技人

[
E
林
一一一
E
C
0
Page-80
E国a振s

p胡2仪4
从面硼(z一z可表示为
z[x(2+闪十O(UAJz~-(2+0CO)Jazz+O(|z[-
E不
E河河芬河沥扬
点)之外,又有两个新的不动点,它们是F,的2周期点(图1-100a)

-训
E
E
e@A
\
E
史EE

0

E
E
的Poincare映射,则当x的值从负到正的瞬间(设a>>0),原有的稚
E
儿乎两倍于原周期的闭轨兰,这种分岔现象称为借周期分岑,它发
E浩东达应2
这带的边界

分岔问题的提法
E
寄x[

E吴技l一江4
Page-81
第一章荣本林伊和醉吉知订

E噩副嘲…0解‖d医力

E2
E

E吴c居2E述
[
E述2园医
是(3.5的不变集,则M一印,且一一9!

[林述
箐=^跷+骗八跷十俨〈蹄)),E沥E达

定义3.3“定理3.2中的不变集卫*称为(3.5)的全局中心流
E

[诊玟明述怀述sss
指出,眙Pr在(3.5)的流下不变,则pEC8CB,)是唯一确定
E唐u沥

E
很难应用.由于0)一0,DfK(0)=0故在奇库<一0附近这个条
件却是自然成立的,因此,利用截断(cut-oftt函数从定理3.2得出
i
E

1,当肥奶1,

k{0yE
E

E/(z)刁[熹〕E(3.9)

此时,为了研究方程(3.5)在x一0附近的中心流形,我们可以考
乐方程

塔<4z+foCam,[
Page-82
Ess

蛀E0

是结构不稳定的,当川人1时,常把(2.10)称寺(2.11的一个
Eu1

[i
7中变动时,弄清系统(2.10的转道结构如何变化?

E余
们对应(2.10的轨道拓扑结构的不同等价类?在例2.6中,一0
Ed东
0

进一歪的问题是。

间题B“能否找到(2.11)的开折,它“包含*了(2.113的任一
开折所能出现的转道结构?

[
d

对上述问题中一些名词的确切含义进行澄清,将会引导到“普
适开扳“和“分岔的余维“这样一些深刻的概念,我们将在85中
介绍,本节先对这些概念给出直观的描述,这里需要指出的是,由
E
Badc
E浩d人2
E

E

茁E0

i
折.我们要证明,(2.1不是满足间题B要求的那种开折E
感(2.127的任意一个C阗E

茁AE
Page-83
EE

Ed许
d
E

林
开集,丁EC“(U,R)并且漪足ft,0)一tg(,其中&E2+,g在
t才
仪8i
E河
E人2

E(红,(omE人

E
到有限光滑性.由上面的定理可知,对于开折(13)存在R““,中
ta
E(zE
E

E
E05仪
玟

黯E林达A225

E应
E

c
+2C08Z,xto=c林

2改写为z则系统(2.15)变为
盖:E怡n
星然,(2.16)所能出现的转道拓扑类型不趣出系统
Page-84
E

E沥25李训0河
uyC3.10)
E东沥明不一圆育2标吴
ACo二0,D/(0)=0,剧3pECKCR,Bh)和z一0在R中的开
E
CD流形
E52着育2
对(3,5)的流局部不变,印
E仪一A人3
E沥c林g5加
E朋
【C2EX河A月
[沥招川
E国育林EEi日技振才f技川[
E朋
沥
Ey
男一方面,由(3.9)可知,英故-(zER国i川
E技0

E述人育沥
现设zEU,Ju(z)一R.则YxER,2(tyz)一t,a)口,

从而着彗…I″^…〈hz)|一co,由(3.6)式,zE佛r,限制在乙内,也就
E宏吴
匹出东河]如果婢仨C^(E(,互*)浓蓼L渊mE兴X00河玟
A林目园技孙芸不
国42
[王a沥n挂u口
Page-85
E

莹沥[
Eppb
们证明了(2.12)的任一开折(可以含有任意多个参数)所能出现
E标沥
c
cd
园沥吊d训

集合为原点O(0,0)和由
伟a人

玲
3

余吴

Eotuteegcutot芸ap

图-11E
岔图1-11.相应于不同的,开折(2,17》的5种转道拓扑分类中有
E
E匹才
E浩育一
Page-86
EE

同的局部中心流形(尽管对每一个截断函数而言,(3.10的全局
中心流形是唯一的.例如,在图!-6(x一0)中的原点附近,取有
0
i
持在U内的任何有界执道(包括奇点.周期转、同宿转,异宿轨等)
都出现在(3.5)的任一局部中心流形上.因此,对于研究分岔现象
00沥
蚺然了的C光滢性保证了吐的C4光滑性,一舫来说,了的C“光
玟
i
IDAal一8#一舫来说,当&一eo时,5x一0,这可能导致p一0

现在我们把局部中心流形、稳定、不稳定流形的结果合写成下
面的定理.

E王5东沥英育5222技
DAC0)一0;相对于4,有如上所述的子空间史,BY和Y,则在R“
中乙一0附近存在开邻域[,和口中的Ct流形WPr,Wv和伟,它们
的维数分别与这三个子空间相同,在z=0点分别与吴,砂和砂
相切,并且在口内是方程(3.5的不变流形iW7“和P“有定义1.12
野
E命巳(

[
E沥e洁
到,系统在契点附近的“复杂现象“发生在它的任一局部中心流形
上.下面的两个定理说月了中心流形的其它重要作用、

定理3.9(渐近性质定理〉设了EC(R“,R“),f(0)一0,
D
i
EA22S5ac2明l伟一东人
Page-87
20E渡s

E沥
E伟江医一
E一
妮

E沥医不
E述E
引出的分岔为标维2的事实上,由于(2.12的任一开折都可等
E木
l刑
E达才
E
tC八月3相肖匹77技2江许河aE办水
E一

2颖兄二莲沥

沥江
E
河
E仪
绍的曲面(相应于
[胡
开折)才能与4模
朐
1-13,而(2.12)的
E
的工中的曲线蛎然闯命
E
相交,但在任意小的批动下,它都可以与口分离.换男话说,至少
E一
【
Page-88
E

玖
E东述l国
0si逊t浩仪
E5沥
E
E吴育沥政
明,在一定条件下,
E
纳
i
型地趋于中心流形
上的某一解(当4
一十co如果ou一
【nice
E砂
定理3.10说明,在
类似条件下,为了
得到局部中心流形上0点附近的执道结构,只需要对它在线性子
E水
0ooxudtui仁e
E述2河
E
E东0月0

E吴见0丞A
|z五)}是(3.5的一个局部中心流形,当且仅当存在B中原

点的开邰域0,侩得VzE,有
E述
Page-89
0国

纳
E

水
E木东木

0
E朋

E沥圆河丿

s
uss述i江
Ey沥朐述i一
E不
E河i月
把结果推广到微分同狐的情形(例如,参见[Wil]).

观察上节图1-6可以发现,二维空间上的分岔现象其实主要
E水
图1-5),而在这不变流形之外的执线,无非是向这不变流形的压
绘.出现这种规律并不是健然的,系统(2.8在(0,0点的线性部
分短阵为

[
1

E
El
[
,近限制在树一个m维的不变流形上,从而使和题的隼度得以降
[
Page-90
第一章焦本概伊和淅备知识-

E河[

E
利用(3.14)式算出它的Taylor展式的前儿项-
为了简单,先把(3.5)化成如下的标准形式:

E

4[
E

医林技人5
D
门善以0]'
0C

上面的C与C:的特征根实部分别为负数与正数,图此,B一[(z,
C2沥
刑技水

u林
E

u仁
[
e
E

利用待定系数法,可逐项计算p(z2.

例3.12考虑二维方程

血`
E汀
E

月
经过扰动在奇点(0,0)附近可能发生的分岔现象,其中8么0注

意(3′17)在(0,0)点的线性部分矩阵为〔彗熹〕ypudseatuu
Page-91
第一章药本李伊和淅备知订

线性情形
先考虑线性方程
蕃E医

玟

5[
的性态完全被矩阵4的特征值的性质所决定.

Ep
E育7河
E
E吴F刑
0

E

记口为R中相应于2Ec的那些特征值的广义特征向量所张成
E
E沥八达述

邵相应的投影
动t酥一,i职一酥t联一阮
E
ker(m)二E′“鼻E翻E
ker(xu)二丘v…鼻Ec申吴,
EE^矗鼻E翼日.
E仪

医
的,而在B中则是指数型“增长“的(t一一eo时的情况相反.所
Page-92
述E

国我们首先设法找出方
E巳p中水
E2理2振3招5

令

圆目
[沥
盯一一岸她+州音(″+讪v,

【、《C3.18)
茁E萨(鹰吴小霄(″E
Eosothetn技5
E.
“E沥水0河逊
E
E

E痨“za

E的第一个方程u

导的方程为
豇一_卑″z+目世0.

E沥一伟
Eatptb河沥
E诊L一广

E[

E
E吊E水江2
Page-93
EE

有对f~士c有界的执道(特别地,所有契点,闭轨)都停国在砂
医s技不园2
Esu
E渡c
E

非线性情形
E
需=Az十八z兆【

E技0
程(3.5的转道结构是否仍然具有方程(3.1)的上述规律?下面
Ed沥i
0浩g途途木

常实用的还是在契点z一0的局部,那些“复杂现象“(特别地,所
有奇点.闭转.同宿转、异宪执等)都发生在W上;在一定条件下,
国O述沥
过对E上诱导的方程的研究而得到.本节的内容主要参考了[V]
E

我们先陈述整佛的结果.

不伟沥园育20技人

[
E
林
一一一
E
C
0
Page-94
E第一章基本梓伊和淅吊知识

E林202月
医东技国
E

E园王
E团

门吴0
E

i
0

附注3.14“为了研究(3.19)的中心洪形,我们把x也视作
[

巳0国
沥(3.21》

玟
而中心子空间为BXRe.故(3.21)的局部稳定流形与不稳定流
形印与WP*的结构与x三0时类似.此时中心流形印一{(zr,当
水

砂.由(3.21)的第二个方程易知,(Cz,471x一常数}是(3.21)
的不变集,从而对固定的x,W<|心希数是(3.21)第一个方程的不
E浩00圭
对于不同的sW“|,e-帕数上的辐道结构可能不同,见下例-

例3.15“设z,y,xER,考虑光滢系统

振一

[蟒一厂一口+JCaoyot,

【。
雹言E技林班35江秀

E
E
Page-95
第一章荣本林伊和醉吉知订

E噩副嘲…0解‖d医力

E2
E

E吴c居2E述
[
E述2园医
是(3.5的不变集,则M一印,且一一9!

[林述
箐=^跷+骗八跷十俨〈蹄)),E沥E达

定义3.3“定理3.2中的不变集卫*称为(3.5)的全局中心流
E

[诊玟明述怀述sss
指出,眙Pr在(3.5)的流下不变,则pEC8CB,)是唯一确定
E唐u沥

E
很难应用.由于0)一0,DfK(0)=0故在奇库<一0附近这个条
件却是自然成立的,因此,利用截断(cut-oftt函数从定理3.2得出
i
E

1,当肥奶1,

k{0yE
E

E/(z)刁[熹〕E(3.9)

此时,为了研究方程(3.5)在x一0附近的中心流形,我们可以考
乐方程

塔<4z+foCam,[
Page-96
E人
3.14和例2.6可知,在(z,y,4)一(0,0,0)的小邻域内,中心流形
e沥一诊
E一
E

c
E沥
Elaeiopk5丶吴纳才a

在下文对所都分岖的讨论中,我们大都假定已经把闭题化归

到它的中心流形上,即对所论方程的线性部分短阵A一言,o(40
E′

$4正规形

E
Eoposeeosueuetiaioggg月加芸s沥余
尽可能简单的形式,以便于研究.这是源于Poincarg时代的一个课
Page-97
E

E沥25李训0河
uyC3.10)
E东沥明不一圆育2标吴
ACo二0,D/(0)=0,剧3pECKCR,Bh)和z一0在R中的开
E
CD流形
E52着育2
对(3,5)的流局部不变,印
E仪一A人3
E沥c林g5加
E朋
【C2EX河A月
[沥招川
E国育林EEi日技振才f技川[
E朋
沥
Ey
男一方面,由(3.9)可知,英故-(zER国i川
E技0

E述人育沥
现设zEU,Ju(z)一R.则YxER,2(tyz)一t,a)口,

从而着彗…I″^…〈hz)|一co,由(3.6)式,zE佛r,限制在乙内,也就
E宏吴
匹出东河]如果婢仨C^(E(,互*)浓蓼L渊mE兴X00河玟
A林目园技孙芸不
国42
[王a沥n挂u口
Page-98
32E招

莲
新引起人们对它的重视,并得出若干计算正规形的新方法.
众所周知,经非退化线性变换z一73,线性微分方程

气亩7C=^薹
变换为

蛊翼E林

54沥沥芒春沥坂步招a命口P
线性系统的辅道结构时,我们无妙假设4为Jordan标准形,除了
E
对非线性部分是否可以作类似的简化?在一定意义下,答案是肯定
e
a沥

微分方程在奇点附近的正规形

考感以乙一0为奇点的C微分方程Cr之3它在<一0附近

可表示为

蔫E河玟伟02沥沥2人

E沥
维&次半次向量多项式所成的空间心一2,...,7一1.
先进行变摄
E关志玟50

E2怀ag
换642)代入(41,并注意

[c
阮
O(ly1),而(ly|5表示X%短阵,它的每一元素都是O(110,
由此把方程(41化为
Page-99
84正规形33

许

的李玟4志人北小6国
其中户与(4.1)中的相同,而共是经过运算得到的新的次齐次
多项式
引入算于ad:技一,
Edn
E

沥

…+弄叶嫩)+O‖川′),0
记毋?为算子at&在H中的值域,而罗“是命*在H8中的一个补
唐

E伟

E林3朐]
n

E5发扬[
这祥,我们把(4.1化为

林林伟

0
E
E
E河或技0
E恩沥浩规i江a育td

E

蔫E李沥汀林述0

E园8
Page-100
EE

同的局部中心流形(尽管对每一个截断函数而言,(3.10的全局
中心流形是唯一的.例如,在图!-6(x一0)中的原点附近,取有
0
i
持在U内的任何有界执道(包括奇点.周期转、同宿转,异宿轨等)
都出现在(3.5)的任一局部中心流形上.因此,对于研究分岔现象
00沥
蚺然了的C光滢性保证了吐的C4光滑性,一舫来说,了的C“光
玟
i
IDAal一8#一舫来说,当&一eo时,5x一0,这可能导致p一0

现在我们把局部中心流形、稳定、不稳定流形的结果合写成下
面的定理.

E王5东沥英育5222技
DAC0)一0;相对于4,有如上所述的子空间史,BY和Y,则在R“
中乙一0附近存在开邻域[,和口中的Ct流形WPr,Wv和伟,它们
的维数分别与这三个子空间相同,在z=0点分别与吴,砂和砂
相切,并且在口内是方程(3.5的不变流形iW7“和P“有定义1.12
野
E命巳(

[
E沥e洁
到,系统在契点附近的“复杂现象“发生在它的任一局部中心流形
上.下面的两个定理说月了中心流形的其它重要作用、

定理3.9(渐近性质定理〉设了EC(R“,R“),f(0)一0,
D
i
EA22S5ac2明l伟一东人
Page-101
0E河

其中g户与(4.7中的相同,而i
洁

Epz国c
E技uP梁i
E6芸
则当交s)E2时,存在危(,使得经变换48可消去[4.7)中
的三次项x否败,只能找到(E8,使(4.7》变为

E林林

Ed

E吊0沥河2n永

玟二
一《,并且乃有表达式(4.1),则在原点附近的邻域内存在一系列
E

E沥连42兰n【

其中伟(Ef,经过这一系列变换(每次变挨后把y换回z),可
把(4.1变成如下形式

蔷E沥沥02沥2芸
E沥林0东l
E亚江
E2技t2
E
E林江
玟

EE
E吴育
Page-102
E

玖
E东述l国
0si逊t浩仪
E5沥
E
E吴育沥政
明,在一定条件下,
E
纳
i
型地趋于中心流形
上的某一解(当4
一十co如果ou一
【nice
E砂
定理3.10说明,在
类似条件下,为了
得到局部中心流形上0点附近的执道结构,只需要对它在线性子
E水
0ooxudtui仁e
E述2河
E
E东0月0

E吴见0丞A
|z五)}是(3.5的一个局部中心流形,当且仅当存在B中原

点的开邰域0,侩得VzE,有
E述
Page-103
$4正规形E

E沥国沥csgt江a振
成收敛的席级数时,这种步骤原则上可以无限地进行下去,闭题在
玟
绪论是胤定的,这就是Poincare-Dulac定理,见[A1],或[CLW].

共振与非共振

E沥林
0
巳化成它的Jordan标准形,并引入共振的概念.

E东园2
Eae

4达
吴D
E

用沥
丶二Cm,加一一DPmai-)
5

c

E肉唐河为育
E仪a

考察(4.11)中晤些gt(z)不出现,就是要考察同伦方程ˇ一

E273一3沥3途命一

E

设是对角矩阵,特征值%互不相同,e是相应于丶的特征
D
E仪

藁”粤_z护…z;"'dE
就是A中元素菪一分量中的最简形式.
医
s
Page-104
第一章焦本概伊和淅备知识-

E河[

E
利用(3.14)式算出它的Taylor展式的前儿项-
为了简单,先把(3.5)化成如下的标准形式:

E

4[
E

医林技人5
D
门善以0]'
0C

上面的C与C:的特征根实部分别为负数与正数,图此,B一[(z,
C2沥
刑技水

u林
E

u仁
[
e
E

利用待定系数法,可逐项计算p(z2.

例3.12考虑二维方程

血`
E汀
E

月
经过扰动在奇点(0,0)附近可能发生的分岔现象,其中8么0注

意(3′17)在(0,0)点的线性部分矩阵为〔彗熹〕ypudseatuu
Page-105
E

羞′4′~[唐巳′z_E盂"

E发人沥
古一方面,由于e是4的相应于%的特征向量,因此
5^箕"馨禽E入jz_′】-
把上面的结果代入(4.15》的左端,得到
E吴利0
这说明ai$也是对角的,并且它的特征值具有[Cm,0一A]的形
式.由此可知,当A的特征值非共振时,adt的所有特征值均非零,
E怡述
Ea
ad也有相应的Jordan坡,并且ad$的特征值仍具有[Cmo一]
的形式.
定义4.5“向量值多项式z“e,称为共摄多项式,如果
0圆[不
E述
技
林技园育沥技
换(4,10),使(411有端的诺gi(z)仅由共振多项式组成,【

E
[木

吴国
佳/
E河t

仪
Page-106
述E

国我们首先设法找出方
E巳p中水
E2理2振3招5

令

圆目
[沥
盯一一岸她+州音(″+讪v,

【、《C3.18)
茁E萨(鹰吴小霄(″E
Eosothetn技5
E.
“E沥水0河逊
E
E

E痨“za

E的第一个方程u

导的方程为
豇一_卑″z+目世0.

E沥一伟
Eatptb河沥
E诊L一广

E[

E
E吊E水江2
Page-107
设a在这组塞下的矩阵为乙至,即
u

0
2
0
E
E
0D
现将空间H与RI等同;一e,其中e,..,ex为R中的标准
D
E吊d一
E绍0吴圭国技一
医纳2沥

E

由此得到

E
其中罗心Spantst,6十25j,由此得出二次正规形为

人

或化成等价形式

d需=箕z十肋蔬,

菩罡E招玟兄江

E
E20
Page-108
E葛
园朐园E
M李H
E沥仪d

沥

蔷E河玟

E
d

[

砦E沥公2述人2

2
E圭ni
dzu

F5

砦E育不玟

E乐芸
E医玟应5en命
E达g应不

空间后,正规形中的系数就唯一确定了.在例47中,求正规形的

方法称为矩阵表示法,由于dimF或一n″十霹_l)匹

E
E庞
形的共树算子法和群表示论法等,见王铐的综述文章[Wd]及其
所引的文献.

例49考虑复方程

d53
林罚

0

[

E
Page-109
E第一章基本梓伊和淅吊知识

E林202月
医东技国
E

E园王
E团

门吴0
E

i
0

附注3.14“为了研究(3.19)的中心洪形,我们把x也视作
[

巳0国
沥(3.21》

玟
而中心子空间为BXRe.故(3.21)的局部稳定流形与不稳定流
形印与WP*的结构与x三0时类似.此时中心流形印一{(zr,当
水

砂.由(3.21)的第二个方程易知,(Cz,471x一常数}是(3.21)
的不变集,从而对固定的x,W<|心希数是(3.21)第一个方程的不
E浩00圭
对于不同的sW“|,e-帕数上的辐道结构可能不同,见下例-

例3.15“设z,y,xER,考虑光滢系统

振一

[蟒一厂一口+JCaoyot,

【。
雹言E技林班35江秀

E
E
Page-110
E人
3.14和例2.6可知,在(z,y,4)一(0,0,0)的小邻域内,中心流形
e沥一诊
E一
E

c
E沥
Elaeiopk5丶吴纳才a

在下文对所都分岖的讨论中,我们大都假定已经把闭题化归

到它的中心流形上,即对所论方程的线性部分短阵A一言,o(40
E′

$4正规形

E
Eoposeeosueuetiaioggg月加芸s沥余
尽可能简单的形式,以便于研究.这是源于Poincarg时代的一个课
Page-111
其中

E

四

0
才

〖g
即A有一对共软纸虚特征根,求它的(形式正规形
解“我们用共振原理求解-记一讨,b一一ia,则共振条件
[
E育小
E规
由定理4.6可知,复正规形为

羞=i跑z十唰z尸z十-"+…z…z十…[
E

ag

考意以z一0为不助点的C眺射F丿3),它在z一0附近
可表示为
E玟仪0河沥2胡述仁人
其中<ER“(或CD,4是线性映射(我们把它在某组基下的矩阵
E林沥
E
E
E沥刑[
E吴林述d沥a刑川[颖丿5月诊3
逆变换
3圆
令
E沥李
则可把(4.23)化为
Page-112
EE

EE林2
[命园许
兰
E园一育
(423),则在原点附近的邻域内存在一系列变换
E水0刑一[
其中任(Ef砺,经过这一系列变换(每次变换后把换回z),可
把(4.23)变成如下形式
E沥沥0河2沥6
玟
E
5邹c沥育仪7
E0
E园沥0园
E
E坂国沥5才一
E二一吴

之0,展′″展祟B
l

0″鼻净…净.0
正数|m|称作共振的阶.

E怀c
E林
E志sd河
EEuuioosyt

深220步武s林述
Page-113
32E招

莲
新引起人们对它的重视,并得出若干计算正规形的新方法.
众所周知,经非退化线性变换z一73,线性微分方程

气亩7C=^薹
变换为

蛊翼E林

54沥沥芒春沥坂步招a命口P
线性系统的辅道结构时,我们无妙假设4为Jordan标准形,除了
E
对非线性部分是否可以作类似的简化?在一定意义下,答案是肯定
e
a沥

微分方程在奇点附近的正规形

考感以乙一0为奇点的C微分方程Cr之3它在<一0附近

可表示为

蔫E河玟伟02沥沥2人

E沥
维&次半次向量多项式所成的空间心一2,...,7一1.
先进行变摄
E关志玟50

E2怀ag
换642)代入(41,并注意

[c
阮
O(ly1),而(ly|5表示X%短阵,它的每一元素都是O(110,
由此把方程(41化为
Page-114
日4正规形朱

E东沥
E
E一园一圆肖水2标23述n
刊六次正规形
t标
E月园二一切

E
[吴
E
【2庞河c浩
其中an6为常数

光滑线性化

E不伟河许不伟2
胚)的双曲奇点(或双曲不动点)为万阶非共振的,如果它的特征
根不满足所有二&阶的共振关系,如果一个奇点(或不动点)是任
意有限阶非共振的,则称它为无穷阶非共振的,或简称非共振

从前面的讨论可以看出,一个&阶非共搬奇点(或不动点)的&
次正规形是线性的,换句语说,在奇点(或不动点的邻域里可以
招到一个多项式的坐标变换,使得在新坐标系下系统可以表示为
e不
通过什么样的坂标变换能把这个&阶小量去掉-

E余
坤场(或徽分同胚)在它的奇点(或不动点)O处可以C#线性化,如
果存在点0的邻域口和Ct徽分同胚一Rr,H(O)一,使得
E

定理4.18[[IY])设是一个自然数或二co,4是一个n
E
Page-115
84正规形33

许

的李玟4志人北小6国
其中户与(4.1)中的相同,而共是经过运算得到的新的次齐次
多项式
引入算于ad:技一,
Edn
E

沥

…+弄叶嫩)+O‖川′),0
记毋?为算子at&在H中的值域,而罗“是命*在H8中的一个补
唐

E伟

E林3朐]
n

E5发扬[
这祥,我们把(4.1化为

林林伟

0
E
E
E河或技0
E恩沥浩规i江a育td

E

蔫E李沥汀林述0

E园8
Page-116
0E河

其中g户与(4.7中的相同,而i
洁

Epz国c
E技uP梁i
E6芸
则当交s)E2时,存在危(,使得经变换48可消去[4.7)中
的三次项x否败,只能找到(E8,使(4.7》变为

E林林

Ed

E吊0沥河2n永

玟二
一《,并且乃有表达式(4.1),则在原点附近的邻域内存在一系列
E

E沥连42兰n【

其中伟(Ef,经过这一系列变换(每次变挨后把y换回z),可
把(4.1变成如下形式

蔷E沥沥02沥2芸
E沥林0东l
E亚江
E2技t2
E
E林江
玟

EE
E吴育
Page-117
E

伟,当蜱<瞻`
EE
使得如果原点是C“微分方程

盖=^罩十…E吴0

【

或微分同胚

Ea园育孙22
的心阶非共振双曲奇点(或非共振双曲不动点),则系统(4.
E吴
[此处没有
E
E一

羞=咖十…`E

或微分同朊
E沥招、

因为待征根2一a(或2一4不满足任意阶共振关系,故由定
2
微分同胚)在它们的双曲奇点(或双曲不动点)处可以C“线性化

例4.21“由于上的向量场在双曲焦点的特征根为2土io,
2
它的双曲焦炉处可以C“线性化

在考虑分岔问顶时,我们常常只需要C“线性化.对此,有下面
E河

E育E述0
胚)吊的双曲奇点(或双曲不动点)如果口在点0的线伯部分算
i命沥

E一月园一月

车健江一t
Page-118
$4正规形E

E沥国沥csgt江a振
成收敛的席级数时,这种步骤原则上可以无限地进行下去,闭题在
玟
绪论是胤定的,这就是Poincare-Dulac定理,见[A1],或[CLW].

共振与非共振

E沥林
0
巳化成它的Jordan标准形,并引入共振的概念.

E东园2
Eae

4达
吴D
E

用沥
丶二Cm,加一一DPmai-)
5

c

E肉唐河为育
E仪a

考察(4.11)中晤些gt(z)不出现,就是要考察同伦方程ˇ一

E273一3沥3途命一

E

设是对角矩阵,特征值%互不相同,e是相应于丶的特征
D
E仪

藁”粤_z护…z;"'dE
就是A中元素菪一分量中的最简形式.
医
s
Page-119
E克8

园

给出.
例423“平面上的C“光滢向量场(或徽分同胚)在它的双曲
e

E一a
根2土泗和一个实根A满足w丿0,A一0则称奇点为鞍焦点.鞍
焦点的特征根显然满足(4.303.故R中C“向量场在它的鞍焦点
处可以C线性化

E东
【

【

则它在该点可以C!线性化

在第五章讨论非局部分岔时,迹到的向量场都是依赖于参数
的.因此,下面我们讨论带参数的向量场或晔射的线性化问题

E标邦沥2江d训s
圆
i
以C线性化,如果存在参数空间中e一e的邺域77相空间R“中
点0的邻域口,以及一个C4晃射上;UX耳一R源趸

0达

(2)对每一个参数5E,一(.,:口一R“是一个徽分同胚,
使得通过依赖于参数e的坐标变换z++上(z,s后,系统在0点的
邻域变成一个线性系统

E林政04浩圆d
医)族,且s一日时点0是系统又s的非共振双曲奇点(或双曲不动

5s
[沥玟技江
t
Page-120
E

羞′4′~[唐巳′z_E盂"

E发人沥
古一方面,由于e是4的相应于%的特征向量,因此
5^箕"馨禽E入jz_′】-
把上面的结果代入(4.15》的左端,得到
E吴利0
这说明ai$也是对角的,并且它的特征值具有[Cm,0一A]的形
式.由此可知,当A的特征值非共振时,adt的所有特征值均非零,
E怡述
Ea
ad也有相应的Jordan坡,并且ad$的特征值仍具有[Cmo一]
的形式.
定义4.5“向量值多项式z“e,称为共摄多项式,如果
0圆[不
E述
技
林技园育沥技
换(4,10),使(411有端的诺gi(z)仅由共振多项式组成,【

E
[木

吴国
佳/
E河t

仪
Page-121
EE摄c

E圭i政
本小节给出的定理是我们在第四章和第五章中讨论问题的基
E

E河河水t

E振沙
接触分岔理论的读者可以晓过本节的内容,只需承认定理5.13的
结果,而不影响对随后章节的学习.、

普适开折的定义

E伟d
E李a
E东ed

E玟圭t沥e
E河河(吊一
示.在考慈局部间题时,利用苓的说法可使陈述简明,附录C中定
dospetdld林t2
给出.

现在考虑向量场族叉E<“CM).在局部情形下,无妨设
Ett生二;e
L述达

盖′E八

园t林命y胡i
E

东伟伟明c吊
为从参数空间ERK在原点的小邻域到向量场空间的春射时,我
Ebakeotoookepu沥s
Page-122
设a在这组塞下的矩阵为乙至,即
u

0
2
0
E
E
0D
现将空间H与RI等同;一e,其中e,..,ex为R中的标准
D
E吊d一
E绍0吴圭国技一
医纳2沥

E

由此得到

E
其中罗心Spantst,6十25j,由此得出二次正规形为

人

或化成等价形式

d需=箕z十肋蔬,

菩罡E招玟兄江

E
E20
Page-123
E葛
园朐园E
M李H
E沥仪d

沥

蔷E河玟

E
d

[

砦E沥公2述人2

2
E圭ni
dzu

F5

砦E育不玟

E乐芸
E医玟应5en命
E达g应不

空间后,正规形中的系数就唯一确定了.在例47中,求正规形的

方法称为矩阵表示法,由于dimF或一n″十霹_l)匹

E
E庞
形的共树算子法和群表示论法等,见王铐的综述文章[Wd]及其
所引的文献.

例49考虑复方程

d53
林罚

0

[

E
Page-124
不5萍适扎抚与切盅的余绮45

E
道2一技e

E凶孝坂t河
E
E
Eaectelll

E东全技腾
户)导出的,如果存在连续晔射g,一p(e),在e的映射芽,使得
E明

E穿坂明刑5振l
t
何一个包告的局部族都与(ozzo,jo)的一个导出族等价,

附注5.7“注意两个向量场族的等价性要求它们含有相同维
数的参数,而导出族的引迹使得同一个退化向量场的普适开折可
以含有不同维数的参数,从而可以迹一步考虑含参数最少的普适
开折.
.附注5.8定义5.4一5.6都取自Arnold的书[Al,p267]定
形林
a
A逊续,则称这种等价为强等价,并可得到强等价意义下的普适
E江余
国江

E王坂河entseimtied河
量场(芽)oo,它的普适开折的存在伯并不是明品的,只有在周密的
讨论之后,才能得出结论,参见第三章$1

读者可以用本节的观点重新考察$2的讨论,在那里利用
Malgrange定理证明了,奇异向量场(2.12)任耿的开折(2,13)都
2
焦是(2.17)的一个导出族,按定义5.6,(2.17)是(2.12)的一个
Page-125
其中

E

四

0
才

〖g
即A有一对共软纸虚特征根,求它的(形式正规形
解“我们用共振原理求解-记一讨,b一一ia,则共振条件
[
E育小
E规
由定理4.6可知,复正规形为

羞=i跑z十唰z尸z十-"+…z…z十…[
E

ag

考意以z一0为不助点的C眺射F丿3),它在z一0附近
可表示为
E玟仪0河沥2胡述仁人
其中<ER“(或CD,4是线性映射(我们把它在某组基下的矩阵
E林沥
E
E
E沥刑[
E吴林述d沥a刑川[颖丿5月诊3
逆变换
3圆
令
E沥李
则可把(4.23)化为
Page-126
0E源s
啧透开折、从这里可以看出导出族的作用.

分岔的余维,几何考虑

我们现在对向量场局部族的分岔问题考虑它的余维,在全体
向量场所成的空间.&“中,奇异(即结构不稳定)向量场e表示一
E余
具有非常复杂的结构.但如果限于考虑么中z邻近的点,它具有与
zv“完全相同的奇异性“,则这样的点集可能具有规则的结构.例
E芸
E医
E胡应肖15
2连e
相交于书或关近旁的a(见图1-17).换句话说,K所具有的o这

22
晋然也能在8中的z点与心相交,但在小扰动下,它就可能与乃
E
0ttoeot
Page-127
EE

EE林2
[命园许
兰
E园一育
(423),则在原点附近的邻域内存在一系列变换
E水0刑一[
其中任(Ef砺,经过这一系列变换(每次变换后把换回z),可
把(4.23)变成如下形式
E沥沥0河2沥6
玟
E
5邹c沥育仪7
E0
E园沥0园
E
E坂国沥5才一
E二一吴

之0,展′″展祟B
l

0″鼻净…净.0
正数|m|称作共振的阶.

E怀c
E林
E志sd河
EEuuioosyt

深220步武s林述
Page-128
5普适开扯与分吴的伟维5

Thom机截定理(见附录C中定理C.15,满足横截条件的向量场
玟
一放为一个通有(generie)族,或一舫族.因此,可以粗略地说,那
种在至少x参数通有族中“不可去“的分岔现象是余维*的.当然,
在空间“中的奶点附近,除了4之外,还可能有余维佗于A的奇
玟
玟
逃择参数族K,使得它在r点与各层次的4都横截,则这个又
就是一个普适开折.此阡,如果把定义开折K的晓射记为@yRt一
幻,则一与各分岔曲面的截痕在8下的象,就形成了参数空间

Rt中的分岔图(见图1-183.这种儿何的考虑有时是方便的.

医沥林中林不育沥沥

考虔一个芽,限制在奇点处的中心流形上,其线性部分
E

[

0

w

团
0

这是一个有二重零特征根的奇异向量场.我们关心的阿题是:
它在z一0附近是否存在普适开折分岔的余维是多少?它的分岔
图如何?其开拙的拓扑结构有骈些不同的类型?它们怎祥随参数的
变动从一种类型变成古一种类型?这些问题的解冶不是轻而易举
b标d
玲
E32标成河当(沥
字,借以介绍向量场分岔的一些基本理论与方法本节主要研究

]'0沥沥吊0

E
Page-129
日4正规形朱

E东沥
E
E一园一圆肖水2标23述n
刊六次正规形
t标
E月园二一切

E
[吴
E
【2庞河c浩
其中an6为常数

光滑线性化

E不伟河许不伟2
胚)的双曲奇点(或双曲不动点)为万阶非共振的,如果它的特征
根不满足所有二&阶的共振关系,如果一个奇点(或不动点)是任
意有限阶非共振的,则称它为无穷阶非共振的,或简称非共振

从前面的讨论可以看出,一个&阶非共搬奇点(或不动点)的&
次正规形是线性的,换句语说,在奇点(或不动点的邻域里可以
招到一个多项式的坐标变换,使得在新坐标系下系统可以表示为
e不
通过什么样的坂标变换能把这个&阶小量去掉-

E余
坤场(或徽分同胚)在它的奇点(或不动点)O处可以C#线性化,如
果存在点0的邻域口和Ct徽分同胚一Rr,H(O)一,使得
E

定理4.18[[IY])设是一个自然数或二co,4是一个n
E
Page-130
05E

这个奇异问量场的余维
E仪c
E述
{E沥
为了对它的奇异性加以限制,这里假设c5万0.
E志辽
E不传
医招技余
医2i

败(,o可以表示为

盖E水(5.2)

因此5.1可以篓单表示为(O,z).我们称($,o)具有与(0vz)相
[述

E

E2沥t

E

伟/
E怀2孝
现在可以把与(5.1有相同奇异性的向量场表示成。

沥如江c林育2不0
E仪
E

[兵王技u水一莲i芸c
c

量场.这符合流形上向量场的一舫定义(见附录B中附注B.14,
由于在R中的钗一点,切空间就是R自身,所以常把向量来与其
主部等同(见附录B中附注B173.我们此处的取法,对描述上面
E沥标
Page-131
E

伟,当蜱<瞻`
EE
使得如果原点是C“微分方程

盖=^罩十…E吴0

【

或微分同胚

Ea园育孙22
的心阶非共振双曲奇点(或非共振双曲不动点),则系统(4.
E吴
[此处没有
E
E一

羞=咖十…`E

或微分同朊
E沥招、

因为待征根2一a(或2一4不满足任意阶共振关系,故由定
2
微分同胚)在它们的双曲奇点(或双曲不动点)处可以C“线性化

例4.21“由于上的向量场在双曲焦点的特征根为2土io,
2
它的双曲焦炉处可以C“线性化

在考虑分岔问顶时,我们常常只需要C“线性化.对此,有下面
E河

E育E述0
胚)吊的双曲奇点(或双曲不动点)如果口在点0的线伯部分算
i命沥

E一月园一月

车健江一t
Page-132
E5

E仪技
数空间的乘积空间中考虑),要注意的是,如果5在“中构成余
园志应不述
tena江
E

0
子流形.记

E人沥
E扬八

2着李5
E心

p途[

E怡招才训
3
Ec一3
的Hesse矩阵(它是6维的.

E东圆河不江沥河浩才
E述标5
子流形.

E圆圆八育

E兰动5刑仁沥林招t仪江李
医林22
E
E林兰5胡

国
27,x5*(mi3)是兰中的光滢子流形;再由定理C16,x5「Cm5)在公
Page-133
E克8

园

给出.
例423“平面上的C“光滢向量场(或徽分同胚)在它的双曲
e

E一a
根2土泗和一个实根A满足w丿0,A一0则称奇点为鞍焦点.鞍
焦点的特征根显然满足(4.303.故R中C“向量场在它的鞍焦点
处可以C线性化

E东
【

【

则它在该点可以C!线性化

在第五章讨论非局部分岔时,迹到的向量场都是依赖于参数
的.因此,下面我们讨论带参数的向量场或晔射的线性化问题

E标邦沥2江d训s
圆
i
以C线性化,如果存在参数空间中e一e的邺域77相空间R“中
点0的邻域口,以及一个C4晃射上;UX耳一R源趸

0达

(2)对每一个参数5E,一(.,:口一R“是一个徽分同胚,
使得通过依赖于参数e的坐标变换z++上(z,s后,系统在0点的
邻域变成一个线性系统

E林政04浩圆d
医)族,且s一日时点0是系统又s的非共振双曲奇点(或双曲不动

5s
[沥玟技江
t
Page-134
EE蒙s

E河定诊伟

E技5伟

E国人育沥

Ealutyyipya仪

[aposeye述
7玟

E标slspsiteis:沥命5命|
现在设

0沥[沥

是心中的向量场族(参见附注5.10),且m就显原来的奇异向量
E芽.因此,又把(5.4)称为(O,oo)的一个
开折;与它相应的微分方程是

羞[河人[

E出人沥6林日Pi河23英

到一的映射
【2

E训
E

n

定理5.13

02扬

E沥50技技
Et林壮才才沥3
E
力一259Couo6),
E水怀达述人沥林八

E
Page-135
EE摄c

E圭i政
本小节给出的定理是我们在第四章和第五章中讨论问题的基
E

E河河水t

E振沙
接触分岔理论的读者可以晓过本节的内容,只需承认定理5.13的
结果,而不影响对随后章节的学习.、

普适开折的定义

E伟d
E李a
E东ed

E玟圭t沥e
E河河(吊一
示.在考慈局部间题时,利用苓的说法可使陈述简明,附录C中定
dospetdld林t2
给出.

现在考虑向量场族叉E<“CM).在局部情形下,无妨设
Ett生二;e
L述达

盖′E八

园t林命y胡i
E

东伟伟明c吊
为从参数空间ERK在原点的小邻域到向量场空间的春射时,我
Ebakeotoookepu沥s
Page-136
E

E

E伟0口
【C3扬

E述

河

乏】E
壬E述河水3朋
[九
玟
【沥发0罚沥1

E
{EE
1沥动
其中@吴滔足与Q,@相同的条件.

定理的结论(1)是定义5.12和定理C,15(Thom定理的jet形
d
Edn林
[

0

E
1吴
3一几十tay十丁十zyG.1

Eotuetrttete当育

[沥
00
的一个普造开折,并且是余维2的,.

在作第(一)步讨论时,除了运用徽分方程定性理论的知识和
n
Page-137
Es

E河U
题.

定理5.13中结论(2)的证明

E木
552AC2eidp3技1
{萼=伽0
E庞玟不2河玟
E

邦E
EEEE
l匹页匹人2刑匹刀2CC0.0刑

Eeli吊王招
项系数
E国园e河

E

{E班2

E沥玟心沥应玟s2
E
林
E

E沥

E

E
{一一口【
E技02庞玟人53命命林3

其中云逼'耿仨C阊,并且
~EE
E刑罡
2n5咤059

|
央

利用上面的条件,可由隐函效方程己@醴)腥)=0碗定C阗函
Page-138
不5萍适扎抚与切盅的余绮45

E
道2一技e

E凶孝坂t河
E
E
Eaectelll

E东全技腾
户)导出的,如果存在连续晔射g,一p(e),在e的映射芽,使得
E明

E穿坂明刑5振l
t
何一个包告的局部族都与(ozzo,jo)的一个导出族等价,

附注5.7“注意两个向量场族的等价性要求它们含有相同维
数的参数,而导出族的引迹使得同一个退化向量场的普适开折可
以含有不同维数的参数,从而可以迹一步考虑含参数最少的普适
开折.
.附注5.8定义5.4一5.6都取自Arnold的书[Al,p267]定
形林
a
A逊续,则称这种等价为强等价,并可得到强等价意义下的普适
E江余
国江

E王坂河entseimtied河
量场(芽)oo,它的普适开折的存在伯并不是明品的,只有在周密的
讨论之后,才能得出结论,参见第三章$1

读者可以用本节的观点重新考察$2的讨论,在那里利用
Malgrange定理证明了,奇异向量场(2.12)任耿的开折(2,13)都
2
焦是(2.17)的一个导出族,按定义5.6,(2.17)是(2.12)的一个
Page-139
E

数e一e(e),再经过变捣
E

把方程(5.13)变为5.11)的形式,并漾尸引理的要求。【

E林技一途育223江2技
i沥c

E园厉
E林

E志班玟仪

0
E

E
余G,,里Gue
{抢一p+ea+明+吊yos+黛0吊圭enoe.

2
E

列(5.157转化成

[怀人不
E{P(E〉+钩(e)″+″Z+

E
7命唐

国此,(5.16)可以写成

E

E吴2述2沥2朋
E
Page-140
0E源s
啧透开折、从这里可以看出导出族的作用.

分岔的余维,几何考虑

我们现在对向量场局部族的分岔问题考虑它的余维,在全体
向量场所成的空间.&“中,奇异(即结构不稳定)向量场e表示一
E余
具有非常复杂的结构.但如果限于考虑么中z邻近的点,它具有与
zv“完全相同的奇异性“,则这样的点集可能具有规则的结构.例
E芸
E医
E胡应肖15
2连e
相交于书或关近旁的a(见图1-17).换句话说,K所具有的o这

22
晋然也能在8中的z点与心相交,但在小扰动下,它就可能与乃
E
0ttoeot
Page-141
E第一章基札概伊和雍备知订

E河p

d
E
下面再迹行一次变换,把(5.17)第二式中的(sJz变成(so
E

则(5.17化奶
Ezz歹[25

E吴林李心
1
E

E吵(s),s),
ET游傍),

P告帷川E

C音矽(鬓)挎〉一告哟(辜)芗〈藁‖e〉,

E鲁砂〈E),E吴
E0703
应许

把,9,,@和雨换同6,p,,Q和申的形式后,方程(5.18)成为
[n

附注5.16“显然,局部族(5.6)(在(y,6)一(0,0)附近的一
i
定义5.4中的映射9与A都可取为相应空间中的恒同映射,在这个
Page-142
5普适开扯与分吴的伟维5

Thom机截定理(见附录C中定理C.15,满足横截条件的向量场
玟
一放为一个通有(generie)族,或一舫族.因此,可以粗略地说,那
种在至少x参数通有族中“不可去“的分岔现象是余维*的.当然,
在空间“中的奶点附近,除了4之外,还可能有余维佗于A的奇
玟
玟
逃择参数族K,使得它在r点与各层次的4都横截,则这个又
就是一个普适开折.此阡,如果把定义开折K的晓射记为@yRt一
幻,则一与各分岔曲面的截痕在8下的象,就形成了参数空间

Rt中的分岔图(见图1-183.这种儿何的考虑有时是方便的.

医沥林中林不育沥沥

考虔一个芽,限制在奇点处的中心流形上,其线性部分
E

[

0

w

团
0

这是一个有二重零特征根的奇异向量场.我们关心的阿题是:
它在z一0附近是否存在普适开折分岔的余维是多少?它的分岔
图如何?其开拙的拓扑结构有骈些不同的类型?它们怎祥随参数的
变动从一种类型变成古一种类型?这些问题的解冶不是轻而易举
b标d
玲
E32标成河当(沥
字,借以介绍向量场分岔的一些基本理论与方法本节主要研究

]'0沥沥吊0

E
Page-143
E

t
E2述
E河王′
E
2CSv6J)|s-e-
工
Ep35nn
E
E
[[
E
E
E
[吴3
E
陶tsse
E沥te
E河林t
2述
E
劝一方面,在厂中(0,oo点附近z:S可申如下方程确定(见
条件CH,)和CH。)》
Ec

E吴吴国莲工
E一0匕52

国t
价于

aE

E

E
Page-144
E源

0【
邹匹
3E
河
医
53E

E

|

引理5.18“设(5.4)是非退化的开折,则存在C“变换一
ACe),p(0)一0,它在5一0附近非退化,并把(5.8)变到(5.9.

0
式给出。

肉二卯(日,肌二日,皂一口,一Eni

则由条件(5.7知AC0)一0又由于(5.8)是非退化的,则由引理
5.17可知,一xC6)在e一0是非退化的,引理得证.方程(5.9)中
EE
Ei技|,

e

习题与怡考题一

1.1考虔R上的动力系统,设它皑辅线分别具有图1-6,图1-8或图1-9
的7种分布,对铁一种分布,取平面上不吾区域的点z,议论极眼集atz)和
z).莲研究系统的汀游茹集D(9).对郧一种分布,你可以断言系统不显结
构穗定的3

1.2利用对(2.13)的讨论方法,证明R上的系绕(2.2)是C“系统

羞=_zz+0(恤忏)

0
匹:
英

2

国

E

13对下列系统,求出与中心诚形相应的诱导方程(3.8》,并由此作出源
玟

八
Page-145
05E

这个奇异问量场的余维
E仪c
E述
{E沥
为了对它的奇异性加以限制,这里假设c5万0.
E志辽
E不传
医招技余
医2i

败(,o可以表示为

盖E水(5.2)

因此5.1可以篓单表示为(O,z).我们称($,o)具有与(0vz)相
[述

E

E2沥t

E

伟/
E怀2孝
现在可以把与(5.1有相同奇异性的向量场表示成。

沥如江c林育2不0
E仪
E

[兵王技u水一莲i芸c
c

量场.这符合流形上向量场的一舫定义(见附录B中附注B.14,
由于在R中的钗一点,切空间就是R自身,所以常把向量来与其
主部等同(见附录B中附注B173.我们此处的取法,对描述上面
E沥标
Page-146
E

E伟
0
c

E述河正规形

1.5设R上的和量杨以(0,0)为帛炭,其线伯鄂分在(0,0)的短阵具有
二重零特征根'而且向量场在翼转角震誓下保持不变,其中疃为正整敷,并且
Et林553加

盖E为河水河浩

其中vepA为复数,m一
E达
16
0医北
[颖
玟
'1.7考虑C“眶射P一R它以一1十5为特征根,|e|人1.证明对
给定的正个数A存在8>0,使当ls|一日时,七可以经过C“变换化为
E沥辽人江达
E河
利用此结果对例2.11的结论给出简单的证明
n
E林政c林达
在小扰动下的分盆蜀律.
Page-147
E5

E仪技
数空间的乘积空间中考虑),要注意的是,如果5在“中构成余
园志应不述
tena江
E

0
子流形.记

E人沥
E扬八

2着李5
E心

p途[

E怡招才训
3
Ec一3
的Hesse矩阵(它是6维的.

E东圆河不江沥河浩才
E述标5
子流形.

E圆圆八育

E兰动5刑仁沥林招t仪江李
医林22
E
E林兰5胡

国
27,x5*(mi3)是兰中的光滢子流形;再由定理C16,x5「Cm5)在公
Page-148
\part{第二章常见的局部与非局部分岔}


E东
岔,Hopf分岑.同宿分岔.Poincare分岔等,其中前三种为局部分岔
闭题,后两种分别为半局部分岔和全局分岔问题,除了奇点分岔
外,本章的大部分讨论都限制在相空间为一维的情形.

E

E
e
分岔现象称为奇点分岖

一舱理论

E怀园s木5
点的线性郯分算子是非奇异的,即它的所有特征根均非零.否则
u

E东育参数的向量场,如果它的奇
点是非退化的,则奇点本身也光滔地依赖于参数

证明ˇ设向量场由微分方程

羞=矾婀…许
巴沥2一诚扬史刑
二乙为(1.1)的非退化奇点,卵o(zoopa一0,一英一

奇异.由隐函数定理,在(zu,附近存在光潘函数z一YC0,使
Page-149
EE蒙s

E河定诊伟

E技5伟

E国人育沥

Ealutyyipya仪

[aposeye述
7玟

E标slspsiteis:沥命5命|
现在设

0沥[沥

是心中的向量场族(参见附注5.10),且m就显原来的奇异向量
E芽.因此,又把(5.4)称为(O,oo)的一个
开折;与它相应的微分方程是

羞[河人[

E出人沥6林日Pi河23英

到一的映射
【2

E训
E

n

定理5.13

02扬

E沥50技技
Et林壮才才沥3
E
力一259Couo6),
E水怀达述人沥林八

E
Page-150
Eu

【dCtocltdtog沥浩|

E玟园怀园东不
数的微小变化下不变,它的位置也光滑地依赖于参数的变化.需
要注意,奇点的非退化性与双曲性是不同的概念.例如,一个们
E
是非退化的,但它是非双曲的.此时在扰动下,蚀然奇点个数(在
小邹域内)不发生变化,但其附近的扬道结构可能变化,出现Hopf
E

E江刑
退化奇点(它们必是孤立奇点)或无奇点的向量场集合形成一个
E

[国圆八沥标亚一者吊622

E

E河述
讨论;考虑投影
E仪

其中了-/C6).具有契炉的向量场集合在空间JoCM,M)中有表
E

E
d
/〔z)在辜点非退化,即噩园l

玟
C,15,得知仅有非退化夹点或无奇点的向量场在<g““CM)中形成
开稽子集,

这个定理说明,向量场的一个退化奇点可以经过任意小的扶
E吊皋尿应
Eouolokai江
Page-151
E

E

E伟0口
【C3扬

E述

河

乏】E
壬E述河水3朋
[九
玟
【沥发0罚沥1

E
{EE
1沥动
其中@吴滔足与Q,@相同的条件.

定理的结论(1)是定义5.12和定理C,15(Thom定理的jet形
d
Edn林
[

0

E
1吴
3一几十tay十丁十zyG.1

Eotuetrttete当育

[沥
00
的一个普造开折,并且是余维2的,.

在作第(一)步讨论时,除了运用徽分方程定性理论的知识和
n
Page-152
EE

stouyzyltisyogp志E
tusktsp志途p
奇点.对一个具体的奇点分盆问题,通常有两种处理方法:一种是
利用中心流形定理,把问题归结到中心流形上,见第一章例3.121ˇ
8n河目J师0沥
method),为了说明这个方法的基本性想,我们先看一种特殊情
E沥2e一

沥

其中《的特征根均为零,而旦的特征根均不为零!人gECr,之
E
E伟

E玟玲
i
a

Ecaent余

E

d
E标逊6E
0

一{Cy史EDICGOoo主二口,马二5门仁二加),
、

则对不同的A,|2|心1,8,结构的变化反晖了奇点个数的变化规
E河2
闰维数得到降低通常称(1.4)为方程(1.2)的分岑函数.

为了应用上的便利,下面在更一舱的框架下讨论这个间题.

Liapunov-Schmidt方法

设又,Z和A为实Banach空间,D和砌分别为又和A中零点
[一e人
Page-153
Es

E河U
题.

定理5.13中结论(2)的证明

E木
552AC2eidp3技1
{萼=伽0
E庞玟不2河玟
E

邦E
EEEE
l匹页匹人2刑匹刀2CC0.0刑

Eeli吊王招
项系数
E国园e河

E

{E班2

E沥玟心沥应玟s2
E
林
E

E沥

E

E
{一一口【
E技02庞玟人53命命林3

其中云逼'耿仨C阊,并且
~EE
E刑罡
2n5咤059

|
央

利用上面的条件,可由隐函效方程己@醴)腥)=0碗定C阗函
Page-154
E

数e一e(e),再经过变捣
E

把方程(5.13)变为5.11)的形式,并漾尸引理的要求。【

E林技一途育223江2技
i沥c

E园厉
E林

E志班玟仪

0
E

E
余G,,里Gue
{抢一p+ea+明+吊yos+黛0吊圭enoe.

2
E

列(5.157转化成

[怀人不
E{P(E〉+钩(e)″+″Z+

E
7命唐

国此,(5.16)可以写成

E

E吴2述2沥2朋
E
Page-155
园国u

E
胡(zi0一00
tt朋
0i
E
(H)-4(4)在万中存在补空间;4(4)是Z中的闭集,并且
在中存在补空间.(当A为Fxedholm算子时,这个假设总是成立
的.在下文的应用中,经常是这种情形.),.
[林匹5技
E林577医3
【志东n不沥河一明沥林河
E林0沥不沥李E
投影九和一乙的值域,显然,方程(L5)等价于,
E河辽0

[沥.7b)
E吴技

E诊
E吴5[阮口
的同构.由隐函数定理,存在X在原点的邺域U。,友_在原点的
邹域,A在原点的邻域印,以及C+酯射史。DXW一V,,侍
吴e

b述,
E耿dLd扬
c
EeCy江江)[
[技7
E(5李厂沥沥刑7伟c顺5唐人
Page-156
E第一章基札概伊和雍备知订

E河p

d
E
下面再迹行一次变换,把(5.17)第二式中的(sJz变成(so
E

则(5.17化奶
Ezz歹[25

E吴林李心
1
E

E吵(s),s),
ET游傍),

P告帷川E

C音矽(鬓)挎〉一告哟(辜)芗〈藁‖e〉,

E鲁砂〈E),E吴
E0703
应许

把,9,,@和雨换同6,p,,Q和申的形式后,方程(5.18)成为
[n

附注5.16“显然,局部族(5.6)(在(y,6)一(0,0)附近的一
i
定义5.4中的映射9与A都可取为相应空间中的恒同映射,在这个
Page-157
EE

U
总结上面的讨论,我们有下面的结果.
E玟玟育s沥0浩
[t
沥y
2
其中加与G分别由C1.8)和(9定义,【
i
E一5沥c沥人5林亚亚小途国
d团hn命
域及值域都作了显著的约化这就是Liapunov-Schmidt方法的核
E
E
s

巳
Fc0

E吊沥八汀
考廊奇点分岗阿题,就是要在R*XR的原点附近考察方程

E5林)沥口CL0
0
【
E吴i吴
人河

医纳t

野
[2沥0
b

[
Page-158
E

t
E2述
E河王′
E
2CSv6J)|s-e-
工
Ep35nn
E
E
[[
E
E
E
[吴3
E
陶tsse
E沥te
E河林t
2述
E
劝一方面,在厂中(0,oo点附近z:S可申如下方程确定(见
条件CH,)和CH。)》
Ec

E吴吴国莲工
E一0匕52

国t
价于

aE

E

E
Page-159
EB

E人5

o“(a,0旦Ca,加满足分岔方程
&(a,加二0,
E不沥一林招浩。
&(aoas一匹一@3F(am十讨(a,加,加.9玟)
玟

我们考虑R中一类更广泛的微分方程

吊
门。2
FFE逊小胡沥287

E2伟

分岔问题,此时线性部分短阵为4一彗真),E东技

n

-园w园-似

纳an2沥
觉(4)一纠(Q)一Spantaoj.
E绍t[]E2

0
E
E伟江
0述2河扬吊
E沥引
E

匹E
相(】_Q)'(0〉=(′_Q)〔

′鲫z)E翼27′JD.
p
Page-160
E源

0【
邹匹
3E
河
医
53E

E

|

引理5.18“设(5.4)是非退化的开折,则存在C“变换一
ACe),p(0)一0,它在5一0附近非退化,并把(5.8)变到(5.9.

0
式给出。

肉二卯(日,肌二日,皂一口,一Eni

则由条件(5.7知AC0)一0又由于(5.8)是非退化的,则由引理
5.17可知,一xC6)在e一0是非退化的,引理得证.方程(5.9)中
EE
Ei技|,

e

习题与怡考题一

1.1考虔R上的动力系统,设它皑辅线分别具有图1-6,图1-8或图1-9
的7种分布,对铁一种分布,取平面上不吾区域的点z,议论极眼集atz)和
z).莲研究系统的汀游茹集D(9).对郧一种分布,你可以断言系统不显结
构穗定的3

1.2利用对(2.13)的讨论方法,证明R上的系绕(2.2)是C“系统

羞=_zz+0(恤忏)

0
匹:
英

2

国

E

13对下列系统,求出与中心诚形相应的诱导方程(3.8》,并由此作出源
玟

八
Page-161
E

E伟
0
c

E述河正规形

1.5设R上的和量杨以(0,0)为帛炭,其线伯鄂分在(0,0)的短阵具有
二重零特征根'而且向量场在翼转角震誓下保持不变,其中疃为正整敷,并且
Et林553加

盖E为河水河浩

其中vepA为复数,m一
E达
16
0医北
[颖
玟
'1.7考虑C“眶射P一R它以一1十5为特征根,|e|人1.证明对
给定的正个数A存在8>0,使当ls|一日时,七可以经过C“变换化为
E沥辽人江达
E河
利用此结果对例2.11的结论给出简单的证明
n
E林政c林达
在小扰动下的分盆蜀律.
Page-162
64第二章常见的局部与非尸郡分岗

加果我们考虑方程(1.13)的C+扰动,拥动参数为,则扰动后
玟
玟
E沥伟
E史I不胡弘
E林575列
E
加一0当ACDDI(0,0)一0时有两个零点,当p(D(0,0)丿0时
E
玟
时,最后的讨论可从第一章定理2.12直接得到.

E河玟育余t
E
化的同时(甚至在奇点个数不变时,见附注1.3),执道结构还可能
玟
i

$2“闭轨分岔

考忠徽分方程族
E黯=训z汹%E

E林命砺扬
【st
有儿条闭转?这就是闭转分岔问题,当?为双曲闭转时,问题是平
凡的(见第一章$13.因此,我们要找到一些方法,来判别7的双
E伟n东
My沥
E
Page-163
第二章常见的局部与非局部分岔

E东
岔,Hopf分岑.同宿分岔.Poincare分岔等,其中前三种为局部分岔
闭题,后两种分别为半局部分岔和全局分岔问题,除了奇点分岔
外,本章的大部分讨论都限制在相空间为一维的情形.

E

E
e
分岔现象称为奇点分岖

一舱理论

E怀园s木5
点的线性郯分算子是非奇异的,即它的所有特征根均非零.否则
u

E东育参数的向量场,如果它的奇
点是非退化的,则奇点本身也光滔地依赖于参数

证明ˇ设向量场由微分方程

羞=矾婀…许
巴沥2一诚扬史刑
二乙为(1.1)的非退化奇点,卵o(zoopa一0,一英一

奇异.由隐函数定理,在(zu,附近存在光潘函数z一YC0,使
Page-164
E65

s
E东颖沥
伟
Ei一
E李22林林E水月河耿认
Ea沥
片(0,0)二0的条件.

在解决具体问题时,图雅在于如何实施上述原则.下面,我们
河tnshu
苗些重要结论

E

[[

E述林0运仪不0
E

2z=棘C)=[

E
瞎0)]′
E扬2
[纳
1b

50=E2
E述医志刑E林学万王东绍

E沥(一E
其中(.,“)表示R中的内积,联坐标变捣

E沥人仪庞述20

其中z在7附近0如s人T,|a|人1.坐标(om3可以这样理解:从
玟
t
E不

1余
Page-165
63E

E

E
25
'
孙
E
0
图2-1

我们先把方程(2.2)转挨成曲线坐标系下的方程,然后建立
E沥i6朱沥

E25AC加述22黯E伟人应逊2盖E途盖′

E
Eoculuoctt水林芸
盖_E

E标命(XC朋L东107扬0

EgrroyarD7
E
E罡c班2口二00颂子1
d醴CX7EXXZORNOZND

E20沥沥
E达
E江

E江2口
Page-166
Eu

【dCtocltdtog沥浩|

E玟园怀园东不
数的微小变化下不变,它的位置也光滑地依赖于参数的变化.需
要注意,奇点的非退化性与双曲性是不同的概念.例如,一个们
E
是非退化的,但它是非双曲的.此时在扰动下,蚀然奇点个数(在
小邹域内)不发生变化,但其附近的扬道结构可能变化,出现Hopf
E

E江刑
退化奇点(它们必是孤立奇点)或无奇点的向量场集合形成一个
E

[国圆八沥标亚一者吊622

E

E河述
讨论;考虑投影
E仪

其中了-/C6).具有契炉的向量场集合在空间JoCM,M)中有表
E

E
d
/〔z)在辜点非退化,即噩园l

玟
C,15,得知仅有非退化夹点或无奇点的向量场在<g““CM)中形成
开稽子集,

这个定理说明,向量场的一个退化奇点可以经过任意小的扶
E吊皋尿应
Eouolokai江
Page-167
EE

stouyzyltisyogp志E
tusktsp志途p
奇点.对一个具体的奇点分盆问题,通常有两种处理方法:一种是
利用中心流形定理,把问题归结到中心流形上,见第一章例3.121ˇ
8n河目J师0沥
method),为了说明这个方法的基本性想,我们先看一种特殊情
E沥2e一

沥

其中《的特征根均为零,而旦的特征根均不为零!人gECr,之
E
E伟

E玟玲
i
a

Ecaent余

E

d
E标逊6E
0

一{Cy史EDICGOoo主二口,马二5门仁二加),
、

则对不同的A,|2|心1,8,结构的变化反晖了奇点个数的变化规
E河2
闰维数得到降低通常称(1.4)为方程(1.2)的分岑函数.

为了应用上的便利,下面在更一舱的框架下讨论这个间题.

Liapunov-Schmidt方法

设又,Z和A为实Banach空间,D和砌分别为又和A中零点
[一e人
Page-168
E

不COvs,0二0,
河(

E
0s

从而(2.7可写成

盖应0

E一一
E

RCasd,加一膘〔epr工〔/氮(震)00
5

0
E0泊
E标
E沥0一不
E吴
E1253
E林仪
i沥许东许不0林一达
E许e林tE一
定义2.1若存在e井0,使VaE(0,e)都有G(a,b一0
0一一不
E林仁技c人河红2一王
E技
Es
从上述定义可知,稳定(不稳定极限环必为颈立闭执.下面
E东
E出孙孙招肖2伟扬医[吴(吊355朋
E河绍c万一t河
Ei述万一
Page-169
EE渡
b

类似可定义内侧复型极限环与内侧周期环域

E沥

E人万儿3一仁浩中仁刑
celetsasogaresssa吴s相河
伟东逊n圆a沥
端函数F(rvs,2解析,从而c她从)解析E
渡八才2振

东朋李口
有

[林门[Ey
Eutt林ag一0
时称为多重极限环.

显然,当&为奇数时,ct一0表明7为稳定的极限环,而c一0
E沥纳e不江技
注意,这里说的穗定性为转道穗定性,而不是结构穗定性,事实
玟
E河泊
重的,我们记

EI工tr寄z(棘(,0)d5Et3)

E梁育d芸
E人
Eod国图M94253

Eexp]′二〔矛/(盒)E1).
Ed

吴DOO
E

4

Eexp丘′矛z(J)dj0
E
Page-170
园国u

E
胡(zi0一00
tt朋
0i
E
(H)-4(4)在万中存在补空间;4(4)是Z中的闭集,并且
在中存在补空间.(当A为Fxedholm算子时,这个假设总是成立
的.在下文的应用中,经常是这种情形.),.
[林匹5技
E林577医3
【志东n不沥河一明沥林河
E林0沥不沥李E
投影九和一乙的值域,显然,方程(L5)等价于,
E河辽0

[沥.7b)
E吴技

E诊
E吴5[阮口
的同构.由隐函数定理,存在X在原点的邺域U。,友_在原点的
邹域,A在原点的邻域印,以及C+酯射史。DXW一V,,侍
吴e

b述,
E耿dLd扬
c
EeCy江江)[
[技7
E(5李厂沥沥刑7伟c顺5唐人
Page-171
E69

劝一方面,把(2.9)式的内积按分量展开,并利用(2,3),(2.8)可
5

E〈芗(占),(P(s),0)芗(s)〉
=t【盂`(孵(裹)】o)一_铲′(s)_z迈

巳河E/
江7

0
E
1
注意到7以T为周期E
′「r蟹艾i伟E
。【L辽F5仪

023

0
[

医a技5
它的稳定性与0的符号之间的关系是显然的,定理得证,〖

E沥沥沥仪红
近为周期环域时,必有v一0.

E北余育2

【2怡chizle沥it刑
E

0坂坂林广二招e
E
为余数时,上述结论可扩充至一0.

EC2se河纺el育动0前团国吴二4
E
Page-172
EE

U
总结上面的讨论,我们有下面的结果.
E玟玟育s沥0浩
[t
沥y
2
其中加与G分别由C1.8)和(9定义,【
i
E一5沥c沥人5林亚亚小途国
d团hn命
域及值域都作了显著的约化这就是Liapunov-Schmidt方法的核
E
E
s

巳
Fc0

E吊沥八汀
考廊奇点分岗阿题,就是要在R*XR的原点附近考察方程

E5林)沥口CL0
0
【
E吴i吴
人河

医纳t

野
[2沥0
b

[
Page-173
EB

E人5

o“(a,0旦Ca,加满足分岔方程
&(a,加二0,
E不沥一林招浩。
&(aoas一匹一@3F(am十讨(a,加,加.9玟)
玟

我们考虑R中一类更广泛的微分方程

吊
门。2
FFE逊小胡沥287

E2伟

分岔问题,此时线性部分短阵为4一彗真),E东技

n

-园w园-似

纳an2沥
觉(4)一纠(Q)一Spantaoj.
E绍t[]E2

0
E
E伟江
0述2河扬吊
E沥引
E

匹E
相(】_Q)'(0〉=(′_Q)〔

′鲫z)E翼27′JD.
p
Page-174
E招

Dn国4
E
E

E

E
E述a木
蘑_Ir订吊`d囊_「(入E吴2林
沥“

这里利用系统的第一个方程得到2zydt一2zdz心d(丿.因此,这
0|

[
0技门
E
13)式中第一个不为零的ct,以便接定义2.4判断7的重次.这时
t

3Hopf分岛

当向量场在奇点的线性部分短阵有一对复特征根,井东随参
数变化而穿越虚轴时,在奇点附近的一个二维中心流形上,奇点的
稳定性发生翻转,从而在奇点附近产生闭转的现象,称为Hopf分
岔,第一章例2.9就是一个典型的实例.既然Hopt分岗发生在二
医

经典的Hopf分岔定理
g
0蕃E河吴公
Page-175
64第二章常见的局部与非尸郡分岗

加果我们考虑方程(1.13)的C+扰动,拥动参数为,则扰动后
玟
玟
E沥伟
E史I不胡弘
E林575列
E
加一0当ACDDI(0,0)一0时有两个零点,当p(D(0,0)丿0时
E
玟
时,最后的讨论可从第一章定理2.12直接得到.

E河玟育余t
E
化的同时(甚至在奇点个数不变时,见附注1.3),执道结构还可能
玟
i

$2“闭轨分岔

考忠徽分方程族
E黯=训z汹%E

E林命砺扬
【st
有儿条闭转?这就是闭转分岔问题,当?为双曲闭转时,问题是平
凡的(见第一章$13.因此,我们要找到一些方法,来判别7的双
E伟n东
My沥
E
Page-176
E理0

Ec江林d小0
线伯部分矩阵4(有特征值a(4士识(,满足条件

【2

0怀纳0

0明

其中a为向量场X。的如下复正规形中的系数(见第一章
一97。“

蓥″E伟

【6H
E林l吊述
Eu前
Ess命d
E标d李
[沥
道是它的唯一闭软.当Rec一0时,它是稳家的;当Reci>0时,
a
1
时,Cmo一0.
E沥医nl沥n
E医沥扬一s
E吴一
Ed标刑
E水
立[z〕=[0^姚]目国
一EE2叉yE“
E人小2发n吴c
E

E林李仪沥)
Page-177
E65

s
E东颖沥
伟
Ei一
E李22林林E水月河耿认
Ea沥
片(0,0)二0的条件.

在解决具体问题时,图雅在于如何实施上述原则.下面,我们
河tnshu
苗些重要结论

E

[[

E述林0运仪不0
E

2z=棘C)=[

E
瞎0)]′
E扬2
[纳
1b

50=E2
E述医志刑E林学万王东绍

E沥(一E
其中(.,“)表示R中的内积,联坐标变捣

E沥人仪庞述20

其中z在7附近0如s人T,|a|人1.坐标(om3可以这样理解:从
玟
t
E不

1余
Page-178
E

1林)

E沥江林沥闵吴

退化Hopf分岖定理
E人

E
E

人

Eodii48沥

引理3.4设在奇点(z,)一(0,0)系绕(3.4的线性鄂分矩
阵有一对复特征根a(4土讨(49,满足条件(H,则对任意自然数
E述
e

0

十C′(#)″/+】歹/E0(|槽/|″十…),

E吴0佳河不芸水
e技2

E沥t0
[沥王一
E技

[p

由条件CH)知,力(0)二(m,A(03)给出一2十2阶的共摄条件,

巳一许
E途202志y万一4途育小应25
0

E
Page-179
63E

E

E
25
'
孙
E
0
图2-1

我们先把方程(2.2)转挨成曲线坐标系下的方程,然后建立
E沥i6朱沥

E25AC加述22黯E伟人应逊2盖E途盖′

E
Eoculuoctt水林芸
盖_E

E标命(XC朋L东107扬0

EgrroyarD7
E
E罡c班2口二00颂子1
d醴CX7EXXZORNOZND

E20沥沥
E达
E江

E江2口
Page-180
E瑶汀150

E沥沥ta|

E出东仪y水
E

E医述

定理3.6“设闯量场X以(0,03点为&阶细焦点;则X在拼
E水

0
的邻域口,使得当|x|一a时,在乙内至多有8个极限环!

E沥林
(z,3一(0,0)的任意邹域口“人口,存在一个开折系绕;,使得
恒在口“内恰有了个极限环,其中|x|一.

E沥(沥俊圆沥t
E达技一技p
意口一z5,e2pi二i,可得到

{E朐
E命

0

El沥

盖g吴达

[
办许
E

t-仪,
2e昙?亘畲))+矾E

e(缥(
0Z室E沥口

沥政沥
Page-181
E

不COvs,0二0,
河(

E
0s

从而(2.7可写成

盖应0

E一一
E

RCasd,加一膘〔epr工〔/氮(震)00
5

0
E0泊
E标
E沥0一不
E吴
E1253
E林仪
i沥许东许不0林一达
E许e林tE一
定义2.1若存在e井0,使VaE(0,e)都有G(a,b一0
0一一不
E林仁技c人河红2一王
E技
Es
从上述定义可知,稳定(不稳定极限环必为颈立闭执.下面
E东
E出孙孙招肖2伟扬医[吴(吊355朋
E河绍c万一t河
Ei述万一
Page-182
国第二章常见的局部与非局部分芸

这里乃j二办j(Py史是pE[0,2z]和p(在0附近)的光滑函数
E
霉…二噜[C3.9)
在乙轴上建立方程(3.8)的Poincare晔射PCz:,4,并令
ELC命52河洁[
显然,
【A水《C3.11》
【怀iss26志752浩胡诊
E
医江玖
E
E人不
E
E
E

E英E余国
F`二〕___】((),0)Ea′(0,2‖)E

0

[达
E朱吴吴gr8
E

E医述2八2技1
「'行日d诊
E

3扎国

0〔(2发+1)丨〕蒂…'E
。

y

0

0E明d理

萨鲜J(′-'2‖)i35一Re乙`鏖
e

《3.13)
Page-183
EE渡
b

类似可定义内侧复型极限环与内侧周期环域

E沥

E人万儿3一仁浩中仁刑
celetsasogaresssa吴s相河
伟东逊n圆a沥
端函数F(rvs,2解析,从而c她从)解析E
渡八才2振

东朋李口
有

[林门[Ey
Eutt林ag一0
时称为多重极限环.

显然,当&为奇数时,ct一0表明7为稳定的极限环,而c一0
E沥纳e不江技
注意,这里说的穗定性为转道穗定性,而不是结构穗定性,事实
玟
E河泊
重的,我们记

EI工tr寄z(棘(,0)d5Et3)

E梁育d芸
E人
Eod国图M94253

Eexp]′二〔矛/(盒)E1).
Ed

吴DOO
E

4

Eexp丘′矛z(J)dj0
E
Page-184
E瑞水5

i技
壬0,当1<m<触+L
〈

2″〔(2桑十1)【〕婶"[李玲

0

注意方程(3E东
E
i
林
s吴玟不e朋
园

PCru,40二@(zruDh(zru,19.

古一方面,Y(0,)二0,且(ru,40对心的正根与负根成对出现
0i
uoapyeeoyoytu不a河A招
对至多有&个正根,定理的结论(1)得证

为了证明结论(2),我们假设Xu以z一0为阶细焦点,即它
具有如下的正规琪

E八林许有述门Rect奂0.
取它的扰动系统
E不
[沥5
E土贝训招中
E
红林述
9
t

E东述林
命
Page-185
E69

劝一方面,把(2.9)式的内积按分量展开,并利用(2,3),(2.8)可
5

E〈芗(占),(P(s),0)芗(s)〉
=t【盂`(孵(裹)】o)一_铲′(s)_z迈

巳河E/
江7

0
E
1
注意到7以T为周期E
′「r蟹艾i伟E
。【L辽F5仪

023

0
[

医a技5
它的稳定性与0的符号之间的关系是显然的,定理得证,〖

E沥沥沥仪红
近为周期环域时,必有v一0.

E北余育2

【2怡chizle沥it刑
E

0坂坂林广二招e
E
为余数时,上述结论可扩充至一0.

EC2se河纺el育动0前团国吴二4
E
Page-186
0第二犊常见的局部与非尾部分岑

53吊2里
E兰吊东
Ed月月汀7租(顶荣

【7河砺7月园0园7
Estyuupuy4,#′′一j,r′一j,1吏得EE
E胺
[余A

[李前=[

[沥7口
由Poincare-Bendixson环域定理,扰动系统(3.15)至少存在个极
限环.

E才[
E
个极限环.若不然,则对任意的a>0和7一0的邺域7,(3.15)在
口内有多于}个极限环,则我们可以仿照上面的方法选取_一,
rk-j-Ly“皋叶以及eroy屋而在口内再获得古外的&一3个极限
E
0|

定理3.1的证明“当一1时,可从定理3.6得到定理3.1.
E3s林浩日(仪2招
一阶细焦点,因此定理3,1的前一部分结论成立,再设条件(Ho)
E命河应图生刑

Ed5沥多庞z_T〉【e2525[途
其中

0【〔expE羞座〕51〕十d述

冉
[
再由(3.14可知
Page-187
E招

Dn国4
E
E

E

E
E述a木
蘑_Ir订吊`d囊_「(入E吴2林
沥“

这里利用系统的第一个方程得到2zydt一2zdz心d(丿.因此,这
0|

[
0技门
E
13)式中第一个不为零的ct,以便接定义2.4判断7的重次.这时
t

3Hopf分岛

当向量场在奇点的线性部分短阵有一对复特征根,井东随参
数变化而穿越虚轴时,在奇点附近的一个二维中心流形上,奇点的
稳定性发生翻转,从而在奇点附近产生闭转的现象,称为Hopf分
岔,第一章例2.9就是一个典型的实例.既然Hopt分岗发生在二
医

经典的Hopf分岔定理
g
0蕃E河吴公
Page-188
医脱u

E

E命71E述c

由(3.17)和条件(Ha)可知

E

蠢(O'O)园湃(O)挝(0)Ealt2E
E存在疃〉0和在0<zl<疃定义的光滑函数″_
[

E[E

至此,定理31的结论(1)得证.为了证明结论(2,从(3.20)求导、
E

E

E
F+萜烬(z】)二

E
E

[
E及条件(H】)可得

,把上面的结果及(3.19)代人(8.21)得到

国国E
Er

冉此得定理的结论(22,【
E招医
E人
一0的邻域口,使得
E
个极限环,当Ree一0(>0时,它是穗定(不稳定)的并且当A
s口
0
Page-189
E理0

Ec江林d小0
线伯部分矩阵4(有特征值a(4士识(,满足条件

【2

0怀纳0

0明

其中a为向量场X。的如下复正规形中的系数(见第一章
一97。“

蓥″E伟

【6H
E林l吊述
Eu前
Ess命d
E标d李
[沥
道是它的唯一闭软.当Rec一0时,它是稳家的;当Reci>0时,
a
1
时,Cmo一0.
E沥医nl沥n
E医沥扬一s
E吴一
Ed标刑
E水
立[z〕=[0^姚]目国
一EE2叉yE“
E人小2发n吴c
E

E林李仪沥)
Page-190
.78第二章常见的尿部与非局部分责

E浩]

附注3.8“在应用定理3.6时,霁要首先判斯未扰动系统X
以0为细焦点的阶数,也就是确定满足条件(3.5的&在实际计
算时,经常应用下面介绍的Liapunov系数法,细节请见[ZDHD]
医

〈E之圭玟
E

E伟d
医达

E兽位2吴成应吴人画江

医述2

E
蔷=针…掳沥江园[沥
E

E

漾足上式的(V,}称为(3.16)的iapunov系数.在下面定理的意义
e

E八北沥朐圆一仁,一0,8.丿0(或一0),当且仅当
E唐5

定理的证明见[BL],下文中,我们把(}或(Re(e,}称为系
统的焦点量.

应用

例3.10“考虑二维系统
壬园【
林沥吴玟江志玟口招玟技招途仪
E振人i林人
(一0附近发生Hopf分岔的可能性.令一乙十1,系统(3.24
.变为
Page-191
E

1林)

E沥江林沥闵吴

退化Hopf分岖定理
E人

E
E

人

Eodii48沥

引理3.4设在奇点(z,)一(0,0)系绕(3.4的线性鄂分矩
阵有一对复特征根a(4土讨(49,满足条件(H,则对任意自然数
E述
e

0

十C′(#)″/+】歹/E0(|槽/|″十…),

E吴0佳河不芸水
e技2

E沥t0
[沥王一
E技

[p

由条件CH)知,力(0)二(m,A(03)给出一2十2阶的共摄条件,

巳一许
E途202志y万一4途育小应25
0

E
Page-192
E
E

红2
E江达0述
根沥5技水沥逊T医
这个系统在(0,0)的线性都分短阵有一对纯虚特征根的条件为
浩胡理1沥
医林d渡仪一仪s途1莲2李浩7
d

d沥E
″75河玟

E河理2吴
E沥江

E圭(的玟

1
E
国育5

v…=訾仰E
由正应用定理3.6,可得下列结论:
E李u[林史王吊不不吊
盆,并东系统(3.24》在原点附近存在唯一极限环的参数区域是
E月
0化2儿沥沥不7林丿史吊不c永
Hopt分岔,(3,.24)在原点附近存在二个极限环的参数区域是
E朱应
D厂才尸

0坂id

[E
E昔#u[河浩二伟口河

D林ss

t河
Page-193
E瑶汀150

E沥沥ta|

E出东仪y水
E

E医述

定理3.6“设闯量场X以(0,03点为&阶细焦点;则X在拼
E水

0
的邻域口,使得当|x|一a时,在乙内至多有8个极限环!

E沥林
(z,3一(0,0)的任意邹域口“人口,存在一个开折系绕;,使得
恒在口“内恰有了个极限环,其中|x|一.

E沥(沥俊圆沥t
E达技一技p
意口一z5,e2pi二i,可得到

{E朐
E命

0

El沥

盖g吴达

[
办许
E

t-仪,
2e昙?亘畲))+矾E

e(缥(
0Z室E沥口

沥政沥
Page-194
第二章常见的局郭与非尾部分岔

EXcuY
E+陲)命
c八沙
E
E
的一种标准形式导出了著名的焦点量公式(见[Ba],并证明了二
技
数值上都有误,在[QL],及[FLLL]中得到纠正.下面介绍的结
E沥
E技
壬2E缥舶翼z十anzy十″叩燮z'
E河园/园
E
E
【757有厂木272
E述
则(不计正数图子》
卯一ha一B,
【01
[E口2振EAgs
E沥王孙芒u刑
(3.28)
在最后一种情形下,系统可积,(0,03为中心点
附注3.12从原则上说,当X以0为细焦点时,总可以通过
有限步运算确定细焦点的阶数&.但是当较大时,对一舫系统息

用(3.28)的方式来表达焦点重公式,计算重非常大.例如,当把方
Easetoakypspp

阮
Page-195
国第二章常见的局部与非局部分芸

这里乃j二办j(Py史是pE[0,2z]和p(在0附近)的光滑函数
E
霉…二噜[C3.9)
在乙轴上建立方程(3.8)的Poincare晔射PCz:,4,并令
ELC命52河洁[
显然,
【A水《C3.11》
【怀iss26志752浩胡诊
E
医江玖
E
E人不
E
E
E

E英E余国
F`二〕___】((),0)Ea′(0,2‖)E

0

[达
E朱吴吴gr8
E

E医述2八2技1
「'行日d诊
E

3扎国

0〔(2发+1)丨〕蒂…'E
。

y

0

0E明d理

萨鲜J(′-'2‖)i35一Re乙`鏖
e

《3.13)
Page-196
[颖水25

的焦点量公式十分复杂;即使利用计算机,目前也仅推导出前儿个
E
点这个固难阿题紧密相关,在这里我们仅列举我国学者在这方面
E吊t招
中心和焦点的较]般方法;杜乃林、曾宪武圃〕给出了计算焦点童
E沥沥
沥
黄文灵“证明,当非线性方程零点的拓扑度变名时,会产生述通
的分岔曲线,等等

E些

【xta2怀e李d吊2
其中口是R:中原点的一个开集,则当条件(H),CH)和(He)成
立时,或者系统(3.1)当一0时以原点为中心,或者当AE(一,
E
d
Ei沥t东p浩
E

例3.14

羞国八述0吴

【
0

s标班2述月

[朋

【
E罚0

au林s一国
u
d′_

L匹
FE浩河玟薯)g
E

医沥
Page-197
E瑞水5

i技
壬0,当1<m<触+L
〈

2″〔(2桑十1)【〕婶"[李玲

0

注意方程(3E东
E
i
林
s吴玟不e朋
园

PCru,40二@(zruDh(zru,19.

古一方面,Y(0,)二0,且(ru,40对心的正根与负根成对出现
0i
uoapyeeoyoytu不a河A招
对至多有&个正根,定理的结论(1)得证

为了证明结论(2),我们假设Xu以z一0为阶细焦点,即它
具有如下的正规琪

E八林许有述门Rect奂0.
取它的扰动系统
E不
[沥5
E土贝训招中
E
红林述
9
t

E东述林
命
Page-198
5第二章帝见的尸部与非尿部分盅

其中0一r一1.故原点是(3.29)。的渐近稳定焦点
E

1
【江sinr一菩)'

图2-2

t林e一睾与E3e一丢之间(见图
E由于

巳技E一
园言c鹏r詹

i
汀东t河才年9
03技[d′【(″〉E取巳刑

E29〉″在原点附近都至少有两条闭轨′因

Ess途伟沥
请凄者验算,此例中对一切正整数A,都有Recx二0.团此,对

【omestsrs2吊e怡

对参数一致的Hopf分岔定理
E
Page-199
E脱u

巳河国匹,
E沥

@张
E林丞

其中函数双一友(z,y7,参数3,xER「,丁$为小参数.设系统有
d
Ed
【5)E

则在一定的附加条件下,当一x(8,8人(0,o时,系统(3.32》
Et沥
3.7可知,存在(8)丿0,使得当|x一A(8)|一e(8且人丿
E
E沥0吊e
s,使得对所有的E(0,8),都有e(8之e这就是所谓对参数
g

[沥

E林振芸s怀

目国园伟芒97扬
[国P

国
Page-200
0第二犊常见的局部与非尾部分岑

53吊2里
E兰吊东
Ed月月汀7租(顶荣

【7河砺7月园0园7
Estyuupuy4,#′′一j,r′一j,1吏得EE
E胺
[余A

[李前=[

[沥7口
由Poincare-Bendixson环域定理,扰动系统(3.15)至少存在个极
限环.

E才[
E
个极限环.若不然,则对任意的a>0和7一0的邺域7,(3.15)在
口内有多于}个极限环,则我们可以仿照上面的方法选取_一,
rk-j-Ly“皋叶以及eroy屋而在口内再获得古外的&一3个极限
E
0|

定理3.1的证明“当一1时,可从定理3.6得到定理3.1.
E3s林浩日(仪2招
一阶细焦点,因此定理3,1的前一部分结论成立,再设条件(Ho)
E命河应图生刑

Ed5沥多庞z_T〉【e2525[途
其中

0【〔expE羞座〕51〕十d述

冉
[
再由(3.14可知
Page-201
医脱u

E

E命71E述c

由(3.17)和条件(Ha)可知

E

蠢(O'O)园湃(O)挝(0)Ealt2E
E存在疃〉0和在0<zl<疃定义的光滑函数″_
[

E[E

至此,定理31的结论(1)得证.为了证明结论(2,从(3.20)求导、
E

E

E
F+萜烬(z】)二

E
E

[
E及条件(H】)可得

,把上面的结果及(3.19)代人(8.21)得到

国国E
Er

冉此得定理的结论(22,【
E招医
E人
一0的邻域口,使得
E
个极限环,当Ree一0(>0时,它是穗定(不稳定)的并且当A
s口
0
Page-202
第二辅常见的尸郭与非尿部分盅

E二沥沥圆胡沥
e
[沥p4李
E加
E吊p

068林2仪
E怡述
5尘

【63i仪2羞彻戚)>伪当a“ci丿0阡,羞慨删
E

0圆技
EaLepa不
E仪73技10272述
[张水育2河z育技
医途园应江[

玟
e
蜇萱鲨z'(0)0,0)荠0.

E李
定理即可.其它推理与定理3.1的证明相同。〖

E技李
i

(当0一8心品,|K一KK81一a且aroi(K一AC8D)一0
E
定(不薹定)的极限环

02林
E

y
Page-203
.78第二章常见的尿部与非局部分责

E浩]

附注3.8“在应用定理3.6时,霁要首先判斯未扰动系统X
以0为细焦点的阶数,也就是确定满足条件(3.5的&在实际计
算时,经常应用下面介绍的Liapunov系数法,细节请见[ZDHD]
医

〈E之圭玟
E

E伟d
医达

E兽位2吴成应吴人画江

医述2

E
蔷=针…掳沥江园[沥
E

E

漾足上式的(V,}称为(3.16)的iapunov系数.在下面定理的意义
e

E八北沥朐圆一仁,一0,8.丿0(或一0),当且仅当
E唐5

定理的证明见[BL],下文中,我们把(}或(Re(e,}称为系
统的焦点量.

应用

例3.10“考虑二维系统
壬园【
林沥吴玟江志玟口招玟技招途仪
E振人i林人
(一0附近发生Hopf分岔的可能性.令一乙十1,系统(3.24
.变为
Page-204
E85

在第三章$1中,我们将看到这种对参数一致的Hopf分岐定
E

E浩二D

E沥an不
医技el芸认水国
上*这种连接鞍点的扬线在拼动下可能破裂,从而改变系统的拓扑
E
en

医

E江泊

盖=训z跚鹑E

其中″eC镳mzXR,R2).设X。的轨线结构如第一章图L9(b)所
示!它有一条初等鞍点的间宿捉,内部是穗定焦点的吸引域
技技
洛形,见图1-9的(a)与Cc).

显然,在图1-9(c)的情形,分界线破瑞的方向与砾褐前了的穗
E伟王刹
东
辅线.因此,我们可以认为闭软是从二经扰动破裂而产生的(或反
B人l运标t技
atetueuupiakspott刑

E

[[
E伟c日顺7林林扬
C
Page-205
E
E

红2
E江达0述
根沥5技水沥逊T医
这个系统在(0,0)的线性都分短阵有一对纯虚特征根的条件为
浩胡理1沥
医林d渡仪一仪s途1莲2李浩7
d

d沥E
″75河玟

E河理2吴
E沥江

E圭(的玟

1
E
国育5

v…=訾仰E
由正应用定理3.6,可得下列结论:
E李u[林史王吊不不吊
盆,并东系统(3.24》在原点附近存在唯一极限环的参数区域是
E月
0化2儿沥沥不7林丿史吊不c永
Hopt分岔,(3,.24)在原点附近存在二个极限环的参数区域是
E朱应
D厂才尸

0坂id

[E
E昔#u[河浩二伟口河

D林ss

t河
Page-206
E第二辅常见的尸郭与非局部分盅

形(b)可类似讨论2.

1吴c

0

离2-4图2-5

e
是重要的。

n
例2.10中,这是利用7内的焦点的稳定性得出来的.我们帝望从

向量场在鞍点和7的特性来获得这个借息.

(3)如何判断X,的稳定流形与不穗定流形的相互位置?

为了解决间题(1),我们先对X在T内侧引入Poincare跚射
[余园小林t
(向内为正3.则在吉附近存在一个邻域,使得VE口史,从
丶出发的辅线9,经过+一T办将再次三I交于一点PCD)一
E28t42志2沥俊d丞
0

丶一助十史m,0河
河t
E逊[
E沥园水技
E河仁
定义4.1乃,的同宿转T称为是渐近稳定(或不稳定)的,如
Page-207
第二章常见的局郭与非尾部分岔

EXcuY
E+陲)命
c八沙
E
E
的一种标准形式导出了著名的焦点量公式(见[Ba],并证明了二
技
数值上都有误,在[QL],及[FLLL]中得到纠正.下面介绍的结
E沥
E技
壬2E缥舶翼z十anzy十″叩燮z'
E河园/园
E
E
【757有厂木272
E述
则(不计正数图子》
卯一ha一B,
【01
[E口2振EAgs
E沥王孙芒u刑
(3.28)
在最后一种情形下,系统可积,(0,03为中心点
附注3.12从原则上说,当X以0为细焦点时,总可以通过
有限步运算确定细焦点的阶数&.但是当较大时,对一舫系统息

用(3.28)的方式来表达焦点重公式,计算重非常大.例如,当把方
Easetoakypspp

阮
Page-208
EE

沥
0英

恤n捌[

P

所决定:当H辖耀Et怡a月也就是b鹫障E吊李市东

[
E

E噩(0'O)l

E东2ttn
EuEsbpag芸i
Eg河
Poincare映射
P:厂研一加,PCB)二8(T(B,切,
E朋
E

e

园伟7朋d5
/′7″。_卜a棘【_『(′】”0,5

E
E
E

d
f口-《疫一

如果记
羞林

E

E沥人仪
Page-209
第一竟腐见的尿部与非屉部分盆

EE
D
E
E命玲22
其中a息一0当a0.男一方面,由于
E人河玟w
技
E

欣国
E【_TU)/″交国

E

园
ag:

才

[lE河玟薯]
Ei

E
deta'>ro一胡玟江人D

其次,我们来计算上步左端的行列式(它与坐标系的选联无关).注
意怡F(t,)是变分方程
琶5垩(PE0222D
a东
E亘P〔妻y/>)〕二expI二n暴/砷【沥d

E
由C4.7),C411)和(412》最后得出
Page-210
[颖水25

的焦点量公式十分复杂;即使利用计算机,目前也仅推导出前儿个
E
点这个固难阿题紧密相关,在这里我们仅列举我国学者在这方面
E吊t招
中心和焦点的较]般方法;杜乃林、曾宪武圃〕给出了计算焦点童
E沥沥
沥
黄文灵“证明,当非线性方程零点的拓扑度变名时,会产生述通
的分岔曲线,等等

E些

【xta2怀e李d吊2
其中口是R:中原点的一个开集,则当条件(H),CH)和(He)成
立时,或者系统(3.1)当一0时以原点为中心,或者当AE(一,
E
d
Ei沥t东p浩
E

例3.14

羞国八述0吴

【
0

s标班2述月

[朋

【
E罚0

au林s一国
u
d′_

L匹
FE浩河玟薯)g
E

医沥
Page-211
5第二章帝见的尸部与非尿部分盅

其中0一r一1.故原点是(3.29)。的渐近稳定焦点
E

1
【江sinr一菩)'

图2-2

t林e一睾与E3e一丢之间(见图
E由于

巳技E一
园言c鹏r詹

i
汀东t河才年9
03技[d′【(″〉E取巳刑

E29〉″在原点附近都至少有两条闭轨′因

Ess途伟沥
请凄者验算,此例中对一切正整数A,都有Recx二0.团此,对

【omestsrs2吊e怡

对参数一致的Hopf分岔定理
E
Page-212
EiE

3
8「Cm=卜Eexp〔鲈槽圣八似岫刑池0

E晕八趴mben

E

E沥05
E
E一厂芒
E不
由附注42立得定理的结论,〖
2

怀

u一唐排
E一技
闻′
E个
门
E
E人伟切
Ec一
Poinc墓【e映射P″=。
E吊cny页c刀用
技口
一
武
、0当o一0,
E
E逊
Page-213
E脱u

巳河国匹,
E沥

@张
E林丞

其中函数双一友(z,y7,参数3,xER「,丁$为小参数.设系统有
d
Ed
【5)E

则在一定的附加条件下,当一x(8,8人(0,o时,系统(3.32》
Et沥
3.7可知,存在(8)丿0,使得当|x一A(8)|一e(8且人丿
E
E沥0吊e
s,使得对所有的E(0,8),都有e(8之e这就是所谓对参数
g

[沥

E林振芸s怀

目国园伟芒97扬
[国P

国
Page-214
筝二章常见的尾部与非尾部分盆

古--方面,两个具有相同稳定性的闭技不可能并列共存,因此
E不朔]

现在,茵下要解决的就是我们在前面所提的间题(2),即X的
Dk棣
形X的相对位置?Melnikov函数就是用以措述P;和x之间
a才

E

d
[光D

E伟伟XR瓜饥r>2'E东
E
EE仪月国余人c

E庞t[″`]二[重_],定义

E
[
〉a

容易验证,对于任意二阶方阵4,有
L述
过卯上的(0)点取截线,使它沿法向a(03一(p「(0))
E2林挂水年江招
y浩
a林
E怀l林2
ten
EzuaRspaeusltey河aae间*缴
E
E5园
6
Page-215
4平国上的合宿分盅

五(p(6)8tCPCC),0)
E仪
E

E技2

E丘钛寰′〈霄【E

E东河技纳玖芸一水水水
E沥不

502河人干一胡0
E

E亡岗D(霉)e一′“〉d′,0

[李1朐
E国胡

E
5

E
52
l志1
E

注意印;(4)是(4.14)的解(丿0),把它代人(4.143,对求导后取

E
林

[

目
E林

E

E32沥钨/「(矽(壹〉〉
Page-216
第二辅常见的尸郭与非尿部分盅

E二沥沥圆胡沥
e
[沥p4李
E加
E吊p

068林2仪
E怡述
5尘

【63i仪2羞彻戚)>伪当a“ci丿0阡,羞慨删
E

0圆技
EaLepa不
E仪73技10272述
[张水育2河z育技
医途园应江[

玟
e
蜇萱鲨z'(0)0,0)荠0.

E李
定理即可.其它推理与定理3.1的证明相同。〖

E技李
i

(当0一8心品,|K一KK81一a且aroi(K一AC8D)一0
E
定(不薹定)的极限环

02林
E

y
Page-217
第二章常见的局部与非尿部分会

E[tr巫胖)4…(Z)45X8吊0标62
E
Ee′(′)[〈廖(0〉E]′二e′′(薹)E林2
0
男一方面,因为t一co时(史一zo,所以p(D)一廿zo)一0,
E林
E
界线兆滑依赖于参数的定理可知,当f芸0时办(有界.从而由
[垩位山则易知
林
E

E

从而由(4.23)式及上式得到
E
EinI"E

[

EE

5

队(4.167,C420)那(421)易知:Q(0)因0
E沥2圆|

由定理48,定理4.4和定理45立即得到

定理4.6设X,由(4-14)给定,X以z为双曲鞍点,昆有顺
E
E

(若j丿0(或一03,则乙,在P的了邻域内恰有一个
从卯分盆出的极限环.当r一0时,它是稳定的!当o之0时,它
是不稳的.
Page-218
E85

在第三章$1中,我们将看到这种对参数一致的Hopf分岐定
E

E浩二D

E沥an不
医技el芸认水国
上*这种连接鞍点的扬线在拼动下可能破裂,从而改变系统的拓扑
E
en

医

E江泊

盖=训z跚鹑E

其中″eC镳mzXR,R2).设X。的轨线结构如第一章图L9(b)所
示!它有一条初等鞍点的间宿捉,内部是穗定焦点的吸引域
技技
洛形,见图1-9的(a)与Cc).

显然,在图1-9(c)的情形,分界线破瑞的方向与砾褐前了的穗
E伟王刹
东
辅线.因此,我们可以认为闭软是从二经扰动破裂而产生的(或反
B人l运标t技
atetueuupiakspott刑

E

[[
E伟c日顺7林林扬
C
Page-219
E5

E坤c沥2en志
E
n
Eacesuesl招是定号的E
l兰
E江
【E述达2
3一QCzo37,

以原点为双曲鞍点y有顺时针定向的同宿轨风,并且姚=茎m胁

E鬟m涮E
{0
小一@Q(z,3)十户PCzo9)
在巩的小邻域内恰有一个极限环(其稳定性由oo的符导决定)而

当jo一0时,(424)在P附近没有极限环.
事实上,

国
.D(灞)=de【〔Q申〕=(尸z十Q【〉」=70)>0'

阮余因此4>0利用定理4.6,上而的结.
论立即可得
E

E
分盅.RoussariePl和Joyal00分别讨论了从(逾化的同宿执分岔
出多个闭软的闰题.他们的基本怡想是在奇点附近利用鞍点性质。
与大范图的微分同胚相结合,得出Poineark映射的表达式,从而在
逃化程度较高时,可以经过逛次适当的扰动,反复改变同宿轨内侧
bnoeaetdn沥
D沥d述2河
Et
Page-220
E第二辅常见的尸郭与非局部分盅

形(b)可类似讨论2.

1吴c

0

离2-4图2-5

e
是重要的。

n
例2.10中,这是利用7内的焦点的稳定性得出来的.我们帝望从

向量场在鞍点和7的特性来获得这个借息.

(3)如何判断X,的稳定流形与不穗定流形的相互位置?

为了解决间题(1),我们先对X在T内侧引入Poincare跚射
[余园小林t
(向内为正3.则在吉附近存在一个邻域,使得VE口史,从
丶出发的辅线9,经过+一T办将再次三I交于一点PCD)一
E28t42志2沥俊d丞
0

丶一助十史m,0河
河t
E逊[
E沥园水技
E河仁
定义4.1乃,的同宿转T称为是渐近稳定(或不稳定)的,如
Page-221
EEtn

d河
E

Ec
厂
叶口得出了在临界情形下判别同宿转或异宿扬的穗定性的方法;
Mourtada““对含两个鞍点的异宿环的分岔问题进行了深入的研
E,

对参数一致的同宿分岔

类似于对参数一致的Hopt分岔问题,现在考虑含双参数8
E

医E
盂r52

4E
亩萱国诊E述

Eaaii不
E
E
的同宿技7s.则在适当的条件下,对每一个国定的$>0,存在
s(8友0,例当|x一p(8D|人e(8)时,疚「的邻域内有定理4.6
的丽条结论,我们关心的是:当8->0时,如何保证s(8)不趋于零,
E

我们不在止给出一舫的定理,只在第三章引理1.6中对一类
特殊的系统介绍这种对参数一致的同宿分岗的结果.

0

5e梁

河述

[儿
Page-222
EE

沥
0英

恤n捌[

P

所决定:当H辖耀Et怡a月也就是b鹫障E吊李市东

[
E

E噩(0'O)l

E东2ttn
EuEsbpag芸i
Eg河
Poincare映射
P:厂研一加,PCB)二8(T(B,切,
E朋
E

e

园伟7朋d5
/′7″。_卜a棘【_『(′】”0,5

E
E
E

d
f口-《疫一

如果记
羞林

E

E沥人仪
Page-223
第一竟腐见的尿部与非屉部分盆

EE
D
E
E命玲22
其中a息一0当a0.男一方面,由于
E人河玟w
技
E

欣国
E【_TU)/″交国

E

园
ag:

才

[lE河玟薯]
Ei

E
deta'>ro一胡玟江人D

其次,我们来计算上步左端的行列式(它与坐标系的选联无关).注
意怡F(t,)是变分方程
琶5垩(PE0222D
a东
E亘P〔妻y/>)〕二expI二n暴/砷【沥d

E
由C4.7),C411)和(412》最后得出
Page-224
0颜E

E伟t沥砂沥t沥技水E
环基,即存在一系列闭辅

c不有
E一
s不
t一芸
Poincare分岔问题.

E

河沥
列(当X为Hamilton系统,且一为相应的Hamilton函数时,这个
E人林c一
化.设
E仪06沥
其中T是7的周期为了考察当x一0时丶,过p(0,4的解能
Ei沥t
E水
户内一8(0,40-设此解当一T时再次与工相交,则由微分
E
(命07二日(工(,0)二T0
定义后继函数
02一3
E沥
u江
且GECr
江d林莲租5
期,可以从(5.4得到
E月0
Page-225
EiE

3
8「Cm=卜Eexp〔鲈槽圣八似岫刑池0

E晕八趴mben

E

E沥05
E
E一厂芒
E不
由附注42立得定理的结论,〖
2

怀

u一唐排
E一技
闻′
E个
门
E
E人伟切
Ec一
Poinc墓【e映射P″=。
E吊cny页c刀用
技口
一
武
、0当o一0,
E
E逊
Page-226
E第二章常见的居部与非局部分岔

[寸胡刑
[(5.5)

E东2
E

[62怀汀2才s标

0东沥
Eurzaelugts命p王
玖
ECLEE4不

E国08怀2p谅
[技
胡园[p一发口5一育i

[芸2述林前
E4林i训育
E
Ee
n
吴

E发东江刑昙?(廖my#m)二0.

沥

为了实际应闵的方便,下面的定理给出从7的表达式与原方
技

定理5.2“对于方程(5.1)和7「的表达式(5.2),有

0
ELt沥林2车
E

『
c二Lr数(Gp)d
Page-227
E沥颖3江

E圆育2
E且G伪卢)|

″【。

〈[az窃r(″,″)

E国上
珈)L动E

E

解,利用(5.3及T为T的周期可得

医E2林2
E
颜3

国沥E达
0
着令
2〈噩亿加腻交Gp0,m7》

一噩化加mAACp0
园
Ey林3T月[
与$中(4.23)式的推导相类似,可得A(,4)的表达式如下

e′【“″)[4〈o,^)十I二e一″“'″〉g(哽【22

E
E
[

E余北沥朐胡沥
[
E为喜阶Md墟嶂。v函量,从它的零真分布可研究扰动
E怀水江
Page-228
筝二章常见的尾部与非尾部分盆

古--方面,两个具有相同稳定性的闭技不可能并列共存,因此
E不朔]

现在,茵下要解决的就是我们在前面所提的间题(2),即X的
Dk棣
形X的相对位置?Melnikov函数就是用以措述P;和x之间
a才

E

d
[光D

E伟伟XR瓜饥r>2'E东
E
EE仪月国余人c

E庞t[″`]二[重_],定义

E
[
〉a

容易验证,对于任意二阶方阵4,有
L述
过卯上的(0)点取截线,使它沿法向a(03一(p「(0))
E2林挂水年江招
y浩
a林
E怀l林2
ten
EzuaRspaeusltey河aae间*缴
E
E5园
6
Page-229
4平国上的合宿分盅

五(p(6)8tCPCC),0)
E仪
E

E技2

E丘钛寰′〈霄【E

E东河技纳玖芸一水水水
E沥不

502河人干一胡0
E

E亡岗D(霉)e一′“〉d′,0

[李1朐
E国胡

E
5

E
52
l志1
E

注意印;(4)是(4.14)的解(丿0),把它代人(4.143,对求导后取

E
林

[

目
E林

E

E32沥钨/「(矽(壹〉〉
Page-230
E第二章常见的局部不非局郭分岔

E河述述
医
i怡连5

E林5李E才沥

垂实际应用中,经常出现(5.1)的一种特殊形式,即Hamitton
D仪

巳李
E

4E
董国芊E达

口
“之2.设当仁一0财,未扰动系统有闭软族,并有表达式
EHC515玖有河c月
E吴李途
E圣r^[菩亘PE箐Q[[沥

0

注意当x二0时,波有
E丞
E沥7林沥
图此,可把(5.11)改写为

E姿′QCzoyvG,吊dz一PCr,y8,bQy,《S.12》
^

玟
E江不

L八东坂国c江沥河芸一吴
王
的零点个数间题称为骏Hilbert第16问题.由于这个问题是V.工
阮
Page-231
第二章常见的局部与非尿部分会

E[tr巫胖)4…(Z)45X8吊0标62
E
Ee′(′)[〈廖(0〉E]′二e′′(薹)E林2
0
男一方面,因为t一co时(史一zo,所以p(D)一廿zo)一0,
E林
E
界线兆滑依赖于参数的定理可知,当f芸0时办(有界.从而由
[垩位山则易知
林
E

E

从而由(4.23)式及上式得到
E
EinI"E

[

EE

5

队(4.167,C420)那(421)易知:Q(0)因0
E沥2圆|

由定理48,定理4.4和定理45立即得到

定理4.6设X,由(4-14)给定,X以z为双曲鞍点,昆有顺
E
E

(若j丿0(或一03,则乙,在P的了邻域内恰有一个
从卯分盆出的极限环.当r一0时,它是稳定的!当o之0时,它
是不稳的.
Page-232
E沥额X公训异河沥大5

年,D,Hilbert00在第二居国际数学家大会上提出了23个数学间
Et
极限环的最小上界HCm是多少?可能出现的极限环相对位置如
何?近一个世纪以来*特别是最近儿个年来,出现了大量的工作.
例如,史松龄““和陈兰草,王明淑“w最先分别举例证明HC2)之
的李继彬\黄其明0举例证明HC3)丿11;叶彦谦、陈兰茹和杨信
玟
证明二次多项式系统中按叶彦谦分类的(!)类方程至多有一个
E李育刑
光的综述文章[Cs]和[CZJ,马知恩的专著[Mz],梁肇军的专著
[Lz],以及Dumortier等人的系列文章[DRR1,2]和[DER]中,读
者可发现大量有趣的绪果、经过IIyashenkom和Eealle02修补证
明后的Dulaem有限性定理指出:一个绘定的n许多项式系统的极
限环个数有限.但是,对全体n次多项式系统而言,其极限环个数
的一致上界如何佼计(郅怕是否有限,即使对n一2这种最简单的
非线性情形,仍是一个未知的问题.S.Smalerem认为,对HGo的
e
作者的一个重大拼战

由(5.5)式可知,Abel积分(5.12)是系统(5.10)在7附近后
E圆2
的极限环个数密切相关.当又是n十!次多项式P和Q为次多
项式时,(5.10)是一类特殊的n汀系统,即Hamilton向量场的扰动
E
Etss
Ea
的形式.此阡及,P,@可能不再是多项式.习惯上仍把(5.12)称为
E

例5.5考虑vanderPol方程
Page-233
E5

E坤c沥2en志
E
n
Eacesuesl招是定号的E
l兰
E江
【E述达2
3一QCzo37,

以原点为双曲鞍点y有顺时针定向的同宿轨风,并且姚=茎m胁

E鬟m涮E
{0
小一@Q(z,3)十户PCzo9)
在巩的小邻域内恰有一个极限环(其稳定性由oo的符导决定)而

当jo一0时,(424)在P附近没有极限环.
事实上,

国
.D(灞)=de【〔Q申〕=(尸z十Q【〉」=70)>0'

阮余因此4>0利用定理4.6,上而的结.
论立即可得
E

E
分盅.RoussariePl和Joyal00分别讨论了从(逾化的同宿执分岔
出多个闭软的闰题.他们的基本怡想是在奇点附近利用鞍点性质。
与大范图的微分同胚相结合,得出Poineark映射的表达式,从而在
逃化程度较高时,可以经过逛次适当的扰动,反复改变同宿轨内侧
bnoeaetdn沥
D沥d述2河
Et
Page-234
英

其中0一|x|奶1.可把它改写为如下的等价形式

[
FPgd

蒜=_z十粼1一菠)丛
述

TeE北
或写成参数方程

0

【招
G

国吴

EI。Ez肋z〔ET)′

显然&一2是它的唯-正零点,并东是简单零点.由定理5.1,当

r洁

l
E

但在多数情况下问题并不如此轻而易举,请看下例.
0一

E
【

E月

E坂标林
Hamilton系统,它有首次积分

3
″(雾黜)=音+Z_言0[

t
E育

E
Page-235
EEtn

d河
E

Ec
厂
叶口得出了在临界情形下判别同宿转或异宿扬的穗定性的方法;
Mourtada““对含两个鞍点的异宿环的分岔问题进行了深入的研
E,

对参数一致的同宿分岔

类似于对参数一致的Hopt分岔问题,现在考虑含双参数8
E

医E
盂r52

4E
亩萱国诊E述

Eaaii不
E
E
的同宿技7s.则在适当的条件下,对每一个国定的$>0,存在
s(8友0,例当|x一p(8D|人e(8)时,疚「的邻域内有定理4.6
的丽条结论,我们关心的是:当8->0时,如何保证s(8)不趋于零,
E

我们不在止给出一舫的定理,只在第三章引理1.6中对一类
特殊的系统介绍这种对参数一致的同宿分岗的结果.

0

5e梁

河述

[儿
Page-236
[圆

为边界.闭软族为
【河{(工'汊)I″(z,汊)E昙<矗<鲁}'

E

利用定理5.1,定理5.2和(5.12),为了研究当0一14|丿1时
n

E招蛋r[命2【

E一
E

n

Picard-Fuchs方程法(例如,见[CS]),和我们最近得到的直接方
E

Pieard-Fuehs方程法

i
E标玑0
E
Page-237
第二章常见的尾邢与非尾部分盅

E林诊河3人
E沥沥

故可定义
2

E
E
T
E诊晓5SCD为工所图成的紧区域的面

一(h)二

0述
E2
定理5.7(1〉当一兽〈颢〈兽时,亭〈P仇)〈h且

^苎瑗填P(克)河

沥E国一着5
1523E3E版政史且^翌珥'…P0
D

许
E「
E丧育水
医移up林n
引理5.8P(满足如下的Riceati方程
n【
E

E
Page-238
0颜E

E伟t沥砂沥t沥技水E
环基,即存在一系列闭辅

c不有
E一
s不
t一芸
Poincare分岔问题.

E

河沥
列(当X为Hamilton系统,且一为相应的Hamilton函数时,这个
E人林c一
化.设
E仪06沥
其中T是7的周期为了考察当x一0时丶,过p(0,4的解能
Ei沥t
E水
户内一8(0,40-设此解当一T时再次与工相交,则由微分
E
(命07二日(工(,0)二T0
定义后继函数
02一3
E沥
u江
且GECr
江d林莲租5
期,可以从(5.4得到
E月0
Page-239
[颜25招水2

0
EE
46052

E

R才沥仪朋
林林
【

古--方面,利用分部积分可得
口

【不水
[月(1

E

E

{伟河2
[

E河标述0
E
t朱3
E

[訾岫E绍
[C77誓盯「

把(5.27)代入P二_』,即得(5.22),』
定理5.7的证明把(5.22)改写为系统
F【沥人沥盖=一9十4《5.28)

EspaEE相g葛口y才刑
P仇〉咔h当力令一鲁因此汨=P仇〉的图形是从系统佤2盼的

l一朐
Page-240
E

的水平等斜线由方程7P:十3RP一5一0给出,它的图形为双曲
医政述

E贵〔′E林0

显然P一P(4)的图形与乜二&(的国形除了两奇点外不可能相
E

巳吴c半,
EE

0
L一一
的图形上方,从而沿P一P有P(D一0lian万(一一“可
E

直接方法

为了判断Abel积分的零点个数,目前使用的方法大都要经过
曲折的推导.下面介绍一个可在一定条件下从原方程判别的直接
E

04颖形iu半2着)沥
Page-241
E第二章常见的居部与非局部分岔

[寸胡刑
[(5.5)

E东2
E

[62怀汀2才s标

0东沥
Eurzaelugts命p王
玖
ECLEE4不

E国08怀2p谅
[技
胡园[p一发口5一育i

[芸2述林前
E4林i训育
E
Ee
n
吴

E发东江刑昙?(廖my#m)二0.

沥

为了实际应闵的方便,下面的定理给出从7的表达式与原方
技

定理5.2“对于方程(5.1)和7「的表达式(5.2),有

0
ELt沥林2车
E

『
c二Lr数(Gp)d
Page-242
E沥颖3江

E圆育2
E且G伪卢)|

″【。

〈[az窃r(″,″)

E国上
珈)L动E

E

解,利用(5.3及T为T的周期可得

医E2林2
E
颜3

国沥E达
0
着令
2〈噩亿加腻交Gp0,m7》

一噩化加mAACp0
园
Ey林3T月[
与$中(4.23)式的推导相类似,可得A(,4)的表达式如下

e′【“″)[4〈o,^)十I二e一″“'″〉g(哽【22

E
E
[

E余北沥朐胡沥
[
E为喜阶Md墟嶂。v函量,从它的零真分布可研究扰动
E怀水江
Page-243
[圆2异河八105

Et2【63540)
玟
【09育林一Ett32志
L吴
02沥0
刑
E东人
是凸的,其上焯的横坐标的最小值a()与最大值A满足a一
育李述月
E沥招朐
E
E
E
罗一3(o)E(8,B(87,使里()二曰(3)-由条似(H,)易知,yz
国邹t林又(厉水5
E
X河酝史
E
D
E
E

E

4

E出林0林诚仪河小招
【5河c志仪2
[林沥
园达厂

Ae3i吴扬3伟3

5Ec沥c
Page-244
E第二章常见的局部不非局郭分岔

E河述述
医
i怡连5

E林5李E才沥

垂实际应用中,经常出现(5.1)的一种特殊形式,即Hamitton
D仪

巳李
E

4E
董国芊E达

口
“之2.设当仁一0财,未扰动系统有闭软族,并有表达式
EHC515玖有河c月
E吴李途
E圣r^[菩亘PE箐Q[[沥

0

注意当x二0时,波有
E丞
E沥7林沥
图此,可把(5.11)改写为

E姿′QCzoyvG,吊dz一PCr,y8,bQy,《S.12》
^

玟
E江不

L八东坂国c江沥河芸一吴
王
的零点个数间题称为骏Hilbert第16问题.由于这个问题是V.工
阮
Page-245
E

国沥265
c力Z1

E吊2林8技(技

定理5.9([LZJ〉设玑(z,y)具有(5.30)的形式,并且条件
【
Ed坂2技亚102

有时所考虑的Hamilton函数具有如下形式

E人伟2【

E
考虑Abel积分之比

0

tE

阮2

E刑50
0仪不
E咤志王
邹c林d
E52河A示EC2Eoxe
[
E林4国
E65
t林技圆

例5.6的第二种解法

C音鹧彻)=′一告孰

E沥沥05c伟
E八2标2
成立.代入(5.37)得`
Page-246
E

E江知盖<0,且由图2-6知乙一一1心守心1从而

答′(冀)=(鞭阜…)z(1一…z)十(1一灞z)羞E

E
D二浩河p2
E
E命仁(河途2

E=踹,而且川脯〉伪当

0
E

E庞玟述
其中a,b,c为常数,一A心加,面且当5丿血时Jo(t)丿0,则可
把7(政写为

Et

医0一
E沥

Eak吴沥a
E述
Ell达a
所以当曲线@一真P)无变曲点(或有&个变曲点3时,T(8)在,
E
D
分相关,一些作者研究平面上与闭轨族相应的周期函数的单调性
E述s

$6“关于Petrov定理的证明
l贾沥
Page-247
E沥额X公训异河沥大5

年,D,Hilbert00在第二居国际数学家大会上提出了23个数学间
Et
极限环的最小上界HCm是多少?可能出现的极限环相对位置如
何?近一个世纪以来*特别是最近儿个年来,出现了大量的工作.
例如,史松龄““和陈兰草,王明淑“w最先分别举例证明HC2)之
的李继彬\黄其明0举例证明HC3)丿11;叶彦谦、陈兰茹和杨信
玟
证明二次多项式系统中按叶彦谦分类的(!)类方程至多有一个
E李育刑
光的综述文章[Cs]和[CZJ,马知恩的专著[Mz],梁肇军的专著
[Lz],以及Dumortier等人的系列文章[DRR1,2]和[DER]中,读
者可发现大量有趣的绪果、经过IIyashenkom和Eealle02修补证
明后的Dulaem有限性定理指出:一个绘定的n许多项式系统的极
限环个数有限.但是,对全体n次多项式系统而言,其极限环个数
的一致上界如何佼计(郅怕是否有限,即使对n一2这种最简单的
非线性情形,仍是一个未知的问题.S.Smalerem认为,对HGo的
e
作者的一个重大拼战

由(5.5)式可知,Abel积分(5.12)是系统(5.10)在7附近后
E圆2
的极限环个数密切相关.当又是n十!次多项式P和Q为次多
项式时,(5.10)是一类特殊的n汀系统,即Hamilton向量场的扰动
E
Etss
Ea
的形式.此阡及,P,@可能不再是多项式.习惯上仍把(5.12)称为
E

例5.5考虑vanderPol方程
Page-248
英

其中0一|x|奶1.可把它改写为如下的等价形式

[
FPgd

蒜=_z十粼1一菠)丛
述

TeE北
或写成参数方程

0

【招
G

国吴

EI。Ez肋z〔ET)′

显然&一2是它的唯-正零点,并东是简单零点.由定理5.1,当

r洁

l
E

但在多数情况下问题并不如此轻而易举,请看下例.
0一

E
【

E月

E坂标林
Hamilton系统,它有首次积分

3
″(雾黜)=音+Z_言0[

t
E育

E
Page-249
E笺二章常见的局部不非属鄯分贫

E途2s沥
原理估算零点的个数,男一方面,为了研究Hamitton向重来的批
动系统所县有的极限环个数的上界,我们霜要把相应的Abet积分
零点个数的佼计,与Hopf分岔、同宿分岔.异宿分岔等所能出现的
E这是本节将戛介绍的另一个问题

考怀Hamilton向量场的扰动系统

盂一一一十册位拟十呱炒′

林

E
E沥诊yE
设Harrilton向髯场(6.1)。有闭转族
F3沥9许中不有
吴江才罚江
软.设口是奇闭转一二a所园成的区域的紧致邻域,则当x充分
32才芸3水
面Abel积分(也吸一阶Melniktov函数,参看(5.12)式)

EL“)Q(罩,汊)d工Exta72o52

E心的应

[形芸江
1沥5
E
「【C3ie技3志5许5途一罚江厂步沥一禅5许灵i
E

[技c沥阮
E河吴e
团孝i9河E逊2
Page-250
[圆

为边界.闭软族为
【河{(工'汊)I″(z,汊)E昙<矗<鲁}'

E

利用定理5.1,定理5.2和(5.12),为了研究当0一14|丿1时
n

E招蛋r[命2【

E一
E

n

Picard-Fuchs方程法(例如,见[CS]),和我们最近得到的直接方
E

Pieard-Fuehs方程法

i
E标玑0
E
Page-251
[浩河E

E
个别情形外,迄今未得到其表达式,对一些特殊的二次或三次
Hamilton向量扬在a一2,3或4的情况下来佶算BCm,a和ZCm,
s沥
使寺于m一一2的一舱情形,问题也远没有後底解欧

EP河唐吾时,卵对Bogdanov-Takens系统,
E05晓7切
E招
幻一3,PMardesiePW0证明,BC2,m一A一1}李宝毅\张芷芬PAm
E
2
的研究中少有的完整结果.本节主要介绍Petrov和Mardesic的感
E故东5河
辅角厝理来直接伙算B(2,m)(见[LbZ])-这种手法可用来研究其
E

Abei积分的构造

s怀一坂2渡5应招
刊霆点个数,必须先研究它的构造,由Gzeen公式,有

EiGQudz,
E命E
其中Q也是<和y的多项式.

医t达i
i

I「〈^)髓】一Z`(′I)'′′z′
雌式成立的一个充分条件是
林河一朐
其中麒和旦都是z和y的多项式,显焊“~-“是等价关系,令
Page-252
第二章常见的尾邢与非尾部分盅

E林诊河3人
E沥沥

故可定义
2

E
E
T
E诊晓5SCD为工所图成的紧区域的面

一(h)二

0述
E2
定理5.7(1〉当一兽〈颢〈兽时,亭〈P仇)〈h且

^苎瑗填P(克)河

沥E国一着5
1523E3E版政史且^翌珥'…P0
D

许
E「
E丧育水
医移up林n
引理5.8P(满足如下的Riceati方程
n【
E

E
Page-253
c

E沥0
E
Dn
而言是一交换群;男外还有
了0DCan二[FCEHJa0u
即

/仇〉IE国2多史
45E

其中/是R的多项式,即丿可在多项式环火上进行乘法运算,一
E人0技园江
0pa0胡江310月
【应2E育达02
0(
[标途
E李n
E训
E
E
E
定理6.1(LP1])考虑Bogdanov-Takens系统

吴
[

羞′=_一覃十zzE吴e庞运

E
E

E仪(02仪一人音,E
EI「伽五汊d艾j2t(技江03
Page-254
[颜25招水2

0
EE
46052

E

R才沥仪朋
林林
【

古--方面,利用分部积分可得
口

【不水
[月(1

E

E

{伟河2
[

E河标述0
E
t朱3
E

[訾岫E绍
[C77誓盯「

把(5.27)代入P二_』,即得(5.22),』
定理5.7的证明把(5.22)改写为系统
F【沥人沥盖=一9十4《5.28)

EspaEE相g葛口y才刑
P仇〉咔h当力令一鲁因此汨=P仇〉的图形是从系统佤2盼的

l一朐
Page-255
E

的水平等斜线由方程7P:十3RP一5一0给出,它的图形为双曲
医政述

E贵〔′E林0

显然P一P(4)的图形与乜二&(的国形除了两奇点外不可能相
E

巳吴c半,
EE

0
L一一
的图形上方,从而沿P一P有P(D一0lian万(一一“可
E

直接方法

为了判断Abel积分的零点个数,目前使用的方法大都要经过
曲折的推导.下面介绍一个可在一定条件下从原方程判别的直接
E

04颖形iu半2着)沥
Page-256
圆肉33

〕'dEg/】(凡)H〔音〕一
我们将证明作为习题留给读者.只要对做归纳法即可迦得

Picard-Fuchs方程

E沥

E人

玟
满足以下的Picard-Fuchs方程

4
0
哚6>d′′Ll]一【

目
【7]y0<凡<6'

E2
6凡

曹力

0

E圆225

在5巾我们已经对Bogdanov-Takens系统的男一种等价形式

Ea河仁02育仪3沥2

《5.27).由于Pieard-Fuchs方程在研究弼Hilbert夙16问题中的重
要性,此处我们将介绍古一种递推公式“0““由于

工矗
E玄十玄5′L,()〈′z<曹
E

E
E河浩刀

E
E鬟E沥算z弓E沥
利门分部积分得
B一In兰z″_3汊m十zdzEIr仇)藁"一1汊mdz一

EE
Page-257
常见的局部与非尾部分盆

0

「(^)E

一dz03
TQ5小

由此得到

E汀
Page-258
[圆2异河八105

Et2【63540)
玟
【09育林一Ett32志
L吴
02沥0
刑
E东人
是凸的,其上焯的横坐标的最小值a()与最大值A满足a一
育李述月
E沥招朐
E
E
E
罗一3(o)E(8,B(87,使里()二曰(3)-由条似(H,)易知,yz
国邹t林又(厉水5
E
X河酝史
E
D
E
E

E

4

E出林0林诚仪河小招
【5河c志仪2
[林沥
园达厂

Ae3i吴扬3伟3

5Ec沥c
Page-259
不6关于Pewow岱理的证明

4
E[E
团E

罡IE[I
「(^)E国′`(炫)2汊EX033y

刑言4dr[
E「【^)

E林氓招
达吴6d′′r(″)z帅

由此得到
[暑茹伍)E告/翼(颢)7

其中0<^<告.E

&技E
′l[′!_畜)′。′(庞〉_〔6互6〕′”(庞)十36z】(庞)'0河

E河河0]

i
医c水
u刑

E
我们需要引入下面的定理.为了遮免行文冗长,我们不引证明,有
E
Ea一
Ey吴
cc
E
i

匹
t沥i沥仁伟
Page-260
E

国沥265
c力Z1

E吊2林8技(技

定理5.9([LZJ〉设玑(z,y)具有(5.30)的形式,并且条件
【
Ed坂2技亚102

有时所考虑的Hamilton函数具有如下形式

E人伟2【

E
考虑Abel积分之比

0

tE

阮2

E刑50
0仪不
E咤志王
邹c林d
E52河A示EC2Eoxe
[
E林4国
E65
t林技圆

例5.6的第二种解法

C音鹧彻)=′一告孰

E沥沥05c伟
E八2标2
成立.代入(5.37)得`
Page-261
第二章常见的尸部与非局部分盆

Ed一一

d

韩茂安和朱德明在专著[HZ]中对非Hamiton系统的线性中
心也证得与定理6.3完全类似的结论.

Et东沥(圆e朋
E口沥园妙如
EJ2n
函数p(有以下展式

002人林运

沥
E
2

t

仿=呈俨…)L,摹>十蔫〕d薹,当窿】=0y
其中积分式前符叶的选择,取决于闭转TCD的定向和当增加时
E园

E王刑刀一
E
E

定理6.4是研究同宿分岔的主要依搬.对于画宿转而言,韩
E沥
E
指敏一分性和三分性理论,深化了不变流形的表示理论,用统一的
方法研究了各类系统(高维或低维,可积或不可积)的同宿扬、异
E.

E林一一
林招一许
Page-262
E浩DE

日
夕D一〔告5凡〕0告'

其deg00一[252]-tdeg0一[孝-k
这个定理的直观含义是,如果当~0时,有?&十1或28个极
cn沥明
E
0

:
″(冀,汊)=誓十言`看=告一归o〈z〈告.

E「(告0不告〉对应系统(6.2),的闭执族,而
n
E

俨(告g2标述友标d林玟仪

其中0一1L,
E东述

EI……汊dzE河或兄人标逊标

EI天“)箕仪d迈河政儿人标述A

氓吊沥仪林肖5林孕河沥孕
E圆0
Page-263
E

E江知盖<0,且由图2-6知乙一一1心守心1从而

答′(冀)=(鞭阜…)z(1一…z)十(1一灞z)羞E

E
D二浩河p2
E
E命仁(河途2

E=踹,而且川脯〉伪当

0
E

E庞玟述
其中a,b,c为常数,一A心加,面且当5丿血时Jo(t)丿0,则可
把7(政写为

Et

医0一
E沥

Eak吴沥a
E述
Ell达a
所以当曲线@一真P)无变曲点(或有&个变曲点3时,T(8)在,
E
D
分相关,一些作者研究平面上与闭轨族相应的周期函数的单调性
E述s

$6“关于Petrov定理的证明
l贾沥
Page-264
第二章常见的局部与非局部分盅

。1一一
一羞z(z_l)(宁十玄)妾dz_沥

已林圆42
[〔_E国
“口(亚十I
月naQ)337
江沥标二E
1[az一EE1

直接运算便可证明引理.
Epn

E/。(告2仪达

E/】(告E吴伟

[达人木

g人
E一仁
t
E
由等式(6.10)立刻得min(m,ma丿心
第二种情形:zm一m:.
0招州沥
2吴沥兄朋
由引理6.6,有
【儿林
【技丶技
玟

E熹俪肠E胡
国o(Z鬓翼,E
Page-265
E笺二章常见的局部不非属鄯分贫

E途2s沥
原理估算零点的个数,男一方面,为了研究Hamitton向重来的批
动系统所县有的极限环个数的上界,我们霜要把相应的Abet积分
零点个数的佼计,与Hopf分岔、同宿分岔.异宿分岔等所能出现的
E这是本节将戛介绍的另一个问题

考怀Hamilton向量场的扰动系统

盂一一一十册位拟十呱炒′

林

E
E沥诊yE
设Harrilton向髯场(6.1)。有闭转族
F3沥9许中不有
吴江才罚江
软.设口是奇闭转一二a所园成的区域的紧致邻域,则当x充分
32才芸3水
面Abel积分(也吸一阶Melniktov函数,参看(5.12)式)

EL“)Q(罩,汊)d工Exta72o52

E心的应

[形芸江
1沥5
E
「【C3ie技3志5许5途一罚江厂步沥一禅5许灵i
E

[技c沥阮
E河吴e
团孝i9河E逊2
Page-266
日6关于Petow定理的证晓

E
E沥y吴林y挂
E林命怡妙bi李|

E达东

扩充到复域

E坂东颖刑伟
到复域

G=C\仙仨疃帆彦告〉,

并且具有以下性质。

E~九昙,E一屋,E林

《237o(07二0,卷(0)奂0z1C03一卯(03一0,00式03
E仪东亚江

E135
(3Imn0E林不蒽(沿上下边界无

定理6.8“考虑方程
E命A233402河a0
8一j是正则奇点的亮玖条件是灰一b分别是(和9的次
转

02振)EE
03阊Z12砂一″”)z/(′′)E沥沥5

E

对无穷远奇点作变挨文一二则有

d国E团
丽一_拐_,腩-″「十胭茁,
[途
Page-267
118第二章胺见的尿部与非局部分皿

其中m)`′(翼〕如果P[〕~″z,疃[半]~陟则由定理6.8,
一0是上面方程的正则奇炙,即无穷远奇点是方程(6.14的正则

定义6.9“方程(6.14)称为Fuchs型方程,如果它只有正则
E

E(东
技
E兰

p光[沥)

E

E吴沥异(达沥

E

E口河师e笑沥
E林达20
Rep乐Repsoas一1令加一f一f,期果办奂0,则功一1.仪
【pssilapoiy
一0.其它ax和仁的确定可参看高维新的解析理论讲义“1.
E林才

E暑岫′05一不00,

国认吴吴
E
【

于QE吴220标
E

Edt3河王丨6″已[最)z。(″)团

国
E(H-磊忡伽一0
Page-268
[浩河E

E
个别情形外,迄今未得到其表达式,对一些特殊的二次或三次
Hamilton向量扬在a一2,3或4的情况下来佶算BCm,a和ZCm,
s沥
使寺于m一一2的一舱情形,问题也远没有後底解欧

EP河唐吾时,卵对Bogdanov-Takens系统,
E05晓7切
E招
幻一3,PMardesiePW0证明,BC2,m一A一1}李宝毅\张芷芬PAm
E
2
的研究中少有的完整结果.本节主要介绍Petrov和Mardesic的感
E故东5河
辅角厝理来直接伙算B(2,m)(见[LbZ])-这种手法可用来研究其
E

Abei积分的构造

s怀一坂2渡5应招
刊霆点个数,必须先研究它的构造,由Gzeen公式,有

EiGQudz,
E命E
其中Q也是<和y的多项式.

医t达i
i

I「〈^)髓】一Z`(′I)'′′z′
雌式成立的一个充分条件是
林河一朐
其中麒和旦都是z和y的多项式,显焊“~-“是等价关系,令
Page-269
EE

由定理6.8=0和互=告都是m′20)和腑′2D的正炊奇点.作

E
dz00〉

十歼[弼十j′。(囊)0沥)

dzz】〔灞)d童】(Z〉

p标班7河〔蕊十j乙O)=0,0

其中姚0)=跖〔鲁],】(:)=/`〔鲁〕′E东

E
(6.zl)的芷则奇点,由定义s.9,佤20)和(6.21)鄙是Fuchs型方
E
下面我们来证明定理6.7.为此先利用定理6.10来讨论
E仪余t
n
E述0
0
E

5一命
榨(/〕一1)十z懈+蕊_洋+严十3s0,
玖

E述沥仁E或打吾,当/咔山即一D~″吾或
吊技认江
2

[巳国

E+3严十蕊一弼十鲈十豁_仇
i认
0动
Page-270
c

E沥0
E
Dn
而言是一交换群;男外还有
了0DCan二[FCEHJa0u
即

/仇〉IE国2多史
45E

其中/是R的多项式,即丿可在多项式环火上进行乘法运算,一
E人0技园江
0pa0胡江310月
【应2E育达02
0(
[标途
E李n
E训
E
E
E
定理6.1(LP1])考虑Bogdanov-Takens系统

吴
[

羞′=_一覃十zzE吴e庞运

E
E

E仪(02仪一人音,E
EI「伽五汊d艾j2t(技江03
Page-271
圆肉33

〕'dEg/】(凡)H〔音〕一
我们将证明作为习题留给读者.只要对做归纳法即可迦得

Picard-Fuchs方程

E沥

E人

玟
满足以下的Picard-Fuchs方程

4
0
哚6>d′′Ll]一【

目
【7]y0<凡<6'

E2
6凡

曹力

0

E圆225

在5巾我们已经对Bogdanov-Takens系统的男一种等价形式

Ea河仁02育仪3沥2

《5.27).由于Pieard-Fuchs方程在研究弼Hilbert夙16问题中的重
要性,此处我们将介绍古一种递推公式“0““由于

工矗
E玄十玄5′L,()〈′z<曹
E

E
E河浩刀

E
E鬟E沥算z弓E沥
利门分部积分得
B一In兰z″_3汊m十zdzEIr仇)藁"一1汊mdz一

EE
Page-272
常见的局部与非尾部分盆

0

「(^)E

一dz03
TQ5小

由此得到

E汀
Page-273
120E

炀吾,当E

国F553633才2标3EE″吾,当方一
E沥a

为求(在&一0附近的展式,先对方程(6.20求奇点一
0的狞定方程p(p一17一0的根,得p一1和p一0.因办一一
EAd
《6.24)中的C东0,伦有7(03一Cm(0十Coos(0)二Cabo式0,

EL伽应E张沥四志
【O70东3
[

地扩宁到G.
i

0前判定方稚p(p一1一p一友一26一0的框,得一2,一

E才5万E

55江沥诊仪0沥

0木L侧E扬
00一
【辽

其它有穷奇点,故工(可单值解析地扩充到G-
0仪

E沥^>告〉E
近的水乎边界,我们断言下列不等式成立;

u07医)

E

E林口
Page-274
E
E达

1[人人
″[^」畜〕国十〔言″+畜〕P一蒽允【
考虑(6,26)的等价系统
目

亚_、_z巳吴
36′〕十〔3九+6)P6尤'

巳网国

E

6.2
【E

E
系统(6.27)的相国如图2-8所示.其中点5:(&,P)二【告j]是

荣点,由方程
E林河玟李2十1

【
筠会定的双曲线是过点$的水平等倾线-利用7(h)和一(8)在h

E
Page-275
不6关于Pewow岱理的证明

4
E[E
团E

罡IE[I
「(^)E国′`(炫)2汊EX033y

刑言4dr[
E「【^)

E林氓招
达吴6d′′r(″)z帅

由此得到
[暑茹伍)E告/翼(颢)7

其中0<^<告.E

&技E
′l[′!_畜)′。′(庞〉_〔6互6〕′”(庞)十36z】(庞)'0河

E河河0]

i
医c水
u刑

E
我们需要引入下面的定理.为了遮免行文冗长,我们不引证明,有
E
Ea一
Ey吴
cc
E
i

匹
t沥i沥仁伟
Page-276
EE

一不陆I的渐近展式,不雅算出当RE*时
[E租命切
E朐
E达
[
E东
p
E一述
E
质(,不难用辐角原理来证明Ja(8)T,(不0,当EG古夺0.
E
E河河a兔璞彗荠0当^仨魍』>方
用反证法.设

m/侦0
E

二工

园

ReCo(胡33Im(二(77一Re(JCf“)3Imto(f「)3二0.
即向量Gmo(h),lmC3)积向最(ReJo(D,Re,(KD7在丁一不
E胡
(6.18)的解,如果它们在一点一f“时成比侧,则它们将在一切
沥

途

lm[艾丽)E

其中c为非零常数,即虚郯不为零回到原变量^=告_Z,此结论

5
Page-277
第二章常见的尸部与非局部分盆

Ed一一

d

韩茂安和朱德明在专著[HZ]中对非Hamiton系统的线性中
心也证得与定理6.3完全类似的结论.

Et东沥(圆e朋
E口沥园妙如
EJ2n
函数p(有以下展式

002人林运

沥
E
2

t

仿=呈俨…)L,摹>十蔫〕d薹,当窿】=0y
其中积分式前符叶的选择,取决于闭转TCD的定向和当增加时
E园

E王刑刀一
E
E

定理6.4是研究同宿分岔的主要依搬.对于画宿转而言,韩
E沥
E
指敏一分性和三分性理论,深化了不变流形的表示理论,用统一的
方法研究了各类系统(高维或低维,可积或不可积)的同宿扬、异
E.

E林一一
林招一许
Page-278
6关于Perov定理的证明
Eea0|

E江
E东(明河朋

E

域的任意紧致邻垣,存在古>0,当||一么时,中极限环个数
E`

Eoao国胡I才祖标2
[

[江告,巳
3乙″'″_>F(告)`E
由定理6.1,
E仪0703告'

E怀00技〔″矽

E〔音〕[L

由定理6.7,75(8D和于(&)可单值解析地扩充刻复城G,故
E
E

n

其中deg/嬴(″)<〔″〕[飚`deg/宙仇)<〔昔〕一1_尤E
0沥东育2

佶_^门…们十腊号/"M))园0
Page-279
E浩DE

日
夕D一〔告5凡〕0告'

其deg00一[252]-tdeg0一[孝-k
这个定理的直观含义是,如果当~0时,有?&十1或28个极
cn沥明
E
0

:
″(冀,汊)=誓十言`看=告一归o〈z〈告.

E「(告0不告〉对应系统(6.2),的闭执族,而
n
E

俨(告g2标述友标d林玟仪

其中0一1L,
E东述

EI……汊dzE河或兄人标逊标

EI天“)箕仪d迈河政儿人标述A

氓吊沥仪林肖5林孕河沥孕
E圆0
Page-280
EE芸e

由定理6.7的性质(23,FCD在G上解析.取区域口人G,其边界
E
E7水

E
一

[林河恤_告_=n0<r《‖,

E告十r<力<倒′
0
E述许主达刑

E一
E技沥标江日阮
的每一部分移劲时,相应的pCF(7)的联值,则

E告.0浩3
E技技
E21
raoe<Sml[e][扎-制+一
E园
E匹不

[咤吴
E
E许

E招3

E
由糯角原理,F(8)在卫中至多有a一24一1个零点,从而F(R)在
0王一河
Page-281
第二章常见的局部与非局部分盅

。1一一
一羞z(z_l)(宁十玄)妾dz_沥

已林圆42
[〔_E国
“口(亚十I
月naQ)337
江沥标二E
1[az一EE1

直接运算便可证明引理.
Epn

E/。(告2仪达

E/】(告E吴伟

[达人木

g人
E一仁
t
E
由等式(6.10)立刻得min(m,ma丿心
第二种情形:zm一m:.
0招州沥
2吴沥兄朋
由引理6.6,有
【儿林
【技丶技
玟

E熹俪肠E胡
国o(Z鬓翼,E
Page-282
E浩125

[技辽一刑

达一一一林敌
一
趴一吴一1个(2)类环,

由定理6.3,系统至多有力个(1)类环

国P
c

E
欢十工个(3)类环.此时由I的展式(6.13),可知不等式(6.28)
E

E着'

E吴
述

E

图皇…P们Ei
E

/y(/)=IE技0<z<玉.
EE

述
李
足等式(6.3中的一切多项式fo(8)和亿(4)都是可以实现的

[暑佯几U)E水江1口
E

E仪水仁00沥水

E
Page-283
日6关于Petow定理的证晓

E
E沥y吴林y挂
E林命怡妙bi李|

E达东

扩充到复域

E坂东颖刑伟
到复域

G=C\仙仨疃帆彦告〉,

并且具有以下性质。

E~九昙,E一屋,E林

《237o(07二0,卷(0)奂0z1C03一卯(03一0,00式03
E仪东亚江

E135
(3Imn0E林不蒽(沿上下边界无

定理6.8“考虑方程
E命A233402河a0
8一j是正则奇点的亮玖条件是灰一b分别是(和9的次
转

02振)EE
03阊Z12砂一″”)z/(′′)E沥沥5

E

对无穷远奇点作变挨文一二则有

d国E团
丽一_拐_,腩-″「十胭茁,
[途
Page-284
第二章常见的尾部与非尾部分岔

EE吊刑
0朐逊ZCo,
扬
EE
0
E
不0,当0人1人工心1[沥)
E李2林水E
和吉是乙(的简单零点,我们将此作为习题留给读者.
E

4
二2l扁`&震zz煌一1一oCl)、
将它和(6.31)右侧相比较,由引理6.6便知农~~0时,一0.同
E林

[仪
国t
),有

0

【

E7明u芸八万李一诊
E江江一4伟吊人厂
E江洁水才仪述余浩夕″`它在(0'告〉园河

E一刀

E

【沥木口沥林沥仪
圆t达a
Page-285
118第二章胺见的尿部与非局部分皿

其中m)`′(翼〕如果P[〕~″z,疃[半]~陟则由定理6.8,
一0是上面方程的正则奇炙,即无穷远奇点是方程(6.14的正则

定义6.9“方程(6.14)称为Fuchs型方程,如果它只有正则
E

E(东
技
E兰

p光[沥)

E

E吴沥异(达沥

E

E口河师e笑沥
E林达20
Rep乐Repsoas一1令加一f一f,期果办奂0,则功一1.仪
【pssilapoiy
一0.其它ax和仁的确定可参看高维新的解析理论讲义“1.
E林才

E暑岫′05一不00,

国认吴吴
E
【

于QE吴220标
E

Edt3河王丨6″已[最)z。(″)团

国
E(H-磊忡伽一0
Page-286
司题与怡考颜二E

0
E吊0沥切

a怦。E'苛〕Z】n/〕+E

Ec李仪8扬
Et沥

构造沟几它在仙告〉ad/(Z)在(0'音)
E圭标刑y回到原变量丸=告_^则

一一

0
E

医

2.1哲量场契点的双曲性与非迹化性有什么联系,有什么区别?光潘向
基场的双曲奇点或非追化奇点在吊量炼的Cr(r21)扰动下有何变化规律?

2.2证明当|心1时,例5.5中的wanderPol方程(5.14)的呐一闭轨
E1

2.3眙(2.27式中的啶量场X是n次多项式系统「试问定理2.7的绪论
[渡d

2.4考虑方程组

E江d江江李

E国技胡许
4一6时的情形来谋明条休(Hl)不能娆少,否则4,0)可既不是方程组的中
E

E

05E沥玟吴

E沥东一一伟
Page-287
EE

由定理6.8=0和互=告都是m′20)和腑′2D的正炊奇点.作

E
dz00〉

十歼[弼十j′。(囊)0沥)

dzz】〔灞)d童】(Z〉

p标班7河〔蕊十j乙O)=0,0

其中姚0)=跖〔鲁],】(:)=/`〔鲁〕′E东

E
(6.zl)的芷则奇点,由定义s.9,佤20)和(6.21)鄙是Fuchs型方
E
下面我们来证明定理6.7.为此先利用定理6.10来讨论
E仪余t
n
E述0
0
E

5一命
榨(/〕一1)十z懈+蕊_洋+严十3s0,
玖

E述沥仁E或打吾,当/咔山即一D~″吾或
吊技认江
2

[巳国

E+3严十蕊一弼十鲈十豁_仇
i认
0动
Page-288
二章帝见的局郭与非尿部分岑

s沥
|一aiz一aay十5
D朋
江E
[
2.7证明定理.15.
E芸沥
功,服有(一0试问显咏必有8「(旭不09
2.9试用定理6.3木求方程组
乏_一j=一z一′】+灼z+炮z仪十灼鹦狮
E李沥
ctutuota
d
2.11试证(6,30)式中的丿和双都是公式(6.29中丁(的简单
E伟江口

2.13对5.15),试用$6中的增推公式,导出a(8)和于(所源足
E

[v昙十z最'E
E
E向量场为)圣E沥的沥招沥北最汉E

0I「…】E兰吴浩沥a

出万DG心013?所满足的Pleard-Fuchs方程
2.16考E逊55朐5才

B
E了zZ园巳招水E招玟

其中为小参数.停设由Absl积分(5.12定义的一阶Mettikov函数不佩为
雳,东其问宿分岔的最高阶数为2,利用定理6.4水其一阶和二防同寇分岑
曲线在(平面上的方程(即对8的一阶近伴方程).
Page-289
120E

炀吾,当E

国F553633才2标3EE″吾,当方一
E沥a

为求(在&一0附近的展式,先对方程(6.20求奇点一
0的狞定方程p(p一17一0的根,得p一1和p一0.因办一一
EAd
《6.24)中的C东0,伦有7(03一Cm(0十Coos(0)二Cabo式0,

EL伽应E张沥四志
【O70东3
[

地扩宁到G.
i

0前判定方稚p(p一1一p一友一26一0的框,得一2,一

E才5万E

55江沥诊仪0沥

0木L侧E扬
00一
【辽

其它有穷奇点,故工(可单值解析地扩充到G-
0仪

E沥^>告〉E
近的水乎边界,我们断言下列不等式成立;

u07医)

E

E林口
Page-290
E
E达

1[人人
″[^」畜〕国十〔言″+畜〕P一蒽允【
考虑(6,26)的等价系统
目

亚_、_z巳吴
36′〕十〔3九+6)P6尤'

巳网国

E

6.2
【E

E
系统(6.27)的相国如图2-8所示.其中点5:(&,P)二【告j]是

荣点,由方程
E林河玟李2十1

【
筠会定的双曲线是过点$的水平等倾线-利用7(h)和一(8)在h

E
Page-291
\part{第三章“几类余维2的平面向量场分岔}


在本章中,我们将综合运用第二章所介绍的儿种典型的向量
E河
类余维2分岔.

E坂d仪[

CKo:坤=<fCao,

E逊
玟
玟
玟

^一′01〕^_[oO)
】一〔00'林
,

|
{
|
E技育李招
若平面呐重场在奇点处的线性部分矩阵有二重零特征根,并

E
E沥园
E沥
一一
E刀一

化条件下,扰动系统X具有双参数的普适开折、主耐参考文献
Page-292
EE

一不陆I的渐近展式,不雅算出当RE*时
[E租命切
E朐
E达
[
E东
p
E一述
E
质(,不难用辐角原理来证明Ja(8)T,(不0,当EG古夺0.
E
E河河a兔璞彗荠0当^仨魍』>方
用反证法.设

m/侦0
E

二工

园

ReCo(胡33Im(二(77一Re(JCf“)3Imto(f「)3二0.
即向量Gmo(h),lmC3)积向最(ReJo(D,Re,(KD7在丁一不
E胡
(6.18)的解,如果它们在一点一f“时成比侧,则它们将在一切
沥

途

lm[艾丽)E

其中c为非零常数,即虚郯不为零回到原变量^=告_Z,此结论

5
Page-293
130第么章儿类仁维2的平国向塔堤分岑

为:e=1的余维2分岔见[Bol,2]和[T],余维3.4的讨论分别见
LDRSL,2]和[LR1];4一2和8一3的余维分岔见[Ho],e一2的
E如江E述才
E
有情形的详细介绍.

$1二重零特征根:Bogdanov-Takens系统
E描
i
gn沥
[

Qz十bry
当史一0时,由第一章定理5.13,系统(1,2)的任一非退化开折可
转化为
dz'
05

E玑

′羞昔E河玟2招沥2庞a

0
其中Q,88EC“,Q(0,0)二土1一sgn(a5),xER,m之2.为确定
起见,取Q(0,0)一1Q(0,0)一一1的情况可类似讨论.

分岑图,轨线的拓扑分类

t
E技

0河t
Page-294
园国

E一
E
0簧赚E标〉0}`

其中SN*,H,HL分别为歌结点分岔曲线,Hopf分岔曲线和同宿
E

E
Eouuaopusetsst河b育
为了证明定理1.1,注意当h之0时,(1.3)在原点附近无奇
E东园沥
p

人人
E不许e沥沥3沥人医莲3扬
Page-295
6关于Perov定理的证明
Eea0|

E江
E东(明河朋

E

域的任意紧致邻垣,存在古>0,当||一么时,中极限环个数
E`

Eoao国胡I才祖标2
[

[江告,巳
3乙″'″_>F(告)`E
由定理6.1,
E仪0703告'

E怀00技〔″矽

E〔音〕[L

由定理6.7,75(8D和于(&)可单值解析地扩充刻复城G,故
E
E

n

其中deg/嬴(″)<〔″〕[飚`deg/宙仇)<〔昔〕一1_尤E
0沥东育2

佶_^门…们十腊号/"M))园0
Page-296
e

一
E沥刑

园沥标人阮

E
我们可以把(1.5看成(1.5)。的扰动系统,后者为Hamilon系
统,它有鞍点4(1,0)的同宿辅,以及该同宿技所园的以点B(一1,
0)为中心的周期环域(见第二章例5.6及图2-6),周期环域中的闭
E

E{(z,汊)_H(渥'仅)=″,_鲁<″<鲁,0
E
|…z唧)=誓十z_誓.医林办
当凡令_吾十0时'厂^缩向奇点B【当凡令音~6时,厂″趋于同

E

注意对任意的8,(1.5)s都以4,B为奇点,且A为鞍点,B为
指标十1的奇点.图此,若(1.5)3存在闭辅,它必定与线段工一
0男一方面,由于「与工的交
E人逊国

0

E阮圆
正向及负向延续分别与z辐(第一次)交于点Q,与@i.记7(&,8,5
E2林

E东林
]

F仇峨痊)…IE人
E
E
Page-297
EE芸e

由定理6.7的性质(23,FCD在G上解析.取区域口人G,其边界
E
E7水

E
一

[林河恤_告_=n0<r《‖,

E告十r<力<倒′
0
E述许主达刑

E一
E技沥标江日阮
的每一部分移劲时,相应的pCF(7)的联值,则

E告.0浩3
E技技
E21
raoe<Sml[e][扎-制+一
E园
E匹不

[咤吴
E
E许

E招3

E
由糯角原理,F(8)在卫中至多有a一24一1个零点,从而F(R)在
0王一河
Page-298
渡

一

E
ue一切

E一
p

林林水

(】.5)′
E吊3河沥沥沥沥9
E
2
EI钨`捡E秉沥9
沥

0
E圆|

我们可以用取极限的方法把F(f,3,5)的定义域扩大到集合
i{(^,贪,玄)_、鲁<″<熹瓜<艘<鳃瓢<芗<鲢卜

E沥沥(d李不
E林匹中吊林训
E

【{(′z,贯,玄)_`鲁<凇<暑』<朦<虬瓢<鳖<鲢}
Page-299
E浩125

[技辽一刑

达一一一林敌
一
趴一吴一1个(2)类环,

由定理6.3,系统至多有力个(1)类环

国P
c

E
欢十工个(3)类环.此时由I的展式(6.13),可知不等式(6.28)
E

E着'

E吴
述

E

图皇…P们Ei
E

/y(/)=IE技0<z<玉.
EE

述
李
足等式(6.3中的一切多项式fo(8)和亿(4)都是可以实现的

[暑佯几U)E水江1口
E

E仪水仁00沥水

E
Page-300
EE浩usssgenlels

E′
E英关莲
E沥医水e

点附近解对参数的光滑依赖性定埋胭在″=_暑关于飙苔Eouoa

E昙
E达

从上面的两个引理可得如下推论.
引理1.4系统(1.5)a(8一0)存在鞍点A的同宿技,当昆仅

当F[兽默店】E
I李

E=I「吊2仪(河仪
D1

E
E沥月3或3yy振一)志1浩河服
E厉
E不2沥2
证明“由引理1.4可得,系统(1.5)s存在同宿转的充要条件

E
人刃孙一刃)-ooa菊招a吴-口分>eam
E育江

E亭并且F〔鲁默忑】稀)】E园

日E政沥吴元沥/发26振力朐
E一河
i
E林李n兄秉7河437
Page-301
第二章常见的尾部与非尾部分岔

EE吊刑
0朐逊ZCo,
扬
EE
0
E
不0,当0人1人工心1[沥)
E李2林水E
和吉是乙(的简单零点,我们将此作为习题留给读者.
E

4
二2l扁`&震zz煌一1一oCl)、
将它和(6.31)右侧相比较,由引理6.6便知农~~0时,一0.同
E林

[仪
国t
),有

0

【

E7明u芸八万李一诊
E江江一4伟吊人厂
E江洁水才仪述余浩夕″`它在(0'告〉园河

E一刀

E

【沥木口沥林沥仪
圆t达a
Page-302
东1二重军案征林,Bogdanor-Takene系给

E一

E
们需要证明同宿分岔对参数3的一致伯(见第二章$4,即证明
存在8丿0和丿丿0,使得对任意和予只要0一一卷,匹一
E

0东el
E

《27(8)在扰动下其稳定流形与不稳定流形具有固定的相对
E

E

E7刑

许

技河沥招玟沥班2
0
E。
国(汊(1不
由于对任意的3,系统的鞅点均在(z,9)一(1,0),而且系统在读点
的发散量为
E氙00

E

67|一多,就有了(8)乙0.结合第二章定理43和宏理4.4可

E刑
E

;力则与第二章(419)式兵似的线“当

|旦(rOCI8|十匹一口(85447

E

M
Page-303
司题与怡考颜二E

0
E吊0沥切

a怦。E'苛〕Z】n/〕+E

Ec李仪8扬
Et沥

构造沟几它在仙告〉ad/(Z)在(0'音)
E圭标刑y回到原变量丸=告_^则

一一

0
E

医

2.1哲量场契点的双曲性与非迹化性有什么联系,有什么区别?光潘向
基场的双曲奇点或非追化奇点在吊量炼的Cr(r21)扰动下有何变化规律?

2.2证明当|心1时,例5.5中的wanderPol方程(5.14)的呐一闭轨
E1

2.3眙(2.27式中的啶量场X是n次多项式系统「试问定理2.7的绪论
[渡d

2.4考虑方程组

E江d江江李

E国技胡许
4一6时的情形来谋明条休(Hl)不能娆少,否则4,0)可既不是方程组的中
E

E

05E沥玟吴

E沥东一一伟
Page-304
136E浩

[理0一朐
(0,加,使只要0一一8,区一咤(8)|一加,就有
E刑罡
rE

匹
所以结论(2)也成立.引理1.6得证,〖
E

0
一万人人品(80,0一8一口时,(1,5)8在旦点的I邻域内恰有
一个闭轨,它是不稳定的极限环;而当吊(3)=<5一多(8)十办0一

E河i
E国沥437河

,目
(沥园穴河友人0

i标
E告殴砧」园江关

E量〔8E

E江个
(8)时,e(8,f(8))一0,8C8,(8))不0.即第二章$3中的条件

(Hi)成立.计算表明,

E
命
[
E
E国林胡
[木
Page-305
二章帝见的局郭与非尿部分岑

s沥
|一aiz一aay十5
D朋
江E
[
2.7证明定理.15.
E芸沥
功,服有(一0试问显咏必有8「(旭不09
2.9试用定理6.3木求方程组
乏_一j=一z一′】+灼z+炮z仪十灼鹦狮
E李沥
ctutuota
d
2.11试证(6,30)式中的丿和双都是公式(6.29中丁(的简单
E伟江口

2.13对5.15),试用$6中的增推公式,导出a(8)和于(所源足
E

[v昙十z最'E
E
E向量场为)圣E沥的沥招沥北最汉E

0I「…】E兰吴浩沥a

出万DG心013?所满足的Pleard-Fuchs方程
2.16考E逊55朐5才

B
E了zZ园巳招水E招玟

其中为小参数.停设由Absl积分(5.12定义的一阶Mettikov函数不佩为
雳,东其问宿分岔的最高阶数为2,利用定理6.4水其一阶和二防同寇分岑
曲线在(平面上的方程(即对8的一阶近伴方程).
Page-306
n

a
E一
E
东,当氧一缸一咤时,一和一肖一育一育一星,其中肖一
Et
n

簧`(凡_,0,茗'〉En

这里要用到y(8“)的有界性.事实上,由第二章(5.23)和本节
E

E朋
T7Ch「)显然是有界的.

t取
0刑东
E

[
8国。
E
一
s
d一
李剧
E关
一

c空述
E怀标
Page-307
第三章“几类余维2的平面向量场分岔

在本章中,我们将综合运用第二章所介绍的儿种典型的向量
E河
类余维2分岔.

E坂d仪[

CKo:坤=<fCao,

E逊
玟
玟
玟

^一′01〕^_[oO)
】一〔00'林
,

|
{
|
E技育李招
若平面呐重场在奇点处的线性部分矩阵有二重零特征根,并

E
E沥园
E沥
一一
E刀一

化条件下,扰动系统X具有双参数的普适开折、主耐参考文献
Page-308
138E

E
Eotcupt蓁伍战…禹)伍E3义

E李达标标水沥砂发东取水述

闭轨是不穗定的极限环,〖

EE沥颂刑河

E玟阮朐沥
E&且介于曲线与曲线HL之间时,系统(1.3)有唯一闭转,它
是不稳定的极限环.当(趋于X时,此闭执缩向奇点B当
0
E

5芸技江

人育
E

转
i胡

为了上系统(1.5)3返回到系统(1.3),由变换(1.4知
E
Eapogzeeaueosy生
Page-309
130第么章儿类仁维2的平国向塔堤分岑

为:e=1的余维2分岔见[Bol,2]和[T],余维3.4的讨论分别见
LDRSL,2]和[LR1];4一2和8一3的余维分岔见[Ho],e一2的
E如江E述才
E
有情形的详细介绍.

$1二重零特征根:Bogdanov-Takens系统
E描
i
gn沥
[

Qz十bry
当史一0时,由第一章定理5.13,系统(1,2)的任一非退化开折可
转化为
dz'
05

E玑

′羞昔E河玟2招沥2庞a

0
其中Q,88EC“,Q(0,0)二土1一sgn(a5),xER,m之2.为确定
起见,取Q(0,0)一1Q(0,0)一一1的情况可类似讨论.

分岑图,轨线的拓扑分类

t
E技

0河t
Page-310
园国

E一
E
0簧赚E标〉0}`

其中SN*,H,HL分别为歌结点分岔曲线,Hopf分岔曲线和同宿
E

E
Eouuaopusetsst河b育
为了证明定理1.1,注意当h之0时,(1.3)在原点附近无奇
E东园沥
p

人人
E不许e沥沥3沥人医莲3扬
Page-311
e

E沥

E木
0

E主
E命吴i

日
uCS耿cn

E

0(″【熹).【
Essau
6sEuzstuaals5标rip
untugod
0胡

【t口

a
E

【
沥

则当y一0,|z|,|y|造当小时,
Page-312
E招d

标国

E技邹
[

水

[
E一
E
c
一一
ult仪李沥

图3-5
E吊sst述
E志江t一
Ei
的情形(图3-5(b)),【
E沥林园
论,得到定理1.!的结论

E

E园形扬
E
岔幽线一与HL的表达式的高阶项O(xi中,这种差异为我们证
E
Page-313
e

一
E沥刑

园沥标人阮

E
我们可以把(1.5看成(1.5)。的扰动系统,后者为Hamilon系
统,它有鞍点4(1,0)的同宿辅,以及该同宿技所园的以点B(一1,
0)为中心的周期环域(见第二章例5.6及图2-6),周期环域中的闭
E

E{(z,汊)_H(渥'仅)=″,_鲁<″<鲁,0
E
|…z唧)=誓十z_誓.医林办
当凡令_吾十0时'厂^缩向奇点B【当凡令音~6时,厂″趋于同

E

注意对任意的8,(1.5)s都以4,B为奇点,且A为鞍点,B为
指标十1的奇点.图此,若(1.5)3存在闭辅,它必定与线段工一
0男一方面,由于「与工的交
E人逊国

0

E阮圆
正向及负向延续分别与z辐(第一次)交于点Q,与@i.记7(&,8,5
E2林

E东林
]

F仇峨痊)…IE人
E
E
Page-314
渡

一

E
ue一切

E一
p

林林水

(】.5)′
E吊3河沥沥沥沥9
E
2
EI钨`捡E秉沥9
沥

0
E圆|

我们可以用取极限的方法把F(f,3,5)的定义域扩大到集合
i{(^,贪,玄)_、鲁<″<熹瓜<艘<鳃瓢<芗<鲢卜

E沥沥(d李不
E林匹中吊林训
E

【{(′z,贯,玄)_`鲁<凇<暑』<朦<虬瓢<鳖<鲢}
Page-315
n

E
2宇辽
标(z,o?下,它们有表达式y一.(z),i=1,2,3,并漪足条件

E23002-E标技肖2
2

d
E行芸洽y1
E林
怀
E述荣挂逊口一工一一
c
改
E扬E
[沥E刑
E朐胺
E
E沥1
【26iatiJ2埕伟沥沥技53圭泓人1
Gronm)一a(g(ruyGm))
是对乙的恒等式,一1,2,3.把上式对z求导,得到
E
再对丿求导一次,得到
E述仪
E达
0
E医途0E
EE林3育
E沥李怡7沥3
Page-316
EE浩usssgenlels

E′
E英关莲
E沥医水e

点附近解对参数的光滑依赖性定埋胭在″=_暑关于飙苔Eouoa

E昙
E达

从上面的两个引理可得如下推论.
引理1.4系统(1.5)a(8一0)存在鞍点A的同宿技,当昆仅

当F[兽默店】E
I李

E=I「吊2仪(河仪
D1

E
E沥月3或3yy振一)志1浩河服
E厉
E不2沥2
证明“由引理1.4可得,系统(1.5)s存在同宿转的充要条件

E
人刃孙一刃)-ooa菊招a吴-口分>eam
E育江

E亭并且F〔鲁默忑】稀)】E园

日E政沥吴元沥/发26振力朐
E一河
i
E林李n兄秉7河437
Page-317
Ep

78吴国酸伟(E国胡
00一273万(g′′(O,0))zy【′〈o)'

E63十7737

E标2如认F
不y2林河育技2余2
Ec江2有沥朐孕77
En

22沥许招(歹_〉,受_(z)E0'受z(z)国'】LTz蔓(z),重〉a(」T)E
E李人技水一根
T

s
E刹8ECr,8C0)一1由于

F

E江仁不人

c
v325
吊区墙DL一{(z,o|z公0.3之乙)U(Czr,9)1z二01,D:一[Cr,
E沥仪林怀仪一
0
E2
水一
t

Edtos0

0

uatsni

设和9一en刹用了EC?和0)一0可知,(1.18)在UfCD,U
E

E林圆

系统族,存在(w)平面上保持原点的C变换,它把其中一个系
Page-318
东1二重军案征林,Bogdanor-Takene系给

E一

E
们需要证明同宿分岔对参数3的一致伯(见第二章$4,即证明
存在8丿0和丿丿0,使得对任意和予只要0一一卷,匹一
E

0东el
E

《27(8)在扰动下其稳定流形与不稳定流形具有固定的相对
E

E

E7刑

许

技河沥招玟沥班2
0
E。
国(汊(1不
由于对任意的3,系统的鞅点均在(z,9)一(1,0),而且系统在读点
的发散量为
E氙00

E

67|一多,就有了(8)乙0.结合第二章定理43和宏理4.4可

E刑
E

;力则与第二章(419)式兵似的线“当

|旦(rOCI8|十匹一口(85447

E

M
Page-319
81二重霖特征松,BogdenowTakens系统

n
E国育c

不英
E

E林二

由引理1.12的统论(2)立即推得本引理成立,【
引理1.13具有(1.3)形式相应于不间Q,申的两个系统族
E
E国圆u
[
F扬

'盖园匹林或坂玟2口朋班一

巳目
E
E
E沥园李林水
0河

界
7

一一

F二3

蒜E怀x坤技
这里2一2C1裕示血一山(4D,吊一加(一p一.

Esusyscpogetukit不国
E胡一
Page-320
136E浩

[理0一朐
(0,加,使只要0一一8,区一咤(8)|一加,就有
E刑罡
rE

匹
所以结论(2)也成立.引理1.6得证,〖
E

0
一万人人品(80,0一8一口时,(1,5)8在旦点的I邻域内恰有
一个闭轨,它是不稳定的极限环;而当吊(3)=<5一多(8)十办0一

E河i
E国沥437河

,目
(沥园穴河友人0

i标
E告殴砧」园江关

E量〔8E

E江个
(8)时,e(8,f(8))一0,8C8,(8))不0.即第二章$3中的条件

(Hi)成立.计算表明,

E
命
[
E
E国林胡
[木
Page-321
EE英

E技
点相应邻域乙(9中极限集间的同胚(,然后把此同胚扩展到
[沥nt沥认不
E

E林沥一圆沥5287

E

FF扬

E。
豇=州十″z汊十z2

是奇异向量李(1.2)(a5>0的一个普适开折,
E吴
变换转化为与下列系统等价的系绕(措在原点的邻域内》

[二
E
「Q

鲁萱E应玟225标河t林述

Ed
2

E
沥
E

E

养中(h,是猎立的参数.如果我们把(1.19)看成是含m十2个
E不不园E东
E沥
unett8u吉
王

巳国
E
目莲2
沥命
Page-322
n

a
E一
E
东,当氧一缸一咤时,一和一肖一育一育一星,其中肖一
Et
n

簧`(凡_,0,茗'〉En

这里要用到y(8“)的有界性.事实上,由第二章(5.23)和本节
E

E朋
T7Ch「)显然是有界的.

t取
0刑东
E

[
8国。
E
一
s
d一
李剧
E关
一

c空述
E怀标
Page-323
E河ssE

拓扑等价,这里把(L,22》看成含参数的系统,特别地,在(1.21)
医起余i园
Di
0怀i技
E伟l伟
统(1.2)的一个普适开折,【

$2“二重零特征根:1:2共振问题

E

玟
黯=仍蕃=僻…十奴钺[
为了陈述俞洁,我们在这里只考虔三次正规形而去掉了高阶项.事
玟
质的影响(可参考12.
医加仪

鲁言=鹏盖=士崴一感扔[

E沥小
E胡

类似于第一章定理5.13,有下面的结果

E东园i东
D胡
开折化成与下列向量场等价的形式

[
F河河刑

刑
D
Page-324
138E

E
Eotcupt蓁伍战…禹)伍E3义

E李达标标水沥砂发东取水述

闭轨是不穗定的极限环,〖

EE沥颂刑河

E玟阮朐沥
E&且介于曲线与曲线HL之间时,系统(1.3)有唯一闭转,它
是不稳定的极限环.当(趋于X时,此闭执缩向奇点B当
0
E

5芸技江

人育
E

转
i胡

为了上系统(1.5)3返回到系统(1.3),由变换(1.4知
E
Eapogzeeaueosy生
Page-325
e

E沥

E木
0

E主
E命吴i

日
uCS耿cn

E

0(″【熹).【
Essau
6sEuzstuaals5标rip
untugod
0胡

【t口

a
E

【
沥

则当y一0,|z|,|y|造当小时,
Page-326
儿美余纵2的乎面问最扬分岔-

Edn圆
训

空间的变换,把(2.3)*化为

万二
d′_叟'

窑E河吴丿北学沥u胡玟仪A技2

[
E林cc道河育[李02才
E沥
Eooo述东

(莹=y'澄=橇z+砀汊士遨_碳拂[

E
E东技莲

E

0沥水
E2口浩沥一不河
E汀行2沥技沥

E

02招

Eel训浩达i
Pnpois12晚沥许2林2
林沥
E

u

E
其中c<0.752,《2.5)-的转线拓扑分类见图3-7.
Page-327
E圆

对(2.5)+的讨论与1很相似,所以下面仅对(2.5的情形
Page-328
E招d

标国

E技邹
[

水

[
E一
E
c
一一
ult仪李沥

图3-5
E吊sst述
E志江t一
Ei
的情形(图3-5(b)),【
E沥林园
论,得到定理1.!的结论

E

E园形扬
E
岔幽线一与HL的表达式的高阶项O(xi中,这种差异为我们证
E
Page-329
E蒙

E河浩明仪

首先,(2.5)-的奇点满尼y一0及az一a一0.与第一章例
玟

其次考察e一0的情形这时(z,y)二(0,0)是(2.5)-的唯
E

E
[小
医余江

程应用第二章公式(3.3〉可得,Re(q)=_告<o.因此,由第二章

定理8.1知,在曲线H上发生Hopf分岔.
E

人一
E河加0
蕃=撕窑=z〔蹴十厥首_麟澈′
E月

当0一8人1时的扰动系统.系统(2.8)有首次积分
H位靓)=誓_吾十誓=允【

Ei

E沥的两条对称闭轨,当颢峥_盖时,它们分别缩

向这两个奇点f当~0*时,它们扩大而形成鞋点(0,0)的对称双

[一

见图3-8.若令工二DU,其中
E
Page-330
E

E
E沥不

【木
园t0园
E

E一园人
0
n招
M
d
1.2一1.4类似可得

E标
E

巳张一
EL你机n醛吴E

而且7成为(2.7)的同宿(双同宿)扔,当且仅当FCO~,8,6一0
0规
E江

EL0
D

其中缥仇)=Lydz,&一0,2.与第二章中(5.18)一(5.20)的
8
Page-331
n

E
2宇辽
标(z,o?下,它们有表达式y一.(z),i=1,2,3,并漪足条件

E23002-E标技肖2
2

d
E行芸洽y1
E林
怀
E述荣挂逊口一工一一
c
改
E扬E
[沥E刑
E朐胺
E
E沥1
【26iatiJ2埕伟沥沥技53圭泓人1
Gronm)一a(g(ruyGm))
是对乙的恒等式,一1,2,3.把上式对z求导,得到
E
再对丿求导一次,得到
E述仪
E达
0
E医途0E
EE林3育
E沥李怡7沥3
Page-332
E

结果类似「容易得到
E李标责E

-目-明a鲸

R
E
E团
1E[许

1当h一二

E

招
E坂仪(d
引理2.5“函数P有如下性质

0乏'E
M

03北a噩<丸<汨E沥扬
E2沥

[音'尸′(′【′)沥匹0[

医芸沥育王
Page-333
E招E

E

E一
从(2,12可知,研究F(t,0,50的零点可用直线5一常数咤去戬
t2

系统的环绕一个奇点的“大极限环”=当告〈魏<1时,截得两个交

E_瓮〈彻<E

的″小极限环”,后者相应于M_言时E

翻双同宿轨,且外侧仍有一个“大极限环墉而当″<巍〈言舶]

fQh“2)时,两个交点的槲垄标,fs心0,系统出现两个“大极限
E
Page-334
Ep

78吴国酸伟(E国胡
00一273万(g′′(O,0))zy【′〈o)'

E63十7737

E标2如认F
不y2林河育技2余2
Ec江2有沥朐孕77
En

22沥许招(歹_〉,受_(z)E0'受z(z)国'】LTz蔓(z),重〉a(」T)E
E李人技水一根
T

s
E刹8ECr,8C0)一1由于

F

E江仁不人

c
v325
吊区墙DL一{(z,o|z公0.3之乙)U(Czr,9)1z二01,D:一[Cr,
E沥仪林怀仪一
0
E2
水一
t

Edtos0

0

uatsni

设和9一en刹用了EC?和0)一0可知,(1.18)在UfCD,U
E

E林圆

系统族,存在(w)平面上保持原点的C变换,它把其中一个系
Page-335
152E

江
D
数学论证要利用(对参数一致的)Hopf分岔定理,同宿分岔定理,
E
对于具有“8字型“双同宿轨的平面系统(见图3-8),在[Lw]
a河

E技河5浩北王口

由第一章习题1.5,以(0,0)为奇点并具有二重零特征根,旋
转誓鼬>盼不变的向量场具有如下的复正规形

羞E2河

E

E扬
E

羞E述江沥

E述u辽述5
Ee沥胡一沥
E,

盖E

塞沥河达人辽达

E
星然,当a一0,6失0时,发生Hopt分岔.若a心0,则由3.3的
第一个方程可知,在r一0的小邻域内的所有轨线当5十co均苔
医
Page-336
81二重霖特征松,BogdenowTakens系统

n
E国育c

不英
E

E林二

由引理1.12的统论(2)立即推得本引理成立,【
引理1.13具有(1.3)形式相应于不间Q,申的两个系统族
E
E国圆u
[
F扬

'盖园匹林或坂玟2口朋班一

巳目
E
E
E沥园李林水
0河

界
7

一一

F二3

蒜E怀x坤技
这里2一2C1裕示血一山(4D,吊一加(一p一.

Esusyscpogetukit不国
E胡一
Page-337
0

罄】=彤,Ez=瞅,r=茆,z=袁'

E咤s李中浩技

城pd-网十(evp,=R(p,b,8,

0技

E木

其中
E林
E达
E6323

羞=爪1-妨),塞=宣十柘障《3.6)

此系统有唯一的不变环不二{Cp,b)1p一1},它是股引的.当幼十
0dt刑朗庞
4
匿>o(相应地`由(3.4),存在勇>o)使得对每】个贪E伟
E加红5s才
n技
E东
E月沥
a标国江政不月
E不江u林(汀河
t沥口53木
E
E林江林一
E於八园中p林技3技
E林国胡
沥
Page-338
EE英

E技
点相应邻域乙(9中极限集间的同胚(,然后把此同胚扩展到
[沥nt沥认不
E

E林沥一圆沥5287

E

FF扬

E。
豇=州十″z汊十z2

是奇异向量李(1.2)(a5>0的一个普适开折,
E吴
变换转化为与下列系统等价的系绕(措在原点的邻域内》

[二
E
「Q

鲁萱E应玟225标河t林述

Ed
2

E
沥
E

E

养中(h,是猎立的参数.如果我们把(1.19)看成是含m十2个
E不不园E东
E沥
unett8u吉
王

巳国
E
目莲2
沥命
Page-339
sg

E
RCp,b,8)一8(p,0,8)一det(
E
det〔a(R,6)仪

E
3(p,百

2KP:外
E
DKp,b,8)一(3fF一Dcos(90)十25zsin(ab)十O(8)。

医述
图此,(3.7)可改写成
E《C3.9)
E
巳仪2李丿
E
E
C
E
由(3.8)知DD|aruco一0在伟附近对9有2个根.因此,集合
EEa春许;

,E
Page-340
E河ssE

拓扑等价,这里把(L,22》看成含参数的系统,特别地,在(1.21)
医起余i园
Di
0怀i技
E伟l伟
统(1.2)的一个普适开折,【

$2“二重零特征根:1:2共振问题

E

玟
黯=仍蕃=僻…十奴钺[
为了陈述俞洁,我们在这里只考虔三次正规形而去掉了高阶项.事
玟
质的影响(可参考12.
医加仪

鲁言=鹏盖=士崴一感扔[

E沥小
E胡

类似于第一章定理5.13,有下面的结果

E东园i东
D胡
开折化成与下列向量场等价的形式

[
F河河刑

刑
D
Page-341
E园

E吊技河满足以卞条件'
林2
5
e
【述胡吴人0

E万木晓发根

0,得刹
E
[逊
整理得
【f尔J河Cu河
E
E标O

E颂工0沥河扬用一一一二().

最后,考虑集合‖仍矾龋谬汁R=0,D=0,6=0),注意

tneC
Ed邦育丿
c吴0育
由对称性知,肖+:(一M,(8,下匹佼计M:(83一M,(8).在等
式(3.12)中取丿一0,1得
E述

E达仪

利用(3.12)和P一1十0(3)整珀上式可得
国cc育园0
E
[0
Page-342
儿美余纵2的乎面问最扬分岔-

Edn圆
训

空间的变换,把(2.3)*化为

万二
d′_叟'

窑E河吴丿北学沥u胡玟仪A技2

[
E林cc道河育[李02才
E沥
Eooo述东

(莹=y'澄=橇z+砀汊士遨_碳拂[

E
E东技莲

E

0沥水
E2口浩沥一不河
E汀行2沥技沥

E

02招

Eel训浩达i
Pnpois12晚沥许2林2
林沥
E

u

E
其中c<0.752,《2.5)-的转线拓扑分类见图3-7.
Page-343
E圆

对(2.5)+的讨论与1很相似,所以下面仅对(2.5的情形
Page-344
156E

总结上面的讨论可得
引理3.1([TJ〉在(p,0,,8)空间中印的邻域内,集合
E

E沥2河吊
(2K(8,8}(j一1,2)组成,漪尽(8)一Mu(3)一$58《十
【江

二林
下列曾线组成:,

人2才八沥命江大2

s沥c沥
E

顺唐n
不p
[沥2

Ea20
Page-345
E蒙

E河浩明仪

首先,(2.5)-的奇点满尼y一0及az一a一0.与第一章例
玟

其次考察e一0的情形这时(z,y)二(0,0)是(2.5)-的唯
E

E
[小
医余江

程应用第二章公式(3.3〉可得,Re(q)=_告<o.因此,由第二章

定理8.1知,在曲线H上发生Hopf分岔.
E

人一
E河加0
蕃=撕窑=z〔蹴十厥首_麟澈′
E月

当0一8人1时的扰动系统.系统(2.8)有首次积分
H位靓)=誓_吾十誓=允【

Ei

E沥的两条对称闭轨,当颢峥_盖时,它们分别缩

向这两个奇点f当~0*时,它们扩大而形成鞋点(0,0)的对称双

[一

见图3-8.若令工二DU,其中
E
Page-346
沥人157

u一才
E标
宁【

办(8一MMo(3)一8一十O(8e3),8一0怡式0.
E园
E
曲缇

0
[
E述芸切月
肖(3)一口(二酥CM(89一Mo(8))一688十O(8p-1,
其中8央0.定理3.2得证,【

题与怡考题三

31在引理1.6和引理1.7中,为什么要分别利用对参数一致的同宿分
岷定理和对参数一致的Hopt分岔定理?这对最终得到定理1.1有什么作用?
3.2证明引理2
$.3证明引理2.5.,
3.4对方程(2.5)7讨论它的分岔现象,并证明分岑图3-6的正确性
E林胡
E技d

E
E一

c
E昔卜《l时,(2^7)在原点的发散量为弼奂0.)
Page-347
E

E
E沥不

【木
园t0园
E

E一园人
0
n招
M
d
1.2一1.4类似可得

E标
E

巳张一
EL你机n醛吴E

而且7成为(2.7)的同宿(双同宿)扔,当且仅当FCO~,8,6一0
0规
E江

EL0
D

其中缥仇)=Lydz,&一0,2.与第二章中(5.18)一(5.20)的
8
Page-348
医沥l李

2
五章中是我们研究空间R中歌点同宿分岔的基础,同E
也有其自身的重要价值.在81中我们证明一个双曲不动点定理.
在2中引进符号助力学的基本概念.在83中给出马蹄存在定理
E
引理.在85中作为$2一$4中诺结果的一个应用,我们将给出E
中Birkhoff-Smale定理的证明.

i园d
Wiggins的书[Wig]中都能挂到,但此处所有定理的证明郭是独立
给出的.我们力图把儿何直观与数学的严密性统一起来,并给予
读者一套易于掌握的方法,用以解决高维空间中其它类似的闰题.

$1双曲不动点定理

定理的陈述

育g
L沥c东技才
E江
Et沥东一一
图象.当一co<a<8<<十co时,称点(af(a)),(8,/(8)为该曲线
E贺伟江
为C的y的函数=二g(y),yECe,切的图象.特别地,当a一一co,
Page-349
E

结果类似「容易得到
E李标责E

-目-明a鲸

R
E
E团
1E[许

1当h一二

E

招
E坂仪(d
引理2.5“函数P有如下性质

0乏'E
M

03北a噩<丸<汨E沥扬
E2沥

[音'尸′(′【′)沥匹0[

医芸沥育王
Page-350
E招E

E

E一
从(2,12可知,研究F(t,0,50的零点可用直线5一常数咤去戬
t2

系统的环绕一个奇点的“大极限环”=当告〈魏<1时,截得两个交

E_瓮〈彻<E

的″小极限环”,后者相应于M_言时E

翻双同宿轨,且外侧仍有一个“大极限环墉而当″<巍〈言舶]

fQh“2)时,两个交点的槲垄标,fs心0,系统出现两个“大极限
E
Page-351
E驱ri

a水纳ta

东
E
E

E

E医一二技
技江
称为边界的水平部分,记作#D:那组A垂直对边称为边界的坩
E

E河05
空间.令

277木人伟
沥07水7

E

定义1.4令DCRIXR`图
E仪a沥n河
n沥林
Eupnicsosisi沥河

3许吴
YED

[刑
E

医

兰7
E仁9

[兄一万3

E沥技
李
Page-352

\part{E第四章双曲不动点及马路存圭定理}

Es
E亚n
Ei匕
E
E怡
202
则了在马中有唯一双曲不动点

上述定理的儿何直观见囹4-2(a).

E
E

E0
E[吴吴17

E罚
E一

e
Page-353
152E

江
D
数学论证要利用(对参数一致的)Hopf分岔定理,同宿分岔定理,
E
对于具有“8字型“双同宿轨的平面系统(见图3-8),在[Lw]
a河

E技河5浩北王口

由第一章习题1.5,以(0,0)为奇点并具有二重零特征根,旋
转誓鼬>盼不变的向量场具有如下的复正规形

羞E2河

E

E扬
E

羞E述江沥

E述u辽述5
Ee沥胡一沥
E,

盖E

塞沥河达人辽达

E
星然,当a一0,6失0时,发生Hopt分岔.若a心0,则由3.3的
第一个方程可知,在r一0的小邻域内的所有轨线当5十co均苔
医
Page-354
E02

第二步证明y一门Di和一一[D-;分别是端点属于3eD和
E标

第三步证明一V是了的一个双曲不动点
E

儿个引理

引理1.6平面R!XRI上一条心水平曳线与一条A垂直曲
E

E
厂
[

[汀刑T
才

[芸c史|

E二
是端点集属于3,D和&D的A8水平曲线及A垂直曲线,则上
E沥

证明交点的存在性可由连续性得到,而唯一性由引理1.6保
E河

E吴7李人
水

5

[一
E

0

E吴页
E命a林人水
E
Page-355
0

罄】=彤,Ez=瞅,r=茆,z=袁'

E咤s李中浩技

城pd-网十(evp,=R(p,b,8,

0技

E木

其中
E林
E达
E6323

羞=爪1-妨),塞=宣十柘障《3.6)

此系统有唯一的不变环不二{Cp,b)1p一1},它是股引的.当幼十
0dt刑朗庞
4
匿>o(相应地`由(3.4),存在勇>o)使得对每】个贪E伟
E加红5s才
n技
E东
E月沥
a标国江政不月
E不江u林(汀河
t沥口53木
E
E林江林一
E於八园中p林技3技
E林国胡
沥
Page-356
E玟

上述引理可参明图4-3.

国4-3

E团医仪江水技
Et水技振沥
0
江
沥不
E述江一石
7DinD:是一个(fa,s)矩形,滔足(fDif1D:)CC3Da.类似地可
E前

医沥2口东9弃锋八河
0

Ey吊4户

2
令

E沥0刑
E
Page-357
sg

E
RCp,b,8)一8(p,0,8)一det(
E
det〔a(R,6)仪

E
3(p,百

2KP:外
E
DKp,b,8)一(3fF一Dcos(90)十25zsin(ab)十O(8)。

医述
图此,(3.7)可改写成
E《C3.9)
E
巳仪2李丿
E
E
C
E
由(3.8)知DD|aruco一0在伟附近对9有2个根.因此,集合
EEa春许;

,E
Page-358
E技t163

E途
E
医沥t仁水述
Eopusdy丞i
E
E
芸E

E
25
P

E东江

E沥
个微分和躯,满尸(Ah,s)锦形条件.设7CD是一条C光溥的A
Eaateyss邦
E

[李仁王

E园0

E河1十觞'
>1是定义1.4中的常数

b善林心一t
诉的证明可用类似方法得到-
E
d
E俊应仪
刑
E
入
国E
E
E技

0日
Page-359
E园

E吊技河满足以卞条件'
林2
5
e
【述胡吴人0

E万木晓发根

0,得刹
E
[逊
整理得
【f尔J河Cu河
E
E标O

E颂工0沥河扬用一一一二().

最后,考虑集合‖仍矾龋谬汁R=0,D=0,6=0),注意

tneC
Ed邦育丿
c吴0育
由对称性知,肖+:(一M,(8,下匹佼计M:(83一M,(8).在等
式(3.12)中取丿一0,1得
E述

E达仪

利用(3.12)和P一1十0(3)整珀上式可得
国cc育园0
E
[0
Page-360
EEag

对河1,令g一乙,则也滔足(x,户3锦形条件,只是其中的扩张
sl
证明引理.〖

E不
E

0园0
E吊5诊
n
E技技才2技

E
上界,则由引理1.10,有

Ed林i
d
江)
E
E
故存在常数C,使得
E
同理可证存在常数C,使得
E浩

定理1.5证明的完成
E
1
Eav…a^D.
E
Page-361
156E

总结上面的讨论可得
引理3.1([TJ〉在(p,0,,8)空间中印的邻域内,集合
E

E沥2河吊
(2K(8,8}(j一1,2)组成,漪尽(8)一Mu(3)一$58《十
【江

二林
下列曾线组成:,

人2才八沥命江大2

s沥c沥
E

顺唐n
不p
[沥2

Ea20
Page-362
E招E

E标an
医故

E沥坂cnmueyf妙口沥政d河沥
方向拉伸,故不动点O是双曲的

E

园
E

符号序列空间及其结构

令
吊二人2oNj,切丿2.
在5上引进如下度量。
【尘
40口二E沥胡二团
E林沥一d东东
阁.仔

E

的双边无穷序列
[
巳
技

这里a5是在分量S上的值.在SY上我们如下引选度重.
Page-363
i

沥

27招

5
Eanatpess
E招泊罚
pouEe真诊
匹a
E命aan国2刑一

E
E集称为完全不连
通的,如果它的每一个连通分支只包含一个点.闷时具有上面两条
Eeaitgpsss医口u

引理2.2空间3Y在度量(2.2)下是

0

E

Et二日
Page-364
沥人157

u一才
E标
宁【

办(8一MMo(3)一8一十O(8e3),8一0怡式0.
E园
E
曲缇

0
[
E述芸切月
肖(3)一口(二酥CM(89一Mo(8))一688十O(8p-1,
其中8央0.定理3.2得证,【

题与怡考题三

31在引理1.6和引理1.7中,为什么要分别利用对参数一致的同宿分
岷定理和对参数一致的Hopt分岔定理?这对最终得到定理1.1有什么作用?
3.2证明引理2
$.3证明引理2.5.,
3.4对方程(2.5)7讨论它的分岔现象,并证明分岑图3-6的正确性
E林胡
E技d

E
E一

c
E昔卜《l时,(2^7)在原点的发散量为弼奂0.)
Page-365
E招

河八
s,使得对其巾任意序列口一(aijiez,古一仁jez,有a二页.下
EoEl志
医沥朱达朐许4
E0沥团E沥
ui志江东5国衍
6
浩
E
U扬一才扬
[史
bnd吴河
心巳经取定,取aL1EKu使得心1世i一0,1,

2

1一
E

E水
人
[

[不

一
c东|

E
E沥
2
Page-366
医沥l李

2
五章中是我们研究空间R中歌点同宿分岔的基础,同E
也有其自身的重要价值.在81中我们证明一个双曲不动点定理.
在2中引进符号助力学的基本概念.在83中给出马蹄存在定理
E
引理.在85中作为$2一$4中诺结果的一个应用,我们将给出E
中Birkhoff-Smale定理的证明.

i园d
Wiggins的书[Wig]中都能挂到,但此处所有定理的证明郭是独立
给出的.我们力图把儿何直观与数学的严密性统一起来,并给予
读者一套易于掌握的方法,用以解决高维空间中其它类似的闰题.

$1双曲不动点定理

定理的陈述

育g
L沥c东技才
E江
Et沥东一一
图象.当一co<a<8<<十co时,称点(af(a)),(8,/(8)为该曲线
E贺伟江
为C的y的函数=二g(y),yECe,切的图象.特别地,当a一一co,
Page-367
第四敦双曲不动点及马路存在定

E
zeanusy
引理2.3移位晔射v是2到自身的同胜.
E国5兄招5
0
令酗=愉h…z,iez)由定义
E

n
E
QKaiyiy颂生1
F
E

E

后@E弛,集合

Ecod2尘
Ee
ECuCouot一水Antpts技g河芸e
期.一个非周期点在的正向及负向迭代下,如果趋于同一个周期
EelE
s

定理2.4对3中的移位映射v下列结论成立.
E坂
Page-368
E驱ri

a水纳ta

东
E
E

E

E医一二技
技江
称为边界的水平部分,记作#D:那组A垂直对边称为边界的坩
E

E河05
空间.令

277木人伟
沥07水7

E

定义1.4令DCRIXR`图
E仪a沥n河
n沥林
Eupnicsosisi沥河

3许吴
YED

[刑
E

医

兰7
E仁9

[兄一万3

E沥技
李
Page-369
EE

slalt王大标江沥沥二
的异宪点集在V中稠密;

[一

E不
的元素构成的一个周期性重复的双边无穷序列用在其重复段上加
一个模线表示.例如(...,1,2,1,2,...}用{12}表示.对左(右)向无
穷的周期重复的序列用一个在其重复段上面向左(有)的箭头表
示,例如卜",1'2,L2〉用〈迳〉表示,〈l,z,1yzy"-)用(迳〉表示.E
E一
E沥水水
E医程c王一吴p刀唐标
E技
伟

E莲中木T
E

现在证明(2).令@一{2“},一他}是阿个周期点,这里a“与
“分别是山和二的周期重复郭分.对任意EV,任意给定s>-0,
EE

E

E沥达
E

E[
Ec林应口林水
EeEososstolpa河
E

c沥

E{鹰′】,勿鬟】,."`窿′N′}′
Page-370
E第四章双曲不动点及马路存圭定理

Es
E亚n
Ei匕
E
E怡
202
则了在马中有唯一双曲不动点

上述定理的儿何直观见囹4-2(a).

E
E

E0
E[吴吴17

E罚
E一

e
Page-371
E玟

E厂育2前许

E〈`“,鹰薯,鹰z,鹰】,鹰z,鹰薯"`_}′
这样。包含任意给定长度的所有可能的序列.我们断言u的转道
E八伟认0沥沥林加招
E广
玟

E沥

故乙的尬道OCa)在3中稔密。〖

E匿浩u

ss河
E
E沥连e
园
性映射的情况,给出所谓马踹存在定理.最后我们将看到马蹄可以
在三维向量杨的Poincare映射中出现.

E

考虔R上的单位正方形D一[0,1]X[0,1].我们引进DD到
E沥t
压缩5倩,然后将所得到的细高矩形在中部夺曲得到马踹形区域,
E江江述c沥

为1家为不的短形V.,V:,而广1CD[1D)是两个高为不宽为1的
t刑
王
E刑国一
【沥
Page-372
E02

第二步证明y一门Di和一一[D-;分别是端点属于3eD和
E标

第三步证明一V是了的一个双曲不动点
E

儿个引理

引理1.6平面R!XRI上一条心水平曳线与一条A垂直曲
E

E
厂
[

[汀刑T
才

[芸c史|

E二
是端点集属于3,D和&D的A8水平曲线及A垂直曲线,则上
E沥

证明交点的存在性可由连续性得到,而唯一性由引理1.6保
E河

E吴7李人
水

5

[一
E

0

E吴页
E命a林人水
E
Page-373
现在我们考虚D中的所有在了的任意次迭代下都不离开D
的点集4,却

E林
E

n
u

E
Eeuies2E
E

明沥水i林E东一[
E

L剧
Page-374
E玟

上述引理可参明图4-3.

国4-3

E团医仪江水技
Et水技振沥
0
江
沥不
E述江一石
7DinD:是一个(fa,s)矩形,滔足(fDif1D:)CC3Da.类似地可
E前

医沥2口东9弃锋八河
0

Ey吊4户

2
令

E沥0刑
E
Page-375
E技t163

E途
E
医沥t仁水述
Eopusdy丞i
E
E
芸E

E
25
P

E东江

E沥
个微分和躯,满尸(Ah,s)锦形条件.设7CD是一条C光溥的A
Eaateyss邦
E

[李仁王

E园0

E河1十觞'
>1是定义1.4中的常数

b善林心一t
诉的证明可用类似方法得到-
E
d
E俊应仪
刑
E
入
国E
E
E技

0日
Page-376
172E

E
技

Dn
E认
是一个非空Cantor集.
l沥怀浩z四林技

故对VzEA

E
注意到ACEuUa,因此存在一个由1,2组成的无限双边序列a
[

E伟

这根我们定义了一个晃射申4>罗,

E
陵沥沥述河i妥上的右移位映射则根据定义有

E技技口
E倭是^到藩的间胚唧人
E丿吊沥el沥
a沥一
E仪
E分

E2月
【unagE志
长度为28--1的中间殷是一样的

E门E
E不
丿中的距离可以充分的小,这便说明了一是连续的-
E
Page-377
EEag

对河1,令g一乙,则也滔足(x,户3锦形条件,只是其中的扩张
sl
证明引理.〖

E不
E

0园0
E吊5诊
n
E技技才2技

E
上界,则由引理1.10,有

Ed林i
d
江)
E
E
故存在常数C,使得
E
同理可证存在常数C,使得
E浩

定理1.5证明的完成
E
1
Eav…a^D.
E
Page-378
E招E

E标an
医故

E沥坂cnmueyf妙口沥政d河沥
方向拉伸,故不动点O是双曲的

E

园
E

符号序列空间及其结构

令
吊二人2oNj,切丿2.
在5上引进如下度量。
【尘
40口二E沥胡二团
E林沥一d东东
阁.仔

E

的双边无穷序列
[
巳
技

这里a5是在分量S上的值.在SY上我们如下引选度重.
Page-379
张理

E
E坤2

E浩
E
E
E沥人木伟五
B竖连一初
E2甫n有
[弛沥月
E
[
这与ax5矛盾.
E技5八育英页肖江胡5述江育
医胺
我们将它作为练习留给读者.、
E途
玟

定理3.1马蹇春射了在D中有一个不变的Cantor集4,漾
E

【68s吊pa

0河5
E沥

【62河坂a

E达

0出李口
E二技技
Page-380
i

沥

27招

5
Eanatpess
E招泊罚
pouEe真诊
匹a
E命aan国2刑一

E
E集称为完全不连
通的,如果它的每一个连通分支只包含一个点.闷时具有上面两条
Eeaitgpsss医口u

引理2.2空间3Y在度量(2.2)下是

0

E

Et二日
Page-381
E第四章双曲不助点及马跨存在宏境

i

一EX是一个微分同胚,并且
木
[木
木
3
则集合

是一个不变的Cantor集.丁限制′羞^上的映射/Etuest

E林
医沥

由定理2.4我们有
推论3.3定理3.2中的春射了有一个不变的Cantor集4,漾

E
noatzseyapl
〈2)周期软道的同宿点及异宿点在4中稚密:
E河沥5

定理3.2证明的恰路
E李
…茅5〈窿′〉o<f<n,命(膘趸)一″<f<u′
aE
如果zE卫漾足fzE.Du,0<Sixsn,则我们记作
_E
加果zED满足广!zED。,,0<Si<smv则我们记作
6
Page-382
E招

河八
s,使得对其巾任意序列口一(aijiez,古一仁jez,有a二页.下
EoEl志
医沥朱达朐许4
E0沥团E沥
ui志江东5国衍
6
浩
E
U扬一才扬
[史
bnd吴河
心巳经取定,取aL1EKu使得心1世i一0,1,

2

1一
E

E水
人
[

[不

一
c东|

E
E沥
2
Page-383
张胡

沥人1技
3许

阮
PDlat)CCDCa_a,D(az)CDCai_0,
E
E
第二步证明

E一
E禺"D(m;)
是--条端点属于,Da的丿垂直曲线
E

a二D(or)nDCar).
E
ERs租s2C)
是一个同胚E
E′1正明等式(3.1〉成立'

定理3.2的证明
令
人+】""`″″)〉
Et
E
[
林水
D“″一i′E
Page-384
第四敦双曲不动点及马路存在定

E
zeanusy
引理2.3移位晔射v是2到自身的同胜.
E国5兄招5
0
令酗=愉h…z,iez)由定义
E

n
E
QKaiyiy颂生1
F
E

E

后@E弛,集合

Ecod2尘
Ee
ECuCouot一水Antpts技g河芸e
期.一个非周期点在的正向及负向迭代下,如果趋于同一个周期
EelE
s

定理2.4对3中的移位映射v下列结论成立.
E坂
Page-385
玟

阮
u
E口
c木
沥

E医

E吴L希技
0
类似地我们有
引理3.5E
e
E宇口0人八5
引理3.6存在一个正常数C,使得
0月
E
E江

Epn沥江弛玟c
b
江
EO月

Un

E_邑。D(m蒙)
E沥一一逊
E剧
E一
Page-386
E沥s

外史:。
E
这样我们得到一个眸射V一A+CoD.
国[
E
Et
玟
人设fzEDa,iEZ,则
E
E
2.是映上的.这挡的是对任意给定zE4,都存在一个点。
EEn
E李
E
E团不
E
团…_巳″【D(m薯〉【ULCsp

3.幼是逊续的.这指的是任络G3N,e>0,都可找到正数8,

使得对任意
c
着4(a,m心8,则Q(ga)「b(m)一s由引理2.1,如果qCaov矶一8,
D沥口2s仪
5
0吴3

医沥林[

0引

E水二
Page-387
EE

slalt王大标江沥沥二
的异宪点集在V中稠密;

[一

E不
的元素构成的一个周期性重复的双边无穷序列用在其重复段上加
一个模线表示.例如(...,1,2,1,2,...}用{12}表示.对左(右)向无
穷的周期重复的序列用一个在其重复段上面向左(有)的箭头表
示,例如卜",1'2,L2〉用〈迳〉表示,〈l,z,1yzy"-)用(迳〉表示.E
E一
E沥水水
E医程c王一吴p刀唐标
E技
伟

E莲中木T
E

现在证明(2).令@一{2“},一他}是阿个周期点,这里a“与
“分别是山和二的周期重复郭分.对任意EV,任意给定s>-0,
EE

E

E沥达
E

E[
Ec林应口林水
EeEososstolpa河
E

c沥

E{鹰′】,勿鬟】,."`窿′N′}′
Page-388
178E

E
E
[

E
男一方面,由(3.5)得
E

这意咋着

E吴
E

闵
定理证毕,

E颠ctEE

本节将给出关于两个线性眼射的复合眸射的双曲性的一个重
要引理,这一引理将在第五章中多次应用.

问颜的提出

第五章将研究者点同宿轨的分岗,这一问题的解决是通过研
E
为两个晔射的复合:一个陋射由奇点的邻域中的向量场决定,男一
E
余
分小时,它治着稳定流形方向的压缩常数和治着不稳定流形方向
的扩张常数分别先分小和充分大,我们对第二个眯射除了知道它
E
余
E
Page-389
E玟

E厂育2前许

E〈`“,鹰薯,鹰z,鹰】,鹰z,鹰薯"`_}′
这样。包含任意给定长度的所有可能的序列.我们断言u的转道
E八伟认0沥沥林加招
E广
玟

E沥

故乙的尬道OCa)在3中稔密。〖

E匿浩u

ss河
E
E沥连e
园
性映射的情况,给出所谓马踹存在定理.最后我们将看到马蹄可以
在三维向量杨的Poincare映射中出现.

E

考虔R上的单位正方形D一[0,1]X[0,1].我们引进DD到
E沥t
压缩5倩,然后将所得到的细高矩形在中部夺曲得到马踹形区域,
E江江述c沥

为1家为不的短形V.,V:,而广1CD[1D)是两个高为不宽为1的
t刑
王
E刑国一
【沥
Page-390
0圆i179

医a吊
两个映射的导算子的积给出的.因而本节我们秽对两个线性映射
的复合晖射的双曲性给出一个判别引理.

E

一
E伟
E
E木仁才

Eu育芸木扬

E
005e
[t

0一刀

这里22>1是一个常数.

Ee
E

壮
E
Ec李p
Page-391
现在我们考虚D中的所有在了的任意次迭代下都不离开D
的点集4,却

E林
E

n
u

E
Eeuies2E
E

明沥水i林E东一[
E

L剧
Page-392
第四章双曲不动点及马蹭存在定理

满足M大0.令[>0是一个常数,使得

L万

仁

[一切E
Ei探丿白
如果下列不等式

万0
匹2团E
[[
a[
a0
Eyyy4扬水23图

E
E吴n育27724247
E
成立的前题下,分三步证明引理.

E义p

令
E
对满尸(4.9)的常数,我们将证明
林应沥
E
c人李
E沥
d水2肉
C41
Page-393
招

E切
t0标)
E

[园一
沥
E|<#汇】M丨t
Et
这些不等式与呱.6)和(4'7)】起推出
Ey
t0

E
I医圆11<鲍》|M茗+1y
由(4.11)`(4′12),〈4′I3)和(4^16〉我们得
厂河胡〔L〉】b瓮)〕-[吴啬乙一外M庸1

0
E
E林
Essoctnpotyi

锥彭区域的不变性
现在证明
E0
江
0
[《19
c刑
Page-394
172E

E
技

Dn
E认
是一个非空Cantor集.
l沥怀浩z四林技

故对VzEA

E
注意到ACEuUa,因此存在一个由1,2组成的无限双边序列a
[

E伟

这根我们定义了一个晃射申4>罗,

E
陵沥沥述河i妥上的右移位映射则根据定义有

E技技口
E倭是^到藩的间胚唧人
E丿吊沥el沥
a沥一
E仪
E分

E2月
【unagE志
长度为28--1的中间殷是一样的

E门E
E不
丿中的距离可以充分的小,这便说明了一是连续的-
E
Page-395
EE

E
E沥玟

由三角不等式

|87|妮lallh-1+1611D8-1十lallB8|+51|
E

E
lall-[十|6|1D8“1东28|42D

进一步由(4.3)
1B5+1妙18|一二|1.

国此由(4.1
la|B1十|阮|妙(十切|纳(22)

将(421)和(4.22)代人(4.20)得
E人

这与(417)一起得
吴t人一

EEasutyyp技圭
便证明了(419,即(4.182.

穗定锥形区域内的压绾性

E
E林伟
我们将证明不等式
[沥

令

沥

余
0

(〗雇
E技

y
Page-396
张理

E
E坤2

E浩
E
E
E沥人木伟五
B竖连一初
E2甫n有
[弛沥月
E
[
这与ax5矛盾.
E技5八育英页肖江胡5述江育
医胺
我们将它作为练习留给读者.、
E途
玟

定理3.1马蹇春射了在D中有一个不变的Cantor集4,漾
E

【68s吊pa

0河5
E沥

【62河坂a

E达

0出李口
E二技技
Page-397
i

E友'}
E
E悬【襄〉】【.
Ef
[
E
E吴

E余
E
s刑7命刀
现在我们证明(4.23).注意到4~.一厂&-1,我们有
E河芸沥荣水乐
E

|红1芸|国[|ag-|一1Beq-|一|B+|一|万B27+|
t

由(426)
仪
由(4.27,(4.25)和(4.8),有
s
E口日
E二

i
由(4.27)和(4.1
E
注意到9EK“,由(4.31)和(432)得
E
Page-398
E第四章双曲不助点及马跨存在宏境

i

一EX是一个微分同胚,并且
木
[木
木
3
则集合

是一个不变的Cantor集.丁限制′羞^上的映射/Etuest

E林
医沥

由定理2.4我们有
推论3.3定理3.2中的春射了有一个不变的Cantor集4,漾

E
noatzseyapl
〈2)周期软道的同宿点及异宿点在4中稚密:
E河沥5

定理3.2证明的恰路
E李
…茅5〈窿′〉o<f<n,命(膘趸)一″<f<u′
aE
如果zE卫漾足fzE.Du,0<Sixsn,则我们记作
_E
加果zED满足广!zED。,,0<Si<smv则我们记作
6
Page-399
E第四章双曲不助点及马踊孛在定理

由(4.247,C4.29),(4.30)和(4.33)得到(4.28)右端的下界佼计.
E
g|交酊一匹十不)一8一止十发8|
E
E

医玟告(乙E沥夜5
E林不

y沥江

医江2沥
Smale定理.

定理的陈述

E日孝c水水一王逊诊
个特征指数,满足|2|一1一|x|.这根的周排执有一维穗定流形
E一国不
期乳5的咏宿转,如果7丿丘YCWr(e)iW*Co).进一步,同宿
敏7称为槲藏的,如果流形Pr(o)与W*Co)沿着执道Y樟戳相交,
E

玟
E
朋标林万玖

事实上,我们将证明比定理5.1更强的结论:在cUY的任意邻
E林
行.,

周期轨道的Poineark映射
医江c
Page-400
张胡

沥人1技
3许

阮
PDlat)CCDCa_a,D(az)CDCai_0,
E
E
第二步证明

E一
E禺"D(m;)
是--条端点属于,Da的丿垂直曲线
E

a二D(or)nDCar).
E
ERs租s2C)
是一个同胚E
E′1正明等式(3.1〉成立'

定理3.2的证明
令
人+】""`″″)〉
Et
E
[
林水
D“″一i′E
Page-401
[渡

E
E林标团达5诊
点处可以〇线性化(参见第一章定理422),故在$上0点的邻域
ELstotesz
【Cpaispsest
E沥技
这里|一1人|A|.直线z一0和y一0分别对庭局部穗定流形和局
E莲
0
E沥L沥
E河
Eocuti
e【
令Dy一P~“名门B表示8中所有在P“作用下眶到吊的那些点所
E沥芸途训i
E
E吴[

E沥吴tn沥

0
Page-402
玟

阮
u
E口
c木
沥

E医

E吴L希技
0
类似地我们有
引理3.5E
e
E宇口0人八5
引理3.6存在一个正常数C,使得
0月
E
E江

Epn沥江弛玟c
b
江
EO月

Un

E_邑。D(m蒙)
E沥一一逊
E剧
E一
Page-403
玟

E

同宿轨道的后继映射

Eapoisky规林s河a水
E怀技技p一沥林玟不沥
我们用与表示这一对应

E[
由常微分方程的解对初值的光滑依赖性知,F是C!微分同贴.令
E2技应2沥诊林t
E
菩a[
E
卫一旦伙二0},灵二书n仁二0}.
取定8充分小,使得
E育
沥
0

Ea
E

D
Page-404
E沥s

外史:。
E
这样我们得到一个眸射V一A+CoD.
国[
E
Et
玟
人设fzEDa,iEZ,则
E
E
2.是映上的.这挡的是对任意给定zE4,都存在一个点。
EEn
E李
E
E团不
E
团…_巳″【D(m薯〉【ULCsp

3.幼是逊续的.这指的是任络G3N,e>0,都可找到正数8,

使得对任意
c
着4(a,m心8,则Q(ga)「b(m)一s由引理2.1,如果qCaov矶一8,
D沥口2s仪
5
0吴3

医沥林[

0引

E水二
Page-405
理J

E深2柳

En一沥一技余扬月刑技
〈5.9)可以取定&充分小,使得

【

[
【

[2二育玖
E
现在我们定义后继映射An:D,-*U如下:
巳吴

E吴

E口
a
林
E

E
和
2孙央
E国2肉
u沥一一
0一
[
标
Page-406
178E

E
E
[

E
男一方面,由(3.5)得
E

这意咋着

E吴
E

闵
定理证毕,

E颠ctEE

本节将给出关于两个线性眼射的复合眸射的双曲性的一个重
要引理,这一引理将在第五章中多次应用.

问颜的提出

第五章将研究者点同宿轨的分岗,这一问题的解决是通过研
E
为两个晔射的复合:一个陋射由奇点的邻域中的向量场决定,男一
E
余
分小时,它治着稳定流形方向的压缩常数和治着不稳定流形方向
的扩张常数分别先分小和充分大,我们对第二个眯射除了知道它
E
余
E
Page-407
玟

E

E
工
朋
E2

E河
EatwciLuco

由(5.10)及与是一个C!徽分同胚,我们有
E
医江
14|一|灵|一0,1M|二17口一0,当一十co,
对充分大的a,(4.3)一(4.8)显然成立-
下面验证边界条件.注意到

E园a团
以及(5.11)和(5.12),便知迈界条件成立.

最后验证相交条件.在D;中有两族直线,即所谓水平直线族
E

【芸标吴汀0朋
[沥林
这里CDiDo一{|Ey一11<8}.注意到
鳜@)卫>B们匣鳐位)逞荨凰,当妻玲+的,
再利用(5.6)得到,当,j充分大时,曲线AC6Cz)与四(o)在外点
附近相交于唯一一点-固此

3
E
Page-408
E源s

E

Eiasnaed|
最后,由引理5.2及定理3.2可推出定理5.1.

医沥沥人明沥不工d
果,可参见[Wig]和〔si11〕.
Page-409
0圆i179

医a吊
两个映射的导算子的积给出的.因而本节我们秽对两个线性映射
的复合晖射的双曲性给出一个判别引理.

E

一
E伟
E
E木仁才

Eu育芸木扬

E
005e
[t

0一刀

这里22>1是一个常数.

Ee
E

壮
E
Ec李p
Page-410
第四章双曲不动点及马蹭存在定理

满足M大0.令[>0是一个常数,使得

L万

仁

[一切E
Ei探丿白
如果下列不等式

万0
匹2团E
[[
a[
a0
Eyyy4扬水23图

E
E吴n育27724247
E
成立的前题下,分三步证明引理.

E义p

令
E
对满尸(4.9)的常数,我们将证明
林应沥
E
c人李
E沥
d水2肉
C41
Page-411
巳江力dE

本章将考虑空间R中鞭点的同宿分岔.在81中将讨论特征
根都为实数的鞍点的同宿分岔.在82中将讨论有复特征根的鞍点
的间宿分岔.在83中将讨论由一个奇点和一条闭尧以及连结它们
E章的学习,读者将会对如
何利用奇点或不动点附近的线性化理论(见第一章$4)来研究非
E

$1具有三个实特征值的鞍点的同宿分岔

河国n沥沥
E

Ep

假设中一光溺向量场有一个特征值都为实数的双曲鞍点
及其间宿技.我们考虑这样一个向量场在一般的单参数扰动下所
能发生的分岔.不失一舫性,我们总可以认为鞍点有两个负特征值
和一个正牺征值(否则考虑其时间反向系统).这样的鞍点具有二
维的稳定流形和一维的不稳定流形我们称最大的负特征值与正
特征值之和为鞍点量

E二园圆沥i
数e==0时,向量场X有一个鞍点0,它具有两个负特征值和一个
d5a
E
E
B
Page-412
招

E切
t0标)
E

[园一
沥
E|<#汇】M丨t
Et
这些不等式与呱.6)和(4'7)】起推出
Ey
t0

E
I医圆11<鲍》|M茗+1y
由(4.11)`(4′12),〈4′I3)和(4^16〉我们得
厂河胡〔L〉】b瓮)〕-[吴啬乙一外M庸1

0
E
E林
Essoctnpotyi

锥彭区域的不变性
现在证明
E0
江
0
[《19
c刑
Page-413
EE

E
E沥玟

由三角不等式

|87|妮lallh-1+1611D8-1十lallB8|+51|
E

E
lall-[十|6|1D8“1东28|42D

进一步由(4.3)
1B5+1妙18|一二|1.

国此由(4.1
la|B1十|阮|妙(十切|纳(22)

将(421)和(4.22)代人(4.20)得
E人

这与(417)一起得
吴t人一

EEasutyyp技圭
便证明了(419,即(4.182.

穗定锥形区域内的压绾性

E
E林伟
我们将证明不等式
[沥

令

沥

余
0

(〗雇
E技

y
Page-414
不具有三个实御征值的糖点的网宝分吴E

E
E

E

在证明定理之前,我们首先解释定理陈述中“一舫“一词的含
Eti汀i沥D命诊5训沥招河
三条是针对E

点0的特征值两两不相同且为非共振.

di
在点0处的线性部分(参见第一章定理418),故在绎性化坐标系
沥
上,除了奇点0和一条通过O点的直线外,所有辅道当二十co时
t
E

0吊p

一
向量张成一个不变平面仁,这一不变平面沿着同宿轨延伸

E
园5

E施t江沥口p刃
E

定理1.1证明的恺路

玟
Eae汀5扎
伟t
u人
E
为初值的正半辉与“的第一个交点.假设(1使我们可以把向量
E志不s沥n
Page-415
i

E友'}
E
E悬【襄〉】【.
Ef
[
E
E吴

E余
E
s刑7命刀
现在我们证明(4.23).注意到4~.一厂&-1,我们有
E河芸沥荣水乐
E

|红1芸|国[|ag-|一1Beq-|一|B+|一|万B27+|
t

由(426)
仪
由(4.27,(4.25)和(4.8),有
s
E口日
E二

i
由(4.27)和(4.1
E
注意到9EK“,由(4.31)和(432)得
E
Page-416
\part{第五章宇间中双曲鞅的问宿分岖}


等函数袁示出来.当鞍点量分别是正和负时,48“8分别具有强双
E
而沿振道我们可以定义从同宿轨与7“的交点的邻域到吟宿技与
工的交点的邺域的晔射,并将它记作46,见图5-2.由解对初值及

Egn沥y沥
诊沥壮标

E
最后,我们分别利用双曲不动点定理和压缉映象原理,讨论当鞍点
i

E玖仪
E
Page-417
朱具市三个实後狄值的鞍炎的问分岔
Ezozsoooiumitieosuctuuuioitdi

B

、′

E

E

E沥水
{汉(0口00
E

E吴

E应
uaeoe林e7
|os|一1不失一舱性,我们还可假设,0点邻域中正=粘是同宿转
园cai

E坂t
Page-418
E第四章双曲不助点及马踊孛在定理

由(4.247,C4.29),(4.30)和(4.33)得到(4.28)右端的下界佼计.
E
g|交酊一匹十不)一8一止十发8|
E
E

医玟告(乙E沥夜5
E林不

y沥江

医江2沥
Smale定理.

定理的陈述

E日孝c水水一王逊诊
个特征指数,满足|2|一1一|x|.这根的周排执有一维穗定流形
E一国不
期乳5的咏宿转,如果7丿丘YCWr(e)iW*Co).进一步,同宿
敏7称为槲藏的,如果流形Pr(o)与W*Co)沿着执道Y樟戳相交,
E

玟
E
朋标林万玖

事实上,我们将证明比定理5.1更强的结论:在cUY的任意邻
E林
行.,

周期轨道的Poineark映射
医江c
Page-419
E

R32203
5
注意到在7上z一1,因此由等式
E
d

哥z(DEzasyJOXG913对

河g

P
然后代入(1.2中前两式,并注意到在7+上z一1,便可得到
E0
EAA

E不不

E口林木
点加一(1,v,0),由解对初值与参数的依赖性可知,存在7“上u
Eeetsee庞街
吊江5加

E
映射4P8是一个光滢依赖于的徽分同胚-令
Eyc吊

不难看出,(Y(e),Z(e))是鞍点的不稳定流形与7的交点,面
E

羞Z(鬓〉L-。亨阜02
玟

0320
Page-420
[渡

E
E林标团达5诊
点处可以〇线性化(参见第一章定理422),故在$上0点的邻域
ELstotesz
【Cpaispsest
E沥技
这里|一1人|A|.直线z一0和y一0分别对庭局部穗定流形和局
E莲
0
E沥L沥
E河
Eocuti
e【
令Dy一P~“名门B表示8中所有在P“作用下眶到吊的那些点所
E沥芸途训i
E
E吴[

E沥吴tn沥

0
Page-421
y

E2怀李国巳河
E初
‖4置i”薯(汉`z)‖″″Etpyan

E
E达
伟一dng:P一,

E

现在我们对鞍点量为负的情况来证明定理1.1.
E林0

国国,
月二″(…)>1`E泓

E河
相E
u涛|[oM〕‖″^
E
E述林a
关于参数一致有界.因此,(1.6)意晖着
L

E

E沥Ea
e<0栗讨论映射4的不动点的存在性.
【692x44
对任意点(y,z)E.Lo,有
U

沥
E江

CL.6)

0
Page-422
玟

E

同宿轨道的后继映射

Eapoisky规林s河a水
E怀技技p一沥林玟不沥
我们用与表示这一对应

E[
由常微分方程的解对初值的光滑依赖性知,F是C!微分同贴.令
E2技应2沥诊林t
E
菩a[
E
卫一旦伙二0},灵二书n仁二0}.
取定8充分小,使得
E育
沥
0

Ea
E

D
Page-423
第五章空间中双曲数点的闯宪分盅

E

ELssetesanotulogoa
0nttdp志t人
过不动点的转道是一个呆引周期轨.而不等式(1.7)意昝着不动点
医余x训
宿软的位置.

C2)eSo.

d

Et

enst<鲁似鹏z)5

g

上边不等式中等叶成立,当丁仅当

E
E
E

E

鞍类周期轨的产生

现在对鞍点量为正的倩况来证明定理.与鞍点量为负的情况
n
E

E
假设(3)意昭督
[
-EE
情况么0和乙一0分别对应所谓叶定向咖宿扔和不可定向闭宿
E志2k史

EE《C1.8)
Page-424
理J

E深2柳

En一沥一技余扬月刑技
〈5.9)可以取定&充分小,使得

【

[
【

[2二育玖
E
现在我们定义后继映射An:D,-*U如下:
巳吴

E吴

E口
a
林
E

E
和
2孙央
E国2肉
u沥一一
0一
[
标
Page-425
$1具有三个宗特征例的鞑点的同分盅

E
沥
[
证明我们用第四章引理41来证明
c
E招r
E
EE
l
EE
吉
伟

【nslt扬g

刀二0,砺一zf-1
由第四章引理41司知+引理1.3成立,如果存在常数[>0,使得

r

【河逊一仪
2
E许技c

E政志仁丞n口
[

U

加(

围E

b水
E

E

E责zl一卢镳E沥厂昔y薪一润伶E
Page-426
玟

E

E
工
朋
E2

E河
EatwciLuco

由(5.10)及与是一个C!徽分同胚,我们有
E
医江
14|一|灵|一0,1M|二17口一0,当一十co,
对充分大的a,(4.3)一(4.8)显然成立-
下面验证边界条件.注意到

E园a团
以及(5.11)和(5.12),便知迈界条件成立.

最后验证相交条件.在D;中有两族直线,即所谓水平直线族
E

【芸标吴汀0朋
[沥林
这里CDiDo一{|Ey一11<8}.注意到
鳜@)卫>B们匣鳐位)逞荨凰,当妻玲+的,
再利用(5.6)得到,当,j充分大时,曲线AC6Cz)与四(o)在外点
附近相交于唯一一点-固此

3
E
Page-427
E源s

E

Eiasnaed|
最后,由引理5.2及定理3.2可推出定理5.1.

医沥沥人明沥不工d
果,可参见[Wig]和〔si11〕.
Page-428
198雉五章空间中双幼骇炯阡问宿分盅

y巳a月
第二个不等式必然成立,引理证毕.‖
i
E李d
E
E
g
eeg
=簪汊z″十a仝予8沥吴

巴E
E
E尚i吊吴A口
E命育c芸3理25命

E

再设e<0,下面用双曲不助点定理来证明,4在T内有唯一

歌类不动点因为4eCy,0)一e一0,我们可以找到这桦一个依赖于
n

医不0

沥_
E怀刑人

则短形D的边界的水平部分为

2L辽
E

2一者许
E

R口3

沥
Page-429
巳江力dE

本章将考虑空间R中鞭点的同宿分岔.在81中将讨论特征
根都为实数的鞍点的同宿分岔.在82中将讨论有复特征根的鞍点
的间宿分岔.在83中将讨论由一个奇点和一条闭尧以及连结它们
E章的学习,读者将会对如
何利用奇点或不动点附近的线性化理论(见第一章$4)来研究非
E

$1具有三个实特征值的鞍点的同宿分岔

河国n沥沥
E

Ep

假设中一光溺向量场有一个特征值都为实数的双曲鞍点
及其间宿技.我们考虑这样一个向量场在一般的单参数扰动下所
能发生的分岔.不失一舫性,我们总可以认为鞍点有两个负特征值
和一个正牺征值(否则考虑其时间反向系统).这样的鞍点具有二
维的稳定流形和一维的不稳定流形我们称最大的负特征值与正
特征值之和为鞍点量

E二园圆沥i
数e==0时,向量场X有一个鞍点0,它具有两个负特征值和一个
d5a
E
E
B
Page-430
1具有三个实待招信的靼点的闭宿分盅

c麾鸭(0,0)iEE

掸'E丨蓥亡卜d途n

注意到|ov|<<1,我们有
E
由(1.10)有
En2n北a
(l,14)与(L10)一起推出
E招
以及
l0
EcptenptyyesR圆7E沥河
兰上满足第四章定理1.5的所有条件.因而4在De内有晴--双
D不胺
沥
E朝
ss
【

E
p
t吊2河述

EE
[

E
〈蔷邈十|园

b玲
Page-431
不具有三个实御征值的糖点的网宝分吴E

E
E

E

在证明定理之前,我们首先解释定理陈述中“一舫“一词的含
Eti汀i沥D命诊5训沥招河
三条是针对E

点0的特征值两两不相同且为非共振.

di
在点0处的线性部分(参见第一章定理418),故在绎性化坐标系
沥
上,除了奇点0和一条通过O点的直线外,所有辅道当二十co时
t
E

0吊p

一
向量张成一个不变平面仁,这一不变平面沿着同宿轨延伸

E
园5

E施t江沥口p刃
E

定理1.1证明的恺路

玟
Eae汀5扎
伟t
u人
E
为初值的正半辉与“的第一个交点.假设(1使我们可以把向量
E志不s沥n
Page-432
EE

0
E吊7
2
再设e>0.因为4z(o,0)一e,故存在一个依赖于参数e的常

E

4z(y,z〉>青,对0<z<Cg′0
E
E<
〈z(二v,言)〈一歹<0′1

E

p
E沥
E
E园
i
EEEe根p
似,我们可以证明映射在De内有唯一双曲不动点.定理证毕,

E沥洁EtoLorkl

我们将R中具有一对复特征值和一实特征值的双曲鞍点称
E
假设鞋焦点的一对复特征值具有负实部,而它的实特征值为正数,
E一
实部的和称作鞍点量.鞋点量为正或为负的鞍焦点所对应的同容
分岗有着本质的不同.当鞍点重为负时,分盆与本章81中相应的
情况类似1而当鞍点重为正时,在鞍焦点的同宿轨的任意邺域中定
Page-433
第五章宇间中双曲鞅的问宿分岖

等函数袁示出来.当鞍点量分别是正和负时,48“8分别具有强双
E
而沿振道我们可以定义从同宿轨与7“的交点的邻域到吟宿技与
工的交点的邺域的晔射,并将它记作46,见图5-2.由解对初值及

Egn沥y沥
诊沥壮标

E
最后,我们分别利用双曲不动点定理和压缉映象原理,讨论当鞍点
i

E玖仪
E
Page-434
朱具市三个实後狄值的鞍炎的问分岔
Ezozsoooiumitieosuctuuuioitdi

B

、′

E

E

E沥水
{汉(0口00
E

E吴

E应
uaeoe林e7
|os|一1不失一舱性,我们还可假设,0点邻域中正=粘是同宿转
园cai

E坂t
Page-435
不2圭闹中驶焦点的同街分彻E

界
5二

具有负鞍点量的鞍焦点的同宿分岐

本节第一个主要结果如下:

定理2.1令X是R中一个一舫的单参数向量场族.假设当
uapa2ssp
的同密轨7,则存在YUO的邻域D和参数空间中零值的邹域7
E途余tt
玟
Ec

在定理21的陈述中“一舫“词的含义是

(向量场X的鞍焦点的特征值非共振:

052技t

Eagtulte

定理2.1证明的直现含义

在给出严格证明以前,为便于读者理解,我们先给出证明的直
观揩述,为此我们将尽量把问题简化-首先我们假设,在鞍焦点的
Rst

E
c
0
EE
Et
E
8二外门I:Croy:切一(000D0.

E

《C2.3)

会
E
E2东林t
Page-436
E

R32203
5
注意到在7上z一1,因此由等式
E
d

哥z(DEzasyJOXG913对

河g

P
然后代入(1.2中前两式,并注意到在7+上z一1,便可得到
E0
EAA

E不不

E口林木
点加一(1,v,0),由解对初值与参数的依赖性可知,存在7“上u
Eeetsee庞街
吊江5加

E
映射4P8是一个光滢依赖于的徽分同胚-令
Eyc吊

不难看出,(Y(e),Z(e))是鞍点的不稳定流形与7的交点,面
E

羞Z(鬓〉L-。亨阜02
玟

0320
Page-437
208第五章空间中双幼鞑怒的问宿分岖

Elgzgi技怀汀
起
E
E5江述汀
Edattuydumdeit
fz一0}上出发的正半扬线的并组成.我们用4fog表示泳系统
《2.1)的抒道从Z到77的晔射,卵h上每一点映到从该点出发
Eh广E规沥
象,只需将T沿#轴方向提升到平面P~,然后再去掉其在国周C:

[i羞之外的部分'有两种情况需要考虚
0沥不c应梁
0怀2东n育梁
见图5-3.

E
在定理2.1中只考虑第一种情况.根据假设同宿执逊结T+的点4
沥林秉北扬水92
不
e
是一个平移:
El5江许3
Page-438
y

E2怀李国巳河
E初
‖4置i”薯(汉`z)‖″″Etpyan

E
E达
伟一dng:P一,

E

现在我们对鞍点量为负的情况来证明定理1.1.
E林0

国国,
月二″(…)>1`E泓

E河
相E
u涛|[oM〕‖″^
E
E述林a
关于参数一致有界.因此,(1.6)意晖着
L

E

E沥Ea
e<0栗讨论映射4的不动点的存在性.
【692x44
对任意点(y,z)E.Lo,有
U

沥
E江

CL.6)

0
Page-439
沥20

E林口沥
E招
E2
E述e口43
Eean
E技
Ed一

定理2.1的证明

我们将证明分成儿步
0标p江
由假设1,应用第一章定理4.27,在鞍焦点的一个邹域[里存
在一个坐标系(z,y,z),使得X:在0内是线性系统
E3朋
哥E怡103E
O
这里A(0)一0人x(0),e一(0)十pC0一0如果必要的话,可对
吴
人史不失一舫性,我们还可进一步假设,当e一0时,[中正轻是
同宿技的一部分、令
i
【2林李圆[朋
厂二{Cruy:切|与十万人1一1
酝c木
z(震)Eexp(入(s)z)〔c。s(m(s)′)堑gsin(m(龌)!)y〕'
庆XCIO3ENCYG3河许2
00
从上面第三个等式解得执道到达7所需时间为4气一xCe)-:Inz.
Page-440
第五章空间中双曲数点的闯宪分盅

E

ELssetesanotulogoa
0nttdp志t人
过不动点的转道是一个呆引周期轨.而不等式(1.7)意昝着不动点
医余x训
宿软的位置.

C2)eSo.

d

Et

enst<鲁似鹏z)5

g

上边不等式中等叶成立,当丁仅当

E
E
E

E

鞍类周期轨的产生

现在对鞍点量为正的倩况来证明定理.与鞍点量为负的情况
n
E

E
假设(3)意昭督
[
-EE
情况么0和乙一0分别对应所谓叶定向咖宿扔和不可定向闭宿
E志2k史

EE《C1.8)
Page-441
E第五章空间中双胡积点的问裂分岔

将其代人前两个等式,得到抒道与P的交点坐标Cz「,y「,1)满趸
[沥技c

E一A簧羞岫.E

E东一
0

E伟E

《2.6)
微分上式得到
Ez”_】【俊Sin辉〕E0办
E
此处

力工E矗忐(入(s)楂。SeBaEX3SX)

E1

0

E河

E朋

E
ERE沥口吴

F
[诊
E述王

E怀c玖
用5,8分别表示咏寇转Y与7r+和-的交点.将坐标系做一
E
E述[
E万圭c沥
M伟
园兰
Page-442
$1具有三个宗特征例的鞑点的同分盅

E
沥
[
证明我们用第四章引理41来证明
c
E招r
E
EE
l
EE
吉
伟

【nslt扬g

刀二0,砺一zf-1
由第四章引理41司知+引理1.3成立,如果存在常数[>0,使得

r

【河逊一仪
2
E许技c

E政志仁丞n口
[

U

加(

围E

b水
E

E

E责zl一卢镳E沥厂昔y薪一润伶E
Page-443
198雉五章空间中双幼骇炯阡问宿分盅

y巳a月
第二个不等式必然成立,引理证毕.‖
i
E李d
E
E
g
eeg
=簪汊z″十a仝予8沥吴

巴E
E
E尚i吊吴A口
E命育c芸3理25命

E

再设e<0,下面用双曲不助点定理来证明,4在T内有唯一

歌类不动点因为4eCy,0)一e一0,我们可以找到这桦一个依赖于
n

医不0

沥_
E怀刑人

则短形D的边界的水平部分为

2L辽
E

2一者许
E

R口3

沥
Page-444
国摄E

E途c
d

00200兰7
则假设(2)的确切含义为

基z佑〉a

E.
E
E用
0
E林a
E林
u

由(2.8)'当/【充分小时,鸟是」个压缩常数小于噩的压缩映射一

河沥
E胡[沥
E

鞍点量为正的鞍焦点的同宿分岔:马跟的存在性

i
形的歌焦点的同宿转.英鞋焦点的鞍点量为正,则在同宿尬的任意
E

定理2.2证明的恺路我们将通过验证同宪承的Poincare映
E小技
0
河
t江t莲
为O,集合4UAMtz一0}有可数个途通分岔,我们考虑那些位
Page-445
E霓五量空间中双曲幔点的同宿分岔

于f一0}上方的连通分岔的原象,这些原象都是曲边短形,晓射4
E

E
n河
E达

止一z一ay
{夕二az十加,
c

E沥

E扬
E仪招加罡

鸡鬣蚯疃(枳'z)E标玟小沥招雇))Y《C2.12)

E

[林二浩[林亚二阮
由(2.12),象肥叩碗是厂_上的以(0,0)为极点的对数嫖线.E
到4是一个微分同胡,故象t=4…〖'鲳叩碗是厂十上以点诙
Page-446
1具有三个实待招信的靼点的闭宿分盅

c麾鸭(0,0)iEE

掸'E丨蓥亡卜d途n

注意到|ov|<<1,我们有
E
由(1.10)有
En2n北a
(l,14)与(L10)一起推出
E招
以及
l0
EcptenptyyesR圆7E沥河
兰上满足第四章定理1.5的所有条件.因而4在De内有晴--双
D不胺
沥
E朝
ss
【

E
p
t吊2河述

EE
[

E
〈蔷邈十|园

b玲
Page-447
0招207

E沥0一如
Eeln浩ga心A
园述
述i心t河河e浩
二是一个曲边矩形.它的上下边属于直线{z一0}的原象,我们晤
医育c江连Es

orml-羯<oosernsaml-献|
0

【<‖腑胖川4肛丿CiexpE814
_m_|

这里b,z)1表示炭(0,m)到点8一(0,0)之间的贸高-为了证明上
述两不等式,在(b,史平面上对点(9,2)EZ我们如此定义其象点
E
E沥

[河0

EE
由212),有

[\羞lnz十0GD.

E
许汀

E牡羞』Z牡十O〈1)'[
E

二2nx十O(D。

由上式便推出(2.13).注意到
Page-448
EE

0
E吊7
2
再设e>0.因为4z(o,0)一e,故存在一个依赖于参数e的常

E

4z(y,z〉>青,对0<z<Cg′0
E
E<
〈z(二v,言)〈一歹<0′1

E

p
E沥
E
E园
i
EEEe根p
似,我们可以证明映射在De内有唯一双曲不动点.定理证毕,

E沥洁EtoLorkl

我们将R中具有一对复特征值和一实特征值的双曲鞍点称
E
假设鞋焦点的一对复特征值具有负实部,而它的实特征值为正数,
E一
实部的和称作鞍点量.鞋点量为正或为负的鞍焦点所对应的同容
分岗有着本质的不同.当鞍点重为负时,分盆与本章81中相应的
情况类似1而当鞍点重为正时,在鞍焦点的同宿轨的任意邺域中定
Page-449
i

a
医招
D

E怀国a沥木
L口

E国ttutfn
En国沥河
E标p汪

E沥伟t河
E一
林工初
E

E

52林沥E2东衍7
E

0
一sin(9十.等式(2.17)两边对9

里鱼暨

″zd朦
E江
团zd″十z〔]友一d″〕

d三51里1里
胁〔E″汊/【命'′z〕

E沥玟2余22
将(2.18)除以z,得

0国吴不巳
丽〔″/′(z)+″zE有z/′(z)十'′]

[
Page-450
不2圭闹中驶焦点的同街分彻E

界
5二

具有负鞍点量的鞍焦点的同宿分岐

本节第一个主要结果如下:

定理2.1令X是R中一个一舫的单参数向量场族.假设当
uapa2ssp
的同密轨7,则存在YUO的邻域D和参数空间中零值的邹域7
E途余tt
玟
Ec

在定理21的陈述中“一舫“词的含义是

(向量场X的鞍焦点的特征值非共振:

052技t

Eagtulte

定理2.1证明的直现含义

在给出严格证明以前,为便于读者理解,我们先给出证明的直
观揩述,为此我们将尽量把问题简化-首先我们假设,在鞍焦点的
Rst

E
c
0
EE
Et
E
8二外门I:Croy:切一(000D0.

E

《C2.3)

会
E
E2东林t
Page-451
E209

El+量/′彻苡。E

注意到当z令0时,量令/′E
E述[
蔫E

再由(2.13),当nrvoo时有z*0.引理证毕,【
E

E{(廖yz)E缙【〕}

E技东技余a技
Anitz一0}-四为鞍点量c一X十/2>0,故当xS>1时

E懈【〉EeXp〈脊了r).

由不等式(2.13)和(2.14),Poincar映射4限制在区域D上类似
一个马路映射,参见图5-5.下面证明Dof1ADn确实包含了一个马
eoiiretainas
理3.2的所有条件.首先,我们证明映射4在区域Dfi14““Dn上
Page-452
bE

E

E国a贾d
Eu沥u7命寸河
E江江诊

E

医e[彗E

E林口许莲人/
EE圆国

因为向童〔盂〕是矩阵靥的特征向量,故

a12
0

[责lnz一>婶。(modEn浩脱白4

由区域D的定义,(2.21)中第一个极限是显然的.现在我们证明
E,故4(6,z)EZ门A由
《2.13)和(2.14)有

E朋
L
E刑

现在我们来计算雅可比阵D4.对(2.12)求微分得

Esin'聋〕^′

sin8cosp

DQreg(a)[

|

E(

E.
门「E帼加仍]_
E圆
因此,由链式法则有
EiC3D4【噜(4血疃(仇z))*D4酶叩(仇z〉
Page-453
208第五章空间中双幼鞑怒的问宿分岖

Elgzgi技怀汀
起
E
E5江述汀
Edattuydumdeit
fz一0}上出发的正半扬线的并组成.我们用4fog表示泳系统
《2.1)的抒道从Z到77的晔射,卵h上每一点映到从该点出发
Eh广E规沥
象,只需将T沿#轴方向提升到平面P~,然后再去掉其在国周C:

[i羞之外的部分'有两种情况需要考虚
0沥不c应梁
0怀2东n育梁
见图5-3.

E
在定理2.1中只考虑第一种情况.根据假设同宿执逊结T+的点4
沥林秉北扬水92
不
e
是一个平移:
El5江许3
Page-454
E

=疃/=〔

a凸)(雅巴
4

E

cosg“一sing

E河5
国E
0麽jE(夕'z))〔蚯n辫匹春

5

E
一DM}'
e
E
玲
E途

二之max(BH-弓a1BM71E
以及

maxt|取一MD4|

E目
成立.由(2.217,对(9,z)En,当z“*oo时有~应或一应.故由
0

1

国医

注意到当w-voo时,8~~0,我们有
团E
1BMT|E

Edp
ua

〈韫L,″》1'E

E

n
b沥述沥

最后来验证,4在Df14““D上满足边界条件和相交条件-
Page-455
沥20

E林口沥
E招
E2
E述e口43
Eean
E技
Ed一

定理2.1的证明

我们将证明分成儿步
0标p江
由假设1,应用第一章定理4.27,在鞍焦点的一个邹域[里存
在一个坐标系(z,y,z),使得X:在0内是线性系统
E3朋
哥E怡103E
O
这里A(0)一0人x(0),e一(0)十pC0一0如果必要的话,可对
吴
人史不失一舫性,我们还可进一步假设,当e一0时,[中正轻是
同宿技的一部分、令
i
【2林李圆[朋
厂二{Cruy:切|与十万人1一1
酝c木
z(震)Eexp(入(s)z)〔c。s(m(s)′)堑gsin(m(龌)!)y〕'
庆XCIO3ENCYG3河许2
00
从上面第三个等式解得执道到达7所需时间为4气一xCe)-:Inz.
Page-456
gi

Etiessiur
医
上下两边属于D的水平边界D的原象4“8%Du,因而HY,F
E

E沥莲仁【
注意到43yHY人%Do,由(2.26)有42HYn(HYUH一幼.由
E沥
E
娆立.'
上面的论证表明,对所有充分大的自熊数,映射4在区域
E磁匹ssta
拓扑共轲.图而4在Ln上有无穷多个周期点.因而向量场在同宿
辅附近有无穷多条双曲周期转.

定理证毕【

转道等价不变量的存在性

医
E达

沥
E林(

表示与友等价的向量场的集合.称XE心“M)(关于等价关系
E沥春[

定义2.5一个复数C称为向量场XE““CM)的模,如果对
玟
上的一个非常值连续函数广漾足

E

E技X园4
Page-457
E第五章空间中双胡积点的问裂分岔

将其代人前两个等式,得到抒道与P的交点坐标Cz「,y「,1)满趸
[沥技c

E一A簧羞岫.E

E东一
0

E伟E

《2.6)
微分上式得到
Ez”_】【俊Sin辉〕E0办
E
此处

力工E矗忐(入(s)楂。SeBaEX3SX)

E1

0

E河

E朋

E
ERE沥口吴

F
[诊
E述王

E怀c玖
用5,8分别表示咏寇转Y与7r+和-的交点.将坐标系做一
E
E述[
E万圭c沥
M伟
园兰
Page-458
EE

E
E达
征根的比值即为模.因而,任意有孤立契点的向量场都不是结构稳
定的.再如,如果我们在拓扑轨道等价中要求相应同背保持时间,
则向量场的孤立周期转的周期便是模.因而,任意有学立周期轨的
向量场不是结构稳定的.显然,上述两种等价关系过于“严格“,使
国2志相
道等价(见第一章定义1.7)下,研究结构稳定问题和分岔问题.寻
找有模的系统是分会理论中一个很有意义的问题

E
Etstusurohgni
t

n
明,是辐道等价不变量.在下而的讨论中,我们仍利用前面的
E

E沥a江0

A(m二min代以N,Zn门A世灭)

E仪江(余

n(吴一pmaxfm仪月,五仁A央如]
E

吴弘
EEE7

b
E
E
E仪
园

根
Page-459
国摄E

E途c
d

00200兰7
则假设(2)的确切含义为

基z佑〉a

E.
E
E用
0
E林a
E林
u

由(2.8)'当/【充分小时,鸟是」个压缩常数小于噩的压缩映射一

河沥
E胡[沥
E

鞍点量为正的鞍焦点的同宿分岔:马跟的存在性

i
形的歌焦点的同宿转.英鞋焦点的鞍点量为正,则在同宿尬的任意
E

定理2.2证明的恺路我们将通过验证同宪承的Poincare映
E小技
0
河
t江t莲
为O,集合4UAMtz一0}有可数个途通分岔,我们考虑那些位
Page-460
E霓五量空间中双曲幔点的同宿分岔

于f一0}上方的连通分岔的原象,这些原象都是曲边短形,晓射4
E

E
n河
E达

止一z一ay
{夕二az十加,
c

E沥

E扬
E仪招加罡

鸡鬣蚯疃(枳'z)E标玟小沥招雇))Y《C2.12)

E

[林二浩[林亚二阮
由(2.12),象肥叩碗是厂_上的以(0,0)为极点的对数嫖线.E
到4是一个微分同胡,故象t=4…〖'鲳叩碗是厂十上以点诙
Page-461
214E

Eugsss
n
由(2.13)和(2.147易知,如果

[

[仪一一
E
[

t

E一一述
李7
E

巳c巴
′e(″)<噜_AE

由(2.30)得
InC2E
E

E
E
幽丨nc】_,E园苴_国矶nC】

E′

E沥或t
E英
E
E木t沥p
伟

工一Aiz一ag,
纠沙人a

5
Page-462
0招207

E沥0一如
Eeln浩ga心A
园述
述i心t河河e浩
二是一个曲边矩形.它的上下边属于直线{z一0}的原象,我们晤
医育c江连Es

orml-羯<oosernsaml-献|
0

【<‖腑胖川4肛丿CiexpE814
_m_|

这里b,z)1表示炭(0,m)到点8一(0,0)之间的贸高-为了证明上
述两不等式,在(b,史平面上对点(9,2)EZ我们如此定义其象点
E
E沥

[河0

EE
由212),有

[\羞lnz十0GD.

E
许汀

E牡羞』Z牡十O〈1)'[
E

二2nx十O(D。

由上式便推出(2.13).注意到
Page-463
Eb
Dnudttiii

通过一个线性变换,我们可使中的点丸一(L,0,0)命于oi的同
E
c2罚月诊

e贺朝
E

s2玖命目医吊

E沥逊怀
园a伟理江
沥i一
E
tto022
月述
[
E达荣p江才
E河口
医ui吴行
E达沥振训
E河p2
的整数,使得
c

Ed

0俊

0沥玟达5万
E

吴
2

E国
Page-464
i

a
医招
D

E怀国a沥木
L口

E国ttutfn
En国沥河
E标p汪

E沥伟t河
E一
林工初
E

E

52林沥E2东衍7
E

0
一sin(9十.等式(2.17)两边对9

里鱼暨

″zd朦
E江
团zd″十z〔]友一d″〕

d三51里1里
胁〔E″汊/【命'′z〕

E沥玟2余22
将(2.18)除以z,得

0国吴不巳
丽〔″/′(z)+″zE有z/′(z)十'′]

[
Page-465
E

才c

552怀e
道兮,满足a(2一史和o(73一ai+lsi心0,1,...,&一1,斧构成的捉
e
E
双曲奇点和一条双曲闭转以及两条连结它们的抒道所构成的环.
值得指出的是,在著名的Lorenz方程中,对某些参数值上述环是
Eseaaizinlaktsppigsa达[d
E

环的分类
巴elc

E沥
0id
稳定流形“(oo.
0沥s沥
E
E沥c
E技t
(4)如果奇点r的特征值都为实数<<A一0一v则c是2阶
E
【志a怀c
Ep
E
E怀d立
满足上面四条假设的环4有三种不合的类型:
pt
Page-466
E209

El+量/′彻苡。E

注意到当z令0时,量令/′E
E述[
蔫E

再由(2.13),当nrvoo时有z*0.引理证毕,【
E

E{(廖yz)E缙【〕}

E技东技余a技
Anitz一0}-四为鞍点量c一X十/2>0,故当xS>1时

E懈【〉EeXp〈脊了r).

由不等式(2.13)和(2.14),Poincar映射4限制在区域D上类似
一个马路映射,参见图5-5.下面证明Dof1ADn确实包含了一个马
eoiiretainas
理3.2的所有条件.首先,我们证明映射4在区域Dfi14““Dn上
Page-467
E腾sE

Essss水
示向量场乃限制在4的棠个邻域U内的非游荡集,这时A称为
E不

E
E

上面三种类珩的环可参见图5-6.下面我们首先证明鞍焦环和
E不述
量场在一般的单参敷扰动下的分岔y还没有被研究渭楚'本节的最
Ed

鞍焦环

定理3.1设光滑向量场丿有一个鞍焦环4,漾足假设(1)一
2

证明定理结论的儿何意义可参见囹5-7-

形沥沥tn述林
E

0

E

02a光

EO扬2

E
E在该邻域里可由下面线
E

宋心矩一a
卡江5[沥
c
吴5江应才
E沥
E命巳匹
Page-468
bE

E

E国a贾d
Eu沥u7命寸河
E江江诊

E

医e[彗E

E林口许莲人/
EE圆国

因为向童〔盂〕是矩阵靥的特征向量,故

a12
0

[责lnz一>婶。(modEn浩脱白4

由区域D的定义,(2.21)中第一个极限是显然的.现在我们证明
E,故4(6,z)EZ门A由
《2.13)和(2.14)有

E朋
L
E刑

现在我们来计算雅可比阵D4.对(2.12)求微分得

Esin'聋〕^′

sin8cosp

DQreg(a)[

|

E(

E.
门「E帼加仍]_
E圆
因此,由链式法则有
EiC3D4【噜(4血疃(仇z))*D4酶叩(仇z〉
Page-469
经
E
[
区
意
国
蟹
梁
国
[
区
E
E
团
E
Page-470
E胡2

E

E命口

E沥林圆一不
(3'1)所导出的从碲E
E沥力

d不辽江园3
E215顺05沥

这里a=A/A<0,婶=_责bz′E

【芸
转道丿连结点加与点,由方程的解对初值的依赖性定理,每一条
E林东
宇廷明小l育0朋
c一
截相交,故曲线WPr(oi)V可以表示成
E52沥一=人罗0才、

E达
Page-471
E

=疃/=〔

a凸)(雅巴
4

E

cosg“一sing

E河5
国E
0麽jE(夕'z))〔蚯n辫匹春

5

E
一DM}'
e
E
玲
E途

二之max(BH-弓a1BM71E
以及

maxt|取一MD4|

E目
成立.由(2.217,对(9,z)En,当z“*oo时有~应或一应.故由
0

1

国医

注意到当w-voo时,8~~0,我们有
团E
1BMT|E

Edp
ua

〈韫L,″》1'E

E

n
b沥述沥

最后来验证,4在Df14““D上满足边界条件和相交条件-
Page-472
220E摄t

EEokt沥s河木洁技5
螺线

【0沥
上述曲线属于何扬r的不穗定流形WP“(o)与平面5的交.下面将
通过验证曲线(3.2)与曲线P“CotfS:z一0)有横戳相交点来证ˇ
明定理3.1.令F一(Fo,Fy),则我们只需诛明方程

y(友(8(a),驯,丁(b(a),z))一0医

a

国史国一

园E,

那么(3.3)可写成
Ed2述3

园,
EE林E刑

E
E李玟人应玟圭玟人河
E
E江2仪才步2河
的函数加上一个小重.因此(3.4有无穷多个根s一0.下面证明
E江应

u水

0

这里当&一oo时,o(1)一0.另一方面
EA2育圆林6育切
_"′(z′)I0亩75+_苏_F万E

团_E江述途
【羞)〕玟沥伟吴23丞通

罡
Page-473
gi

Etiessiur
医
上下两边属于D的水平边界D的原象4“8%Du,因而HY,F
E

E沥莲仁【
注意到43yHY人%Do,由(2.26)有42HYn(HYUH一幼.由
E沥
E
娆立.'
上面的论证表明,对所有充分大的自熊数,映射4在区域
E磁匹ssta
拓扑共轲.图而4在Ln上有无穷多个周期点.因而向量场在同宿
辅附近有无穷多条双曲周期转.

定理证毕【

转道等价不变量的存在性

医
E达

沥
E林(

表示与友等价的向量场的集合.称XE心“M)(关于等价关系
E沥春[

定义2.5一个复数C称为向量场XE““CM)的模,如果对
玟
上的一个非常值连续函数广漾足

E

E技X园4
Page-474
医胡e

5兽土z…『】(肃E6

上面最后一个等式是由(3.4和(3.5)推出的,定理证毕,〖

E

E东志2us怡a租沥河
Et东5i丞志朐

E国英
E
E怀0

源万水P。擅黜崖E

医沥a兰一公i
邹基可以定义闭轨口的Poincare映射九因为@点是一维晔射P的
E00招
E

E

0

0林

E4芒

[
E沥
E
国
E
玟
令
E
Page-475
EE

E
E达
征根的比值即为模.因而,任意有孤立契点的向量场都不是结构稳
定的.再如,如果我们在拓扑轨道等价中要求相应同背保持时间,
则向量场的孤立周期转的周期便是模.因而,任意有学立周期轨的
向量场不是结构稳定的.显然,上述两种等价关系过于“严格“,使
国2志相
道等价(见第一章定义1.7)下,研究结构稳定问题和分岔问题.寻
找有模的系统是分会理论中一个很有意义的问题

E
Etstusurohgni
t

n
明,是辐道等价不变量.在下而的讨论中,我们仍利用前面的
E

E沥a江0

A(m二min代以N,Zn门A世灭)

E仪江(余

n(吴一pmaxfm仪月,五仁A央如]
E

吴弘
EEE7

b
E
E
E仪
园

根
Page-476
第五蜕空间中双曲粤炎的间宿分盆

E技
E
医沥L
E沥2技

E

国t

由假设(43,我们可以在鞍炒%的一个邻域外中选取一个C!
E河t、

0沥

2向量场X由下列线性系统给出

E
{仪E林

5

E

E

沥

E沥22技林江一“
E

E
Euztse河E唐芸
有形式
[

E

C卫到广的映射

E
E沥命林沥东
Page-477
214E

Eugsss
n
由(2.13)和(2.147易知,如果

[

[仪一一
E
[

t

E一一述
李7
E

巳c巴
′e(″)<噜_AE

由(2.30)得
InC2E
E

E
E
幽丨nc】_,E园苴_国矶nC】

E′

E沥或t
E英
E
E木t沥p
伟

工一Aiz一ag,
纠沙人a

5
Page-478
E脱y

Egoam河y
【a02673
由假设(3)流形P*(ou)与W“(oo)沿.朱截相交,故有

E

面a

囹此,可选联充分小,使得.
E林朋

a

颜e

[一林许
E木
技__

E
E_

闵0E

[

浩′)姜0,阊[

曦>0和碗<o分别对应非游荡环和游荡环′

E膏e江许沥
令
砺一G人HKn一PeK8口丞n
Ed吊洁玟
〈愫”:F”配岫匹
令
@二{ED|Rral十切一5丢不叶3G一动j
Page-479
224l

则有
[吴2

令
E
则映射4在区域DYUDY上是马蹄映射,见图5-8-

E林国胺
E
E
Es2
b沥
[6t32

因为(&,EQn+a,故
大[沥

0=壬

E
Page-480
Eb
Dnudttiii

通过一个线性变换,我们可使中的点丸一(L,0,0)命于oi的同
E
c2罚月诊

e贺朝
E

s2玖命目医吊

E沥逊怀
园a伟理江
沥i一
E
tto022
月述
[
E达荣p江才
E河口
医ui吴行
E达沥振训
E河p2
的整数,使得
c

Ed

0俊

0沥玟达5万
E

吴
2

E国
Page-481
E浩2

EE
E

E

E
EE

命砌)门
明
玄剔<破<2j腥′y对一余1

这与(3.11)一起推出
4

(

现在我们计算雅可比阵D4.由链式法则有
E沥3

孕
E[0

E

E

y

E
。潺〕e达一

园东林
E【(z抛+叮抛%
Page-482
226d

E江
maxtlDRF[IDPF-ia1BM-|一,(8.13)
E一沥

E东
max(|4|“|取M1D|,14一BMTD|,
c[浩5
E水
E途沥沥d丿厂2余技沥0办育3
E告蔚一婶(二菖…厂】z箐E咿彗a
1沥
国5
E
E江当鹫+o时5

M[一肉RmFTmar-ece驿土心骆0一o,

t夏_离一-|(一)一】(一)|一0.

E
林李
D,
[一木标
E达

1
P雇_″〔E谚〕崛翼。】肯互血+”

(1+盘_市湟…」+…_〉玮0y门
园技沥沥圆
Page-483
E

才c

552怀e
道兮,满足a(2一史和o(73一ai+lsi心0,1,...,&一1,斧构成的捉
e
E
双曲奇点和一条双曲闭转以及两条连结它们的抒道所构成的环.
值得指出的是,在著名的Lorenz方程中,对某些参数值上述环是
Eseaaizinlaktsppigsa达[d
E

环的分类
巴elc

E沥
0id
稳定流形“(oo.
0沥s沥
E
E沥c
E技t
(4)如果奇点r的特征值都为实数<<A一0一v则c是2阶
E
【志a怀c
Ep
E
E怀d立
满足上面四条假设的环4有三种不合的类型:
pt
Page-484
不3环的分盆

颠L

沥
图
E林7

【关许

[怀沥c技胡

《3)边界条件,4aeDf门D?一史,ADFM13,D9一奶心j一1,2.

a
盘的一部分,而它的上卞两边分别是Q″+z上下边在映射4下的原
象;因而是18水平曲线,故条件(1)成立,其次,相交条件不难由述
E东E伟行

【述

交,故49.DFTIDY一亿.另一方面,注意到

E22沥
E训沥扬达江王月

E

E|

游茆环

4
所发生的分岗.设光滑向最场X有一个游荧环4,满尼假设(0)一
E江林n仪
和闭轨u是双曲的,故向量场Xe在ca,oi及软道4附近有双曲奇
沥
E7朐

定义3.4称参数值e为向量场族Xe的同宣分岔值(环分岔
Page-485
E腾sE

Essss水
示向量场乃限制在4的棠个邻域U内的非游荡集,这时A称为
E不

E
E

上面三种类珩的环可参见图5-6.下面我们首先证明鞍焦环和
E不述
量场在一般的单参敷扰动下的分岔y还没有被研究渭楚'本节的最
Ed

鞍焦环

定理3.1设光滑向量场丿有一个鞍焦环4,漾足假设(1)一
2

证明定理结论的儿何意义可参见囹5-7-

形沥沥tn述林
E

0

E

02a光

EO扬2

E
E在该邻域里可由下面线
E

宋心矩一a
卡江5[沥
c
吴5江应才
E沥
E命巳匹
Page-486
经
E
[
区
意
国
蟹
梁
国
[
区
E
E
团
E
Page-487
228i

朐东仪uy5
Ecouysi应招3李
E力

园
E
沥扬
E人
i
医
E才
E汪

(双曲周期转e)有一条横截同宿转:

(2)集合Bc的闭包B是一个包含s一0的Cantor集,且满足
yogpz吊浩s途2

E述一余21
道的双曲不变集},
E

E半m(〈0<s<叶s途2
D巳

这里_m(“)表示RI上的集合(.)的Lebesgue测度.

Ebb
E述gpy泓
utyc林e
E

医ep标育述水E技
E吴

E林s

10

定理3.5的证明
Page-488
E胡2

E

E命口

E沥林圆一不
(3'1)所导出的从碲E
E沥力

d不辽江园3
E215顺05沥

这里a=A/A<0,婶=_责bz′E

【芸
转道丿连结点加与点,由方程的解对初值的依赖性定理,每一条
E林东
宇廷明小l育0朋
c一
截相交,故曲线WPr(oi)V可以表示成
E52沥一=人罗0才、

E达
Page-489
E踊c
证明将分成儿步进行

团腾l

医林林仪达园技
E林东
参数5的有限光滑坂标卡(x,o),使得

【690途胡2

(2){Goo|jx|,|a|妇3人(@),闭辅e)的Poinears晖
射在UCQ)上有形式

Ent2有
这里0一XCe)心1<<RCej
E林572东沥
E
E2一人c
和

E
E李
E

p

因为鞍点c非共振,故在c的一个邻域仁中,可以取一个依
E述
E
并且向量场族Xe有形步
E
河
E
E从喊<唰翰<0<″胤E
El),
Page-490
220E摄t

EEokt沥s河木洁技5
螺线

【0沥
上述曲线属于何扬r的不穗定流形WP“(o)与平面5的交.下面将
通过验证曲线(3.2)与曲线P“CotfS:z一0)有横戳相交点来证ˇ
明定理3.1.令F一(Fo,Fy),则我们只需诛明方程

y(友(8(a),驯,丁(b(a),z))一0医

a

国史国一

园E,

那么(3.3)可写成
Ed2述3

园,
EE林E刑

E
E李玟人应玟圭玟人河
E
E江2仪才步2河
的函数加上一个小重.因此(3.4有无穷多个根s一0.下面证明
E江应

u水

0

这里当&一oo时,o(1)一0.另一方面
EA2育圆林6育切
_"′(z′)I0亩75+_苏_F万E

团_E江述途
【羞)〕玟沥伟吴23丞通

罡
Page-491
第五章空间中双曲数点的间宿分盆

E

Et动
E

而转道与7+相交于丿点

吴加月动二(Cyom二Ooo0,|y|一1

Ey技

T有形式

【net3月
E

团吴1C助吴刹
D″(s)〈″〈s)_m

25河u

Epotauastests芸e河沥3沥北砂河
Enetiip
0t200吊3罡03
由假设(3),不穗定流形W7“(ot)与稳定流形WP“(oo)泰7模截相交,
E

[

G

t

Epoeutusduesle育0
E描
Et23
E林技
E
Page-492
医胡e

5兽土z…『】(肃E6

上面最后一个等式是由(3.4和(3.5)推出的,定理证毕,〖

E

E东志2us怡a租沥河
Et东5i丞志朐

E国英
E
E怀0

源万水P。擅黜崖E

医沥a兰一公i
邹基可以定义闭轨口的Poincare映射九因为@点是一维晔射P的
E00招
E

E

0

0林

E4芒

[
E沥
E
国
E
玟
令
E
Page-493
E脱p
游草环对应情泓do<<0.

Eyy

假设(6)的确切含意是

E
张[n

E

Ey剧

F<0时结论的证明
对e<0,我们有
E巴

Ere2颐仪′E苔歹z′

a

国s+票燮z″十誓剽=E吊265李人

E
EairpsueetateyyE河春
s
所有的执道都将离开A的邻域

G结论(1)的证明

现在证明,当e>0时,闭执ci(e)有横截同宝转.这一结论从图
沥
E志改s江仪
E河
的在o处与丶,仁的特征方向相切的不资流形)在7。上的任意
处相切,故由假设(43的(a),它与流形WV“Co泪7模截相交.另一
Epn李一
n
E
Page-494
第五蜕空间中双曲粤炎的间宿分盆

E技
E
医沥L
E沥2技

E

国t

由假设(43,我们可以在鞍炒%的一个邻域外中选取一个C!
E河t、

0沥

2向量场X由下列线性系统给出

E
{仪E林

5

E

E

沥

E沥22技林江一“
E

E
Euztse河E唐芸
有形式
[

E

C卫到广的映射

E
E沥命林沥东
Page-495
空间中双尔敏炉的闭宿分吴

E国Lt

为了证明结论(2),我们需要下面的
引理3.7存在正规同宿分岑值序列哎和正规环分岔值序列

E

a

Ek一p2
Elan
d
Page-496
E脱y

Egoam河y
【a02673
由假设(3)流形P*(ou)与W“(oo)沿.朱截相交,故有

E

面a

囹此,可选联充分小,使得.
E林朋

a

颜e

[一林许
E木
技__

E
E_

闵0E

[

浩′)姜0,阊[

曦>0和碗<o分别对应非游荡环和游荡环′

E膏e江许沥
令
砺一G人HKn一PeK8口丞n
Ed吊洁玟
〈愫”:F”配岫匹
令
@二{ED|Rral十切一5丢不叶3G一动j
Page-497
224l

则有
[吴2

令
E
则映射4在区域DYUDY上是马蹄映射,见图5-8-

E林国胺
E
E
Es2
b沥
[6t32

因为(&,EQn+a,故
大[沥

0=壬

E
Page-498
[贺c

此,曲线Ci可表示为
〉[国A吊502达(C3.19)
标r罪
技
A《C3.20)
因此,如果
吊2
野
巳A友At河
[述)
E沥
EAz不园
颂水沥
噩町徇1_皇v懈〉

p

E(E)‖〈e〉十翌」夏(罄)'″′(E)十誓〉〕
E,【
E国述25顺朐吊a标
y
2团
EY5

-[
E胡

23seet5
的重要附注
E怀政仪
Page-499
E浩2

EE
E

E

E
EE

命砌)门
明
玄剔<破<2j腥′y对一余1

这与(3.11)一起推出
4

(

现在我们计算雅可比阵D4.由链式法则有
E沥3

孕
E[0

E

E

y

E
。潺〕e达一

园东林
E【(z抛+叮抛%
Page-500
E第五定空间中双兽敲点的名宪分岑

Euatnsy命t2吊河
e园s切
EL0芸河吊林命|
速度接近及离开曲线Pef15.

pgr浩
国
D

b余林22明Iea3y

这里(0,0)二0,四目哪>0.E(E))n
沥伟林一

EeE
医技

5一′l〈仅辜聂yE)0
令(oo,0)是点兮门7的土标,我们对方程(3.Z们庄黄(y试黯)_
Ey

z=互(汊,s)歹零嬗屿赋E
我们断言,函数东有下面的性质
Page-501
226d

E江
maxtlDRF[IDPF-ia1BM-|一,(8.13)
E一沥

E东
max(|4|“|取M1D|,14一BMTD|,
c[浩5
E水
E途沥沥d丿厂2余技沥0办育3
E告蔚一婶(二菖…厂】z箐E咿彗a
1沥
国5
E
E江当鹫+o时5

M[一肉RmFTmar-ece驿土心骆0一o,

t夏_离一-|(一)一】(一)|一0.

E
林李
D,
[一木标
E达

1
P雇_″〔E谚〕崛翼。】肯互血+”

(1+盘_市湟…」+…_〉玮0y门
园技沥沥圆
Page-502
E

为了得到(3.28)中后两个估计,我们对(3.25)分别关于3和求
医

EE
J
i

伟韶z婶丨nz一萝誓(瞰一l

医沥扬EE
E
0吴3[沥
E才
E江3育
医

E

59FE

EE
哥O(s『′〉D
令Cn一P-oCS,则曲线C,n上的点(ovo7满尸

azcynAluceaE
因为曲线C.n腾于闭轨ri()的稳定流形W“(o(6)),故若点
E沥3国
Page-503
236第二章宇间中双幼敌炳的闯分盆

E技b丶林江
林
哥EAcuAC1IC7
令<<1是一个小正数,滔足

万E
E
E

](1溃)浑″[

日

工〕(l十谬)E技

E
分大时有

E【
E沥
证明史是单根,即

0沥

〔″雇(氨)一“「1业王璧v八“庐)O切〔″互(E)″_】E

EP

E河_李

十X)EE“=又(罄)_″(星)yE=s二
E

EPE
n兰园

由(3.33)可知,如果8S1,则有
D
00E吴s辜二

医一口

i
Page-504
不3环的分盆

颠L

沥
图
E林7

【关许

[怀沥c技胡

《3)边界条件,4aeDf门D?一史,ADFM13,D9一奶心j一1,2.

a
盘的一部分,而它的上卞两边分别是Q″+z上下边在映射4下的原
象;因而是18水平曲线,故条件(1)成立,其次,相交条件不难由述
E东E伟行

【述

交,故49.DFTIDY一亿.另一方面,注意到

E22沥
E训沥扬达江王月

E

E|

游茆环

4
所发生的分岗.设光滑向最场X有一个游荧环4,满尼假设(0)一
E江林n仪
和闭轨u是双曲的,故向量场Xe在ca,oi及软道4附近有双曲奇
沥
E7朐

定义3.4称参数值e为向量场族Xe的同宣分岔值(环分岔
Page-505
E源s237

E|庭(s)一″s_霹撞』=g

_雇(辱)园

“_X(鲁〉″″(E)E

0
严(

″一I卯1
E一PE
〈1)[雇(0)一1凌)_膘互(″(o)E
二0[RKo)一薯…锦」〕″一0
E
[elt
芸工河口一

巳A0(l)燮羞'[沥
j28技公扬知汀江
E
E标
E仪0河达0
Eudttucsucayuny

E0(1)灭(莓)_″(铭)羞.

[
一OCDR(s-at加n
E

i
Eey
E
Page-506
228i

朐东仪uy5
Ecouysi应招3李
E力

园
E
沥扬
E人
i
医
E才
E汪

(双曲周期转e)有一条横截同宿转:

(2)集合Bc的闭包B是一个包含s一0的Cantor集,且满足
yogpz吊浩s途2

E述一余21
道的双曲不变集},
E

E半m(〈0<s<叶s途2
D巳

这里_m(“)表示RI上的集合(.)的Lebesgue测度.

Ebb
E述gpy泓
utyc林e
E

医ep标育述水E技
E吴

E林s

10

定理3.5的证明
Page-507
第五章空间中双曲鞍灰阡同街分动

E赖b62p

由引理3.7,集合Bw和Be非空.对任意e,EBevt0}由假设
〈3)和结论(1),流形IP*Coo(so)和IPr(oCe))与流形W*(ou(ey)》
Ece浩水
E东
明B是Cantor集.为此只霁证明它不含任何开区间.令s,EB,
则系统丿的奇点%(e)有一条咖宿转.因此,对e的紫一侧e,附近
E
应
E

了结论(3)的证明

ENC2

2C:
一0.
河子

E庆
医吴cnb招

到区域Do.令Ku表示G与C$之间的区域,
医一(Guoo)P乐一付GusJ,0丿4一酊.
令

EstcR2一吊一李李胡
巳
E一3
s吴育
医不t园

r吴
则由口,咤的渐近表达式可得a一.令Dn表示区间(8n,气.对
Page-508
E踊c
证明将分成儿步进行

团腾l

医林林仪达园技
E林东
参数5的有限光滑坂标卡(x,o),使得

【690途胡2

(2){Goo|jx|,|a|妇3人(@),闭辅e)的Poinears晖
射在UCQ)上有形式

Ent2有
这里0一XCe)心1<<RCej
E林572东沥
E
E2一人c
和

E
E李
E

p

因为鞍点c非共振,故在c的一个邻域仁中,可以取一个依
E述
E
并且向量场族Xe有形步
E
河
E
E从喊<唰翰<0<″胤E
El),
Page-509
E颖

于eEUn_-1,我们定义由网二KoULn刨万的映射如下。
E东

E

E

令
E
则4在ZrUZn上看上去像一个马踹眺射.参见图5-10.

1Eoale
E一
沥命史诊2
不二三

7
s

Enu
E才d
E
<1,它就满足(/avo)锥形条件.对于(uoo)E,有
Ezn
在利用第四章引理4.1来验证4源足锦形条件时,所有运算
与引理3.3的证明中完全一样,从那里可以看出我们只需验证
Page-510
第五章空间中双曲数点的间宿分盆

E

Et动
E

而转道与7+相交于丿点

吴加月动二(Cyom二Ooo0,|y|一1

Ey技

T有形式

【net3月
E

团吴1C助吴刹
D″(s)〈″〈s)_m

25河u

Epotauastests芸e河沥3沥北砂河
Enetiip
0t200吊3罡03
由假设(3),不穗定流形W7“(ot)与稳定流形WP“(oo)泰7模截相交,
E

[

G

t

Epoeutusduesle育0
E描
Et23
E林技
E
Page-511
玟

[
E东不沥木
E园

E一
E
因此
E训

s
国C羞一1雇(s)_″〔亘(0)(1一羞)(1+羞】〕一】
r
这里r一F(o30-劳Q+声/FCe.注意到
E
故3.40)成立.引理证毕,【

s浩t
E=′gz4塞(乙;Uucopeomeeie

ELn
E述
E河i
E达n吴s余2n
E

E
E

E
c

E
Page-512
E脱p
游草环对应情泓do<<0.

Eyy

假设(6)的确切含意是

E
张[n

E

Ey剧

F<0时结论的证明
对e<0,我们有
E巴

Ere2颐仪′E苔歹z′

a

国s+票燮z″十誓剽=E吊265李人

E
EairpsueetateyyE河春
s
所有的执道都将离开A的邻域

G结论(1)的证明

现在证明,当e>0时,闭执ci(e)有横截同宝转.这一结论从图
沥
E志改s江仪
E河
的在o处与丶,仁的特征方向相切的不资流形)在7。上的任意
处相切,故由假设(43的(a),它与流形WV“Co泪7模截相交.另一
Epn李一
n
E
Page-513
E河e

[
湟够0′l+玄焊;

医沥4林c4技丞
命1
E林
1沥沥

E0(])亘(0)_u+圭】″,
故

E()(1〉雇(o)一“+竟‖.
D

男一方面
B

E

因而

R3

-【量

Hm

E[〗(′″)〕】+硐

EE

凸(3.43)不难导出(3.42).定理证毕,〖

E述a不一c
的敌点量为负,则使系统X具有稳定周期转的参数值集是一个
边界包含原点的开集

E
邰域UCeo),使得对<EU(e2vteo},Xe有一个稳定周期轨.另一方
E

E刑e兰s
论也成立,读者可参见[Sil2,3]和[ILJ

E
Page-514
空间中双尔敏炉的闭宿分吴

E国Lt

为了证明结论(2),我们需要下面的
引理3.7存在正规同宿分岑值序列哎和正规环分岔值序列

E

a

Ek一p2
Elan
d
Page-515
[贺c

此,曲线Ci可表示为
〉[国A吊502达(C3.19)
标r罪
技
A《C3.20)
因此,如果
吊2
野
巳A友At河
[述)
E沥
EAz不园
颂水沥
噩町徇1_皇v懈〉

p

E(E)‖〈e〉十翌」夏(罄)'″′(E)十誓〉〕
E,【
E国述25顺朐吊a标
y
2团
EY5

-[
E胡

23seet5
的重要附注
E怀政仪
Page-516

\part{第六章“实二次单峥映射族的吸引子}


从60年代兴起的动力系统的现代研究,其中心课题之一是双
曲理论.一个系统(流,微分间胚,胥射)是双曲的,如果它的极限
集是双曲的,即极限集中所有转道的Liapunov指数一致非0(在讨
论洪的双曲性时,不考虑沿流方向的Liapunow指数,它为0.从那
E东
E技标
古外,当时人们曾相信动力系统基本上是由双曲系统构成的.这一
看法的根本转变是由于在70年代受到物理、天文学等领域的一些
E林技人育y育沥林55尿技y命i招s
Lorenz(他们的工作在70年代开始才受到数学家的重视等人对
这些模型的算工作表明,这些模型具有极其复杂的动力学
行为,它们似乎不具有双曲结构,相反它们应当履于非双曲苑畴

河
Jakobsonp在80年代初取得的,他证明了对实二次映射族的一个
E途c北二江标
E
2
Jakobson的结果和方法,并在此基础上,讨论了Hnon映射在具有
Ett
E江河dt
D
E沥
E沥E
对菪个系统有很好的了解(例如,Logistie映射族在a一2时HHenon
Page-517
E第五定空间中双兽敲点的名宪分岑

Euatnsy命t2吊河
e园s切
EL0芸河吊林命|
速度接近及离开曲线Pef15.

pgr浩
国
D

b余林22明Iea3y

这里(0,0)二0,四目哪>0.E(E))n
沥伟林一

EeE
医技

5一′l〈仅辜聂yE)0
令(oo,0)是点兮门7的土标,我们对方程(3.Z们庄黄(y试黯)_
Ey

z=互(汊,s)歹零嬗屿赋E
我们断言,函数东有下面的性质
Page-518
E

为了得到(3.28)中后两个估计,我们对(3.25)分别关于3和求
医

EE
J
i

伟韶z婶丨nz一萝誓(瞰一l

医沥扬EE
E
0吴3[沥
E才
E江3育
医

E

59FE

EE
哥O(s『′〉D
令Cn一P-oCS,则曲线C,n上的点(ovo7满尸

azcynAluceaE
因为曲线C.n腾于闭轨ri()的稳定流形W“(o(6)),故若点
E沥3国
Page-519
0243

E
E技
Eyy吴E
和证明方法.应当指出,这里介绍的证明思想在当前这一方向的研
究中是十分重要的.

E沥刑心余c

江
t

小不园i江园c浩江
e

20尿

0

005仪坂政EcdM

E沥技t

【UUD明志t林一北不

医沥怀e沥许玟东步2江
E亚招
b学述玲技振
E,有静噩荤阗/″P(仪)今zE

E

现在我们要讨论的问题;一个单峰昭射可以有多少条穗定的
E兰一i水椿中
玟
E

E春应述

国u河秉扬A
A浩河′(′)〕「
Page-520
236第二章宇间中双幼敌炳的闯分盆

E技b丶林江
林
哥EAcuAC1IC7
令<<1是一个小正数,滔足

万E
E
E

](1溃)浑″[

日

工〕(l十谬)E技

E
分大时有

E【
E沥
证明史是单根,即

0沥

〔″雇(氨)一“「1业王璧v八“庐)O切〔″互(E)″_】E

EP

E河_李

十X)EE“=又(罄)_″(星)yE=s二
E

EPE
n兰园

由(3.33)可知,如果8S1,则有
D
00E吴s辜二

医一口

i
Page-521
E源s237

E|庭(s)一″s_霹撞』=g

_雇(辱)园

“_X(鲁〉″″(E)E

0
严(

″一I卯1
E一PE
〈1)[雇(0)一1凌)_膘互(″(o)E
二0[RKo)一薯…锦」〕″一0
E
[elt
芸工河口一

巳A0(l)燮羞'[沥
j28技公扬知汀江
E
E标
E仪0河达0
Eudttucsucayuny

E0(1)灭(莓)_″(铭)羞.

[
一OCDR(s-at加n
E

i
Eey
E
Page-522
244E

E不医伟

[沥明北

【明志s

【仪

0李伟22

《Se一(0)一0.

E林M国育u沥招技t
E李a余

Eos国圆D河2河
的,此处不再赘述,,

2
Sg(z.

E沥技e伟2
【仪

3.|一|在(一1,1中没有正的局部最小.

E
E仪d
医芸2沥伟
d
2(天士1)是了的不动点并东|乜(97|一1,那么这个不助点至少
在一边是稳定的.

沥政2技
2沥王一
E途

如若不然,令g一户并设有无穷多=E,满足g(z)一z.由
t东
医e伟
E

5.如果a一&一c是8一八的相邻不动点,并且在区间[avc]
Page-523
第五章空间中双曲鞍灰阡同街分动

E赖b62p

由引理3.7,集合Bw和Be非空.对任意e,EBevt0}由假设
〈3)和结论(1),流形IP*Coo(so)和IPr(oCe))与流形W*(ou(ey)》
Ece浩水
E东
明B是Cantor集.为此只霁证明它不含任何开区间.令s,EB,
则系统丿的奇点%(e)有一条咖宿转.因此,对e的紫一侧e,附近
E
应
E

了结论(3)的证明

ENC2

2C:
一0.
河子

E庆
医吴cnb招

到区域Do.令Ku表示G与C$之间的区域,
医一(Guoo)P乐一付GusJ,0丿4一酊.
令

EstcR2一吊一李李胡
巳
E一3
s吴育
医不t园

r吴
则由口,咤的渐近表达式可得a一.令Dn表示区间(8n,气.对
Page-524
E颖

于eEUn_-1,我们定义由网二KoULn刨万的映射如下。
E东

E

E

令
E
则4在ZrUZn上看上去像一个马踹眺射.参见图5-10.

1Eoale
E一
沥命史诊2
不二三

7
s

Enu
E才d
E
<1,它就满足(/avo)锥形条件.对于(uoo)E,有
Ezn
在利用第四章引理4.1来验证4源足锦形条件时,所有运算
与引理3.3的证明中完全一样,从那里可以看出我们只需验证
Page-525
t

E李林

E
训林
Ehean怀租

一
E刑

E标东荣
E人一
t沥
E芸
0

[

[

E李

【
林
E

E
沥

木一
2
招

【d月仪育尘E

E伟
E
2
水
玟2
E5王
Page-526
玟

[
E东不沥木
E园

E一
E
因此
E训

s
国C羞一1雇(s)_″〔亘(0)(1一羞)(1+羞】〕一】
r
这里r一F(o30-劳Q+声/FCe.注意到
E
故3.40)成立.引理证毕,【

s浩t
E=′gz4塞(乙;Uucopeomeeie

ELn
E述
E河i
E达n吴s余2n
E

E
E

E
c

E
Page-527
E河e

[
湟够0′l+玄焊;

医沥4林c4技丞
命1
E林
1沥沥

E0(])亘(0)_u+圭】″,
故

E()(1〉雇(o)一“+竟‖.
D

男一方面
B

E

因而

R3

-【量

Hm

E[〗(′″)〕】+硐

EE

凸(3.43)不难导出(3.42).定理证毕,〖

E述a不一c
的敌点量为负,则使系统X具有稳定周期转的参数值集是一个
边界包含原点的开集

E
邰域UCeo),使得对<EU(e2vteo},Xe有一个稳定周期轨.另一方
E

E刑e兰s
论也成立,读者可参见[Sil2,3]和[ILJ

E
Page-528
246第六章宗二次单峰胸射旌的吟引孔

E关人一兰e招
乙,使得8(一1.如榔有yE(d,z),使得8(9》一0,则利用性
质6,吞则由情况(iit),我们完成了性质7的证明,并因此证明了定
E

定理1.3有以下儿个推论.

E江园
期点,加上在区间[一1,f(1)]中的一个可能的稳定不动点

E园2
玟
园达L8芸s
a口i
如果是第一种情况,z炜引[0,z]并因此吸引0点!如果是第二种
E
论证.

如果有一个周期p之2的穗定周期轨,那么用完全类似的讨
E
引0点或者它吸引1.于是,我们证明了予在J(丿中至多有一条稿
定周期转.另外,从上面的证明我们也可以看到,在(户中没有稿
E

现在我们考虑了的稳定周期软,它不盼引0或1.由定理1.3,
E不仪

下面我们证明,这样的穗定周期软是JC+一(一1,A(1)中
的一个穗定不动点.如同我们已经看到的,如果这条转道有一个点
在J丿中,那么整条执道在(丿中并豚引0和1因此,这桦的轨
道必在JCA5+中.由性质3,丫在JC/)+中至多有两个不动点,因为
了在[一1,0]中至多有两个不动点.如果丨在(乙+中没有不动
P技ie唐八i
没有周期炭!如果丁在T一中只有一个不动焯,当人(z)丿1时,
E
Page-529
第六章“实二次单峥映射族的吸引子

从60年代兴起的动力系统的现代研究,其中心课题之一是双
曲理论.一个系统(流,微分间胚,胥射)是双曲的,如果它的极限
集是双曲的,即极限集中所有转道的Liapunov指数一致非0(在讨
论洪的双曲性时,不考虑沿流方向的Liapunow指数,它为0.从那
E东
E技标
古外,当时人们曾相信动力系统基本上是由双曲系统构成的.这一
看法的根本转变是由于在70年代受到物理、天文学等领域的一些
E林技人育y育沥林55尿技y命i招s
Lorenz(他们的工作在70年代开始才受到数学家的重视等人对
这些模型的算工作表明,这些模型具有极其复杂的动力学
行为,它们似乎不具有双曲结构,相反它们应当履于非双曲苑畴

河
Jakobsonp在80年代初取得的,他证明了对实二次映射族的一个
E途c北二江标
E
2
Jakobson的结果和方法,并在此基础上,讨论了Hnon映射在具有
Ett
E江河dt
D
E沥
E沥E
对菪个系统有很好的了解(例如,Logistie映射族在a一2时HHenon
Page-530
不1关孔单峰咏射租定闸期点的存在性247

c
一1,团此JCD+中没有其它的稳定周期轨.最后,假设广在J一
E口沥3
E

E育
E

E河utni技2才
i

推论1.5可以看成是推论1.4的部分证明过程,用它可以断
E

推论!6“存在没有穗定周期转的5单峰映射.

D国圆2
Neumann在1947年给出的.容易验证,丫是5单峰晏射.经0点的
扬
弓林不pen林y1
个点.因此,它们不能被其它周期扬道妓引.但是,由于一(一1)一
E沥沥22标

附注1.7“在本节我们讨论了5单峻眸射的穗定周期轨的存
Eouaogsi浩园东i江许永途沥3
沥i技
els河沥招圆
Et园
E
e
e芬
E
E河技85深
Page-531
E

e技芸一一

玟

(一8,8)的方式,其中8一exp(一Ve)特别要讨论它们是怎样
带近原点的,以及靠近原点的速度.而FCz,a)的送代的一些基本
性质对于理解这些问题起着重要作用.应当指出,尽管我们只对这
一特殊眨射展开讨论,但是这种以Jakobson开始,Benedicks和
E育沥林E
g=l′器J>″/″,

巳沥22n林庆不
E厂不余

E东莲育2
如果加E[一L,1J和满趸

3才52明
E
e丞3河
E
EC2.1

n河=sin昔廖.E
为

E熹m恤0b奎廖).
团
0s认史

E育2
4C。sz音没苛_〔s_n昔核)[1歹)

i
Page-532
0243

E
E技
Eyy吴E
和证明方法.应当指出,这里介绍的证明思想在当前这一方向的研
究中是十分重要的.

E沥刑心余c

江
t

小不园i江园c浩江
e

20尿

0

005仪坂政EcdM

E沥技t

【UUD明志t林一北不

医沥怀e沥许玟东步2江
E亚招
b学述玲技振
E,有静噩荤阗/″P(仪)今zE

E

现在我们要讨论的问题;一个单峰昭射可以有多少条穗定的
E兰一i水椿中
玟
E

E春应述

国u河秉扬A
A浩河′(′)〕「
Page-533
E浩uE

E育
园2育2江口

dP′(枳′)T[a′雇〈″”瓜)薹俨′【(%)II(_2a87,C2.2
E

E<1E江,
团因此cos告庞尝0E育

日
Y1一办

沥…|<缪以及_旺沪蝴)卜春

I
e二
引理2.2“存在常数7丿0和A(8》又0,当m充分靠近2时,
E
t
10沥c
(2)log13cF(z,a)|交

E一e一青,而东
E沥
0
[

0

玟

一(2a327(一52标切(一4522纳](一270

人
t
[水
n
Page-534
244E

E不医伟

[沥明北

【明志s

【仪

0李伟22

《Se一(0)一0.

E林M国育u沥招技t
E李a余

Eos国圆D河2河
的,此处不再赘述,,

2
Sg(z.

E沥技e伟2
【仪

3.|一|在(一1,1中没有正的局部最小.

E
E仪d
医芸2沥伟
d
2(天士1)是了的不动点并东|乜(97|一1,那么这个不助点至少
在一边是稳定的.

沥政2技
2沥王一
E途

如若不然,令g一户并设有无穷多=E,满足g(z)一z.由
t东
医e伟
E

5.如果a一&一c是8一八的相邻不动点,并且在区间[avc]
Page-535
第六竟宗二次单峰哈尊旋的媒引孔

最后,取&(8)一(log2)71E
东
区间(ao,23中存在一个小区间4,使得映射k:心一工是一一对
应的,并东保持指数扩张.
E不技宏
E
s木一用
[td
《383|8砂(La)|交(1.9一史二2一工
E园
[一
E改才月

E告[河河述鲁UE命

E吴述沥一命
E

EE
「蒜_^鬓_脓龟盂<一毓

可归纳地证明a一史(a是单调下降的.由这些事实,结论(1)和
〈2)得证.这是因为对上述的8和丶,我们逃出包含2的小参数区
[oteyaE连驱述沥t
t林
一,其中B是。的左端点-注意到Sm(a),aE。是单调下陆
的,所以存在心一4o,得当eE心时,Sn:4一厂是一一映射.最
后我们证明指数扩张伯.图为ˇ

a
Page-536
一2PCrva]一1一a的葛本性质E

E园理
E

园
E江江河0盼

E技
和
E

E

Enal
E胺

并归纳地得到
E
7E
[壶墓u十

0

s
[命
引理2.4“存在充分小的>0和m一2,m帝近2,如果
0技
[扬
0

s
2

E园
E

M
[不1
I伟y

E

z2
Page-536
t

E李林

E
训林
Ehean怀租

一
E刑

E标东荣
E人一
t沥
E芸
0

[

[

E李

【
林
E

E
沥

木一
2
招

【d月仪育尘E

E伟
E
2
水
玟2
E5王
Page-538
第六章“实二次单峰胥寺族的哉引二

[吊]
0

n_

E

E[

E
l一酝韧廷卜十

E一

[
其它情况可类似证明.

5九东5
E>仍羞扁>仪E
5

E朐

E命谅
D之0.所以1十
医
Z
E沥圭a′庐5责exp基″气园
E

EB
E
最后,我们来归纳地证明引理结论.一7时无需证明,设7女?一
E

E

[
忐<‖l命

a「
回为
Page-539
246第六章宗二次单峰胸射旌的吟引孔

E关人一兰e招
乙,使得8(一1.如榔有yE(d,z),使得8(9》一0,则利用性
质6,吞则由情况(iit),我们完成了性质7的证明,并因此证明了定
E

定理1.3有以下儿个推论.

E江园
期点,加上在区间[一1,f(1)]中的一个可能的稳定不动点

E园2
玟
园达L8芸s
a口i
如果是第一种情况,z炜引[0,z]并因此吸引0点!如果是第二种
E
论证.

如果有一个周期p之2的穗定周期轨,那么用完全类似的讨
E
引0点或者它吸引1.于是,我们证明了予在J(丿中至多有一条稿
定周期转.另外,从上面的证明我们也可以看到,在(户中没有稿
E

现在我们考虑了的稳定周期软,它不盼引0或1.由定理1.3,
E不仪

下面我们证明,这样的穗定周期软是JC+一(一1,A(1)中
的一个穗定不动点.如同我们已经看到的,如果这条转道有一个点
在J丿中,那么整条执道在(丿中并豚引0和1因此,这桦的轨
道必在JCA5+中.由性质3,丫在JC/)+中至多有两个不动点,因为
了在[一1,0]中至多有两个不动点.如果丨在(乙+中没有不动
P技ie唐八i
没有周期炭!如果丁在T一中只有一个不动焯,当人(z)丿1时,
E
Page-540
医浩a

E忡】丨
[
E
E
[
a巍痿`′+I12
萜廷善洲<m1

E
阮
i
根据/的送代的这种特性,分别讨论了当参数攀近2时,/a进入
乙前和走出习后,在厂外的扩张性质.引理2.3将作为我们归络
地得到正Lebesgue参数集4..的基础.我们将FCz,a)看成二元函
数,引理2.4说明FCz,a)的送代对<=和a是等度增长的,这一结果
的好处在于对参数或对变量的估算可以相互转化.在$3定理3.1
Ee胡
引理的意义还不仅仅如化-对于一舫的眺射族FCz,a),我们可以
玲
E沥
结中闸述.

(l十

$3(rva)不存在穗定屉期轨问题

a
e
n

伟[
E

2沥林一
Page-541
EE河s2

而在[F(L,a),1]中至多有一条穗定周期豹;并且偎如x一0没有
被吴引到一条周朝轨,那么F(z,a)没有穗定周期轨.于恬,定理
3.1可以从下面的定理3.2得到.

定理3.2存在4CC(0,2),4.具有正Lebesgue测度,以及
二

[刑国仪命

E图胡a扬巳
E

A我们根据之(0,a)不断返回到“的特点,讨论返回的形式.
与此同时用自由返回概念,归纳地将参数区间4分类.

由引理2.3,存在一个最小整数za(>>N),使得

E
是丿到T“上的1一1映射.称m是映射的首次自由返回的指标,
相应的参数首欣分类为
[t

h途
E二d八厂5
使得0点是F(z,eo的周期为z的超稳定周期点.为了保证定理
匹
数区间4中,我们全部排除了使/可能有小于等于m的稳定周
期转的参数

设口是第&次分类的任意一个构成区闽,丿1.下面我们定义
第十1欣自由返回以及相应的分类ˇ

E2技
t伟

[一_占′(″)_
成立的最大值.更确切地讲,对所有窿…m和所有″仨a
Page-542
不1关孔单峰咏射租定闸期点的存在性247

c
一1,团此JCD+中没有其它的稳定周期轨.最后,假设广在J一
E口沥3
E

E育
E

E河utni技2才
i

推论1.5可以看成是推论1.4的部分证明过程,用它可以断
E

推论!6“存在没有穗定周期转的5单峰映射.

D国圆2
Neumann在1947年给出的.容易验证,丫是5单峰晏射.经0点的
扬
弓林不pen林y1
个点.因此,它们不能被其它周期扬道妓引.但是,由于一(一1)一
E沥沥22标

附注1.7“在本节我们讨论了5单峻眸射的穗定周期轨的存
Eouaogsi浩园东i江许永途沥3
沥i技
els河沥招圆
Et园
E
e
e芬
E
E河技85深
Page-543
E

e技芸一一

玟

(一8,8)的方式,其中8一exp(一Ve)特别要讨论它们是怎样
带近原点的,以及靠近原点的速度.而FCz,a)的送代的一些基本
性质对于理解这些问题起着重要作用.应当指出,尽管我们只对这
一特殊眨射展开讨论,但是这种以Jakobson开始,Benedicks和
E育沥林E
g=l′器J>″/″,

巳沥22n林庆不
E厂不余

E东莲育2
如果加E[一L,1J和满趸

3才52明
E
e丞3河
E
EC2.1

n河=sin昔廖.E
为

E熹m恤0b奎廖).
团
0s认史

E育2
4C。sz音没苛_〔s_n昔核)[1歹)

i
Page-544
述胡

16(a一RCya1一官髌赫二2力G8.D)

Eabosuta一

[
cr[

E林沥孝0
0

林胺
E林
E沥达
E
0

沥怀d政
E一吴
医

述怀芸林一命
0根t东沥仪才
E河芸c才b浩莲达江水一河
C
EJ商
E
c东
E朐沥

5林江标木木沥才医伟
那么71心a此时,PoeheC0,m)未必可以严裁地表为区间加的并.
团

EF瓢蟹+…(0y唧〉\〈(7』J鬣′7)U(一eakgHFlycsHH1D7.
Page-545
E浩uE

E育
园2育2江口

dP′(枳′)T[a′雇〈″”瓜)薹俨′【(%)II(_2a87,C2.2
E

E<1E江,
团因此cos告庞尝0E育

日
Y1一办

沥…|<缪以及_旺沪蝴)卜春

I
e二
引理2.2“存在常数7丿0和A(8》又0,当m充分靠近2时,
E
t
10沥c
(2)log13cF(z,a)|交

E一e一青,而东
E沥
0
[

0

玟

一(2a327(一52标切(一4522纳](一270

人
t
[水
n
Page-546
256第六章“实二次单峰吹射族的呆引孔

E李2,

E沥
的gz,此时我们只要稍稍扩展相应的参数区间4,r,使得Frvhi(0,
7)一丁U匕.这桦的构造意显着对恰当的7和符号土有

Eg兄15[
由此,我们可以定义第史十1欣自由返回.

[63述u河异t
仪
林芸
[nyc水

4′+lEUm′+l(酗).
E

E不技莲江林
Po,都与工中的一个区间关联.这样的区间分成两类,一类区间是
In,而另一类区间除包含形奶Z的区间外,它的端点葛入I毗邻
的区间内(见公式(3.3为了暮免引入复杂的记号,在下面的讨
玟
可能性.读者可以仿照证明思想,对第二类区间得到同桦的结论.

E怡3015EsEupotsyna的沥技s
面的事实成立.

(D如果m一mv(a)是第%次返回的指标,志之1「那么

[林

G途

e

0达东木

E育江
【李

下面我们对&十1证明(,(ii)和(ii)成立.令是心中一个
Page-547
第六竟宗二次单峰哈尊旋的媒引孔

最后,取&(8)一(log2)71E
东
区间(ao,23中存在一个小区间4,使得映射k:心一工是一一对
应的,并东保持指数扩张.
E不技宏
E
s木一用
[td
《383|8砂(La)|交(1.9一史二2一工
E园
[一
E改才月

E告[河河述鲁UE命

E吴述沥一命
E

EE
「蒜_^鬓_脓龟盂<一毓

可归纳地证明a一史(a是单调下降的.由这些事实,结论(1)和
〈2)得证.这是因为对上述的8和丶,我们逃出包含2的小参数区
[oteyaE连驱述沥t
t林
一,其中B是。的左端点-注意到Sm(a),aE。是单调下陆
的,所以存在心一4o,得当eE心时,Sn:4一厂是一一映射.最
后我们证明指数扩张伯.图为ˇ

a
Page-548
E浩i257

E
sd
E沥伟
(a一砺(9,a)二F1Ct,a)一R1(1一af,a)
E医述
E
E

u伟
园吴55E
a1

E
E

12cFV(1一a8a)|
c

E音.[
技
E江2刃

[<羞陶伽一卯|万

&Z
d

只要j一1丿m,邦么由(iD,exp(2V乐exp(j4).但是由引理.
E林林

E沥述5

[15′】(音矶】5告胍薹)E林E0

武式也说明,当了增加时不等式了<<23x4在一劝之前就破坏掉
E

E[
Page-549
一2PCrva]一1一a的葛本性质E

E园理
E

园
E江江河0盼

E技
和
E

E

Enal
E胺

并归纳地得到
E
7E
[壶墓u十

0

s
[命
引理2.4“存在充分小的>0和m一2,m帝近2,如果
0技
[扬
0

s
2

E园
E

M
[不1
I伟y

E

z2
Page-550
258E

令山一(a,5).我们讨论史二Pov+f+1(0,的长度的下
乙可以表为
[张
E[

[咤

沥'忐expm乏"'E

医c
E比较是微不足道的(也可参考〔TTY〕).因此

E暑2″|左m′(亿)|[

E辽t[

其中1一a丿8女1
由(3.2)和〈3.4),存在仍俨,0〈劝<v<知_n使得
国吴
d沥
E门江2沥技
18Fp(ga1乐不18FpG一aqoom1.

将(3.11)和上式代入(3.10

[

我们要进一步证明|2|乐exp(一2144).事实上,由假设(口),
[人

log2〔啬)Eitsssut

日

P二exp(一YR一1一exp(一YR)乐ce-,
仪0【达″12/歹
Page-551
第六章“实二次单峰胥寺族的哉引二

[吊]
0

n_

E

E[

E
l一酝韧廷卜十

E一

[
其它情况可类似证明.

5九东5
E>仍羞扁>仪E
5

E朐

E命谅
D之0.所以1十
医
Z
E沥圭a′庐5责exp基″气园
E

EB
E
最后,我们来归纳地证明引理结论.一7时无需证明,设7女?一
E

E

[
忐<‖l命

a「
回为
Page-552
道志河d

王
6pHlCa1芸exp[一(1十硕砺)07,

EJEdcpe
E

王
E1壬/_(]十z号#昙)_zexPG〈″)E水
命5

荣砺自
E[吴n
EEz/〖〕_
u林述tn
1asPm+pCLa1节exp|2Cmx十刑才十圭涣暑〕.G3.16
E人
EE吊22鹰左″_(鹰))*

E命c沥石

吴

G
之2aexp(一YR)。玉exp(2E愤0

王一Ass
>zoexp(Y十2医(P+l)z

[一扬明沥

E
e7

yeomorA>exp〈啬潍菩〉.

医玟5
E

[噩P菩〕

>叠xp〔E戍)鲁E忐棘吾],
Page-553
医浩a

E忡】丨
[
E
E
[
a巍痿`′+I12
萜廷善洲<m1

E
阮
i
根据/的送代的这种特性,分别讨论了当参数攀近2时,/a进入
乙前和走出习后,在厂外的扩张性质.引理2.3将作为我们归络
地得到正Lebesgue参数集4..的基础.我们将FCz,a)看成二元函
数,引理2.4说明FCz,a)的送代对<=和a是等度增长的,这一结果
的好处在于对参数或对变量的估算可以相互转化.在$3定理3.1
Ee胡
引理的意义还不仅仅如化-对于一舫的眺射族FCz,a),我们可以
玲
E沥
结中闸述.

(l十

$3(rva)不存在穗定屉期轨问题

a
e
n

伟[
E

2沥林一
Page-554
260第大章实二次单珠胥射旌的呆引孔

E
由分类定义,设对f一丶,JomtAHt(0,a与工相交.那么
[6命弘林扬
E又技
0
2RhT1C8)|交1978一1

E
[沥59)f】一】exp[E2击涮

之exp(2Cmy十力十趸】)晏)′
E许标许刑
E标一
E一一
迷回时,我们有
12Rma1Ca|>exp(z″置).

如此继缓下去直至到性时,返团是自由的时侯为止.自由返回在
E东
我们在A中看到的一桦,在迷回时是非常快地增长的.我们完成了
i`

zn育
E不

E沥铁沥林技5沥朐
那么我们有

肖

gn吴c顶
E一
Page-555
EE河s2

而在[F(L,a),1]中至多有一条穗定周期豹;并且偎如x一0没有
被吴引到一条周朝轨,那么F(z,a)没有穗定周期轨.于恬,定理
3.1可以从下面的定理3.2得到.

定理3.2存在4CC(0,2),4.具有正Lebesgue测度,以及
二

[刑国仪命

E图胡a扬巳
E

A我们根据之(0,a)不断返回到“的特点,讨论返回的形式.
与此同时用自由返回概念,归纳地将参数区间4分类.

由引理2.3,存在一个最小整数za(>>N),使得

E
是丿到T“上的1一1映射.称m是映射的首次自由返回的指标,
相应的参数首欣分类为
[t

h途
E二d八厂5
使得0点是F(z,eo的周期为z的超稳定周期点.为了保证定理
匹
数区间4中,我们全部排除了使/可能有小于等于m的稳定周
期转的参数

设口是第&次分类的任意一个构成区闽,丿1.下面我们定义
第十1欣自由返回以及相应的分类ˇ

E2技
t伟

[一_占′(″)_
成立的最大值.更确切地讲,对所有窿…m和所有″仨a
Page-556
医源s

E林沥加肉
[许
>EXp〈2泅詹5、/页E吴′″翼)昙)>exP著鲁'
如果又&十力,由(8.147
E
E命3E吊
E蛊擒吾〉[吴吊
E
3
[
医
』命51怀
E林述r沥木
从而(iD得证.
E
b园仪沥人东15
伟
[一
[

[不

E
测度.更确切地讲,我们要证环
n[

其中0<鲇<1且置u_鲇)〉0.如果(3.16)得以证明,那么由
l
E
Page-557
第六章实二次单峰咏射族的吴引扎

1al芸lldl节小丿真d口a
注意到4+iC心,我们有国
E」亘叫>飙
并完或了定理3.2的证明.在证明(3.16)之前,我们暂时便设对所

E沥

E
Pn5
E团汞

人
Ln0的|一|PowuC0,a一Fou0
E2
E
与推导(3.10》相似,我们得到
E

z一
i

Ed
5沥水沥沥网孙
[沥E弘
河
E

沥

E

二鲁(LE7
我们也有

[刑E发小刀
E林跃江林
Page-558
述胡

16(a一RCya1一官髌赫二2力G8.D)

Eabosuta一

[
cr[

E林沥孝0
0

林胺
E林
E沥达
E
0

沥怀d政
E一吴
医

述怀芸林一命
0根t东沥仪才
E河芸c才b浩莲达江水一河
C
EJ商
E
c东
E朐沥

5林江标木木沥才医伟
那么71心a此时,PoeheC0,m)未必可以严裁地表为区间加的并.
团

EF瓢蟹+…(0y唧〉\〈(7』J鬣′7)U(一eakgHFlycsHH1D7.
Page-559
医胡s

王
园in
1尹[日

男一方面,令仁表示中可以包含于@+的小区间全体.根
据+i定义;记仁+一v,则

e沥

00
[E
其中wEs因此

,|二sgC-vETKFI
[

利用(3.17),我们可以佼计从Ax中排除的参数集的测度
I吴E史扬视园X王35E
团1E
团此,有
_4叠|g|4蘑+l|Eexp(_霜斗_媲-卜)
1“E

1g+l交C1一aD|Q|.
E述

H许5

容易验证Ha因此II[

耐注34读者可以看到,在证明(316)时,我们仅考虑了从
E招
英

玟
t
Page-560
256第六章“实二次单峰吹射族的呆引孔

E李2,

E沥
的gz,此时我们只要稍稍扩展相应的参数区间4,r,使得Frvhi(0,
7)一丁U匕.这桦的构造意显着对恰当的7和符号土有

Eg兄15[
由此,我们可以定义第史十1欣自由返回.

[63述u河异t
仪
林芸
[nyc水

4′+lEUm′+l(酗).
E

E不技莲江林
Po,都与工中的一个区间关联.这样的区间分成两类,一类区间是
In,而另一类区间除包含形奶Z的区间外,它的端点葛入I毗邻
的区间内(见公式(3.3为了暮免引入复杂的记号,在下面的讨
玟
可能性.读者可以仿照证明思想,对第二类区间得到同桦的结论.

E怡3015EsEupotsyna的沥技s
面的事实成立.

(D如果m一mv(a)是第%次返回的指标,志之1「那么

[林

G途

e

0达东木

E育江
【李

下面我们对&十1证明(,(ii)和(ii)成立.令是心中一个
Page-561
E

[
不伟心口
E沥林招沥

历
E

E【

E刑

E刃

E圆
E园沥)林林技2

[2
团n

EE途

由归纳的结论(,得
E
[

[
p沥
|el园0
P<〔1十茼]廷h+曲岖_m川E

E述

春
u
E

0逵叶]园愉砧)_锹叭‖

[l

所以我们只要佶计
Page-562
68颜uesposssaey

t
国O
s“Rar
国22E吴
E
iw
吴
2

E

乙林沥东顶仁取洁0
成两部分

园s沥
国0

与
n

首免佶计第一个和式,当y一一时,r
E伟
20|F…(左′′(′z)EF窜(虔′翼(′′),丑〉|
e
同时
E
之|严(0,a)|一】F醴(蔓z′(″〉`鹰)g
E一T认
r
E

江

l

【

月
国2

|十26‖′茎〕日2
Page-563
E浩i257

E
sd
E沥伟
(a一砺(9,a)二F1Ct,a)一R1(1一af,a)
E医述
E
E

u伟
园吴55E
a1

E
E

12cFV(1一a8a)|
c

E音.[
技
E江2刃

[<羞陶伽一卯|万

&Z
d

只要j一1丿m,邦么由(iD,exp(2V乐exp(j4).但是由引理.
E林林

E沥述5

[15′】(音矶】5告胍薹)E林E0

武式也说明,当了增加时不等式了<<23x4在一劝之前就破坏掉
E

E[
Page-564
E

layH门aRFeTTCLa|

我们把不等式有边的和式分成两部分
00
国

EE蠹_′j′+1
E
壬L

d
第二部分我们利用(3.1),(3.4)和(3.5)推导出

l]oz|6s(镳)|`

E吴沥训
0技[颂闯木c
x

E技

0
许
DE

E
[opos达

[
其中9在&.(a)与吕(8)之间,我们有
Page-565
258E

令山一(a,5).我们讨论史二Pov+f+1(0,的长度的下
乙可以表为
[张
E[

[咤

沥'忐expm乏"'E

医c
E比较是微不足道的(也可参考〔TTY〕).因此

E暑2″|左m′(亿)|[

E辽t[

其中1一a丿8女1
由(3.2)和〈3.4),存在仍俨,0〈劝<v<知_n使得
国吴
d沥
E门江2沥技
18Fp(ga1乐不18FpG一aqoom1.

将(3.11)和上式代入(3.10

[

我们要进一步证明|2|乐exp(一2144).事实上,由假设(口),
[人

log2〔啬)Eitsssut

日

P二exp(一YR一1一exp(一YR)乐ce-,
仪0【达″12/歹
Page-566
医溥s

16.(4b一051〈鱼惮ruCa一咤、
[l

E一一个一
t林2浩
于是
16,(a一6(01万暑〔罡]′]+】一”|怠'…1洁工
[
人玑国
,=′鲈熹′+]国X

国
EEzssorsoyooon
19]1

,(一品,(8)|

a

限2
[

c'量】E
s<m瞄〗国驿上让'
/[命
为完成(3.18)的估计,我们只要佶计上面不等式右边和式的
界.完全类似于(3.13)的佼计,我们有

u(3.21)
E沥

5协仁为此,先证明在有限的时间段1<i廷m中,胁
Page-567
道志河d

王
6pHlCa1芸exp[一(1十硕砺)07,

EJEdcpe
E

王
E1壬/_(]十z号#昙)_zexPG〈″)E水
命5

荣砺自
E[吴n
EEz/〖〕_
u林述tn
1asPm+pCLa1节exp|2Cmx十刑才十圭涣暑〕.G3.16
E人
EE吊22鹰左″_(鹰))*

E命c沥石

吴

G
之2aexp(一YR)。玉exp(2E愤0

王一Ass
>zoexp(Y十2医(P+l)z

[一扬明沥

E
e7

yeomorA>exp〈啬潍菩〉.

医玟5
E

[噩P菩〕

>叠xp〔E戍)鲁E忐棘吾],
Page-568
E
′T[E汀

u(

C25月吉

E

E
E

E
d

我们得到
[

lD”z〈|左y(亿】)|c命Enc

1(1zeL,eEa

和
FPCruvao)|之告_尸(0鸪z)卜

E

cn
dcao

6ao一a,16.Ca一6Cao|
m讲[2十L)

由归纳的结论佃%引璎2E

[刑
Ee

<mn鱿(

因止
[3
E
Page-569
E浩a

妙constexp(8丫内)|8一马|,

)悼[仙)_
0慨码e
江

E
团园
_az矗寸(z】E
E
E
田引理2.1和上述佶计,我们得到
5莲n
P_尸…_(′′十p′+1)〈左′′十′′+】(“),′z)1
E
E
E

E

Esdaou

E

吊园
因为8移小,所以(3.22)意昧着lov+i|丿5|oy|,最后,我们估计
和式

【
内

E某些叨可能同时位于某个区间驷中Eaulne
E李永有阮河t
Page-570
260第大章实二次单珠胥射旌的呆引孔

E
由分类定义,设对f一丶,JomtAHt(0,a与工相交.那么
[6命弘林扬
E又技
0
2RhT1C8)|交1978一1

E
[沥59)f】一】exp[E2击涮

之exp(2Cmy十力十趸】)晏)′
E许标许刑
E标一
E一一
迷回时,我们有
12Rma1Ca|>exp(z″置).

如此继缓下去直至到性时,返团是自由的时侯为止.自由返回在
E东
我们在A中看到的一桦,在迷回时是非常快地增长的.我们完成了
i`

zn育
E不

E沥铁沥林技5沥朐
那么我们有

肖

gn吴c顶
E一
Page-571
270第失章_官二次单峰眸射消的吟引孔

最大下标,由此有

U是剩余指标集E
吴巳刑
′…/】厂‖|

E
E

a乏】歹廷叩n歇,

E林沥述
lex|芸const|ajlexpCV为)exp|一z#量〕

艺constV内羞‖吟恤帐伽募|
3

乃exp(一3序).
沥罪一孙李人厉
Page-572
医源s

E林沥加肉
[许
>EXp〈2泅詹5、/页E吴′″翼)昙)>exP著鲁'
如果又&十力,由(8.147
E
E命3E吊
E蛊擒吾〉[吴吊
E
3
[
医
』命51怀
E林述r沥木
从而(iD得证.
E
b园仪沥人东15
伟
[一
[

[不

E
测度.更确切地讲,我们要证环
n[

其中0<鲇<1且置u_鲇)〉0.如果(3.16)得以证明,那么由
l
E
Page-573
J…J】^/趸0
E5`从而定理3i国L

E
玟

目
E青〗鳃[
c

E吴i
的基本定理

E林木水
r王述
E林

附注4.2这个定理表明,测度x不是很特殊的,并且临界点
[水沥力邹河租途运
国
才
E林浩国口月
E

E林一一
巳阮邦院才s江沥史江
沥个u
本质返回(sw}的分布具有有界密度.我们最后证明有界返回{}
E一玟

E

1.自由返回的分布问题的讨论:
Page-574
第六章实二次单峰咏射族的吴引扎

1al芸lldl节小丿真d口a
注意到4+iC心,我们有国
E」亘叫>飙
并完或了定理3.2的证明.在证明(3.16)之前,我们暂时便设对所

E沥

E
Pn5
E团汞

人
Ln0的|一|PowuC0,a一Fou0
E2
E
与推导(3.10》相似,我们得到
E

z一
i

Ed
5沥水沥沥网孙
[沥E弘
河
E

沥

E

二鲁(LE7
我们也有

[刑E发小刀
E林跃江林
Page-575
En

E
我们可以把它看成概率测度.在此基础上,我们可以引入期望三和
2

E

E如
或者.“一4由前面的讨论我们知道,这种形式的参数集合可以分
成一列子集.一G,其中每个心(相应于一条不同的“路径吊,该
E沥才a东河儿孙
E振化工0志=仪

0人0一5朐
颂匹
对每个路往2由(3.13),可扩增为一个区间.(5),其中
E[途
由分类的意义,我们知道00.(5)可以分成一些区间,的五,由前面

的引理3.5,我们得到

,命,
E茎〕m4耿0).[

为完成情形1的证明,我们首先给出两个事实,并把它们归纳
为下面两个引理.
E东吴阮才
E
D
对所有的y成立.郁么,对所有x我们有
E二月
E根水
Page-576
医胡s

王
园in
1尹[日

男一方面,令仁表示中可以包含于@+的小区间全体.根
据+i定义;记仁+一v,则

e沥

00
[E
其中wEs因此

,|二sgC-vETKFI
[

利用(3.17),我们可以佼计从Ax中排除的参数集的测度
I吴E史扬视园X王35E
团1E
团此,有
_4叠|g|4蘑+l|Eexp(_霜斗_媲-卜)
1“E

1g+l交C1一aD|Q|.
E述

H许5

容易验证Ha因此II[

耐注34读者可以看到,在证明(316)时,我们仅考虑了从
E招
英

玟
t
Page-577
g

E,d
FF05许2仪技,
E标

E3仪E水李河
B笋忐菖m彻们一呈m膘、

E

不
P刑应3

E07

其中,之(2e/Zo).于是
】′D~^′′′<[譬E暑〕′Z′]′】【】】′<Q|′″|m】_5_
D江沥国
引理43的一种特殊情况为
引理4.5“假如-
E怀-有沥不明训
鄱么对所有的,
E
E圆e
2.3可得
E
如果上述不等式对n时成立,下面证明
E
事实上,如果我们记心一U,dm,那么
EP
a
人
其中Q一(mw)一,二是,由引理43,我们立得
Page-578
E

[
不伟心口
E沥林招沥

历
E

E【

E刑

E刃

E圆
E园沥)林林技2

[2
团n

EE途

由归纳的结论(,得
E
[

[
p沥
|el园0
P<〔1十茼]廷h+曲岖_m川E

E述

春
u
E

0逵叶]园愉砧)_锹叭‖

[l

所以我们只要佶计
Page-579
E

有

a
t园育

匹el牛I^E

E沥
Eut东沥技伟i
密度.因此

E亩E0
其中第二个不等式由引理4.5得到.因此,如果定义
林

那么
E

医i门

E(R二一N一寸E发0吊伟
s^

我们要证明
fE
E命a扬形2江2
t木u许
并注意骁
L棚勾R…唰驯)一一1

Ee4河
u李林技
Page-580
68颜uesposssaey

t
国O
s“Rar
国22E吴
E
iw
吴
2

E

乙林沥东顶仁取洁0
成两部分

园s沥
国0

与
n

首免佶计第一个和式,当y一一时,r
E伟
20|F…(左′′(′z)EF窜(虔′翼(′′),丑〉|
e
同时
E
之|严(0,a)|一】F醴(蔓z′(″〉`鹰)g
E一T认
r
E

江

l

【

月
国2

|十26‖′茎〕日2
Page-581
医圆2E

ECulyyi芸
un林林目剧
s东院沥辽
时,我们在读引理中取Q一|令
E不朋

由引理43,我们有

E朝
[pulol阮伟
E朋

E朝月

E

E水
E

E
E沥

E
t

E沥
E沥2庞
pC47

(4.7)式意显着自由返回的极限分布在.一0外有有界密度(界为
c).由于在x不0处密度一致有界,以及对所有的x,zs久0,所以
E沥t

利用非本质返回的定义和性质,类似地可以证明,非本质返回
的极限分布具有有界密度.

E沥p
Page-582
E

layH门aRFeTTCLa|

我们把不等式有边的和式分成两部分
00
国

EE蠹_′j′+1
E
壬L

d
第二部分我们利用(3.1),(3.4)和(3.5)推导出

l]oz|6s(镳)|`

E吴沥训
0技[颂闯木c
x

E技

0
许
DE

E
[opos达

[
其中9在&.(a)与吕(8)之间,我们有
Page-583
uE

我们要讨论有界返回{ow},n一.18,,N,一1,2,..的分
E晓noltereaole
JJ固定)的分布,然后对所有了>>J进行分析.

E东育
t不
E
为
[
[

E={

E
口
E李3
E

沥
E
土_i)二I由中值定理和(3.5式,我们有

[刀团于
这里我们要特别指出,如果{6,}8-,是一个奇(偶)接序列,那么a;
E振

E
因此,一个点znET的第了次有界返回属于(a)的“条件概
E标纳a
Page-584
医溥s

16.(4b一051〈鱼惮ruCa一咤、
[l

E一一个一
t林2浩
于是
16,(a一6(01万暑〔罡]′]+】一”|怠'…1洁工
[
人玑国
,=′鲈熹′+]国X

国
EEzssorsoyooon
19]1

,(一品,(8)|

a

限2
[

c'量】E
s<m瞄〗国驿上让'
/[命
为完成(3.18)的估计,我们只要佶计上面不等式右边和式的
界.完全类似于(3.13)的佼计,我们有

u(3.21)
E沥

5协仁为此,先证明在有限的时间段1<i廷m中,胁
Page-585
E
′T[E汀

u(

C25月吉

E

E
E

E
d

我们得到
[

lD”z〈|左y(亿】)|c命Enc

1(1zeL,eEa

和
FPCruvao)|之告_尸(0鸪z)卜

E

cn
dcao

6ao一a,16.Ca一6Cao|
m讲[2十L)

由归纳的结论佃%引璎2E

[刑
Ee

<mn鱿(

因止
[3
E
Page-586
4分布问题

爽<垩const丽,L
0,a一
E沥i河

3
C
3

AEM0

其中

、E灿愤>删跳仰「圩
Eots水

,招
E。

从而得

2秉沥E
玟
形式

E皇大圆A
0
玟

Eatteohiet
到,对充分大的丿,我们有

E92东e^尝昔″~[大
由(4'9〉式我们得到,相应于y″j,极限分布对几乎所有的窿仨^有
2

一

原
I″(″)g/(况)ED

Etgsnatlp规d
E胡
Page-587
E浩a

妙constexp(8丫内)|8一马|,

)悼[仙)_
0慨码e
江

E
团园
_az矗寸(z】E
E
E
田引理2.1和上述佶计,我们得到
5莲n
P_尸…_(′′十p′+1)〈左′′十′′+】(“),′z)1
E
E
E

E

Esdaou

E

吊园
因为8移小,所以(3.22)意昧着lov+i|丿5|oy|,最后,我们估计
和式

【
内

E某些叨可能同时位于某个区间驷中Eaulne
E李永有阮河t
Page-588
E

E页

E
,一
″′〈“)_{0'E
E

〗缆E沥用

E由上式还可以得到″、有懈|>J医途
20

E〗E湘eEe一告`/7'

E

E刑医p
E时分两种情况】吨n岫H庞木

渡A李t

e个,
E人吊述渡吴P水扬
定理4.6“对儿乎所有的aEA,自由返回和非本质自由返固
的极限分布具有一致有界的密度,而有界返回极限分布具有密度
&(z)、g(z)满足不等式
[E

Ea
57

0
E
Page-589
270第失章_官二次单峰眸射消的吟引孔

最大下标,由此有

U是剩余指标集E
吴巳刑
′…/】厂‖|

E
E

a乏】歹廷叩n歇,

E林沥述
lex|芸const|ajlexpCV为)exp|一z#量〕

艺constV内羞‖吟恤帐伽募|
3

乃exp(一3序).
沥罪一孙李人厉
Page-590
团

*E
E

69E0,
,E史
E渡2
玟
i

2En‖
1
3林3林3林林33
EEE2h

0
E引i0许技

0

医达
_m'嚷萝E林

E

E沥伟李
E
E一育

E
ERCD

(z)二

E

EI″U〉′L(算)E

E

医梁

其中ACz)二const〗吾亓工.E
E

上式的右端具形式

8
E50
I″U(一″)′【(箕)EL幄碧颢(F【0怡沥
Page-591
J…J】^/趸0
E5`从而定理3i国L

E
玟

目
E青〗鳃[
c

E吴i
的基本定理

E林木水
r王述
E林

附注4.2这个定理表明,测度x不是很特殊的,并且临界点
[水沥力邹河租途运
国
才
E林浩国口月
E

E林一一
巳阮邦院才s江沥史江
沥个u
本质返回(sw}的分布具有有界密度.我们最后证明有界返回{}
E一玟

E

1.自由返回的分布问题的讨论:
Page-592
280l
E庆

z
Ltocoytnyt标c
[

E

1
gEfoal12江E
I|RCF-Gm))1F-i(ay|dz二I…z)|…彻)怦

<c呱【I…叫′_缸<伽酞〗_[|′′(z〉|P_d丈

pca

3医l)/_IL

P

国

E
1I[
林8

E
E
-盲e一″盖扒<[朋英

E
E艾`^舌″宜]_z′′_歹zze兰′"Ee【″”_一″T〕

用积分佼计式,我们容易证明
0限刃dz
E吴
c0【ISt'=〔′熹…〕+】ez[I_丨恤一灼瞎]

3

E
EEeseity怡招5556芸河5国子氏标淅
E【
Page-593
En

E
我们可以把它看成概率测度.在此基础上,我们可以引入期望三和
2

E

E如
或者.“一4由前面的讨论我们知道,这种形式的参数集合可以分
成一列子集.一G,其中每个心(相应于一条不同的“路径吊,该
E沥才a东河儿孙
E振化工0志=仪

0人0一5朐
颂匹
对每个路往2由(3.13),可扩增为一个区间.(5),其中
E[途
由分类的意义,我们知道00.(5)可以分成一些区间,的五,由前面

的引理3.5,我们得到

,命,
E茎〕m4耿0).[

为完成情形1的证明,我们首先给出两个事实,并把它们归纳
为下面两个引理.
E东吴阮才
E
D
对所有的y成立.郁么,对所有x我们有
E二月
E根水
Page-594
E

E
P〗麟枫岫[达)

我们已经知道对于z[〖az,l)2伟3林沥沥久
沥
[肖沥[途)
用此事实以及完全类伸前面的方法,我们有下面的估计

E<I7355浩兰技技技
其中8xjjs二constXr,力一2,由此推得

0<I〗gz(z)dz,

其中〗例…猷E

小结40技生s途达
a一2附近的动力学行为,证明过程也说明,[Yo]中闸述的关于当
前如何研究非双曲系统的看法的实用性.下面我们简单地介绍,关
于一舫单峰映射族/:(z的新结果,以此作为对上面特例的总结。

E怀训述t国伟
[沥育印2技沥吴ndUi技

林顶

[芸林a芸
[沥75
林仪不

O2R河c技u
有rEI

E

以及
Page-595
g

E,d
FF05许2仪技,
E标

E3仪E水李河
B笋忐菖m彻们一呈m膘、

E

不
P刑应3

E07

其中,之(2e/Zo).于是
】′D~^′′′<[譬E暑〕′Z′]′】【】】′<Q|′″|m】_5_
D江沥国
引理43的一种特殊情况为
引理4.5“假如-
E怀-有沥不明训
鄱么对所有的,
E
E圆e
2.3可得
E
如果上述不等式对n时成立,下面证明
E
事实上,如果我们记心一U,dm,那么
EP
a
人
其中Q一(mw)一,二是,由引理43,我们立得
Page-596
第六辅“实二次单峰吾射焊的吴引孔

E:.|井
E

人招

Es东不
动参数的概念-

木
动参数,如果它满足如下条件:

E林沥东胡2
周期转;

E2坂u沥沥(沥
E技0林沥江林芸0李林沥庆
E刑co不
E沥G沥才
翦丽EQ'姜0

E
临界点c一0是非回复的;面么没有稳定周期转可以保证在临界
点的任意邻域外的任何足够长的执道段的指数增长,CCEu)条件
是证明进程中技术上的要求,它保证软道的增长指数(比方说忍
伟
Eaze规沥口扬汀

沥
E尘林园河5

0沥人
b

其中|Q|表示集合的Lebesgue测度.特别,如果该极限值为1,
则称a,为Lebesgue全翟点.
Es沥
Page-597
E

有

a
t园育

匹el牛I^E

E沥
Eut东沥技伟i
密度.因此

E亩E0
其中第二个不等式由引理4.5得到.因此,如果定义
林

那么
E

医i门

E(R二一N一寸E发0吊伟
s^

我们要证明
fE
E命a扬形2江2
t木u许
并注意骁
L棚勾R…唰驯)一一1

Ee4河
u李林技
Page-598
E283

E
E

E河

E仪
园
E

EouEyAsst才

E技ddn

08东孙2

E才2
0技野25许
0

由于对每个aE,务滢足条件(NS),(CE1)和CCE2),利用
DNS]的结果,立得对乙存在一个关于Lebesgue测度绝对连绩的
技途

近年来,在非双曲系统研究中不断有较深刻的结果出现,我们
在本章中不准备介绍了.但有一点要指出,我们在这里介绍的
Benedicks和Carleson的证明思想(简称BC方法),近儿年来在这个
d
射族还是对平面间宿相切(例如Henon映射,敲结分岔甚至高维
的同宿分岔现象所得到的主要绪果的证明方法,基本上是BC方
法,或者是从这一方法中派生出来的-
Page-599
医圆2E

ECulyyi芸
un林林目剧
s东院沥辽
时,我们在读引理中取Q一|令
E不朋

由引理43,我们有

E朝
[pulol阮伟
E朋

E朝月

E

E水
E

E
E沥

E
t

E沥
E沥2庞
pC47

(4.7)式意显着自由返回的极限分布在.一0外有有界密度(界为
c).由于在x不0处密度一致有界,以及对所有的x,zs久0,所以
E沥t

利用非本质返回的定义和性质,类似地可以证明,非本质返回
的极限分布具有有界密度.

E沥p
Page-600
附“录

即袖我们只考虑R上的向量场,在讨论分岔的佘维数时,也
ELe
述
在附录4~C中介绍一些有关微分流形与徽分拓扑的概念、名词
E
医
D

[

微分流形是欧氏空间中光滑曲面椿念的抽象和推广.它的基
s
立微分同背而引迹相应的代数和拓扑结构,然后再把这些局部结
构光滑地粘接起来.因此,我们可以把Banach空间中的运算推广
到微分流形上.

征分流形的定义

朋江标国育李i门
Banach空间.假设口是M的开子集,p是从口到8中开子集CDU)
E

切的两个坐标卡(U,g),(7,43称为是C“相容的,如果当口
【绍

E
Page-601
uE

我们要讨论有界返回{ow},n一.18,,N,一1,2,..的分
E晓noltereaole
JJ固定)的分布,然后对所有了>>J进行分析.

E东育
t不
E
为
[
[

E={

E
口
E李3
E

沥
E
土_i)二I由中值定理和(3.5式,我们有

[刀团于
这里我们要特别指出,如果{6,}8-,是一个奇(偶)接序列,那么a;
E振

E
因此,一个点znET的第了次有界返回属于(a)的“条件概
E标纳aP
age-602
4分布问题

爽<垩const丽,L
0,a一
E沥i河

3
C
3

AEM0

其中

、E灿愤>删跳仰「圩
Eots水

,招
E。

从而得

2秉沥E
玟
形式

E皇大圆A
0
玟

Eatteohiet
到,对充分大的丿,我们有

E92东e^尝昔″~[大
由(4'9〉式我们得到,相应于y″j,极限分布对几乎所有的窿仨^有
2

一

原
I″(″)g/(况)ED

Etgsnatlp规d
E胡
Page-603
E

是旦上的C微分同胚.

E
为一个C坐标綦,如果

【江c

0

p
版的Cr坐标系.队上C“坐标系的一个等价类三称作元的一个C“
E吴沥a
的-个极火Cr坐标系,而(U,8》E-必称作一个容许坐标卡.

如果在吊上给定了一个C“微分结构则称6一(史,3是一
Et
E育
(注意,由吊的连通性和坐标卡的相容性易知,5与坂标卡的选取
无关).特别,当8为Hilbert空间时,称M为Hilbert流渺;而当5
Eoratstsaediuhet

E玟吴a
以把与a等价的全部坐标系合起来而得到一个极大坐标系,从而
生成队上的一个微分结构.图此,只须给定M上一个特定的坐标
系,就可以决定这个微分流形.

E
应地得到C“微分流形.C“微分流形也称为光滑流形
国0n
(是旦的佰同映射),这个坐标卡显然就构成日上的一个C“坐标.
系.因此,任何Banach空间都是装备在它自身上的一个光谊
Banach流形

(2)8一tzERr++llia一1}是一个n雕流形.事实上,记
足一人,0,.0},5二{一一0,.0}分别是“的北极与家极,取
坐标卡(SeNtN},),《SeMtS},,其中
Page-604
E

E页

E
,一
″′〈“)_{0'E
E

〗缆E沥用

E由上式还可以得到″、有懈|>J医途
20

E〗E湘eEe一告`/7'

E

E刑医p
E时分两种情况】吨n岫H庞木

渡A李t

e个,
E人吊述渡吴P水扬
定理4.6“对儿乎所有的aEA,自由返回和非本质自由返固
的极限分布具有一致有界的密度,而有界返回极限分布具有密度
&(z)、g(z)满足不等式
[E

Ea
57

0
E
Page-605
EE吴

E
0

E叽〈渥】,""艾”鼻】〉=

。

沥招
容易得出

8RM0)一RM(0j,80一命
是C“微分同胚.

EE绅_〕
Ei左

流形间的映射

定义A.5“设友,分别是装备在Banach空间4,上的C
E
氓53诊252技沥i
Eu坤林f

E
是C“的.

E

E胡2东技
医5标g
E

E丶沥标育2林述林标江
E沥|

定义A.8“设肌,义是Cr徽分流形,称M~丶为C微分
同胚,如果它是一一的C映射,并且其逆映射广!:N一吊也是Cr
的.如果两个流形间存在一个C徽分同胚,则称这两个流形C“微
分同胚.

子流形与积流形
类似于向量空间的子空间与乘积空间,微分流形也存在予流
Page-606
团

*E
E

69E0,
,E史
E渡2
玟
i

2En‖
1
3林3林3林林33
EEE2h

0
E引i0许技

0

医达
_m'嚷萝E林

E

E沥伟李
E
E一育

E
ERCD

(z)二

E

EI″U〉′L(算)E

E

医梁

其中ACz)二const〗吾亓工.E
E

上式的右端具形式

8
E50
I″U(一″)′【(箕)EL幄碧颢(F【0怡沥
Page-607
280l
E庆

z
Ltocoytnyt标c
[

E

1
gEfoal12江E
I|RCF-Gm))1F-i(ay|dz二I…z)|…彻)怦

<c呱【I…叫′_缸<伽酞〗_[|′′(z〉|P_d丈

pca

3医l)/_IL

P

国

E
1I[
林8

E
E
-盲e一″盖扒<[朋英

E
E艾`^舌″宜]_z′′_歹zze兰′"Ee【″”_一″T〕

用积分佼计式,我们容易证明
0限刃dz
E吴
c0【ISt'=〔′熹…〕+】ez[I_丨恤一灼瞎]

3

E
EEeseity怡招5556芸河5国子氏标淅
E【
Page-608
E标

E
E林
E林c园
国江2p
河江
E
E技5
【a浩
E应辽述逊s
CU,外,和8的直和分解8=B申Ba,使得ED,干一
E林02030相
p林林国吊
E一
E沥d
个微分流形,并旦它的微分结构可由下面的坂标系生成:
d
s吴
E水
流形的方法
Ezi朐
t小不
ER西一明春述
分流形,称它为M与N的积流形,仍记为友XN-

附星B“切丛与切映射,向量场及其流,漫人与漫盐

向量丛是积流形的推广,而流形上的向量场,则是作为一个特
Eu
Page-609
E

E
P〗麟枫岫[达)

我们已经知道对于z[〖az,l)2伟3林沥沥久
沥
[肖沥[途)
用此事实以及完全类伸前面的方法,我们有下面的估计

E<I7355浩兰技技技
其中8xjjs二constXr,力一2,由此推得

0<I〗gz(z)dz,

其中〗例…猷E

小结40技生s途达
a一2附近的动力学行为,证明过程也说明,[Yo]中闸述的关于当
前如何研究非双曲系统的看法的实用性.下面我们简单地介绍,关
于一舫单峰映射族/:(z的新结果,以此作为对上面特例的总结。

E怀训述t国伟
[沥育印2技沥吴ndUi技

林顶

[芸林a芸
[沥75
林仪不

O2R河c技u
有rEI

E

以及
Page-610
E附录B“切屿与切际驯,名量圭友其浇浦人与深范

Banach空间中的隐函数(反函数)定理以及局部潍入和浸盎定理
E沥吊c胡
E

向量丛

E孙林团
开集,秒乙X日为局部向量丛,并称U为廉空间,它可以等同于
Ei国林
团伟
人

E园一一
一个开子流形.

定义B.2设DX旦和DXX都是局部向量仰.如果眸射
E「

E
E
这个映射还是一一的,则称它为C局邵向量丛同构

E
t
E东

类似于微分流形的定义,我们可以把局部向量丛粘接起来,得
E

定义B.3设5是集合.称(P7,旭是5的一个局部向重仰卡,
E沥
(U,B可能与g有关)称这桦的卡集畅一(CWo,8)|aE4是5

的C7向量丛垫标系,如果
E05林tA沥
(2)著(Wav8),(Wo8)E男,JanW夺切,则(
Page-611
第六辅“实二次单峰吾射焊的吴引孔

E:.|井
E

人招

Es东不
动参数的概念-

木
动参数,如果它满足如下条件:

E林沥东胡2
周期转;

E2坂u沥沥(沥
E技0林沥江林芸0李林沥庆
E刑co不
E沥G沥才
翦丽EQ'姜0

E
临界点c一0是非回复的;面么没有稳定周期转可以保证在临界
点的任意邻域外的任何足够长的执道段的指数增长,CCEu)条件
是证明进程中技术上的要求,它保证软道的增长指数(比方说忍
伟
Eaze规沥口扬汀

沥
E尘林园河5

0沥人
b

其中|Q|表示集合的Lebesgue测度.特别,如果该极限值为1,
则称a,为Lebesgue全翟点.
Es沥
Page-612
E283

E
E

E河

E仪
园
E

EouEyAsst才

E技ddn

08东孙2

E才2
0技野25许
0

由于对每个aE,务滢足条件(NS),(CE1)和CCE2),利用
DNS]的结果,立得对乙存在一个关于Lebesgue测度绝对连绩的
技途

近年来,在非双曲系统研究中不断有较深刻的结果出现,我们
在本章中不准备介绍了.但有一点要指出,我们在这里介绍的
Benedicks和Carleson的证明思想(简称BC方法),近儿年来在这个
d
射族还是对平面间宿相切(例如Henon映射,敲结分岔甚至高维
的同宿分岔现象所得到的主要绪果的证明方法,基本上是BC方
法,或者是从这一方法中派生出来的-
Page-613
附源B切世与切险射,名量行及其流,清人与渥敏289

L林tt

称5的两个向量从坂标系88与田蔽价,如果幺lU口;还是
l吴
E应不不不i
伟
医i2

[

0微分流形

ECutopasaopo志st

E胡c李
达

0
E

E江
E述
EnU
Ep,a线性拓扑同构(作为线性空间是同构的,作为拓扑空间是同
胃的.在这个意义下,可以认为E与8无关,可记为E

【坂孙育的婵…M(底空间)E
可定义投影

Ei
i
有的Banach结构称为纤维型.在有些书上,把上面的性质作为向
Eatpystt
E

从几何上粗略地说,以流形M为底空间的向丛y就是在M

上的每一点“附着“一个以该点为零元素的Banach空间,而不囡点
E
Page-614
附“录

即袖我们只考虑R上的向量场,在讨论分岔的佘维数时,也
ELe
述
在附录4~C中介绍一些有关微分流形与徽分拓扑的概念、名词
E
医
D

[

微分流形是欧氏空间中光滑曲面椿念的抽象和推广.它的基
s
立微分同背而引迹相应的代数和拓扑结构,然后再把这些局部结
构光滑地粘接起来.因此,我们可以把Banach空间中的运算推广
到微分流形上.

征分流形的定义

朋江标国育李i门
Banach空间.假设口是M的开子集,p是从口到8中开子集CDU)
E

切的两个坐标卡(U,g),(7,43称为是C“相容的,如果当口
【绍

E
Page-615
EE

列两种方式构造出不同的向量丛:Y#E5,取过的法线为纤维
七;或取过的切平面为Ep,前者的纤维型是R「,而后者是R「、

切空间与切丛

我们可以通过坐标卡把Banach空间中曲线相切的概含诱导
到流形上,从而建立切空间与切东.

空
江江32技45林2标史江E
力则称曲线以为基点.设c,es是以8为基点的两条曲线,并且
d

E
(即Banach空间中的曲线8“e与c:在0点相切,则称切上的
曲线c与cs在丿点相切.

注意,利用流形史上坐标卡的相宰性,ei与c在点的相切性
医沥
标卡,ED[Da-设D(p。c(0)二DCp。c(0).由于.

画6二(8pD王一工2
E
Et

一DC8。877Cp(力)D。co(0)一D。67(0).

这桦,我们在同基点的曲线之间规定了一个等价关系:a!一c
E
力点的一个切向量

E
E

E林口
E沥朐
E
E
Page-616
E

是旦上的C微分同胚.

E
为一个C坐标綦,如果

【江c

0

p
版的Cr坐标系.队上C“坐标系的一个等价类三称作元的一个C“
E吴沥a
的-个极火Cr坐标系,而(U,8》E-必称作一个容许坐标卡.

如果在吊上给定了一个C“微分结构则称6一(史,3是一
Et
E育
(注意,由吊的连通性和坐标卡的相容性易知,5与坂标卡的选取
无关).特别,当8为Hilbert空间时,称M为Hilbert流渺;而当5
Eoratstsaediuhet

E玟吴a
以把与a等价的全部坐标系合起来而得到一个极大坐标系,从而
生成队上的一个微分结构.图此,只须给定M上一个特定的坐标
系,就可以决定这个微分流形.

E
应地得到C“微分流形.C“微分流形也称为光滑流形
国0n
(是旦的佰同映射),这个坐标卡显然就构成日上的一个C“坐标.
系.因此,任何Banach空间都是装备在它自身上的一个光谊
Banach流形

(2)8一tzERr++llia一1}是一个n雕流形.事实上,记
足一人,0,.0},5二{一一0,.0}分别是“的北极与家极,取
坐标卡(SeNtN},),《SeMtS},,其中
Page-617
E291

EEt园

E
E林2

2

E
E河
人
空间.这桦,Y[c]JpETp(M3,有二DCp。c)(0)E为与之对应.
E2怡52
下而DGp。c)(0)二o,从c可得[c].于是,得到T「(肌)到B,的一
沥
E不育玟

再来证明(TM,z,M)是向量丛,其中投影rxTM-M通过
0p
外)|aE4},则出下文的定理B10可知,吊一{(TUe,7W)laE4}
E沥李沥
t的切晃封-

切映射

E木用
E绍匹tt许儿应沥希余江逊
lt永P林史日
22seG30
注意

伟东3才
则由Banach空间中导算子的链式法则可得
DC8。一67(0)二D。丁g(6))。DCp。c2(0),
0
Page-618
EE吴

E
0

E叽〈渥】,""艾”鼻】〉=

。

沥招
容易得出

8RM0)一RM(0j,80一命
是C“微分同胚.

EE绅_〕
Ei左

流形间的映射

定义A.5“设友,分别是装备在Banach空间4,上的C
E
氓53诊252技沥i
Eu坤林f

E
是C“的.

E

E胡2东技
医5标g
E

E丶沥标育2林述林标江
E沥|

定义A.8“设肌,义是Cr徽分流形,称M~丶为C微分
同胚,如果它是一一的C映射,并且其逆映射广!:N一吊也是Cr
的.如果两个流形间存在一个C徽分同胚,则称这两个流形C“微
分同胚.

子流形与积流形
类似于向量空间的子空间与乘积空间,微分流形也存在予流
Page-619
用录B“切坚与奶限射,向重姬及其沛,描入与抒堵

J【
D技8(坂沥30

E
E

玟
为了的切晖射.有时也把T记为4或了.

E诊0ed江
E

E92592中东技2

国【
E0日
许

[河江5林玟

月

林一人一目

沥

是T一TK的C一晓射.1

E河
射.

万
沥

Et

[李
木刀一一刑
E林

E万
0

E
Page-620
贺录卫“切此与切院射,向量李及其流,揉人与抒故2

n
n
E
E
医沥伟林才
[国,t圆19顺阮

玟
东结构,例如,叟AC{#}XB)为过E伟的纤维,则恶是TW与
L志Eopettiot2s圆

引理B.9_设仁和伟「分别是Banach空间日和B的开子集,
乙印一仁「是Cr徽分同胚,则Tf:V7X8一IP「XB是局部向量
东同构映射.

证明“因为

【095才62

E林林
玟
也是一个C“局部向量丛映射,从而T/是一向量丛同松映射,‖

现在可以证明

E二209才
}是胆的一个Cr坐标系,则T.ar二((T(Ua),TCR2)[aE心是
切丛TM的C!向量丛坐标系,从而TWU是C““*向量仰.此外,如
E沥李沥江
团0园人)
E肖技川E沥

E732述

是局部向量丛,并且

E|T'″(7〈U-】门7(U′〉)一工(8“秀|T′"(7(″″)门T【U′〉〉
(见定义B7下的性质(3)和(5)).注意8。8是9(U。)所在的
Page-621
E标

E
E林
E林c园
国江2p
河江
E
E技5
【a浩
E应辽述逊s
CU,外,和8的直和分解8=B申Ba,使得ED,干一
E林02030相
p林林国吊
E一
E沥d
个微分流形,并旦它的微分结构可由下面的坂标系生成:
d
s吴
E水
流形的方法
Ezi朐
t小不
ER西一明春述
分流形,称它为M与N的积流形,仍记为友XN-

附星B“切丛与切映射,向量场及其流,漫人与漫盐

向量丛是积流形的推广,而流形上的向量场,则是作为一个特
Eu
Page-622
EEt

Banach空间五中的开二集9(7CUa)门T(Do))到自身的〇映射.
EEi2匹a北2圆政林s吊

E一
E沥沥李b

附注B.11“前面已经提到,流形间的切映射是Banach空间
中导算子的推广.闷此,有时把它狞为导昭射-设M和丶是充分光
玟
江仁ae
续求二防导晃射7(TA一72f7(TM3一(CTN),并可通推地定
Eolss

向量场及其流

王
6sneaeololsl

玟

E万
E
量场.M上一切Cr向量场的集合记为-gT“C2.

[达
E应泓
园
椿p

E林Ba李梁d
王
映射

E林林i

E肖吴0李水5沥技0一

E
E
Page-623
E附录B“切屿与切际驯,名量圭友其浇浦人与深范

Banach空间中的隐函数(反函数)定理以及局部潍入和浸盎定理
E沥吊c胡
E

向量丛

E孙林团
开集,秒乙X日为局部向量丛,并称U为廉空间,它可以等同于
Ei国林
团伟
人

E园一一
一个开子流形.

定义B.2设DX旦和DXX都是局部向量仰.如果眸射
E「

E
E
这个映射还是一一的,则称它为C局邵向量丛同构

E
t
E东

类似于微分流形的定义,我们可以把局部向量丛粘接起来,得
E

定义B.3设5是集合.称(P7,旭是5的一个局部向重仰卡,
E沥
(U,B可能与g有关)称这桦的卡集畅一(CWo,8)|aE4是5

的C7向量丛垫标系,如果
E05林tA沥
(2)著(Wav8),(Wo8)E男,JanW夺切,则(
Page-624
E78刑E

E才c木
d王tpu莲E林命
E
E
E0020[沥
技
E伟团沥林5罡伟技
朱5林a技伯
E
蛊0
p
含0的开区间,都有JCC厂且al,=.
取切上的坐标卡(C,9),poEU设向量场&:伦一TWU有局部
玟

表示相对照,就得到Banach空间中关于(p。)的微分方程初值闭
5

青E
命(201水
浩
玟
E

利用Banach空间中微分方程初值问题解的存在和唯一性定
理,可以得到

E一吴沥孝小
E.

E55林招人
沥达
出的切向量相君合,这与欧氏空间中向量场的流的概念相一致

[怀ue扬
Page-625
附源B切世与切险射,名量行及其流,清人与渥敏289

L林tt

称5的两个向量从坂标系88与田蔽价,如果幺lU口;还是
l吴
E应不不不i
伟
医i2

[

0微分流形

ECutopasaopo志st

E胡c李
达

0
E

E江
E述
EnU
Ep,a线性拓扑同构(作为线性空间是同构的,作为拓扑空间是同
胃的.在这个意义下,可以认为E与8无关,可记为E

【坂孙育的婵…M(底空间)E
可定义投影

Ei
i
有的Banach结构称为纤维型.在有些书上,把上面的性质作为向
Eatpystt
E

从几何上粗略地说,以流形M为底空间的向丛y就是在M

上的每一点“附着“一个以该点为零元素的Banach空间,而不囡点
E
Page-626
EE

E兰达
E
E许达一颂技一t
[一才
满足微分方程
D沥伟3
E林i李
E水
E梁y关达
、友一Tg*X(g(za)),乙E伊口巴
E沥g
E沥林吴八技
Da(o)一E0
E汀沥
u诊李才一
【吴2s沥技沥3振
它成为积分曲线的条件是

a
因此可以说,对于维流形上的向量场而言;其积分曲线的局部表
E
E
隐函数定理是徽分学中最重要的定理之一.设R“-~R“是
E伟c仪nt
Page-627
EE

列两种方式构造出不同的向量丛:Y#E5,取过的法线为纤维
七;或取过的切平面为Ep,前者的纤维型是R「,而后者是R「、

切空间与切丛

我们可以通过坐标卡把Banach空间中曲线相切的概含诱导
到流形上,从而建立切空间与切东.

空
江江32技45林2标史江E
力则称曲线以为基点.设c,es是以8为基点的两条曲线,并且
d

E
(即Banach空间中的曲线8“e与c:在0点相切,则称切上的
曲线c与cs在丿点相切.

注意,利用流形史上坐标卡的相宰性,ei与c在点的相切性
医沥
标卡,ED[Da-设D(p。c(0)二DCp。c(0).由于.

画6二(8pD王一工2
E
Et

一DC8。877Cp(力)D。co(0)一D。67(0).

这桦,我们在同基点的曲线之间规定了一个等价关系:a!一c
E
力点的一个切向量

E
E

E林口
E沥朐
E
E
Page-628
E291

EEt园

E
E林2

2

E
E河
人
空间.这桦,Y[c]JpETp(M3,有二DCp。c)(0)E为与之对应.
E2怡52
下而DGp。c)(0)二o,从c可得[c].于是,得到T「(肌)到B,的一
沥
E不育玟

再来证明(TM,z,M)是向量丛,其中投影rxTM-M通过
0p
外)|aE4},则出下文的定理B10可知,吊一{(TUe,7W)laE4}
E沥李沥
t的切晃封-

切映射

E木用
E绍匹tt许儿应沥希余江逊
lt永P林史日
22seG30
注意

伟东3才
则由Banach空间中导算子的链式法则可得
DC8。一67(0)二D。丁g(6))。DCp。c2(0),
0
Page-629
E2标E

E
E门逊
E'

王2颂木朐
E胡2阮
s沥ct万玟唐
E

E沥林

这称为(局部)正则漫羞(投影).

一
E不
过坐标卡推广到微分草形上,并田此得到构造(或判断)于流形的
方法.对于无穷维的Banach空间,仅有DfCz)的单射(或满射)条“
件是不够的,要附加适当的可裂性条件.首先给出下面的

EL余林二t国
ap

E林L玟不沥
E
c
E
[人才
E

n
[述
E0水
[许芸
E伟沥扬
Ee沥
Page-630
用录B“切坚与奶限射,向重姬及其沛,描入与抒堵

J【
D技8(坂沥30

E
E

玟
为了的切晖射.有时也把T记为4或了.

E诊0ed江
E

E92592中东技2

国【
E0日
许

[河江5林玟

月

林一人一目

沥

是T一TK的C一晓射.1

E河
射.

万
沥

Et

[李
木刀一一刑
E林

E万
0

E
Page-631
E附录B“切坤与切陋财,吊探埃及其旁,淑人不清盎

[仪2

E亚园芸人0技沥育
炼(CDJCuo))人七的代数与拓扑同构.令
E李李e林林沥动03河
E沥

E
E

0国
是口X卷一吊田P一丁的Cr徽分同背,由隐函数定理,存在开
水2E021
[7一技一7
[

寸
E肖口林
E8应n沥
E
n
E

E国0

E技月
EE

D[

0Dz/(吻)〕
0E
王
[2
EiSE
[吊小沥2
E林2
Page-632
贺录卫“切此与切院射,向量李及其流,揉人与抒故2

n
n
E
E
医沥伟林才
[国,t圆19顺阮

玟
东结构,例如,叟AC{#}XB)为过E伟的纤维,则恶是TW与
L志Eopettiot2s圆

引理B.9_设仁和伟「分别是Banach空间日和B的开子集,
乙印一仁「是Cr徽分同胚,则Tf:V7X8一IP「XB是局部向量
东同构映射.

证明“因为

【095才62

E林林
玟
也是一个C“局部向量丛映射,从而T/是一向量丛同松映射,‖

现在可以证明

E二209才
}是胆的一个Cr坐标系,则T.ar二((T(Ua),TCR2)[aE心是
切丛TM的C!向量丛坐标系,从而TWU是C““*向量仰.此外,如
E沥李沥江
团0园人)
E肖技川E沥

E732述

是局部向量丛,并且

E|T'″(7〈U-】门7(U′〉)一工(8“秀|T′"(7(″″)门T【U′〉〉
(见定义B7下的性质(3)和(5)).注意8。8是9(U。)所在的
Page-633
E林E

现在把上面的结果推广到徽分流形之间的晔射广M-,
医罡医命述
件是合理的.

E尘史希7述355河招秉5
沥林

E
是满射,而且ker(TA7)作为TAM的闭子空间是可裂的.在流形
n
氓

定理B.23“设传,W是Banach流形,/:M一N是C“映射(r
E

【化;江i

0沥en2t
El标林技t
包吾际射A*(,03;

[是NE
【

证明“CU)与(2)的等价性可由定理B.20得到.(2与(3)的
等价性可由子流形定义A.9得到,注意Y是N中的开集【

E
E伟
E林i
的子流形.事实上,涉入的局部单射性不能保证整体的单射性.
例如,由平面极坐标方程r一cos2g定义的映射八S「-R是一个
E技
E政吊
i江吊p沥
(见图2).上面两例中的问题都出在R原点附近的邻域内、
Page-634
E

图1图2
丁D利用了从友诱导的拓扑与/CM)作为*的孙集而获得的承
扑是不同的,在前一种拓扑下,f(M)作成一微分流形,因而有时
称单射浸入的象集fCM)是一个漫人子流形;但是在后一种拓扑
2园
E江唐

E
相关拓扑下)的同胚,则称它是一个崴人.此时,称一M是N的
E志八汪

容易证明下面的

定理B.26“设广Mf一N是单值的漫入,若它是M到f)
E育

E胡2
0沥1河
一个C“子流形

53圆圆丿P孙32育5江江林仪|

E沥
E述林一
R为了的一切正则值的集合*则定理B:27有如下等价的陈述:

E沥2
Page-635
EEt

Banach空间五中的开二集9(7CUa)门T(Do))到自身的〇映射.
EEi2匹a北2圆政林s吊

E一
E沥沥李b

附注B.11“前面已经提到,流形间的切映射是Banach空间
中导算子的推广.闷此,有时把它狞为导昭射-设M和丶是充分光
玟
江仁ae
续求二防导晃射7(TA一72f7(TM3一(CTN),并可通推地定
Eolss

向量场及其流

王
6sneaeololsl

玟

E万
E
量场.M上一切Cr向量场的集合记为-gT“C2.

[达
E应泓
园
椿p

E林Ba李梁d
王
映射

E林林i

E肖吴0李水5沥技0一

E
E
Page-636
E江浩育水江页大

E

附录CThom横戬定理

在寻拖结构不稳定向量场的普适开折时,有时要确定开折中
的通有族(genericfamily),或称为一般族,非退化族.为此,零要利
用Thom模截定理(它是从著名的Sard定理导出的.事实上,为丁
玟
e
E

E′日

E
E

定友C.1(骏拓扑,即compact-open拓扑》设孔ECrCM,
),(U,9和(C7,办分别是肖和义的容许坐标卡;令史口口是紧
E林伟关途技5江2扬广

ac

E0

E

E红L木c
包含有限个这种集司交集的集合都是了的一个邻域.所得的拓扑
室间记为Cr(M,N).

E出伟c
{GUn,8o|aE4}是余的一个局部有限坐标系,即M的每一点都
有一个邹域,它只与有限个Dn相京;记2一(KaleE4},Ku人
Ue是胡上的紧集;画二{CVaugp)|B}是丶的坐标系,对任一
D
Page-637
E78刑E

E才c木
d王tpu莲E林命
E
E
E0020[沥
技
E伟团沥林5罡伟技
朱5林a技伯
E
蛊0
p
含0的开区间,都有JCC厂且al,=.
取切上的坐标卡(C,9),poEU设向量场&:伦一TWU有局部
玟

表示相对照,就得到Banach空间中关于(p。)的微分方程初值闭
5

青E
命(201水
浩
玟
E

利用Banach空间中微分方程初值问题解的存在和唯一性定
理,可以得到

E一吴沥孝小
E.

E55林招人
沥达
出的切向量相君合,这与欧氏空间中向量场的流的概念相一致

[怀ue扬
Page-638
EE

E兰达
E
E许达一颂技一t
[一才
满足微分方程
D沥伟3
E林i李
E水
E梁y关达
、友一Tg*X(g(za)),乙E伊口巴
E沥g
E沥林吴八技
Da(o)一E0
E汀沥
u诊李才一
【吴2s沥技沥3振
它成为积分曲线的条件是

a
因此可以说,对于维流形上的向量场而言;其积分曲线的局部表
E
E
隐函数定理是徽分学中最重要的定理之一.设R“-~R“是
E伟c仪nt
Page-639
EE

E伟技
u

0

E不
d

L孝莲园口河
Et
沥

E
E余c江
2水吴
C“胥射在“无穷远“的伯质,强拓扑就成为需要的了,注意,在办ˇ
玟
E
集在其中稠密.这对研究通有性质是很重要的,如无牺别声明,
下文中都取强拓扑,并把CS(MM,简记为CrCM,MD.

U途i

一怀国0
1位7妙svdimM一十eo.则Gr(M,N)在Gr(MM,N)中稠密,其中
E

g扬玖河厂标
紧集5【

2
胚于一个C“微分流形,

0
它们C微分同胚,

定理C.7(Whitney定理〉设1口r口十co,则任何a维的C“
Page-640
E

微分流形都C7微分同胚于R*“「的一个闭子流形,【

E0河沥i沥
到C“,二不是一个很严重的事情.今后我们将经常作这样的假定.
虽然^维微分流形是维欧氏空间非常一舱的推广,但定理C.7说
basaeetupotey林水d

射式e6)流形

EE
E伟
E0

E吊
E应

林医一
t

显然,上面的定义与坐标卡的选取无关,从而也与口的选取无
关.因此记

E仪
E林c政b河c
记力(M,N)为全体从此到为的r-jet之集合,我们把一(M,
E沥沙技招朐
Epeuuepnutisgu
现在考虑一个特殊情形:M一R“,N一R“.此时简记
Ed
E5林L2育t5
式就给出了在z点的r-jet一种自然的表示.这个从R“到R“的多

项式眺射可由一在z点直到r阶(包括0阶)导算予所唯一决定.这
医江志
Page-641
E2标E

E
E门逊
E'

王2颂木朐
E胡2阮
s沥ct万玟唐
E

E沥林

这称为(局部)正则漫羞(投影).

一
E不
过坐标卡推广到微分草形上,并田此得到构造(或判断)于流形的
方法.对于无穷维的Banach空间,仅有DfCz)的单射(或满射)条“
件是不够的,要附加适当的可裂性条件.首先给出下面的

EL余林二t国
ap

E林L玟不沥
E
c
E
[人才
E

n
[述
E0水
[许芸
E伟沥扬
Ee沥
Page-642
E

E振目沥
E

医林
量空间.这说明,对Pr(m,中的每一个元素,都有且仅有一个
E木cl园
Et之
E林
[阮河
E伟
E
E
技t技李
Eddtotd
EdD沥t
氏空间同构,所以,yvCLD,V))就是一(M,N)的一个坐标卡,这
E
河t仪2
Eapyteu2沥河

Thom槲戢定理

Earitasese沥5
p

E吴i
E
空间中树子流形的机截性.

E北伟05
E技
E李述
Page-643
E附录B“切坤与切陋财,吊探埃及其旁,淑人不清盎

[仪2

E亚园芸人0技沥育
炼(CDJCuo))人七的代数与拓扑同构.令
E李李e林林沥动03河
E沥

E
E

0国
是口X卷一吊田P一丁的Cr徽分同背,由隐函数定理,存在开
水2E021
[7一技一7
[

寸
E肖口林
E8应n沥
E
n
E

E国0

E技月
EE

D[

0Dz/(吻)〕
0E
王
[2
EiSE
[吊小沥2
E林2
Page-644
EE

Etp
Ed化明,
E心c林5浩a

(D(一(TpM)十Tyek二TyeN,面东

0芸李
E怀吴沥

林
E江s

E
E应

102怡u园
EE
Et述仪沥e纳d

由于机蒙相交性与土标卡的选取无关,所以当史与丶都为有
E

英
Et江5脱
心点附近有局部坐标z「,.,z“,X在zo)点附近有局部坐标

E沥ui

沥扬2
区

E

江
E

E林0汀

E达
定理C.14CThom横戳定理).w“(M,Ny4)是CrCM,N)中

Esi沥
Page-645
E林E

现在把上面的结果推广到徽分流形之间的晔射广M-,
医罡医命述
件是合理的.

E尘史希7述355河招秉5
沥林

E
是满射,而且ker(TA7)作为TAM的闭子空间是可裂的.在流形
n
氓

定理B.23“设传,W是Banach流形,/:M一N是C“映射(r
E

【化;江i

0沥en2t
El标林技t
包吾际射A*(,03;

[是NE
【

证明“CU)与(2)的等价性可由定理B.20得到.(2与(3)的
等价性可由子流形定义A.9得到,注意Y是N中的开集【

E
E伟
E林i
的子流形.事实上,涉入的局部单射性不能保证整体的单射性.
例如,由平面极坐标方程r一cos2g定义的映射八S「-R是一个
E技
E政吊
i江吊p沥
(见图2).上面两例中的问题都出在R原点附近的邻域内、
Page-646
EE

果么是闭子流形,则它还是开的,

E
E志技e医
成为横截,而原来樟截的晔射休可保持横戳.因此“利用槲戳性可
e个中一

Thom横截定理可以推广到jet形式.这样就可把无限维空间
CrQd,中的通有性闭题,转化到有限维空间J7CM,N)中的模
截闭题`

国ut
水
E}

E技t朋

E河园
|

证明可见[Hirpp80一81]下面的定理对于确定子流形的余
维数是重要的.

E李
了与么横戳,则广“(4)是仁的子流形.如果4在丶中有有限余
维,则

codim(广1(4D)二eodim(4)。【
特别地,当人育一丶是浸草时,条件“历与横截“总是满趸
医
Page-647
E

图1图2
丁D利用了从友诱导的拓扑与/CM)作为*的孙集而获得的承
扑是不同的,在前一种拓扑下,f(M)作成一微分流形,因而有时
称单射浸入的象集fCM)是一个漫人子流形;但是在后一种拓扑
2园
E江唐

E
相关拓扑下)的同胚,则称它是一个崴人.此时,称一M是N的
E志八汪

容易证明下面的

定理B.26“设广Mf一N是单值的漫入,若它是M到f)
E育

E胡2
0沥1河
一个C“子流形

53圆圆丿P孙32育5江江林仪|

E沥
E述林一
R为了的一切正则值的集合*则定理B:27有如下等价的陈述:

E沥2
Page-648
医招怀生人

[A1]ArmoldVLGeometrieMethodsintheTheoryofOrdinaryDiflerental
E
E东不e
E
〖nouoe志怀坂园0技砂
E

E
E林日
E
ProgramforSciTranal,Wiley,1973
[AMRJAbrahamR,Marsden]E,RatiuTManilolds,TensorAnalysis,
E口
PublishingCompany,Inc,1983
E江技
ofthecoefticientsfromanequitbriamPositionalfocusorceotertype、
E沥沥E李应一
E
E
E沥
[BC2]一一.ThedynamicsoftheHtnonmag,AanMath。1991,133:73一
E
E
3印永X前日
[BL]BoninGLegaltJComparisiondelmethodedesconstantesdeLia-
PunovetlaMfreationdeHopf,Canad,Math、Bull1988,31(2):200
g
E林伟
Page-649
E江浩育水江页大

E

附录CThom横戬定理

在寻拖结构不稳定向量场的普适开折时,有时要确定开折中
的通有族(genericfamily),或称为一般族,非退化族.为此,零要利
用Thom模截定理(它是从著名的Sard定理导出的.事实上,为丁
玟
e
E

E′日

E
E

定友C.1(骏拓扑,即compact-open拓扑》设孔ECrCM,
),(U,9和(C7,办分别是肖和义的容许坐标卡;令史口口是紧
E林伟关途技5江2扬广

ac

E0

E

E红L木c
包含有限个这种集司交集的集合都是了的一个邻域.所得的拓扑
室间记为Cr(M,N).

E出伟c
{GUn,8o|aE4}是余的一个局部有限坐标系,即M的每一点都
有一个邹域,它只与有限个Dn相京;记2一(KaleE4},Ku人
Ue是胡上的紧集;画二{CVaugp)|B}是丶的坐标系,对任一
D
Page-650
E卷考文欣

吴
012技移2圆
E

[水
玟
Sov.1981,1+373一387GinEnglish》

[C]陈郎炎,含参数循分方程的周排解与极限环,数学学报,1963,13(42:

807一609

[李述扬

30刑c
Sinica(SeriesA7y1995,38:29一35

E
Springer-Verlag,1982,

[CLW]ChowShuiNee,LiChengzhi,WangDuoMormalFormsandBifar-

0沥
Press1994

[CMJ蔡暗林,马晔.广义Litnard方程的奇点的中心焦点判定闭题.浙江大
E技

[Ca]蔡腾林,二次系统研究近况.数学进展,1989,18(+5一21

[口水沥沥
E水园广

[CW]陈兰荪,王明淑,二次微分系统极限环的相对位置和数目数学学
E振E莲

[
玟

E
E玲汀沥5
DelaeH,Sarlescycleslinites,BullSoc,Math,Fr,1923,51+柯一
E
E

ECICL32SSL2EXISu
Page-651
EE

E伟技
u

0

E不
d

L孝莲园口河
Et
沥

E
E余c江
2水吴
C“胥射在“无穷远“的伯质,强拓扑就成为需要的了,注意,在办ˇ
玟
E
集在其中稠密.这对研究通有性质是很重要的,如无牺别声明,
下文中都取强拓扑,并把CS(MM,简记为CrCM,MD.

U途i

一怀国0
1位7妙svdimM一十eo.则Gr(M,N)在Gr(MM,N)中稠密,其中
E

g扬玖河厂标
紧集5【

2
胚于一个C“微分流形,

0
它们C微分同胚,

定理C.7(Whitney定理〉设1口r口十co,则任何a维的C“
Page-652
E
e
quadraticvectorfelds,J.ReineAngew、Math,1987,3825165一180
[DL]丁同仁,李承治.常微分方程敬程.北京:高等教育出版社,1991
[DLZ]DaumaortierF,LiChengzhi,ZhangZhi-Fen、Unfoldingofaquadreti
integrablesywtemWmithtmocentersandtwoumboundedhetroctinie
loops,Preprit,1996
CDRR1JDumortierF,RousearieR,RousseseC。Hilberth16thproblemfor
E圆圆一木
E

1994,786一133、
[DRSL]DumortierF,Roussare叉,SotomayorJGeneric3-parameterfamily

吴
Hnearpart,Thecepcaseofcodimension3,ErgodieTheoryendDy-
E
[DRS5]一一,Genetie3-parameterfamilyotplanarvectorfields,anfoldingof
E
tureNotesinMeth,1991,1480:1一164
CDz]杜之林,曾宪武.计算焦点量的一粤通推公式,科学通报,1994,39
Eb
Ecalle一丁Finiudedeseyoesbmites吊acotlero-sommationde
Dapplicationderetour,、LectureNotesinMath。1990,1455174一159
n
E
[F]冯贝名,临界情泓下奇环的猪定伯-数学学报,1990,33(1):113--134
[皇E育河
玟
Anal.1989,20:13一30
E
andBifurcationsafVWectorFields,NewYotk:SpringerVeriag,1983
E
E沥途浩浩
Page-653
E

微分流形都C7微分同胚于R*“「的一个闭子流形,【

E0河沥i沥
到C“,二不是一个很严重的事情.今后我们将经常作这样的假定.
虽然^维微分流形是维欧氏空间非常一舱的推广,但定理C.7说
basaeetupotey林水d

射式e6)流形

EE
E伟
E0

E吊
E应

林医一
t

显然,上面的定义与坐标卡的选取无关,从而也与口的选取无
关.因此记

E仪
E林c政b河c
记力(M,N)为全体从此到为的r-jet之集合,我们把一(M,
E沥沙技招朐
Epeuuepnutisgu
现在考虑一个特殊情形:M一R“,N一R“.此时简记
Ed
E5林L2育t5
式就给出了在z点的r-jet一种自然的表示.这个从R“到R“的多

项式眺射可由一在z点直到r阶(包括0阶)导算予所唯一决定.这
医江志
Page-654
E动考文献

E河

Math,1990,1455:160一196′

[Go]TouesoaBILSummaneuricorscovweterE
E沥E吴E
E

E

计

E沥一园一工刑
E

[Ha]HaysshiS.OnthesolutionofCtstabtlityconjectureforfloms、Preprint、

D
turbationsofquadretieHamitoniansystems,JDif,Eq、1994,113
y

Euulor吴林浩b沥沥工
Springer-Verlag,1976

[HL.zJ韩苗安,罗定军,朱循明,奇闯捉分支出极限环的唯一性(I.C王,

玟

LHmJ韩茂安.周期抢动系统的不变环面与亚调和解的分支.中国科学A
n

[Ho]HorozoyEVerealdeformetionsafeduivanantvectorficldsinthecaseof
E一
一192GinRussian)

[Biw]HuangWenzao,Thebifureationtheoryfornonlinearequations,Lecture
E
E

E

招
Ei

E沥林沥林口工一一力孙水门吊才才用河
E

E

E门林河

Sarveys1990,40148一200
Page-655
E

E振目沥
E

医林
量空间.这说明,对Pr(m,中的每一个元素,都有且仅有一个
E木cl园
Et之
E林
[阮河
E伟
E
E
技t技李
Eddtotd
EdD沥t
氏空间同构,所以,yvCLD,V))就是一(M,N)的一个坐标卡,这
E
河t仪2
Eapyteu2沥河

Thom槲戢定理

Earitasese沥5
p

E吴i
E
空间中树子流形的机截性.

E北伟05
E技
E李述
Page-656
EE

[助育2技
腾史当
E江英0胡
E
河
[JaJIakobsonMVAbsoiatelycontinuousinvariantmesstreforone-parame-
n
E

DJojJoyslPGeneratizedHopEbifurcationanditsdualgeneralizedhomoelinic
Hifurceation、SIAMJMath,1988,48,481一486
]KhovanslkeyAG,Realanalyticmanifoldswithfinitenesspropertiesand
complexAbetianintegrals,Funct,Anal,AppL1984,18:119一128
Ee

5兄口un胡y

tons,hiChineseMathematesinto21etCentury-,PekingUniversity
Press,1992.
[Le]李承治.关于平面二次系统的两个阿题.中国科学(A辑)+1982(12。
E
2胡玲王3江2埕u
thewealkened16thHilbertprablem,]MAA1995,190:489一516
[胡八沥河技
3
阮
玟
[URI]LiChengzhi,RouseeauC。Asystemwiththreeliniteyeseppearing
训sHopfbifureationanddying记ahomoclinicbiforcation,thecuspof
order4丁Diff,Eq,1989,78:132一167
[UR2]一一,Codimension2symmetrichomodiniebiforeation,Can,了
Math、1980,42:191一212
[Lnw]LiWeiga,ThebHforeationof“eighture“ofseparatizofsaddlewith
zeroseddlealueinthePlane,PreprintofPekingUniversity,Research
ReportNo46,1995
Page-657
EE

Etp
Ed化明,
E心c林5浩a

(D(一(TpM)十Tyek二TyeN,面东

0芸李
E怀吴沥

林
E江s

E
E应

102怡u园
EE
Et述仪沥e纳d

由于机蒙相交性与土标卡的选取无关,所以当史与丶都为有
E

英
Et江5脱
心点附近有局部坐标z「,.,z“,X在zo)点附近有局部坐标

E沥ui

沥扬2
区

E

江
E

E林0汀

E达
定理C.14CThom横戳定理).w“(M,Ny4)是CrCM,N)中

Esi沥
Page-658
312参考文献

E
E

i

玲
E
MaReRAproofaftheCtstabiityconjecttzeInst,Hautes,Sci,PubiL
E吴E
E
Hiamiltonignveotorfald,ErgodTh,吊Dynsm,Sys,1990,10:523
国J「
[Me]MourtadaA,Degenerateandnon-trivialhypetbolicpolyeycleswithwo
E吴园振n
E国园
河
[Ma]马知恩.种羯生怪学的数学建模与研究合肥:安微救育出版社,1996
[NS]Nowiki个,StrienYS,AbsclatelycontinuousinvariantmeasuresforC+
E育诊汀
E
E振
E
[
1988,22:72一78
[沥
EE沥弋2沥江10
yyslioyxXEHEAXLL3IIX335SLuy
n
[PTJPaisJ,TikensF,Hyperbolicityandsensitivechaoticdyaemicstho-
meotinicbifureationsCambridgeUniversityPress,1992
[Q]秽元助、微分方程所定义的积分曲缇,上下册,北京:科学出版社,
1956.1959【
3
E
Page-659
EE

果么是闭子流形,则它还是开的,

E
E志技e医
成为横截,而原来樟截的晔射休可保持横戳.因此“利用槲戳性可
e个中一

Thom横截定理可以推广到jet形式.这样就可把无限维空间
CrQd,中的通有性闭题,转化到有限维空间J7CM,N)中的模
截闭题`

国ut
水
E}

E技t朋

E河园
|

证明可见[Hirpp80一81]下面的定理对于确定子流形的余
维数是重要的.

E李
了与么横戳,则广“(4)是仁的子流形.如果4在丶中有有限余
维,则

codim(广1(4D)二eodim(4)。【
特别地,当人育一丶是浸草时,条件“历与横截“总是满趸
医
Page-660
医招怀生人

[A1]ArmoldVLGeometrieMethodsintheTheoryofOrdinaryDiflerental
E
E东不e
E
〖nouoe志怀坂园0技砂
E

E
E林日
E
ProgramforSciTranal,Wiley,1973
[AMRJAbrahamR,Marsden]E,RatiuTManilolds,TensorAnalysis,
E口
PublishingCompany,Inc,1983
E江技
ofthecoefticientsfromanequitbriamPositionalfocusorceotertype、
E沥沥E李应一
E
E
E沥
[BC2]一一.ThedynamicsoftheHtnonmag,AanMath。1991,133:73一
E
E
3印永X前日
[BL]BoninGLegaltJComparisiondelmethodedesconstantesdeLia-
PunovetlaMfreationdeHopf,Canad,Math、Bull1988,31(2):200
g
E林伟
Page-661
E313

E
E
choPaciticoed,Longman,ScientificandTechnial,PitmanResearch
NotesinMath,Series180,1987:377一385
b技
E莲E
E

RoassesuC,UniversaluntoldingfaSingularityfasymumetricvector

E林圆
EE

NotesinMath、1989,1455,334一354
e
tionstolanarqundraticgystems,Ann,Pclon,Math.1988,49:1一16
n
Phys1971,20:167一192
SingerDStableofbitsandbifurcationofmapsoftheinterval,SIAMAp-
E朋
[Sh]ShoashitaishvifANBfurcetionoftopalogicaltypeasngularpointsf
psrameterizedveetorfalda,Fumet,AnalAppl1972(2):169一170
E
E吴技唐E沥
[SHj]SijbrandJ,Propertiesofcentermantfolda,Trans,AMS1985,289:431
E
E园e招
353一371
[SH2]一一.Onthegenerationofaperiodicmotionfzomatrajectorydoubly
aspmptotictanedilibriamstateafsnddletype,MathUSSRSb1968,
沥江
[
neighborhoodcfaroughequitibriumstatefsaddls-fooustype,Math
E吴
[SJ]ShenJaqi,JingZhjun,Anewdetectingmethodaconditionsfortheex-
Page-662
E卷考文欣

吴
012技移2圆
E

[水
玟
Sov.1981,1+373一387GinEnglish》

[C]陈郎炎,含参数循分方程的周排解与极限环,数学学报,1963,13(42:

807一609

[李述扬

30刑c
Sinica(SeriesA7y1995,38:29一35

E
Springer-Verlag,1982,

[CLW]ChowShuiNee,LiChengzhi,WangDuoMormalFormsandBifar-

0沥
Press1994

[CMJ蔡暗林,马晔.广义Litnard方程的奇点的中心焦点判定闭题.浙江大
E技

[Ca]蔡腾林,二次系统研究近况.数学进展,1989,18(+5一21

[口水沥沥
E水园广

[CW]陈兰荪,王明淑,二次微分系统极限环的相对位置和数目数学学
E振E莲

[
玟

E
E玲汀沥5
DelaeH,Sarlescycleslinites,BullSoc,Math,Fr,1923,51+柯一
E
E

ECICL32SSL2EXISu
Page-663
EE

istenceafHopfbifurcationin“DynamicalSystetms“,NankaiSeies训
E朋一罡途
E吴5团一不院日

[芸

NonlinearScienceTheNextDecade,见*数学注林“,动力系统学的团
顾;重大闭题,失败前尝谈,1993(47:262一269,.

[Ss]史松龄.平面二次系统存在四个极限环的具体值子中国科学,1979
《LDy1081一1056

[TJTkensF,Forcedoseilhtionsandbiftrcstions:Applicationsofglobal
E

[TTYJThisullenP,TresserC+YoungL-S,PositiveLispunovexbonentfor

gereieone-parameterfamilesafunimodaltmape,Peprint
E林E
E吴一芸
E
[Va]VarchenkoANEstimateofthenumberofzerosotanAbetianintegeal
E莲
E技n
月
E
[Wd]WengDao。Anintroduetiontathenormaljormtheoryofcdinarydif-
ferentalequations,Adyances训Math,1990,19(1738一仁

[WiH]WigginsS8,IntroductiontoAppliedNonlinearDynamicalSystemsand
i

[Wia]一一.GlbalBifureationsandChaoe,AnalyticalMethode,NewYork
E

[W0王兰字,多峰晔射的动力学.北京大学博士论文,1996

[X]肖冬梅.一类余维3歌点坦平面向量杨的分支,中国科学CA辑)1993
0

[XY1叶咤谥等,极限环论,第二版,上海;上海科学技术出版社,1964

[Y2]叶彦谦,多项式徽分系统定性理论.上海;上海科学技术出版社,1995

E
Page-664
E
e
quadraticvectorfelds,J.ReineAngew、Math,1987,3825165一180
[DL]丁同仁,李承治.常微分方程敬程.北京:高等教育出版社,1991
[DLZ]DaumaortierF,LiChengzhi,ZhangZhi-Fen、Unfoldingofaquadreti
integrablesywtemWmithtmocentersandtwoumboundedhetroctinie
loops,Preprit,1996
CDRR1JDumortierF,RousearieR,RousseseC。Hilberth16thproblemfor
E圆圆一木
E

1994,786一133、
[DRSL]DumortierF,Roussare叉,SotomayorJGeneric3-parameterfamily

吴
Hnearpart,Thecepcaseofcodimension3,ErgodieTheoryendDy-
E
[DRS5]一一,Genetie3-parameterfamilyotplanarvectorfields,anfoldingof
E
tureNotesinMeth,1991,1480:1一164
CDz]杜之林,曾宪武.计算焦点量的一粤通推公式,科学通报,1994,39
Eb
Ecalle一丁Finiudedeseyoesbmites吊acotlero-sommationde
Dapplicationderetour,、LectureNotesinMath。1990,1455174一159
n
E
[F]冯贝名,临界情泓下奇环的猪定伯-数学学报,1990,33(1):113--134
[皇E育河
玟
Anal.1989,20:13一30
E
andBifurcationsafVWectorFields,NewYotk:SpringerVeriag,1983
E
E沥途浩浩
Page-665
城考文献

E职。
[

n
【
E
[Za1]ZhuDeming,Meiaikovvectorandheteroctiniemanifolds。Sciencein
D
E
E
[Z4s]一一.TranaersalhetrocinicorbitsingSneraldegenerateceses,Science
E
[zDHD]张芷芬,丁同仁,黄文灶,董镐善,徽分方程定性理论,北京:科学
出版社,1985
[Zg]张恭庆,临界点理论及其应用上海,上海科学技术出版社,1886
[Z永张镯粉,常微分方程儿何理论与分支闭题(修订本》,北京:北京大学出
版社,1987
0
ProgressinNaturalScience,1996,6(旦,401一407
[zQ]张锦炎,钱敏,徽分动力系统导引、北京,北京大学出版社,1991
[Zol]2oladekH,Outheversalityofceriainfamilyofvectorfeldsonthe
E
E唐0沥刑沥沥江目
E发压巩玟2吴y
[N]涨梅斯基BB,四十年李的苏联数学(1917一1957),常微分方程部分,
饶生惠译,北京:科学出版社,1960
[Zafl]ZhangZhifen,Ontheuniquenessolthelimitcyclesofsomenoninear
E
e
b
Dignardequations,ApplicableAnalysis,1986,2383一67.
gZze]张筑生,微分助力系统原理,北京,科学出版社,1987
Page-666
索

l

中文词条拳首字的笔下排列,西文开头的词条接字母颂序排列.

E

t

医
四

双公韶点

双曾闭辅

双曲不动点定理

分员

E

分盆值

分岔图

分岔曲线

分岔方程

分岔函数

分岔的余维

无穷阶非共振

无限C水平曲线

无限C垂直曲线

E

E

切向量

切空间

E

E

7,46一50,286
E
173

63
53

159

4

E

E

E

E
291292

E
3sE
EE
aE
对参数一致的Hop分岔定理82
p282
EE
边翔的水平部分159,198,200
边界的垂宇部分E
边弈条件E
主穗定方向E
E
E
10
165
15,85
10,23,114
E
146
Page-667
E动考文献

E河

Math,1990,1455:160一196′

[Go]TouesoaBILSummaneuricorscovweterE
E沥E吴E
E

E

计

E沥一园一工刑
E

[Ha]HaysshiS.OnthesolutionofCtstabtlityconjectureforfloms、Preprint、

D
turbationsofquadretieHamitoniansystems,JDif,Eq、1994,113
y

Euulor吴林浩b沥沥工
Springer-Verlag,1976

[HL.zJ韩苗安,罗定军,朱循明,奇闯捉分支出极限环的唯一性(I.C王,

玟

LHmJ韩茂安.周期抢动系统的不变环面与亚调和解的分支.中国科学A
n

[Ho]HorozoyEVerealdeformetionsafeduivanantvectorficldsinthecaseof
E一
一192GinRussian)

[Biw]HuangWenzao,Thebifureationtheoryfornonlinearequations,Lecture
E
E

E

招
Ei

E沥林沥林口工一一力孙水门吊才才用河
E

E

E门林河

Sarveys1990,40148一200
Page-668
E园E

自由返团E园E
a国巳唐
E万既
全局中心流形定理途8
E口圆英E
沥EE
E十画

颖玟
bE45
E通有族网
E116
E0E
b税资形
更荐法渥入

EE
E
E

坐标卡

u

移位映射

符号动加系统
E

E、

尿朝辅腑适开折

E焦炉敦
环E
河超稳定周婵软道
非游济环逗历
E园刹
Es
E

单峰眯射

仪

E

组歉点
Page-669
EE

[助育2技
腾史当
E江英0胡
E
河
[JaJIakobsonMVAbsoiatelycontinuousinvariantmesstreforone-parame-
n
E

DJojJoyslPGeneratizedHopEbifurcationanditsdualgeneralizedhomoelinic
Hifurceation、SIAMJMath,1988,48,481一486
]KhovanslkeyAG,Realanalyticmanifoldswithfinitenesspropertiesand
complexAbetianintegrals,Funct,Anal,AppL1984,18:119一128
Ee

5兄口un胡y

tons,hiChineseMathematesinto21etCentury-,PekingUniversity
Press,1992.
[Le]李承治.关于平面二次系统的两个阿题.中国科学(A辑)+1982(12。
E
2胡玲王3江2埕u
thewealkened16thHilbertprablem,]MAA1995,190:489一516
[胡八沥河技
3
阮
玟
[URI]LiChengzhi,RouseeauC。Asystemwiththreeliniteyeseppearing
训sHopfbifureationanddying记ahomoclinicbiforcation,thecuspof
order4丁Diff,Eq,1989,78:132一167
[UR2]一一,Codimension2symmetrichomodiniebiforeation,Can,了
Math、1980,42:191一212
[Lnw]LiWeiga,ThebHforeationof“eighture“ofseparatizofsaddlewith
zeroseddlealueinthePlane,PreprintofPekingUniversity,Research
ReportNo46,1995
Page-670
BE

EE
锋分绪构E
徽分同酗E
iE

十四画

E
稳定周期扔道
E

十五面
E四
EE
E′国
E216
E46,304

E99v0L,106一109
E沥E
Bogdanov-Takens系统47,109,130
Biekhof-Smale定理E
EE
EE
[159
C垂直带域E
E

E0
ED
E6
Hilbert第16问题E
E园
EE

引

吴234
【吴4
0园
00
江0
防细焦点[
Lebesgue测度242
EE
Lebestgoe全稿点E
Liapunov系数法78,79
E
Maigrange定理18
E河刑90,97,108
E282
E

Pichfork分员

Pieari-Fuchs方程

E

Pioncare分纭

Pliss约化原理

E

Smale马路

EE
EtE
【扬33斧江江E
0形达江E
E育江162,168
E159
E228
Page-671

Page-672
京)112号

伟习分为六耿,各章内容分泉是;基本概伊积准备知识,帝见的局部与非局部分
岔,儿兴余绵2的平面暨圭分岔,双丞不劲炕及驯晃存在定璐,空间中友曲驿炳的固
E的铁章之后,都配备了一定数最的习

E
木书可作为高等学校数学考业高年镁本移生的法修谅教材,或相关专业研究生的

药磁课教材以司供莲振了焊分盆理论这门学科的学生.敏帝或科技人员作为坂考书.

固书在版编目(CIP)数据

i
育出版社,1997
E

a途的

旦

高等救育出版社出版
北京沙滩后街55号
E2玟
新华书店总店北京发行所发行
固货工业出版社印刷厂印刷

E

玟
E的1997年10月第1次卯刹
印数0001-1715
定价10.20元

凡购买高等敏育出版社的图书,如有狱页.倒页,育页等
E

、圃版扑所有,不得番印
Page-673

Page-674

Page-675

Page-676
312参考文献

E
E

i

玲
E
MaReRAproofaftheCtstabiityconjecttzeInst,Hautes,Sci,PubiL
E吴E
E
Hiamiltonignveotorfald,ErgodTh,吊Dynsm,Sys,1990,10:523
国J「
[Me]MourtadaA,Degenerateandnon-trivialhypetbolicpolyeycleswithwo
E吴园振n
E国园
河
[Ma]马知恩.种羯生怪学的数学建模与研究合肥:安微救育出版社,1996
[NS]Nowiki个,StrienYS,AbsclatelycontinuousinvariantmeasuresforC+
E育诊汀
E
E振
E
[
1988,22:72一78
[沥
EE沥弋2沥江10
yyslioyxXEHEAXLL3IIX335SLuy
n
[PTJPaisJ,TikensF,Hyperbolicityandsensitivechaoticdyaemicstho-
meotinicbifureationsCambridgeUniversityPress,1992
[Q]秽元助、微分方程所定义的积分曲缇,上下册,北京:科学出版社,
1956.1959【
3
E
Page-677
E313

E
E
choPaciticoed,Longman,ScientificandTechnial,PitmanResearch
NotesinMath,Series180,1987:377一385
b技
E莲E
E

RoassesuC,UniversaluntoldingfaSingularityfasymumetricvector

E林圆
EE

NotesinMath、1989,1455,334一354
e
tionstolanarqundraticgystems,Ann,Pclon,Math.1988,49:1一16
n
Phys1971,20:167一192
SingerDStableofbitsandbifurcationofmapsoftheinterval,SIAMAp-
E朋
[Sh]ShoashitaishvifANBfurcetionoftopalogicaltypeasngularpointsf
psrameterizedveetorfalda,Fumet,AnalAppl1972(2):169一170
E
E吴技唐E沥
[SHj]SijbrandJ,Propertiesofcentermantfolda,Trans,AMS1985,289:431
E
E园e招
353一371
[SH2]一一.Onthegenerationofaperiodicmotionfzomatrajectorydoubly
aspmptotictanedilibriamstateafsnddletype,MathUSSRSb1968,
沥江
[
neighborhoodcfaroughequitibriumstatefsaddls-fooustype,Math
E吴
[SJ]ShenJaqi,JingZhjun,Anewdetectingmethodaconditionsfortheex-
Page-678
EE

istenceafHopfbifurcationin“DynamicalSystetms“,NankaiSeies训
E朋一罡途
E吴5团一不院日

[芸

NonlinearScienceTheNextDecade,见*数学注林“,动力系统学的团
顾;重大闭题,失败前尝谈,1993(47:262一269,.

[Ss]史松龄.平面二次系统存在四个极限环的具体值子中国科学,1979
《LDy1081一1056

[TJTkensF,Forcedoseilhtionsandbiftrcstions:Applicationsofglobal
E

[TTYJThisullenP,TresserC+YoungL-S,PositiveLispunovexbonentfor

gereieone-parameterfamilesafunimodaltmape,Peprint
E林E
E吴一芸
E
[Va]VarchenkoANEstimateofthenumberofzerosotanAbetianintegeal
E莲
E技n
月
E
[Wd]WengDao。Anintroduetiontathenormaljormtheoryofcdinarydif-
ferentalequations,Adyances训Math,1990,19(1738一仁

[WiH]WigginsS8,IntroductiontoAppliedNonlinearDynamicalSystemsand
i

[Wia]一一.GlbalBifureationsandChaoe,AnalyticalMethode,NewYork
E

[W0王兰字,多峰晔射的动力学.北京大学博士论文,1996

[X]肖冬梅.一类余维3歌点坦平面向量杨的分支,中国科学CA辑)1993
0

[XY1叶咤谥等,极限环论,第二版,上海;上海科学技术出版社,1964

[Y2]叶彦谦,多项式徽分系统定性理论.上海;上海科学技术出版社,1995

E
Page-679
城考文献

E职。
[

n
【
E
[Za1]ZhuDeming,Meiaikovvectorandheteroctiniemanifolds。Sciencein
D
E
E
[Z4s]一一.TranaersalhetrocinicorbitsingSneraldegenerateceses,Science
E
[zDHD]张芷芬,丁同仁,黄文灶,董镐善,徽分方程定性理论,北京:科学
出版社,1985
[Zg]张恭庆,临界点理论及其应用上海,上海科学技术出版社,1886
[Z永张镯粉,常微分方程儿何理论与分支闭题(修订本》,北京:北京大学出
版社,1987
0
ProgressinNaturalScience,1996,6(旦,401一407
[zQ]张锦炎,钱敏,徽分动力系统导引、北京,北京大学出版社,1991
[Zol]2oladekH,Outheversalityofceriainfamilyofvectorfeldsonthe
E
E唐0沥刑沥沥江目
E发压巩玟2吴y
[N]涨梅斯基BB,四十年李的苏联数学(1917一1957),常微分方程部分,
饶生惠译,北京:科学出版社,1960
[Zafl]ZhangZhifen,Ontheuniquenessolthelimitcyclesofsomenoninear
E
e
b
Dignardequations,ApplicableAnalysis,1986,2383一67.
gZze]张筑生,微分助力系统原理,北京,科学出版社,1987
Page-680
索

l

中文词条拳首字的笔下排列,西文开头的词条接字母颂序排列.

E

t

医
四

双公韶点

双曾闭辅

双曲不动点定理

分员

E

分盆值

分岔图

分岔曲线

分岔方程

分岔函数

分岔的余维

无穷阶非共振

无限C水平曲线

无限C垂直曲线

E

E

切向量

切空间

E

E

7,46一50,286
E
173

63
53

159

4

pE

E

E

E
291292

E
3sE
EE
aE
对参数一致的Hop分岔定理82
p282
EE
边翔的水平部分159,198,200
边界的垂宇部分E
边弈条件E
主穗定方向E
E
E
10
165
15,85
10,23,114
E
146
Page-681
E园E

自由返团E园E
a国巳唐
E万既
全局中心流形定理途8
E口圆英E
沥EE
E十画

颖玟
bE45
E通有族网
E116
E0E
b税资形
更荐法渥入

EE
E
E

坐标卡

u

移位映射

符号动加系统
E

E、

尿朝辅腑适开折

E焦炉敦
环E
河超稳定周婵软道
非游济环逗历
E园刹
Es
E

单峰眯射

仪

E

组歉点
Page-682
BE

EE
锋分绪构E
徽分同酗E
iE

十四画

E
稳定周期扔道
E

十五面
E四
EE
E′国
E216
E46,304

E99v0L,106一109
E沥E
Bogdanov-Takens系统47,109,130
Biekhof-Smale定理E
EE
EE
[159
C垂直带域E
E

E0
ED
E6
Hilbert第16问题E
E园
EE

引

吴234
【吴4
0园
00
江0
防细焦点[
Lebesgue测度242
EE
Lebestgoe全稿点E
Liapunov系数法78,79
E
Maigrange定理18
E河刑90,97,108
E282
E

Pichfork分员

Pieari-Fuchs方程

E

Pioncare分纭

Pliss约化原理

E

Smale马路

EE
EtE
【扬33斧江江E
0形达江E
E育江162,168
E159
E228
Page-683

Page-684
京)112号

伟习分为六耿,各章内容分泉是;基本概伊积准备知识,帝见的局部与非局部分
岔,儿兴余绵2的平面暨圭分岔,双丞不劲炕及驯晃存在定璐,空间中友曲驿炳的固
E的铁章之后,都配备了一定数最的习

E
木书可作为高等学校数学考业高年镁本移生的法修谅教材,或相关专业研究生的

药磁课教材以司供莲振了焊分盆理论这门学科的学生.敏帝或科技人员作为坂考书.

固书在版编目(CIP)数据

i
育出版社,1997
E

a途的

旦

高等救育出版社出版
北京沙滩后街55号
E2玟
新华书店总店北京发行所发行
固货工业出版社印刷厂印刷

E

玟
E的1997年10月第1次卯刹
印数0001-1715
定价10.20元

凡购买高等敏育出版社的图书,如有狱页.倒页,育页等
E

、圃版扑所有,不得番印
Page-685

Page-686

Page-687
